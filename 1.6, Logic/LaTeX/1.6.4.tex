\documentclass[a4paper, 12pt]{article}
\usepackage[utf8]{inputenc}
\usepackage[english]{babel}
\usepackage{amssymb, amsmath, amsthm}
\theoremstyle{plain}
\newtheorem*{theorem*}{Theorem}
\newtheorem{theorem}{Theorem}

\usepackage{mathtools}
\renewcommand\qedsymbol{$\blacksquare$}

\begin{document}
	
	\begin{theorem*}[1.6.4]
		The additive inverse of an even number is an even number.
	\end{theorem*}
	
	\begin{proof}
		Let $n$ be an even number. There exists an integer $k$ such that 
		\newline $n = 2k$, by definition. The additive inverse of $n$ is $-2k$. By commutativity of addition, $-2k = 2(-k)$, and $-k$ is an integer because the product of integers is an integer $\therefore$ $\space$ $-n$ is an even number by definition.
	\end{proof}

\end{document}
