\documentclass[a4paper, 12pt]{article}
\usepackage[utf8]{inputenc}
\usepackage[english]{babel}
\usepackage{amssymb, amsmath, amsthm}
\theoremstyle{plain}
\newtheorem*{theorem*}{Theorem}
\newtheorem{theorem}{Theorem}

\usepackage{mathtools}
\renewcommand\qedsymbol{$\blacksquare$}

\begin{document}
	
	\begin{theorem*}[1.6.8]
		If n is a perfect square, then n + 2 is not a perfect square.
	\end{theorem*}
	
	\begin{proof}
		Assume the contrapositive, that $n + 2$ is a perfect square. By definition there exists an integer $m$ such that $mm = n + 2$. Obviously $n = mm - 2$. This definition yields the following relationship $\sqrt{n} = \frac{n}{m} = m - \frac{2}{m}$.
		
		\noindent \newline $(i)$ If $m > 2$ or $m < -2$, then clearly $\frac{2}{m} \notin \mathbb{Z}$ because $0 < \frac{2}{m} < 1$ or $0 > \frac{2}{m} > -1$; meaning that $n$ has a rational root not in integers. Thus proving the contrapositive for all roots $m > 2$ and $m < -2$.
		
		\noindent \newline $(ii)$ If $m \in \{-2, -1, 1, 2\}$, then $\sqrt{n} = (-1 \lor 1)$ meaning that $n + 2 = 3$,  contradicting the assumption that $n + 2$ is a perfect square. 
		
		\noindent \newline $(iii)$ Suppose $m = 0$. Then quite trivially $\sqrt{n + 2}$ is identical to $\sqrt{2}$. Again, contradicting the contrapositive hypothesis. 
		
		\noindent \newline $\therefore$ \space if $n$ is a perfect square, then $n + 2$ is not a perfect square.
	\end{proof}

\end{document}
