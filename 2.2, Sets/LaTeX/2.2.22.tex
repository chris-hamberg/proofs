\documentclass[a4paper, 12pt]{article}
\usepackage[utf8]{inputenc}
\usepackage[english]{babel}
\usepackage{amssymb, amsmath, amsthm}
\theoremstyle{plain}
\newtheorem*{theorem*}{Theorem}
\newtheorem{theorem}{Theorem}

\usepackage{mathtools}
\renewcommand\qedsymbol{$\blacksquare$}

\begin{document}
	
	\begin{theorem*}[2.2.22]
		Let A, B, and C be sets. $A \cap (B \cap C) = (A \cap B) \cap C$, such that set intersection is associative.
	\end{theorem*}
	
	\begin{proof}
		Let $x$ be an element in $A \cap (B \cap C)$. The logical definition being \newline $(x \in A) \land [(x \in B) \land (x \in C)]$. It trivially follows from the logical law of association for conjunction that $[(x \in A) \land (x \in B)] \land (x \in C)$. Therefore by definition $x$ is an element of $(A \cap B) \cap C$.
		
		Proving the converse, let $x$ be an element in $(A \cap B) \cap C$. The logical definition being $[(x \in A) \land (x \in B)] \land (x \in C)$. It trivially follows from the logical law of association for conjunction that \newline $(x \in A) \land [(x \in B) \land (x \in C)]$. Therefore by definition $x$ is an element of $A \cap (B \cap C)$.
		
		Since $A \cap (B \cap C) \subseteq (A \cap B) \cap C$ and $(A \cap B) \cap C \subseteq A \cap (B \cap C)$ it immediately follows from the definition for set equality that \newline $A \cap (B \cap C) = (A \cap B) \cap C$. Thus, the intersection of three sets is associative.
	\end{proof}

\end{document}
