\documentclass[a4paper, 12pt]{article}
\usepackage[utf8]{inputenc}
\usepackage[english]{babel}
\usepackage{amssymb, amsmath, amsthm}
\theoremstyle{plain}
\newtheorem*{theorem*}{Theorem}
\newtheorem{theorem}{Theorem}

\usepackage{mathtools}
\renewcommand\qedsymbol{$\blacksquare$}

\begin{document}
	
	\begin{theorem*}[2.2.9a]
		Let A be a set with universal set U. $A \cup \overline{A} = U$.
	\end{theorem*}
	
	\begin{proof}
		Let $x$ be an element in $A \cup \overline{A}$. By the definition for set union and the complement of sets we have $(x \in A) \lor (x \notin A)$. The right-hand side of this disjunction is equivalent to $x \in (U \cap \overline{A})$ according to the definition for set complementation. That is, $(x \in U) \land (x \notin A)$. So the original disjunction is the same as $(x \in A) \lor [(x \in U) \land (x \notin A)]$. We must distribute the left-hand side of this disjunction over the conjunction occurring in the right-hand side. We get $[(x \in A) \lor (x \in U)] \land [(x \in A) \lor (x \notin A)]$. By the logical law of negation the identity for the right-hand side of this conjunction is true. The left-hand side of this conjunction is dominated by $U$, according to Theorem 2.2.7a. Therefore the statement $x \in A \cup \overline{A}$ can be equivalently stated as $(x \in U) \land T$; the logical identity of which is $x \in U$. Thus proving the set complementation law for the union of sets, $A \cup \overline{A} = U$.

		
	\end{proof}

\end{document}
