\documentclass[a4paper, 12pt]{article}
\usepackage[utf8]{inputenc}
\usepackage[english]{babel}
\usepackage{amssymb, amsmath, amsthm}
\theoremstyle{plain}
\newtheorem*{theorem*}{Theorem}
\newtheorem{theorem}{Theorem}

\usepackage{mathtools}
\renewcommand\qedsymbol{$\blacksquare$}

\begin{document}
	
	\begin{theorem*}[2.3.25]
		Let $f$ be a function $f: \mathbb{R} \implies \mathbb{R}$ defined by $f(x) = |x|$. $f(x)$ is not invertible.
	\end{theorem*}
	
	\begin{proof}
		Let $x$ be a postive real number. $f(-x) = f(x) = x$. If $f$ had an inverse then $f^{-1}(x) = x = -x$, but this is not a function by definition. Also, $f^{-1}$ is not a function by contradiction since $\lnot \exists x ((x \in \mathbb{R}) \land (x = -x))$. Thus, $f$ is not a bijection, and $f$ is not invertible.
	\end{proof}

\end{document}
