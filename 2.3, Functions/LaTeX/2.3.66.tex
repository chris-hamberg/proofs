\documentclass[a4paper, 12pt]{article}
\usepackage[utf8]{inputenc}
\usepackage[english]{babel}
\usepackage{amssymb, amsmath, amsthm}
\theoremstyle{plain}
\newtheorem*{theorem*}{Theorem}
\newtheorem{theorem}{Theorem}

\usepackage{mathtools}
\renewcommand\qedsymbol{$\blacksquare$}
\DeclarePairedDelimiter{\floor}{\lfloor}{\rfloor}
\DeclarePairedDelimiter{\ceil}{\lceil}{\rceil}

\begin{document}
	
	\begin{theorem*}[2.3.66]
		Let f be the invertible function $f: Y \implies Z$, and let \newline g be the invertible function $g: X \implies Y$. The inverse of the composition $f \circ g$ is given by $(f \circ g)^{-1} = g^{-1} \circ f^{-1}$.
	\end{theorem*}
	
	\begin{proof}
		By Theorem 2.3.29a and Theorem 2.3.29b, and by the definition for bijective functions, $f \circ g$ is invertible. Thus, $(f \circ g)^{-1} \circ (f \circ g) = \iota_X$. 
		
		What remains to be  determined is whether $(g^{-1} \circ f^{-1}) \circ (f \circ g) = \iota_X$. Let $x$ be an element in the domain of $g$ such that $((g^{-1} \circ f^{-1}) \circ (f \circ g))(x) = x$. By the definition for the composition of functions, that is $g^{-1}(f^{-1}(f(g(x)))) = x$. Clearly, $(g^{-1} \circ f^{-1}) \circ (f \circ g) = \iota_X$.
		
		Thus, the inverse of the composition $f \circ g$ is indeed given by \newline $(f \circ g)^{-1} = g^{-1} \circ f^{-1}$.
		
	\end{proof}

\end{document}
