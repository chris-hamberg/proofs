\documentclass[a4paper, 12pt]{article}
\usepackage[utf8]{inputenc}
\usepackage[english]{babel}
\usepackage{amssymb, amsmath, amsthm}
\theoremstyle{plain}
\newtheorem*{theorem*}{Theorem}
\newtheorem{theorem}{Theorem}

\usepackage{mathtools}
\renewcommand\qedsymbol{$\blacksquare$}
\DeclarePairedDelimiter{\floor}{\lfloor}{\rfloor}
\DeclarePairedDelimiter{\ceil}{\lceil}{\rceil}

\begin{document}
	
	\begin{theorem*}[2.3.70c]
		Let x be a real number. $\ceil{\ceil{\frac{x}{2}} \div 2} = \ceil{\frac{x}{4}}$.
	\end{theorem*}
	
	\begin{proof}
		Let $n$ be an integer satisfying the properties for ceiling functions with respect to $x$ such that
		$\ceil{\frac{x}{4}} = n$. Thus establishes the fact, $4n - 4 < x \le 4n$. We shall proceed by analyzing the statement $\ceil{\ceil{\frac{x}{2}} \div 2} = n$. If $\ceil{\ceil{\frac{x}{2}} \div 2} = \ceil{\frac{x}{4}}$ is true, then $\ceil{\ceil{\frac{x}{2}} \div 2} = n$ will be defined, by the properties of ceiling functions, as $4n - 4 < x \le 4n$; since this is the case for $\ceil{\frac{x}{4}} = n$.
		\newline
		\newline
		First $\ceil{\ceil{\frac{x}{2}} \div 2} = n$ says that $2n - 2 < \ceil{\frac{x}{2}} \le 2n$. By the properties for ceiling functions, this is equivalently stated as $(i)$ $\ceil{\frac{x}{2}} = 2n - 1$, or (logical) $(ii)$ $\ceil{\frac{x}{2}} = 2n$.
		\newline
		\newline
		\indent $(i)$ $\ceil{\frac{x}{2}} = 2n - 1$, by the properties for ceiling functions, states that \indent $4n - 4 < x \le 4n - 2$.
		\newline
		\newline
		\indent $(ii)$ $\ceil{\frac{x}{2}} = 2n$, by the properties for ceiling functions, states that \newline \indent $4n - 2 < x \le 4n$.
		\newline
		\newline
		The statement $4n - 4 < x \le 4n - 2$ or (logical) $4n - 2 < x \le 4n$ is the same as 
		$4n - 4 < x \le 4n$. Thus, $\ceil{\ceil{\frac{x}{2}} \div 2} = n$, is indeed defined by $4n - 4 < x \le 4n$. Because both sides of the equation have the same definition, the statement $\ceil{\ceil{\frac{x}{2}} \div 2} = \ceil{\frac{x}{4}}$, is true.
		
	\end{proof}

\end{document}
