\documentclass[a4paper, 12pt]{article}
\usepackage[utf8]{inputenc}
\usepackage[english]{babel}
\usepackage{amssymb, amsmath, amsthm}
\theoremstyle{plain}
\newtheorem*{theorem*}{Theorem}
\newtheorem{theorem}{Theorem}

\usepackage{mathtools}
\renewcommand\qedsymbol{$\blacksquare$}
\DeclarePairedDelimiter{\floor}{\lfloor}{\rfloor}
\DeclarePairedDelimiter{\ceil}{\lceil}{\rceil}

\begin{document}
	
	\begin{theorem*}[2.4.22]
		The sum of squares from 1 to n is $\frac{n(n+1)(2n+1)}{6}$.
	\end{theorem*}
	
	\begin{proof}
		Let $n$ and $k$ be integers. By the Binomial Theorem, \newline $(k-1)^{3} = k^{3} - 3k^{2} + 3k - 1$. This means that $k^{3} - (k - 1)^{3} = 3k^{2} - 3k + 1$. By Theorem 2.4.19, $n^{3} = \sum_{k=1}^{n} k^{3} - (k - 1)^{3}$. Thus, $n^{3} = \sum_{k=1}^{n} 3k^{2} - 3k + 1$. This statement is equivalent to $n^{3} = (3\sum_{k=1}^{n} k^{2}) - (3\sum_{k=1}^{n} k) + (\sum_{k=1}^{n} 1)$. By Theorem 2.4.21a, and by Theorem 2.4.21b, that is \newline  $n^{3} = (3\sum_{k=1}^{n} k^{2}) - 3 \frac{n(n+1)}{2} + n$; or rather $n^{3} + 3\frac{n(n+1)}{2} - n = 3\sum_{k=1}^{n} k^{2}$. Eliminating the coefficient $3$ from the right-hand side, all that is left to do is to simplify the left-hand side $\frac{1}{3}[n^{3} + 3\frac{n(n+1)}{2} - n] = \sum_{k=1}^{n} k^{2}$. On the left-hand side we have $\frac{2n^{3} + 3n^{2} + 3n - 2n}{6} = \frac{2n^{3} + 3n^{2} + n}{6}$. Factoring $\frac{1}{6}n$ out of this expression gives $\frac{1}{6}n(2n^{2} + 3n + 1) = \frac{1}{6}n(2n^{2} + 2n + n + 1)$. Factoring the first two terms in the sum, $\frac{1}{6}n[2n(n + 1) + (n + 1)]$. The simplification process is completed by factoring $(n+1)$ out of the sum, $\frac{1}{6}n(n+1)(2n+1)$. That is, $\sum_{k=1}^{n} k^{2} = \frac{n(n+1)(2n+1)}{6}$.
	\end{proof}

\end{document}
