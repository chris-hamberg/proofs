\documentclass[a4paper, 12pt]{article}
\usepackage[utf8]{inputenc}
\usepackage[english]{babel}
\usepackage{amssymb, amsmath, amsthm}
\theoremstyle{plain}
\newtheorem*{theorem*}{Theorem}
\newtheorem{theorem}{Theorem}

\usepackage{mathtools}
\renewcommand\qedsymbol{$\blacksquare$}
\DeclarePairedDelimiter{\floor}{\lfloor}{\rfloor}
\DeclarePairedDelimiter{\ceil}{\lceil}{\rceil}

\begin{document}
	
	\begin{theorem*}[2.4.40]
		The union of two countable sets is countable.
	\end{theorem*}
	
	\begin{proof}
		By cases. Let $A$, and $B$ be countable sets. There are three cases that must be considered. $(i)$ $A$ and $B$ are finite, $(ii)$ exclusively $A$ or $B$ is finite and the other is countably infinite, $(iii)$ $A$ and $B$ are both countably infinite.
		\newline
		\newline
		$(i)$ Suppose $A$ and $B$ are finite. There exist natural numbers $m$, and $n$ such that $|A| = m$ and $|B| = n$. The maximum cardinality for $A \cup B$ occurs when $A$ and $B$ are disjoint, where the cardinality is $m + n$. $m+n$ is a natural number less than $\aleph_0$. Thus, $A \cup B$ is finite and countable by definition. 
		\newline
		\newline
		$(ii)$ Without loss of generality suppose $A$ is finite with cardinality $n$, and $B$ is countably infinite. It must be that a sequence exists $\{a_i\} = \{a_0, a_1, \dots a_n\}$ containing all elements in $A$. Since a bijection exists between $B$ and $\mathbb{N}$ by the definition for countability, a sequence exists $\{b_i\} = \{b_0, b_1, b_2, \dots \}$ containing all elements in $B$. Clearly, for the union of $A$ and $B$ a sequence exists $\{c_i\} = \{a_0, a_1, \dots, a_n, b_0, b_1, b_2, \dots \}$. Infinite sequences are countable by definition, so $A \cup B$ is a countably infinite set.
		\newline
		\newline
		$(iii)$ Suppose $A$ and $B$ are infinitely countable sets. Since the set cardinalities are $\aleph_0$, $A$ and $B$ are bijective with $\mathbb{N}$. Thus, the elements in $A$ can be ordered by the sequence $\{a_i\} = \{a_0, a_1, a_2, \dots\}$, and the elements in $B$ can be ordered by the sequence $\{b_i\} = \{b_0, b_1, b_2, \dots \}$. The union of $A$ and $B$ can be ordered by the sequence $\{c_i\} = \{a_0, b_0, a_1, b_1, a_2, b_2, \dots\}$. Thus a bijection exists between the union of $A$ and $B$ and the cardinality of the union is $\aleph_0$.
	\end{proof}

\end{document}
