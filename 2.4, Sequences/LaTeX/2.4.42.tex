\documentclass[a4paper, 12pt]{article}
\usepackage[utf8]{inputenc}
\usepackage[english]{babel}
\usepackage{amssymb, amsmath, amsthm}
\theoremstyle{plain}
\newtheorem*{theorem*}{Theorem}
\newtheorem{theorem}{Theorem}

\usepackage{mathtools}
\renewcommand\qedsymbol{$\blacksquare$}
\DeclarePairedDelimiter{\floor}{\lfloor}{\rfloor}
\DeclarePairedDelimiter{\ceil}{\lceil}{\rceil}
\DeclarePairedDelimiter{\ord}{\langle}{\rangle}

\begin{document}
	
	\begin{theorem*}[2.4.42]
		The cardinality of $\mathbb{Z^{+}} \times \mathbb{Z^{+}}$ is aleph null.
	\end{theorem*}
	
	\begin{proof}
		$\mathbb{Z^{+}} \times \mathbb{Z^{+}}$ is defined as $\{\ord{x, y} | (x \in \mathbb{Z^{+}}) \land (y \in \mathbb{Z^{+}})\}$. Since $x$ and $y$ are positive integers, for every ordered pair $\ord{x, y}$ in $\mathbb{Z^{+}} \times \mathbb{Z^{+}}$, $\ord{x, y}$ exists if and only if the rational number $\frac{x}{y}$ exists. Thus, $\frac{x}{y}$ exists, and all elements in $\mathbb{Z^{+}} \times \mathbb{Z^{+}}$ can be represented by the two dimensional list
		\newline \indent $\ord{1,1} \iff \frac{1}{1}$, $\ord{1,2} \iff \frac{1}{2}$, $\ord{1,3} \iff \frac{1}{3}$, $\dots$
		\newline \indent $\ord{2,1} \iff \frac{2}{1}$, $\ord{2,2} \iff \frac{2}{2}$, $\ord{2,3} \iff \frac{2}{3}$, $\dots$
		\newline \indent $\ord{3,1} \iff \frac{3}{1}$, $\ord{3,2} \iff \frac{3}{2}$, $\ord{3, 3} \iff \frac{3}{3}$, $\dots$
		\newline \indent \indent $\vdots$
		\newline 
		The hypotheses in the biconditional converse statements for each list entry are the list elements in the proof for the countability of rational numbers. That means $\mathbb{Z^{+}} \times \mathbb{Z^{+}}$ is countable if and only if the rational numbers are countable. We know the rational numbers are countable. Therefore the cardinality of $\mathbb{Z^{+}} \times \mathbb{Z^{+}}$ is $\aleph_0$.
	\end{proof}

\end{document}
