\documentclass[a4paper, 12pt]{article}
\usepackage[utf8]{inputenc}
\usepackage[english]{babel}
\usepackage{amssymb, amsmath, amsthm}
\theoremstyle{plain}
\newtheorem*{theorem*}{Theorem}
\newtheorem{theorem}{Theorem}

\usepackage{mathtools}
\renewcommand\qedsymbol{$\blacksquare$}

\setcounter{page}{7}

\begin{document}

\section*{2.2 Set Operations}
\begin{center}
    \rule{5.4in}{1pt}
\end{center}


% ============================== Theorem 2.2.5 ================================
\begin{theorem*}[\textbf{2.2.5}]
    Let A be a subset of U. $\overline{\overline{A}} = A$.
\end{theorem*}

\begin{proof}
    Suppose $x$ is an element in $\overline{\overline{A}}$. By the definition of set complementation 
    $x \in \lnot \overline{A}$, and of course by the same reasoning $x \in \lnot ( \lnot A)$. By the 
    logical law of double negation $x \in A$. Thus it follows directly that 
    $\overline{\overline{A}} = A$.
\end{proof}
\begin{center}
    \rule{5.4in}{1pt}
\end{center}


% ============================= Theorem 2.2.6a ================================
\begin{theorem*}[\textbf{2.2.6a}]
    Let A be a set. The set identity for A is $A \cup \emptyset = A$.
\end{theorem*}

\begin{proof}
    Let $x$ be an element in $A \cup \emptyset$. By the definition of set union \newline 
    $(x \in A) \lor (x \in \emptyset)$. But $x \in \emptyset$ is $\bot$ because $\emptyset$ is 
    empty. Therefore $x$ must be in $A$. It follows directly that $A \cup \emptyset = A$. Thus 
    proves the set identity law for set union.
\end{proof}
\begin{center}
    \rule{5.4in}{1pt}
\end{center}


% ============================= Theorem 2.2.6b ================================
\begin{theorem*}[\textbf{2.2.6b}]
    Let A be a set with universal set U. The set identity for A is $A \cap U = A$.
\end{theorem*}

\begin{proof}
    Let $x$ be an element in $A \cap U$. By the definition for set intersection, 
    $(x \in A) \land (x \in U)$. We know that $x \in U$ is true because $U$ is the 
    universe. The logical law of identity has it that $x \in A$. Therefore it follows 
    directly that $A \cap U = A$. Thus proves the set identity law for set intersection.
\end{proof}
\begin{center}
    \rule{5.4in}{1pt}
\end{center}


% ============================= Theorem 2.2.7a ================================
\begin{theorem*}[\textbf{2.2.7a}]
    Let A be a set with universal set U. U dominates set union such that $A \cup U = U$.
\end{theorem*}

\begin{proof}
    Let $x$ be an element in $A \cup U$. By the definition of set union, \newline 
    $(x \in A) \lor (x \in U)$. Regardless of the truth value for $x \in A$, we know 
    $x \in U$ is always true because $U$ is the universe. Therefore by logical domination 
    $(x \in A) \lor (x \in U) \equiv x \in U$. It directly follows from the definitions 
    that $A \cup U = U$. Thus proves the set domination law for the union of sets.
\end{proof}

\pagebreak


% ============================= Theorem 2.2.7b ================================
\begin{theorem*}[\textbf{2.2.7b}]
    Let A be a set. The empty set dominates set intersection such that $A \cap \emptyset = \emptyset$.
\end{theorem*}

\begin{proof}
    Let $x$ be an element in $A \cap \emptyset$. By the definition for set intersection we have 
    $(x \in A) \land (x \in \emptyset)$. We know that $(x \in \emptyset) \equiv \bot$ because 
    the empty set is empty. The logical law of domination gives us that 
    $(x \in A) \land \bot \equiv \bot$. It immediately follows that 
    $(x \in A) \land (x \in \emptyset) \equiv (x \in \emptyset)$, which is the definition of 
    $A \cap \emptyset = \emptyset$. Thus proves the law of set domination for the intersection 
    of sets. 
\end{proof}
\begin{center}
    \rule{5.4in}{1pt}
\end{center}


% ============================= Theorem 2.2.8a ================================
\begin{theorem*}[\textbf{2.2.8a}]
    Let A be a set. A is idempotent such that $A \cup A = A$.
\end{theorem*}

\begin{proof}
    Let $x$ be an element in $A \cup A$. By the definition of set union we have, \newline 
    $(x \in A) \lor (x \in A)$. The logical idempotent law says \newline 
    $(x \in A) \lor (x \in A) \equiv (x \in A)$ which is precisely $A \cup A = A$ by definition. 
    Thus proves the idempotent law for the union of sets.
\end{proof}
\begin{center}
    \rule{5.4in}{1pt}
\end{center}


% ============================= Theorem 2.2.8b ================================
\begin{theorem*}[\textbf{2.2.8b}]
    Let A be a set. A is idempotent such that $A \cap A = A$.
\end{theorem*}

\begin{proof}
    Let $x$ be an element in $A \cap A$. By the definition for set intersection we have 
    $(x \in A) \land (x \in A)$. The logical idempotent law says \newline 
    $(x \in A) \land (x \in A) \equiv (x \in A)$. That is the definition of $A \cap A = A$. 
    Thus proves the idempotent law for the intersection of sets.
\end{proof}
\begin{center}
    \rule{5.4in}{1pt}
\end{center}


% ============================= Theorem 2.2.9a ================================
\begin{theorem*}[\textbf{2.2.9a}]
    Let A be a set with universal set U. $A \cup \overline{A} = U$.
\end{theorem*}

\begin{proof}
    Let $x$ be an element in $A \cup \overline{A}$. By the definition for set union and the complement 
    of sets we have $(x \in A) \lor (x \notin A)$. The right-hand side of this disjunction is 
    equivalent to $x \in (U \cap \overline{A})$ according to the definition for set complementation. 
    That is, $(x \in U) \land (x \notin A)$. So the original disjunction is the same as 
    $(x \in A) \lor [(x \in U) \land (x \notin A)]$. We must distribute the left-hand side of this 
    disjunction over the conjunction occurring in the right-hand side. We get 
    $[(x \in A) \lor (x \in U)] \land [(x \in A) \lor (x \notin A)]$. By the logical law of negation 
    the identity for the right-hand side of this conjunction is true. The left-hand side of this 
    conjunction is dominated by $U$, according to Theorem 2.2.7a. Therefore the statement 
    $x \in A \cup \overline{A}$ can be equivalently stated as $(x \in U) \land T$; the logical identity 
    of which is $x \in U$. Thus proving the set complementation law for the union of sets, 
    $A \cup \overline{A} = U$.
\end{proof}

\pagebreak


% ============================= Theorem 2.2.9b ================================
\begin{theorem*}[\textbf{2.2.9b}]
    Let A be a set. $A \cap \overline{A} = \emptyset$.
\end{theorem*}

\begin{proof}
    Let $x$ be an element in $A \cap \overline{A}$. By definition, 
    $(x \in A) \land (x \in \overline{A})$. The right-hand side of this conjunction, according to the 
    definitions of set complementation and logical negation, can be restated as 
    \newline $(x \notin A) \equiv \lnot (x \in A)$. Again, by logical negation, we have 
    \newline $(x \in A) \land \lnot (x \in A) \equiv \bot$. This is the very meaning of 
    $A \cap \overline{A} = \emptyset$. Thus proves the set complementation law for the intersection of 
    sets.
\end{proof}
\begin{center}
    \rule{5.4in}{1pt}
\end{center}


% ============================= Theorem 2.2.10a ===============================
\begin{theorem*}[\textbf{2.2.10a}]
    Let A be a set. $A - \emptyset = A$.
\end{theorem*}

\begin{proof}
    Let $x$ be an element in $A - \emptyset \equiv A \cap \overline{\emptyset}$, the logical 
    equivalence for which is established by Theorem 2.2.19. By definition we have, 
    \newline $(x \in A) \land (x \notin \emptyset)$. Because by supposition 
    $\exists x (x \in (A - \emptyset))$, we know that the statement $(x \notin \emptyset)$ must 
    be true. Therefore, by logical identity, the statement $x \in (A - \emptyset)$ is defined 
    as $x \in A$. So $A - \emptyset = A.$
\end{proof}
\begin{center}
    \rule{5.4in}{1pt}
\end{center}


% ============================= Theorem 2.2.10b ===============================
\begin{theorem*}[\textbf{2.2.10b}]
    Let A be a set. $\emptyset - A = \emptyset$.
\end{theorem*}

\begin{proof}
    Let $x$ be an element in $\emptyset - A \equiv \emptyset \cap \overline{A}$, the logical 
    equivalence for which is established by Theorem 2.2.19. Because this expression is defined 
    as $(x \in \emptyset) \land (x \notin A)$ the supposition 
    $\exists x (x \in (\emptyset - A))$ immediately contradicts $x \in \emptyset$. Meaning that 
    no such $x$ could possibly exist. It follows that $(\emptyset - A)$ must be empty. Hence, 
    $\emptyset - A = \emptyset$
\end{proof}
\begin{center}
    \rule{5.4in}{1pt}
\end{center}


% ============================= Theorem 2.2.11a ===============================
\begin{theorem*}[\textbf{2.2.11a}]
    Let A and B be sets. The union of A and B is commutative.
\end{theorem*}

\begin{proof}
    Let $x$ be an element in $A \cup B$. This is defined as $(x \in A) \lor (x \in B)$. Because logical 
    disjunction is commutative \newline $(x \in A) \lor (x \in B) \equiv (x \in B) \lor (x \in A)$. This 
    of course means $x \in (B \cup A)$. Therefore $A \cup B = B \cup A$, and the union of two sets is 
    indeed commutative.
\end{proof}

\pagebreak


% ============================= Theorem 2.2.11b ===============================
\begin{theorem*}[\textbf{2.2.11b}]
    Let A and B be sets. The intersection of A and B is commutative.
\end{theorem*}

\begin{proof}
    Let $x$ be an element in $A \cap B$. By the definition of intersection, $(x \in A) \land (x \in B)$. 
    Because logical conjunction is commutative, the definition is equivalently stated as 
    $(x \in B) \land (x \in A)$. Meaning that $x \in (B \cap A)$. So $A \cap B = B \cap A$, and indeed 
    the intersection of two sets is commutative.
\end{proof}
\begin{center}
    \rule{5.4in}{1pt}
\end{center}


% ============================= Theorem 2.2.12 ================================
\begin{theorem*}[\textbf{2.2.12}]
    Let A and B be sets. $A \cup (A \cap B) = A$.
\end{theorem*}

\begin{proof}
    Let $x$ be an element in $A \cup (A \cap B)$. By the definitions for set union and set intersection 
    we have $(x \in A) \lor [(x \in A) \land (x \in B)]$. Treating each parenthesized object as a 
    discrete object we can immediately apply the logical law of absorption to this definition. The result 
    is $x \in A$. Thus it follows directly from the definition that $A \cup (A \cap B) = A$; the consequence 
    of which proves the absorption law for the union of a set with the intersection of itself and another set.
\end{proof}
\begin{center}
    \rule{5.4in}{1pt}
\end{center}


% ============================= Theorem 2.2.13 ================================
\begin{theorem*}[\textbf{2.2.13}]
    Let A and B be sets. $A \cap (A \cup B) = A$.
\end{theorem*}

\begin{proof}
    Let $x$ be an element in $A \cup (A \cap B)$. By the definitions for set union and set intersection 
    we have $(x \in A) \land [(x \in A) \lor (x \in B)]$. Treating each parenthesized object as a 
    discrete object we can immediately apply the logical law of absorption to this definition. The 
    result is $x \in A$. Thus it follows directly from the definition that $A \cap (A \cup B) = A$; the 
    consequence of which proves the absorption law for the intersection of a set with the union of 
    itself and another set.
\end{proof}

\pagebreak


% ============================= Theorem 2.2.15 ================================
\begin{theorem*}[\textbf{2.2.15}]
    Let A and B be sets. $\overline{A \cup B} = \overline{A} \cap \overline{B}.$
\end{theorem*}

\begin{proof}
    Let $x$ be an element in $\overline{A \cup B}$. By the definition of set \newline complementation we 
    have $\lnot [x \in (A \cup B)]$. By the definition of set union, \newline 
    $\lnot [(x \in A) \lor (x \in B)]$. Applying DeMorgans law (from logic) to the logical operations we 
    get $\lnot(x \in A) \land \lnot(x \in B) \equiv (x \in \overline{A}) \land (x \in \overline{B})$. 
    This is the definition of $x \in (\overline{A} \cap \overline{B})$. Therefore 
    $\overline{A \cup B} \subseteq (\overline{A} \cap \overline{B})$. 
    
    Now suppose $x$ were an element in $\overline{A} \cap \overline{B}$. Then by definition \newline 
    $(x \in \overline{A}) \land (x \in \overline{B}) \equiv \lnot (x \in A) \land \lnot (x \in B)$. 
    By DeMorgans law (from logic) we have $\lnot [(x \in A) \lor (x \in B)]$. Since this is the 
    definition for set union it follows that $\lnot [x \in (A \cup B)]$. Finally, applying the definition 
    of set complementation we arrive at $x \in \overline{A \cup B}$. Therefore 
    $(\overline{A} \cap \overline{B}) \subseteq \overline{A \cup B}$.
    
    Because $\overline{A \cup B} \subseteq (\overline{A} \cap \overline{B})$ and 
    $(\overline{A} \cap \overline{B}) \subseteq \overline{A \cup B}$ the sets are equivalent by definition. 
    That is,  $\overline{A \cup B} = (\overline{A} \cap \overline{B})$. Thereby proving DeMorgans law for sets, 
    that the complement of the union of two sets is equivalent to the intersection of those set complements.		
\end{proof}
\begin{center}
    \rule{5.4in}{1pt}
\end{center}


% ============================= Theorem 2.2.16a ===============================
\begin{theorem*}[\textbf{2.2.16a}]
    Let A and B be sets. $(A \cap B) \subseteq A$.
\end{theorem*}

\begin{proof}
    Let $x$ be an element in $(A \cap B)$. Then by definition, $(x \in A) \land (x \in B)$. 
    It trivially follows from the definition of logical conjunction that $x \in A$ whenever 
    $x \in (A \cap B)$. So $(A \cap B) \subseteq A$. 
\end{proof}
\begin{center}
    \rule{5.4in}{1pt}
\end{center}


% ============================= Theorem 2.2.16b ===============================
\begin{theorem*}[\textbf{2.2.16b}]
    Let A and B bet sets. $A \subseteq (A \cup B)$
\end{theorem*}

\begin{proof}
    All of the elements in A are a subset of $A \cup B$ by the definition of set union. 
    Therefore it trivially follows that $A \subseteq (A \cup B)$.
\end{proof}
\begin{center}
    \rule{5.4in}{1pt}
\end{center}


% ============================= Theorem 2.2.16c ===============================
\begin{theorem*}[\textbf{2.2.16c}]
    Let A and B be sets. $(A - B) \subseteq A$.
\end{theorem*}

\begin{proof}
    Let $x$ be an element in $A - B$. By Theorem $2.2.19$ $A - B$ is equivalent to the 
    statement $A \cap \overline{B}$, and thus by definition we have 
    $(x \in A) \land (x \notin B)$. We can infer by the simplification rule that 
    $x \in A$. It therefore follows immediately from the definition that 
    $(A - B) \subseteq A$.
\end{proof}

\pagebreak


% ============================= Theorem 2.2.16d ===============================
\begin{theorem*}[\textbf{2.2.16d}]
    Let A and B be sets. $A \cap (B - A) = \emptyset$.
\end{theorem*}

\begin{proof}
    Let $x$ be an element in $A \cap (B - A)$. By Theorem 2.2.19  
    $(B - A)$ is equivalentt to $(B \cap \overline{A})$. Because set intersection is associative we can 
    drop the parentheses, giving us $A \cap B \cap \overline{A}$. This is logically defined 
    as $(x \in A) \land (x \in B) \land (x \notin A) \equiv \bot$. Because this statement is 
    false $\forall x$ in the domain, $A \cap (B - A)$ is empty.
\end{proof}
\begin{center}
    \rule{5.4in}{1pt}
\end{center}


% ============================= Theorem 2.2.16e ===============================
\begin{theorem*}[\textbf{2.2.16e}]
    Let A and B be sets. $A \cup (B - A) = A \cup B$.
\end{theorem*}

\begin{proof}
    Let $x$ be an element in $A \cup (B - A)$. By Theorem 2.2.19  
    $(B - A)$ is equivalent to $(B \cap \overline{A})$. So by definition we have 
    $(x \in A) \lor [(x \in B) \land (x \notin A)]$. Distributing logical disjunction over 
    logical conjunction yields \newline 
    $[(x \in A) \lor (x \in B)] \land [(x \in A) \lor (x \notin A)]$. Which by logical negation 
    and by logical identity reduces to $(x \in A) \lor (x \in B)$, that is the very definition for 
    $A \cup B$.
    
    Suppose the converse case in which  $x$ is an element of $A \cup B$. That is, of course as 
    already stated, defined as $(x \in A) \lor (x \in B)$. Note the fact that the conjunction of 
    this proposition with another true proposition is true. Let $p$ be that proposition, 
    $(x \in A)$. Then $p \lor \lnot p \equiv (x \in A) \lor (x \notin A) \equiv T$, and thus we 
    can make the following statement 
    $[(x \in A) \lor (x \in B)] \land [(x \in A) \lor (x \notin A)]$, which holds. Factoring the 
    term $(x \in A)$ out on the logical operators gives the form 
    $(x \in A) \lor [(x \in B) \land (x \notin A)]$. This statement is the definition for 
    $A \cup (B - A)$.
    
    Since $A \cup (B - A) \subseteq A \cup B$ and 
    $A \cup B \subseteq A \cup (B - A)$, 
    \newline 
    $A \cup (B - A) = A \cup B$ by definition.
\end{proof}

\pagebreak


% ============================= Theorem 2.2.17 ================================
\begin{theorem*}[\textbf{2.2.17}]
    Let A, B, and C be sets. 
    $\overline{A \cap B \cap C} = \overline{A} \cup \overline{B} \cup \overline {C}$.
\end{theorem*}

\begin{proof}
    Let $x$ be an element in $\overline{A \cap B \cap C}$. By the definitions for set 
    \newline complementation and logical negation we have \newline 
    $x \notin (A \cap B \cap C) \equiv \lnot (x \in A \cap B \cap C)$. By the definition of 
    set intersection that is $\lnot [(x \in A) \land (x \in B) \land (x \in C)]$. Using 
    DeMorgans law (from logic) we can distribute the logical negation across the conjunctions 
    giving us the expression $\lnot (x \in A) \lor \lnot (x \in B) \lor \lnot (x \in C)$. 
    Carrying out those logical negations on each respective term, and again by the definition 
    for set complementation we have 
    $(x \in \overline{A}) \lor (x \in \overline{B}) \lor (x \in \overline{C})$. This is the 
    definition of $\overline{A} \cup \overline{B} \cup \overline{C}$. Hence, 
    $\overline{A \cap B \cap C} \subseteq \overline{A} \cup \overline{B} \cup \overline{C}$.
    
    Now suppose the converse case, where x is an element of 
    $\overline{A} \cup \overline{B} \cup \overline{C}$. As already stated in the previous 
    paragraph the definition for this expression is 
    $(x \in \overline{A}) \lor (x \in \overline{B}) \lor (x \in \overline{C})$. By the 
    definitions for complementation and logical negation that is 
    $\lnot (x \in A) \lor \lnot (x \in B) \lor \lnot (x \in C)$. Using DeMorgans law 
    (from logic) we can factor out the logical negations such that 
    $\lnot [(x \in A) \land (x \in B) \land (x \in C)]$. Which, by the arguments given in the 
    first paragraph we know is equivalent to $\overline{A \cap B \cap C}$. Thus, 
    $\overline{A} \cup \overline{B} \cup \overline{C} \subseteq \overline{A \cap B \cap C}$.
    
    It immediately follows from the definition of set equivalence that \newline 
    $\overline{A \cap B \cap C} = \overline{A} \cup \overline{B} \cup \overline{C}$
\end{proof}
\begin{center}
    \rule{5.4in}{1pt}
\end{center}


% ============================= Theorem 2.2.18a ===============================
\begin{theorem*}[\textbf{2.2.18a}]
    Let A, B, and C be sets. $(A \cup B) \subseteq (A \cup B \cup C)$.
\end{theorem*}

\begin{proof}
    Let $x$ be an element in $(A \cup B)$. We have $(x \in A) \lor (x \in B)$, by definition. 
    Let this definition statement be represented by $p$. Trivially, the fact of $p$ would be 
    unaffected were $p$ disjunct any proposition $q$. Supposing such a $q$ existed, we would have 
    $p \lor q \equiv T$ by logical domination (because the hypothetical supposition assumes 
    $p \equiv T$.) Let $q$ be the proposition $(x \in C)$. Then 
    $p \lor q \equiv (x \in A) \lor (x \in B) \lor (x \in C)$ is the well formed statement 
    defining the superset under interrogation. Since we know that $x$ is in the union of $A$ and 
    $B$ by the hypothesis, and because $p \lor q \equiv T$ means that $x$ is in the union of 
    $A$, $B$, and $C$, it immediately follows that \newline $(A \cup B) \subseteq (A \cup B \cup C)$.
\end{proof}

\pagebreak


% ============================= Theorem 2.2.18b ===============================
\begin{theorem*}[\textbf{2.2.18b}]
    Let A, B and C be sets. $(A \cap B \cap C) \subseteq (A \cap B)$.
\end{theorem*}

\begin{proof}
    Let $x$ be an element in $(A \cap B \cap C)$. The definition of this expression is 
    $(x \in A) \land (x \in B) \land (x \in C)$. From that, obviously 
    $(x \in A) \land (x \in B)$, being the definition for $(A \cap B)$. It necessarily 
    follows that $(A \cap B \cap C) \subseteq (A \cap B)$.
\end{proof}
\begin{center}
    \rule{5.4in}{1pt}
\end{center}


% ============================= Theorem 2.2.18c ===============================
\begin{theorem*}[\textbf{2.2.18c}]
    Let A, B, and C be sets. $(A - B) - C \subseteq (A - C)$.
\end{theorem*}

\begin{proof}
    Let $x$ be an element in $(A - B) - C$. By Theorem 2.2.19 $A - B \equiv A \cap \overline{B}$ and 
    $(A \cap \overline{B}) - C \equiv (A \cap \overline{B}) \cap \overline{C}$. By the 
    associative laws and the commutative laws for the intersection of sets we have, 
    $x \in (A \cap \overline{C}) \cap \overline{B}$. By definition that is 
    $[(x \in A) \land (x \in \overline{C})] \land (x \notin B)$. Or rather, 
    $x \in (A - C) \land (x \notin B)$. By logical identity $x \in (A - C)$. Since 
    $x \in [(A - B) - C] \implies x \in (A - C)$, $(A - B) - C \subseteq (A - C)$.
\end{proof}
\begin{center}
    \rule{5.4in}{1pt}
\end{center}


% ============================= Theorem 2.2.18d ===============================
\begin{theorem*}[\textbf{2.2.18d}]
    Let A, B, and C be sets. $(A - C) \cap (C - B) = \emptyset$.
\end{theorem*}

\begin{proof}
    Let $x$ be an element in $(A - C) \cap (C - B)$. By Theorem 2.2.19 this is equivalently stated as 
    $x \in [(A \cap \overline{C}) \cap (C \cap \overline{B})]$. Since set intersection is 
    associative the inner parentheses can be eliminated, 
    $x \in (A \cap \overline{C} \cap C \cap \overline{B})$. An expression the definition for which 
    is $(x \in A) \land (x \in \overline{C}) \land (x \in C) \land (x \in \overline{B})$. But by the 
    logical law of negation that is 
    $(x \in A) \land \bot \land (x \in \overline{B}) \equiv \bot$. Meaning 
    $\lnot \exists x (x \in (A - C) \cap (C - B))$. In other words, the intersection is indeed empty.
\end{proof}

\pagebreak


% ============================= Theorem 2.2.18e ===============================
\begin{theorem*}[\textbf{2.2.18e}]
    Let A, B, and C be sets. $(B - A) \cup (C - A) = (B \cup C) - A$.
\end{theorem*}

\begin{proof}
    Let $x$ be an element in $(B - A) \cup (C - A)$. By Theorem 2.2.19 this is the same as  
    $x \in [(B \cap \overline{A}) \cup (C \cap \overline{A})]$. By definition, that is, 
    \newline 
    $[(x \in B) \land (x \in \overline{A})] \lor [(x \in C) \land (x \in \overline{A})]$. By 
    the associative property for logical conjunction, and factoring out the term 
    $(x \in \overline{A})$, we get \newline 
    $[(x \in B) \lor (x \in C)] \land (x \in \overline{A})$. This statement is the definition 
    for \newline $(B \cup C) - A$.
    
    Proving the converse case, suppose that $x$ is an element in $(B \cup C) - A$. Note that 
    the expression is equivalent to $(B \cup C) \cap \overline{A}$. Thus we have the following 
    definition, $[(x \in B) \lor (x \in C)] \land (x \in \overline{A})$. By logical distribution 
    for conjunction over disjunction 
    $[(x \in B) \land (x \in \overline{A})] \lor [(x \in C) \land (x \in \overline{A})]$. This 
    statement defines the expression $x \in [(B \cap \overline{A}) \cup (C \cap \overline{A})]$. 
    Which, as argued in the first paragraph, is logically equivalent to 
    $x \in [(B - A) \cup (C - A)]$.
    
    Since $[(B - A) \cup (C - A)] \subseteq [(B \cup C) - A]$ and \newline 
    $[(B \cup C) - A] \subseteq [(B - A) \cup (C - A)]$. It immediately follows from the definition 
    of set equivalence that $(B - A) \cup (C - A) = (B \cup C) - A$.
\end{proof}
\begin{center}
    \rule{5.4in}{1pt}
\end{center}


% ============================= Theorem 2.2.19 ================================
\begin{theorem*}[\textbf{2.2.19}]
    Let A, and B be sets. $A - B = A \cap \overline{B}$.
\end{theorem*}

\begin{proof}
    Let $x$ be an element in $A - B$. By the definition for set difference, 
    $(x \in A) \land (x \notin B)$. By the definition for set complementation this is the 
    same as $(x \in A) \land (x \in \overline{B})$. Which is exactly the definition for 
    $x \in (A \cap \overline{B})$.
    
    Proving the converse trivially follows by reversing our steps in the direct form. Suppose 
    there exists an element $x$ such that $x \in (A \cap \overline{B})$. By the definition for 
    set intersection we have $(x \in A) \land (x \in \overline{B})$. By the definition for set 
    complementation we arrive at the definition for set difference \newline 
    $(x \in A) \land (x \notin B)$. Therefore $x \in (A - B)$.
    
    Since $(A - B) \subseteq (A \cap \overline{B})$ and $(A \cap \overline{B}) \subseteq (A - B)$, 
    by the definition for set equality we have $(A - B) = (A \cap \overline{B})$.
\end{proof}

\pagebreak


% ============================= Theorem 2.2.20 ================================
\begin{theorem*}[\textbf{2.2.20}]
    Let A, and B be sets. $(A \cap B) \cup (A \cap \overline{B}) = A$.
\end{theorem*}

\begin{proof}
    Let $x$ be an element in $(A \cap B) \cup (A \cap \overline{B})$. By definition then, we 
    have $[(x \in A) \land (x \in B)] \lor [(x \in A) \land (x \notin B)]$. By the logical 
    law of distribution for conjunction over disjunction we can factor the term $(x \in A)$ 
    out on the conjunction. Thus, $(x \in A) \land [(x \in B) \lor (x \notin B)]$. By the 
    logical law of negation we have $(x \in A) \land T$ which is equivalent to $x \in A$ by 
    the identity law of logic.
    
    Proving the converse, let $x$ be an element in $A$. That is, $(x \in A)$. By the logical 
    law of identity, $(x \in A) \land T \equiv (x \in A)$. Since $(x \in B) \lor (x \notin B)$ 
    is true by the logical law of negation, we can, by logical equivalence, construct the 
    following statement while retaining the logical identity for the statement $(x \in A)$; that 
    is $(x \in A) \land [(x \in B) \lor (x \notin B)]$. By the logical law of distribution for 
    conjunction over disjunction we have the definition of our original expression, 
    $[(x \in A) \land (x \in B)] \lor [(x \in A) \land (x \notin B)]$. Hence, 
    $x \in [(A \cap B) \cup (A \cap \overline{B})]$.
    
    Because $(A \cap B) \cup (A \cap \overline{B}) \subseteq A$ and 
    $A \subseteq  (A \cap B) \cup (A \cap \overline{B})$ it follows from the definition of set 
    equivalence that $(A \cap B) \cup (A \cap \overline{B}) = A$.
\end{proof}
\begin{center}
    \rule{5.4in}{1pt}
\end{center}


% ============================= Theorem 2.2.21 ================================
\begin{theorem*}[\textbf{2.2.21}]
    Let A, B, and C be sets. $A \cup (B \cup C) = (A \cup B) \cup C$, such that set union is 
    associative.
\end{theorem*}

\begin{proof}
    Let $x$ be an element in $A \cup (B \cup C)$. The logical definition being \newline 
    $(x \in A) \lor [(x \in B) \lor (x \in C)]$. It trivially follows from the associative 
    law for logical disjunction that $[(x \in A) \lor (x \in B)] \lor (x \in C)$. Hence, $x$ 
    is an element of $(A \cup B) \cup C$.
    
    In the converse case, let $x$ be an element of $(A \cup B) \cup C$. The logical definition 
    being $[(x \in A) \lor (x \in B)] \lor (x \in C)$. It trivially follows from the associative 
    law for logical disjunction that $(x \in A) \lor [(x \in B) \lor (x \in C)]$. Hence, $x$ is 
    an element of $A \cup (B \cup C) $.
    
    Since $A \cup (B \cup C) \subseteq (A \cup B) \cup C$ and 
    $(A \cup B) \cup C \subseteq A \cup (B \cup C)$ it follows immediately from the definition 
    for set equality that \newline $A \cup (B \cup C) = (A \cup B) \cup C$. Thus, the union of 
    three sets is indeed associative.
\end{proof}

\pagebreak


% ============================= Theorem 2.2.22 ================================
\begin{theorem*}[\textbf{2.2.22}]
    Let A, B, and C be sets. $A \cap (B \cap C) = (A \cap B) \cap C$, such that set 
    intersection is associative.
\end{theorem*}

\begin{proof}
    Let $x$ be an element in $A \cap (B \cap C)$. The logical definition being \newline 
    $(x \in A) \land [(x \in B) \land (x \in C)]$. It trivially follows from the logical 
    law of association for conjunction that $[(x \in A) \land (x \in B)] \land (x \in C)$. 
    Therefore by definition $x$ is an element of $(A \cap B) \cap C$.
    
    Proving the converse, let $x$ be an element in $(A \cap B) \cap C$. The logical 
    definition being $[(x \in A) \land (x \in B)] \land (x \in C)$. It trivially follows 
    from the logical law of association for conjunction that \newline 
    $(x \in A) \land [(x \in B) \land (x \in C)]$. Therefore by definition $x$ is an element of 
    $A \cap (B \cap C)$.
    
    Since $A \cap (B \cap C) \subseteq (A \cap B) \cap C$ and 
    $(A \cap B) \cap C \subseteq A \cap (B \cap C)$ it immediately follows from the definition 
    for set equality that \newline $A \cap (B \cap C) = (A \cap B) \cap C$. Thus, the 
    intersection of three sets is associative.
\end{proof}
\begin{center}
    \rule{5.4in}{1pt}
\end{center}


% ============================= Theorem 2.2.23 ================================
\begin{theorem*}[\textbf{2.2.23}]
    Let A, B, and C be sets. $A \cup (B \cap C) = (A \cup B) \cap (A \cup C)$, such that set 
    union is distributive over set intersection.
\end{theorem*}

\begin{proof}
    Let $x$ be an element in $A \cup (B \cap C)$. The logical definition being \newline 
    $(x \in A) \lor [(x \in B) \land (x \in C)]$. By the logical law for distribution of 
    disjunction over conjunction we have 
    $[(x \in A) \lor (x \in B)] \land [(x \in A) \lor (x \in C)]$. By definition, $x$ is an 
    element in $(A \cup B) \cap (A \cup C)$.
    
    Proving the converse, let $x$ be an element in $(A \cup B) \cap (A \cup C)$. The logical 
    definition being $[(x \in A) \lor (x \in B)] \land [(x \in A) \lor (x \in C)]$. By the 
    logical law for distribution we can factor out the term $x \in A$ over the conjunction. 
    Thus, $(x \in A) \lor [(x \in B) \land (x \in C)]$, and $x$ is an element in 
    $A \cup (B \cap C)$ by definition.
    
    Since $A \cup (B \cap C) \subseteq (A \cup B) \cap (A \cup C)$ and \newline 
    $(A \cup B) \cap (A \cup C) \subseteq A \cup (B \cap C)$, it follows immediately from the 
    definition of set equality that $A \cup (B \cap C) = (A \cup B) \cap (A \cup C)$. Therefore, 
    set union is indeed distributive over set intersection.
\end{proof}

\pagebreak


% ============================= Theorem 2.2.24 ================================
\begin{theorem*}[\textbf{2.2.24}]
    Let A, B, and C be sets. $(A - B) - C = (A - C) - (B - C)$.
\end{theorem*}

\begin{proof}
    Let $x$ be an element in $(A - B) - C$. By the definition for set difference we have 
    $[(x \in A) \land (x \notin B)] \land (x \notin C)$.  Now, the logical identity for the 
    proposition $(x \notin B)$ is $(x \notin B) \lor \bot \equiv (x \notin B)$, and since 
    \newline $(x \in C) \equiv \bot$ by reason of the hypothesis, it necessarily follows 
    from logical identity that $(x \notin B) \equiv [(x \notin B) \lor (x \in C)]$. Thus, 
    by these facts, the law of logical commutativity, and the law of logical association we 
    make the logically equivalent statement with respect to the definition given by the first 
    expression, $[(x \in A) \land (x \notin C)] \land [(x \notin B) \lor (x \in C)]$. Since 
    the right-hand side of the conjunction is true and in its proper logical identity, if it 
    were double negated, it would remain intact by the law of double negation. Thus applying 
    first a negation by DeMorgans law, and second a negation directly on the propositional 
    statement, we have 
    $[(x \in A) \land (x \notin C)] \land \lnot [(x \in B) \land (x \notin C)]$. By the 
    definition for set difference and set complementation $x$ is an element in 
    $(A - C) \cap \overline{(B - C)}$. Which is equivalent to the expression 
    $(A - C) - (B - C)$, by Theorem 2.2.19.
    
    To prove the converse case, let $x$ be an element in $(A - C) - (B - C)$. By Theorem 2.2.19  
    $(A - C) \cap \overline{(B - C)}$ is an equivalent expression. This expression containing 
    the element $x$ is defined by 
    \newline $[(x \in A) \land (x \notin C)] \land \lnot [(x \in B) \land (x \notin C)]$. 
    Applying DeMorgans law to the statement on the right-hand side of the conjunction we get 
    \newline $[(x \in A) \land (x \notin C)] \land [(x \notin B) \lor (x \in C)]$. Note that as 
    demonstrated in the first paragraph we have the following identity 
    $[(x \notin B) \lor (x \in C)] \equiv (x \notin B)$. Thus, by that fact, the law of logical 
    commutativity, and by the law of logical association it follows that the identity of our 
    statement is \newline $[(x \in A) \land (x \notin B)] \land (x \notin C)$. This is the 
    definition for $x \in [(A - B) - C]$.
    
    Since $(A - B) - C \subseteq (A - C) - (B - C)$ and 
    \newline $(A - C) - (B - C) \subseteq (A - B) - C$, it follows immediately from the 
    definition for set equality that $(A - B) - C = (A - C) - (B - C)$
\end{proof}

\pagebreak


% ============================= Theorem 2.2.31 ================================
\begin{theorem*}[\textbf{2.2.31}]
    Let A, and B be subsets of a universal set U. \newline 
    $A \subseteq B \iff \overline{B} \subseteq \overline{A}$.
\end{theorem*}

\begin{proof}
    The proposition $A \subseteq B$ is equivalent to the universally quantified statement 
    $\forall x ( x \in A \implies x \in B)$. It is tautological that the propositional 
    function in this statement is logically equivalent to its contrapositive form 
    (satisfying the biconditional requirement.) That is,  
    $\forall x (x \notin B \implies x \notin A)$ is the logically equivalent statement. 
    By the definition for set complementation that is 
    $\forall x (x \in \overline{B} \implies x \in \overline{A})$. By universal generalization 
    \newline $A \subseteq B \iff \overline{B} \subseteq \overline{A}$.
\end{proof}

\pagebreak


% ============================= Theorem 2.2.35 ================================
\begin{theorem*}[\textbf{2.2.35}]
    Let A, and B be sets. $A \oplus B = (A \cup B) - (A \cap B)$.
\end{theorem*}

\begin{proof}
    Let $x$ be an element in $A \oplus B$. This is logically defined as \newline 
    $[(x \in A) \land (x \notin B)] \lor [(x \notin A) \land (x \in B)]$. By the definition 
    for set difference $x$ is an element in $(A - B) \cup (B - A)$ which by Theorem 2.2.19 
    can be expressed as 
    $(A \cap \overline{B}) \cup (B \cap \overline{A})$. By Theorem 2.2.23, which proves that 
    set unions are distributive over set intersections, the following expression is equivalent 
    $[A \cup (B \cap \overline{A})] \cap [\overline{B} \cup (B \cap \overline{A})]$. Again, 
    by Theorem 2.2.23, we have 
    $[(A \cup B) \cap (A \cup \overline{A})] \cap [(\overline{B} \cup B) \cap (\overline{B} 
        \cup \overline{A})]$. 
    Let the logical definition for this expression be represented in two propositional variables 
    $p \land q$ such that: \newline \indent 
    $p \equiv \{[(x \in A) \lor (x \in B)] \land [(x \in A) \lor (x \notin A)]\} 
    \newline \indent q \equiv \{[(x \notin B) \lor (x \in B)] \land 
    [(x \notin B) \lor (x \notin A)]\}$. 
    \newline By the logical law of identity $p \equiv (x \in A) \lor (x \in B)$, and 
    \newline $q \equiv (x \notin B) \lor (x \notin A)$. So we have 
    $[(x \in A) \lor (x \in B)] \land [(x \notin B) \lor (x \notin A)]$. Applying DeMorgans law 
    to the right-hand side of the conjunction we get \newline 
    $[(x \in A) \lor (x \in B)] \land \lnot [(x \in B) \land (x \in A)]$. Then by definition, 
    $x$ is an element in $(A \cup B) \cap \overline{(A \cap B)}$, and according to Theorem 
    2.2.19 $x$ is an element in $(A \cup B) - (A \cap B)$.
    
    Proving the converse case is trivial. Let $x$ be an element in \newline 
    $(A \cup B) - (A \cap B)$. By Theorem 2.2.19 $x$ is an element in 
    $(A \cup B) \cap \overline{(A \cap B)}$. By definition we have, 
    $[(x \in A) \lor (x \in B)] \land \lnot [(x \in B) \land (x \in A)]$. By DeMorgans law, 
    $[(x \in A) \lor (x \in B)] \land [(x \notin B) \lor (x \notin A)]$. Let this logical formula 
    be represented in two propositional variables $p \land q$ such that 
    $p \equiv (x \in A) \lor (x \in B)$ and $q \equiv (x \notin B) \lor (x \notin A)$. By the 
    logical law of identity $p \land T \equiv T$, and $q \land T \equiv T$. By the negation law 
    of logic, $(x \notin B) \lor (x \in B) \equiv T$, and $(x \in A) \lor (x \notin A) \equiv T$. 
    Therefore, \newline \indent 
    $p \equiv \{[(x \in A) \lor (x \in B)] \land [(x \in A) \lor (x \notin A)]\} 
    \newline \indent q \equiv \{[(x \notin B) \lor (x \in B)] \land 
    [(x \notin B) \lor (x \notin A)]\}$. 
    \newline By definition, $x$ is an element in 
    $[(A \cup B) \cap (A \cup \overline{A})] \cap 
    [(\overline{B} \cup B) \cap (\overline{B} \cup \overline{A})]$. By Theorem 2.2.23, 
    factoring $A$ out of the left-hand side of the intersection, and factoring $\overline{B}$ 
    out of the right-hand side of the intersection, the following expression is equivalent 
    $[A \cup (B \cap \overline{A})] \cap [\overline{B} \cup (B \cap \overline{A})]$. Again, by 
    Theorem 2.2.23, factoring $(B \cap \overline{A})$ out of the intersection we have the following 
    equivalent expression $(A \cap \overline{B}) \cup (B \cap \overline{A})$. Which, by 
    Theorem 2.2.19 is equivalently stated as $(A - B) \cup (B - A)$, defined by 
    $[(x \in A) \land (x \notin B)] \lor [(x \notin A) \land (x \in B)]$. But this is the formal 
    definition for the symmetric difference of sets, so $x$ must be an element in $A \oplus B$.
    
    Since $A \oplus B \subseteq (A \cup B) - (A \cap B)$ and 
    $(A \cup B) - (A \cap B) \subseteq A \oplus B$, it follows immediately that 
    $A \oplus B = (A \cup B) - (A \cap B)$.
\end{proof}

\pagebreak


% ============================= Theorem 2.2.36 ================================
\begin{theorem*}[\textbf{2.2.36}]
    Let A, and B be sets. $A \oplus B = (A - B) \cup (B - A)$
\end{theorem*}

\begin{proof}
    Let $x$ be an element in $A \oplus B$. Then by the definition for symmetric 
    \newline difference $[(x \in A) \land (x \notin B)] \lor [(x \notin A) \land (x \in B)]$. 
    Because logical \newline conjunction is associative, the statement is equivalent to 
    \newline $[(x \in A) \land (x \notin B)] \lor [(x \in B) \land (x \notin A)]$. According 
    to the definition for set difference, and by the definition for set union, it follows 
    that $x$ is an element in $(A - B) \cup (B - A)$.
    
    Proving the converse, suppose $x$ were an element in $(A - B) \cup (B - A)$. The logical 
    definition being $[(x \in A) \land (x \notin B)] \lor [(x \in B) \land (x \notin A)]$. 
    By the associative law for logical conjunction the following statement is equivalent 
    $[(x \in A) \land (x \notin B)] \lor [(x \notin A) \land (x \in B)]$. Since this is the 
    definition for symmetric difference, $x$ is an element in $A \oplus B$.
    
    Since $A \oplus B \subseteq (A - B) \cup (B - A)$ and 
    $(A - B) \cup (B - A) \subseteq A \oplus B$ it immediately follows from the definition of 
    set equality that \newline $A \oplus B = (A - B) \cup (B - A)$.
\end{proof}
\begin{center}
    \rule{5.4in}{1pt}
\end{center}


% ============================= Theorem 2.2.37a ===============================
\begin{theorem*}[\textbf{2.2.37a}]
    Let A be a subset of the universal set U. $A \oplus A = \emptyset$.
\end{theorem*}

\begin{proof}
    By Theorem 2.2.35, $A \oplus A = (A \cup A) - (A \cap A)$. By the set idempotent law, 
    that is $A - A$, and by Theorem 2.2.19, equivalent to $A \cap \overline{A}$. It follows 
    immediately from the set complementation law that $A \oplus A = \emptyset$.
\end{proof}
\begin{center}
    \rule{5.4in}{1pt}
\end{center}


% ============================= Theorem 2.2.37b ===============================
\begin{theorem*}[\textbf{2.2.37b}]
    Let A be a subset of the universal set U. $A \oplus \emptyset = A$.
\end{theorem*}

\begin{proof}
    By Theorem 2.2.35, $A \oplus \emptyset = (A \cup \emptyset) - (A \cap \emptyset)$. By set 
    identity, and by set domination, that is $A - \emptyset$, which by Theorem 2.2.19 means 
    $A \cap \overline{\emptyset}$. Because $\overline{\emptyset} = U$, we have by set identity 
    that $A \oplus \emptyset = A$.
\end{proof}
\begin{center}
    \rule{5.4in}{1pt}
\end{center}


% ============================= Theorem 2.2.37c ===============================
\begin{theorem*}[\textbf{2.2.37c}]
    Let A be a subset of the universal set U. $A \oplus U = \overline{A}$.
\end{theorem*}

\begin{proof}
    By Theorem 2.2.35, $A \oplus U = (A \cup U) - (A \cap U)$. By set domination, and by set 
    identity, that is $U - A$. By Theorem 2.2.19, $U \cap \overline{A}$. By the identity law 
    for sets $A \oplus U = \overline{A}$.
\end{proof}

\pagebreak


% ============================= Theorem 2.2.37d ===============================
\begin{theorem*}[\textbf{2.2.37d}]
    Let A be a subset of a universal set U. $A \oplus \overline{A} = U$.
\end{theorem*}

\begin{proof}
    By Theorem 2.2.35, $A \oplus \overline{A} = (A \cup \overline{A}) - (A \cap \overline{A})$. 
    By the set complementation laws that is $U - \emptyset$. Rather, 
    $U \cap \overline{\emptyset}$ by Theorem 2.2.19. Since $\overline{\emptyset} = U$ we have 
    $U \cap U$, which is obviously $U$, by the idempotent law. Thus, 
    $A \oplus \overline{A} = U$.
\end{proof}
\begin{center}
    \rule{5.4in}{1pt}
\end{center}


% ============================= Theorem 2.2.38a ===============================
\begin{theorem*}[\textbf{2.2.38a}]
    Let A, and B be sets. The symmetric difference of sets is associative such that 
    $(A \oplus B) = (B \oplus A)$.
\end{theorem*}

\begin{proof}
    By Theorem 2.2.35, $A \oplus B = (A \cup B) - (A \cap B)$. Because set union is 
    associative, and because set intersection is associative, we have 
    \newline $(B \cup A) - (B \cap A)$. Again, according to Theorem 2.2.35, that is 
    $B \oplus A$.
\end{proof}
\begin{center}
    \rule{5.4in}{1pt}
\end{center}


% ============================= Theorem 2.2.38b ===============================
\begin{theorem*}[\textbf{2.2.38b}]
    Let A, and B be sets. The symmetric difference of sets is subject to absorption such that 
    $(A \oplus B) \oplus B = A$.
\end{theorem*}

\begin{proof}
    By Theorem 2.2.35, $(A \oplus B) \oplus B = [(A \oplus B) \cup B] - [(A \oplus B) \cap B]$. 
    By Theorem 2.2.19 that is $[(A \oplus B) \cup B] \cap \overline{[(A \oplus B) \cap B]}$, 
    and by DeMorgans law $[(A \oplus B) \cup B] \cap [\overline{(A \oplus B)} \cup \overline{B}]$. 
    Again, by Theorem 2.2.35, that is \newline 
    $\{[(A \cup B) - (A \cap B)] \cup B\} \cap 
    \{\overline{[(A \cup B) - (A \cap B)]} \cup \overline{B}\}$. By Theorem 2.2.19, 
    $\{[(A \cup B) \cap \overline{(A \cap B)}] \cup B\} \cap 
    \{\overline{[(A \cup B) \cap \overline{(A \cap B)}]} \cup \overline{B}\}$. Applying 
    DeMorgans law, three times in succession we get \newline 
    $\{[(A \cup B) \cap (\overline{A} \cup \overline{B})] \cup B\} \cap 
    \{[(\overline{A} \cap \overline{B}) \cup (A \cap B)] \cup \overline{B}\}$. By distribution 
    and association we make the following equivalent statement, \newline 
    $[(A \cup B \cup B) \cap (\overline{A} \cup \overline{B} \cup B)] \cap 
    \{[(\overline{A} \cap \overline{B}) \cup \overline{B}] \cup [(A \cap B) \cup \overline{B}]\}$. 
    Now this expression shall be reduced. By the idempotent law, the complementation law, 
    set absorption, and distribution we have \newline 
    $[(A \cup B) \cap (\overline{A} \cup U)] \cap 
    \{\overline{B} \cup [(A \cup \overline{B}) \cap (B \cup \overline{B})]\}$. By complementation 
    and domination we make the following equivalent statement, \newline 
    $[(A \cup B) \cap U] \cap \{\overline{B} \cup [(A \cup \overline{B}) \cap U]\}$. By set 
    identity, and the idempotent law that is $(A \cup B) \cap (A \cup \overline{B})$. Factoring 
    out $A$ by the law of distribution, and by the set complementation laws, and by set identity 
    we find that \newline $A \cup (B \cap \overline{B}) \equiv A \cup \emptyset \equiv A$
    
    $\therefore (A \oplus B) \oplus B = A$.
\end{proof}

\pagebreak


% ============================= Theorem 2.2.40 ================================
\begin{theorem*}[\textbf{2.2.40}]
    Let A, B, and C be sets. The symmetric difference for sets is associative such that 
    $(A \oplus B) \oplus C = A \oplus (B \oplus C)$.
\end{theorem*}

\begin{proof}
    Let $x$ be an element in $(A \oplus B) \oplus C$. By the definition for the symmetric 
    difference of sets,
    $x \in [(A \oplus B) \cap \overline{C}] \cup [\overline{(A \oplus B)} \cap C]$. 
    By Theorem 2.2.36, $x$ is an element in 
    $\{[(A - B) \cup (B - A)] \cap \overline{C}\} \cup 
    \{\overline{[(A - B) \cup (B - A)]} \cap C\}$. 
    And by Theorem 2.2.19,
    $\{[(A \cap \overline{B}) \cup (B \cap \overline{A})] \cap \overline{C}\} \cup 
    \{\overline{[(A \cap \overline{B}) \cup (B \cap \overline{A})]} \cap C]\}$. 
    Applying DeMorgans law for sets to the subset that is the right-hand side of the union 
    in this superset, twice, produces the following logical superset equivalence,
    $x \in \{[(A \cap \overline{B}) \cup (B \cap \overline{A})] \cap \overline{C}\} \cup 
    \{[(\overline{A} \cup B) \cap (\overline{B} \cup A)] \cap C]\}$.
    Then, distributing the terms in the subset that is the right-hand side of the union of 
    this superset gives the logical subset equivalence,
    $[(\overline{A} \cup B) \cap (\overline{B} \cup A)] \cap C \equiv 
    \{[\overline{A} \cap (\overline{B} \cup A)] \cup [B \cap (\overline{B} \cup A)]\} \cap C$. 
    The subset terms must be distributed further,
    $\{[(\overline{A} \cap \overline{B}) \cup (\overline{A} \cap A)] \cup 
    [(B \cap \overline{B}) \cup (B \cap A)]\} \cap C$.
    By the negation, and identity laws,
    $[(\overline{A} \cup B) \cap (\overline{B} \cup A)] \cap C \equiv 
    [(\overline{A} \cap \overline{B}) \cup (B \cap A)] \cap C$. Distributing $C$, 
    $[(\overline{A} \cup B) \cap (\overline{B} \cup A)] \cap C \equiv 
    (C \cap \overline{A} \cap \overline{B}) \cup (C \cap B \cap A)$.
    Finally, carrying out distribution on the subset that is the left-hand side of the union 
    of the superset gives us the logically equivalent superset statement \newline
    $(A \cap \overline{B} \cap \overline{C}) \cup (B \cap \overline{A} \cap \overline{C}) \cup 
    (C \cap \overline{A} \cap \overline{B}) \cup (C \cap B \cap A)$.
    
    Now let $x$ be an element in $A \oplus (B \oplus C)$. By the definition for the symmetric 
    difference of sets,
    $x \in [A \cap \overline{(B \oplus C)}] \cup [\overline{A} \cap (B \oplus C)]$. 
    By Theorem 2.2.36, $x$ is an element in 
    $\{A \cap \overline{[(B-C)\cup(C-B)]} 
    \} \cup \{\overline{A} \cap [(B-C)\cup(C-B)]\}$.
    And by Theorem 2.2.19, $\{A \cap \overline{[(B \cap \overline{C}) \cup 
    (C \cap \overline{B})]} 
    \} \cup \{\overline{A} \cap [(B \cap \overline{C})\cup(C \cap \overline{B})]\}$.
    Applying DeMorgans laws for sets to the subset that is the left-hand side of the union of 
    this superset, twice, produces the following logical superset equivalence,
    $x \in \{A \cap [(\overline{B} \cup C) \cap (\overline{C} \cup B)] 
    \} \cup \{\overline{A} \cap [(B \cap \overline{C})\cup(C \cap \overline{B})]\}$.
    Then, distributing the terms in the subset that is the left-hand side of the union of this 
    superset gives the logical subset equivalence
    $A \cap [(\overline{B} \cup C) \cap (\overline{C} \cup B)] \equiv 
    A \cap \{[\overline{B} \cap (\overline{C} \cup B)] \cup [C \cap (\overline{C} \cup B)]
    \}$.
    The subset terms must be distributed further, 
    $A \cap \{[(\overline{B} \cap \overline{C}) \cup (\overline{B} \cap B)] \cup 
    [(C \cap \overline{C}) \cup (C \cap B)]\}$. By the negation, and identity laws, 
    $A \cap [(\overline{B} \cup C) \cap (\overline{C} \cup B)] \equiv 
    A \cap [(\overline{B} \cap \overline{C}) \cup (C \cap B)]$.
    Distributing $A$, $A \cap [(\overline{B} \cup C) \cap (\overline{C} \cup B)] \equiv 
    (A \cap \overline{B} \cap \overline{C}) \cup (A \cap C \cap B)$.
    Finally, carrying out distribution on the subset that is the right-hand side of the union 
    of the superset gives us the logically equivalent superset statement \newline
    $(A \cap \overline{B} \cap \overline{C}) \cup (A \cap C \cap B) \cup 
    (\overline{A} \cap B \cap \overline{C}) \cup (\overline{A} \cap C \cap \overline{B})$.
    
    Because $x \in [(A \cap \overline{B} \cap \overline{C}) \cup (A \cap C \cap B) \cup 
    (\overline{A} \cap B \cap \overline{C}) \cup (\overline{A} \cap C \cap \overline{B})]$, 
    whenever $x$ is an element in $(A \oplus B) \oplus C$ and $x$ is an element in 
    $A \oplus (B \oplus C)$, it follows that $(A \oplus B) \oplus C= A \oplus (B \oplus C) 
    \therefore$ the symmetric difference for sets is indeed, associative.
\end{proof}

\pagebreak


% ============================= Theorem 2.2.41 ================================
\begin{theorem*}[\textbf{2.2.41}]
    Let A, B, and C be sets. If $A \oplus C = B \oplus C$, then $A = B$.
\end{theorem*}

\begin{proof} 
By contraposition. Note that the statement $A \oplus C = B \oplus C$ is by 
definition $(A \cap \overline{C}) \cup (\overline{A} \cap C) = 
(B \cap \overline{C}) \cup (\overline{B} \cap C)$. Assume there exists an element $x$ 
such that $x \in A$ and $x \notin B$. Thus, $A \not\subseteq B$. By the hypothesis, $x$ 
has to be in $(A \cap \overline{C})$ and cannot be in $(\overline{A} \cap C)$. This means 
that $x$ is not in $C$. Neither can $x$ be in $(B \cap \overline{C})$. And since 
$x \notin C$, $x$ cannot be in $(\overline{B} \cap C)$. So $x$ is in  $A \oplus C$ but not 
$B \oplus C$. Therefore, $A \oplus C \not\subseteq B \oplus C$. The implication, if 
$B \not\subseteq A$, then $B \oplus C \not\subseteq A \oplus C$, trivially follows without 
loss of generality. Conclusively, $A \neq B$ implies $A \oplus C \neq B \oplus C$.
\end{proof}

\end{document}