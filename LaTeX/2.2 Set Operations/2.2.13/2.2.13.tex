\documentclass[a4paper, 12pt]{article}
\usepackage[utf8]{inputenc}
\usepackage[english]{babel}
\usepackage{amssymb, amsmath, amsthm}
\theoremstyle{plain}
\newtheorem*{theorem*}{Theorem}
\newtheorem{theorem}{Theorem}

\usepackage{mathtools}
\renewcommand\qedsymbol{$\blacksquare$}

\begin{document}
	
	\begin{theorem*}[2.2.13]
		Let A and B be sets. $A \cap (A \cup B) = A$.
	\end{theorem*}
	
	\begin{proof}
		Let $x$ be an element in $A \cup (A \cap B)$. By the definitions for set union and set intersection 
		we have $(x \in A) \land [(x \in A) \lor (x \in B)]$. Treating each parenthesized object as a 
		discrete object we can immediately apply the logical law of absorption to this definition. The 
		result is $x \in A$. Thus it follows directly from the definition that $A \cap (A \cup B) = A$; the 
		consequence of which proves the absorption law for the intersection of a set with the union of 
		itself and another set.
	\end{proof}

\end{document}
