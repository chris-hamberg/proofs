\documentclass[a4paper, 12pt]{article}
\usepackage[utf8]{inputenc}
\usepackage[english]{babel}
\usepackage{amssymb, amsmath, amsthm}
\theoremstyle{plain}
\newtheorem*{theorem*}{Theorem}
\newtheorem{theorem}{Theorem}

\usepackage{mathtools}
\renewcommand\qedsymbol{$\blacksquare$}

\begin{document}
	
	\begin{theorem*}[2.2.16c]
		Let A and B be sets. $(A - B) \subseteq A$.
	\end{theorem*}
	
	\begin{proof}
		Let $x$ be an element in $A - B$. $A - B$ is equivalent to the statement 
		$A \cap \overline{B}$, and thus by definition we have $(x \in A) \land (x \notin B)$. 
		We can infer by the simplification rule that $x \in A$. It therefore follows immediately 
		from the definition that $(A - B) \subseteq A$.
	\end{proof}

\end{document}
