\documentclass[a4paper, 12pt]{article}
\usepackage[utf8]{inputenc}
\usepackage[english]{babel}
\usepackage{amssymb, amsmath, amsthm}
\theoremstyle{plain}
\newtheorem*{theorem*}{Theorem}
\newtheorem{theorem}{Theorem}

\usepackage{mathtools}
\renewcommand\qedsymbol{$\blacksquare$}

\begin{document}
	
	\begin{theorem*}[2.2.40]
		Let A, B, and C be sets. The symmetric difference for sets is associative such that 
		$(A \oplus B) \oplus C = A \oplus (B \oplus C)$.
	\end{theorem*}

	\begin{proof}
		Let $x$ be an element in $(A \oplus B) \oplus C$. By the definition for the symmetric 
		difference of sets,
		$x \in [(A \oplus B) \cap \overline{C}] \cup [\overline{(A \oplus B)} \cap C]$. 
		By Theorem 2.2.36, $x$ is an element in 
		$\{[(A - B) \cup (B - A)] \cap \overline{C}\} \cup 
		\{\overline{[(A - B) \cup (B - A)]} \cap C\}$. 
		And by Theorem 2.2.19,
		$\{[(A \cap \overline{B}) \cup (B \cap \overline{A})] \cap \overline{C}\} \cup 
		\{\overline{[(A \cap \overline{B}) \cup (B \cap \overline{A})]} \cap C]\}$. 
		Applying DeMorgans law for sets to the subset that is the right-hand side of the union 
		in this superset, twice, produces the following logical superset equivalence,
		$x \in \{[(A \cap \overline{B}) \cup (B \cap \overline{A})] \cap \overline{C}\} \cup 
		\{[(\overline{A} \cup B) \cap (\overline{B} \cup A)] \cap C]\}$.
		Then, distributing the terms in the subset that is the right-hand side of the union of 
		this superset gives the logical subset equivalence,
		$[(\overline{A} \cup B) \cap (\overline{B} \cup A)] \cap C \equiv 
		\{[\overline{A} \cap (\overline{B} \cup A)] \cup [B \cap (\overline{B} \cup A)]\} \cap C$. 
		The subset terms must be distributed further,
		$\{[(\overline{A} \cap \overline{B}) \cup (\overline{A} \cap A)] \cup 
		[(B \cap \overline{B}) \cup (B \cap A)]\} \cap C$.
		By the negation, and identity laws,
		$[(\overline{A} \cup B) \cap (\overline{B} \cup A)] \cap C \equiv 
		[(\overline{A} \cap \overline{B}) \cup (B \cap A)] \cap C$. Distributing $C$, 
		$[(\overline{A} \cup B) \cap (\overline{B} \cup A)] \cap C \equiv 
		(C \cap \overline{A} \cap \overline{B}) \cup (C \cap B \cap A)$.
		Finally, carrying out distribution on the subset that is the left-hand side of the union 
		of the superset gives us the logically equivalent superset statement \newline
		$(A \cap \overline{B} \cap \overline{C}) \cup (B \cap \overline{A} \cap \overline{C}) \cup 
		(C \cap \overline{A} \cap \overline{B}) \cup (C \cap B \cap A)$.
		
		Now let $x$ be an element in $A \oplus (B \oplus C)$. By the definition for the symmetric 
		difference of sets,
		$x \in [A \cap \overline{(B \oplus C)}] \cup [\overline{A} \cap (B \oplus C)]$. 
		By Theorem 2.2.36, $x$ is an element in 
		$\{A \cap \overline{[(B-C)\cup(C-B)]} 
		\} \cup \{\overline{A} \cap [(B-C)\cup(C-B)]\}$.
		And by Theorem 2.2.19, $\{A \cap \overline{[(B \cap \overline{C}) \cup 
		(C \cap \overline{B})]} 
		\} \cup \{\overline{A} \cap [(B \cap \overline{C})\cup(C \cap \overline{B})]\}$.
		Applying DeMorgans laws for sets to the subset that is the left-hand side of the union of 
		this superset, twice, produces the following logical superset equivalence,
		$x \in \{A \cap [(\overline{B} \cup C) \cap (\overline{C} \cup B)] 
		\} \cup \{\overline{A} \cap [(B \cap \overline{C})\cup(C \cap \overline{B})]\}$.
		Then, distributing the terms in the subset that is the left-hand side of the union of this 
		superset gives the logical subset equivalence
		$A \cap [(\overline{B} \cup C) \cap (\overline{C} \cup B)] \equiv 
		A \cap \{[\overline{B} \cap (\overline{C} \cup B)] \cup [C \cap (\overline{C} \cup B)]
		\}$.
		The subset terms must be distributed further, 
		$A \cap \{[(\overline{B} \cap \overline{C}) \cup (\overline{B} \cap B)] \cup 
		[(C \cap \overline{C}) \cup (C \cap B)]\}$. By the negation, and identity laws, 
		$A \cap [(\overline{B} \cup C) \cap (\overline{C} \cup B)] \equiv 
		A \cap [(\overline{B} \cap \overline{C}) \cup (C \cap B)]$.
		Distributing $A$, $A \cap [(\overline{B} \cup C) \cap (\overline{C} \cup B)] \equiv 
		(A \cap \overline{B} \cap \overline{C}) \cup (A \cap C \cap B)$.
		Finally, carrying out distribution on the subset that is the right-hand side of the union 
		of the superset gives us the logically equivalent superset statement \newline
		$(A \cap \overline{B} \cap \overline{C}) \cup (A \cap C \cap B) \cup 
		(\overline{A} \cap B \cap \overline{C}) \cup (\overline{A} \cap C \cap \overline{B})$.
		
		Because $x \in [(A \cap \overline{B} \cap \overline{C}) \cup (A \cap C \cap B) \cup 
		(\overline{A} \cap B \cap \overline{C}) \cup (\overline{A} \cap C \cap \overline{B})]$, 
		whenever $x$ is an element in $(A \oplus B) \oplus C$ and $x$ is an element in 
		$A \oplus (B \oplus C)$, it follows that $(A \oplus B) \oplus C= A \oplus (B \oplus C) 
		\therefore$ the symmetric difference for sets is indeed, associative.
	\end{proof}

\end{document}
