\documentclass[a4paper, 12pt]{article}
\usepackage[utf8]{inputenc}
\usepackage[english]{babel}
\usepackage{amssymb, amsmath, amsthm}
\theoremstyle{plain}
\newtheorem*{theorem*}{Theorem}
\newtheorem{theorem}{Theorem}

\usepackage{mathtools}
\renewcommand\qedsymbol{$\blacksquare$}

\begin{document}
	
\begin{theorem*}[2.2.41]
    Let A, B, and C be sets. If $A \oplus C = B \oplus C$, then $A = B$.
\end{theorem*}

\begin{proof} By contraposition. Note that the statement $A \oplus C = B \oplus C$ is by 
definition $(A \cap \overline{C}) \cup (\overline{A} \cap C) = 
(B \cap \overline{C}) \cup (\overline{B} \cap C)$. Assume there exists an element $x$ 
such that $x \in A$ and $x \notin B$. Thus, $A \not\subseteq B$. By the hypothesis, $x$ 
has to be in $(A \cap \overline{C})$ and cannot be in $(\overline{A} \cap C)$. This means 
that $x$ is not in $C$. Neither can $x$ be in $(B \cap \overline{C})$. And since 
$x \notin C$, $x$ cannot be in $(\overline{B} \cap C)$. So $x$ is in  $A \oplus C$ but not 
$B \oplus C$. Therefore, $A \oplus C \not\subseteq B \oplus C$. The implication, if 
$B \not\subseteq A$, then $B \oplus C \not\subseteq A \oplus C$, trivially follows without 
loss of generality. Conclusively, $A \neq B$ implies $A \oplus C \neq B \oplus C$.
\end{proof}

\end{document}
