\documentclass[a4paper, 12pt]{article}
\usepackage[utf8]{inputenc}
\usepackage[english]{babel}
\usepackage{amssymb, amsmath, amsthm}
\theoremstyle{plain}
\newtheorem*{theorem*}{Theorem}
\newtheorem{theorem}{Theorem}

\usepackage{mathtools}
\renewcommand\qedsymbol{$\blacksquare$}

\begin{document}
	
	\begin{theorem*}[2.2.7b]
		Let A be a set. The empty set dominates set intersection such that $A \cap \emptyset = \emptyset$.
	\end{theorem*}
	
	\begin{proof}
		Let $x$ be an element in $A \cap \emptyset$. By the definition for set intersection we have 
		$(x \in A) \land (x \in \emptyset)$. We know that $(x \in \emptyset) \equiv \bot$ because 
		the empty set is empty. The logical law of domination gives us that 
		$(x \in A) \land \bot \equiv \bot$. It immediately follows that 
		$(x \in A) \land (x \in \emptyset) \equiv (x \in \emptyset)$, which is the definition of 
		$A \cap \emptyset = \emptyset$. Thus proves the law of set domination for the intersection 
		of sets. 
	\end{proof}

\end{document}
