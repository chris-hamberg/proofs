\documentclass[a4paper, 12pt]{article}
\usepackage[utf8]{inputenc}
\usepackage[english]{babel}
\usepackage{amssymb, amsmath, amsthm}
\theoremstyle{plain}
\newtheorem*{theorem*}{Theorem}
\newtheorem{theorem}{Theorem}

\usepackage{mathtools}
\renewcommand\qedsymbol{$\blacksquare$}

\begin{document}
	
	\begin{theorem*}[2.2.8a]
		Let A be a set. A is idempotent such that $A \cup A = A$.
	\end{theorem*}
	
	\begin{proof}
		Let $x$ be an element in $A \cup A$. By the definition of set union we have, \newline 
		$(x \in A) \lor (x \in A)$. The logical idempotent law says \newline 
		$(x \in A) \lor (x \in A) \equiv (x \in A)$ which is precisely $A \cup A = A$ by definition. 
		Thus proves the idempotent law for the union of sets.
	\end{proof}

\end{document}
