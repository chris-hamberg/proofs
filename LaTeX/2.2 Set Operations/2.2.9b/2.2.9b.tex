\documentclass[a4paper, 12pt]{article}
\usepackage[utf8]{inputenc}
\usepackage[english]{babel}
\usepackage{amssymb, amsmath, amsthm}
\theoremstyle{plain}
\newtheorem*{theorem*}{Theorem}
\newtheorem{theorem}{Theorem}

\usepackage{mathtools}
\renewcommand\qedsymbol{$\blacksquare$}

\begin{document}
	
	\begin{theorem*}[2.2.9b]
		Let A be a set. $A \cap \overline{A} = \emptyset$.
	\end{theorem*}
	
	\begin{proof}
		Let $x$ be an element in $A \cap \overline{A}$. By definition, 
		$(x \in A) \land (x \in \overline{A})$. The right-hand side of this conjunction, according to the 
		definitions of set complementation and logical negation, can be restated as 
		\newline $(x \notin A) \equiv \lnot (x \in A)$. Again, by logical negation, we have 
		\newline $(x \in A) \land \lnot (x \in A) \equiv \bot$. This is the very meaning of 
		$A \cap \overline{A} = \emptyset$. Thus proves the set complementation law for the intersection of 
		sets.
	\end{proof}
	
\end{document}
