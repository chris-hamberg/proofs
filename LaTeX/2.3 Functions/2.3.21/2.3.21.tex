\documentclass[a4paper, 12pt]{article}
\usepackage[utf8]{inputenc}
\usepackage[english]{babel}
\usepackage{amssymb, amsmath, amsthm}
\theoremstyle{plain}
\newtheorem*{theorem*}{Theorem}
\newtheorem{theorem}{Theorem}

\usepackage{mathtools}
\renewcommand\qedsymbol{$\blacksquare$}

\begin{document}
	
	\begin{theorem*}[2.3.21]
		Let $f$ be the function f: $\mathbb{R} \implies \mathbb{R}$, such that \newline $\forall x ((x \in \mathbb{R}) \implies (f(x) > 0))$. Let g be the function $g: \mathbb{R} \implies \mathbb{R}$ defined by g(x) = 1/f(x). f(x) is strictly decreasing if and only if g(x) is strictly increasing.
	\end{theorem*}
	
	\begin{proof}
		Suppose there exist real numbers $x$ and $y$ such that $x < y$, and suppose that $f(x) > f(y)$. $f$ is a strictly decreasing real-valued function by definition. It follows that $g(x) = 1/f(x) < g(y) = 1/f(y)$, which is the definition for strictly increasing real-valued functions.
		
		Conversely, suppose there exist real numbers $x$ and $y$ such that $x < y$, and suppose that $g(x) < g(y)$. $g$ is a strictly increasing real-valued function by definition. It follows that $f(x) = 1/g(x) > f(y) = 1/f(y)$, which is the definition for strictly decreasing real-valued functions.
		
		Thus, $f(x)$ is strictly decreasing if and only if $g(x)$ is strictly increasing.
	\end{proof}

\end{document}
