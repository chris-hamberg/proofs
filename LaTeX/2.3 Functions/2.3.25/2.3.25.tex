\documentclass[a4paper, 12pt]{article}
\usepackage[utf8]{inputenc}
\usepackage[english]{babel}
\usepackage{amssymb, amsmath, amsthm}
\theoremstyle{plain}
\newtheorem*{theorem*}{Theorem}
\newtheorem{theorem}{Theorem}

\usepackage{mathtools}
\renewcommand\qedsymbol{$\blacksquare$}

\begin{document}
	
	\begin{theorem*}[2.3.25]
		Let $f$ be a function $f: \mathbb{R} \implies \mathbb{R}$ defined by $f(x) = |x|$. $f(x)$ 
		is not invertible.
	\end{theorem*}
	
	\begin{proof}
		Let $x$ be a postive real number. $f(x) = y$ and $f(-x) = y$. If $f$ had an inverse then 
		$f^{-1}(y) = x$ or $f^{-1}(y) = -x$, so $f^{-1}$ is not a function by definition. Which
		concludes the proof.
	\end{proof}

\end{document}
