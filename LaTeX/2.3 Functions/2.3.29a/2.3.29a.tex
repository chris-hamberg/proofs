\documentclass[a4paper, 12pt]{article}
\usepackage[utf8]{inputenc}
\usepackage[english]{babel}
\usepackage{amssymb, amsmath, amsthm}
\theoremstyle{plain}
\newtheorem*{theorem*}{Theorem}
\newtheorem{theorem}{Theorem}

\usepackage{mathtools}
\renewcommand\qedsymbol{$\blacksquare$}

\begin{document}
	
	\begin{theorem*}[2.3.29a]
		Let f be a function $f: B \implies C$, and let g be a function $g: A \implies B$. If both 
		f and g are injective, then $f \circ g$ is injective.
	\end{theorem*}
	
	\begin{proof}
		By the contrapositive. Let the domain of discourse be $A$. 
		\newline Suppose it were not the case that $(f \circ g)$ were injective. Then by the 
		\newline definition for injective functions, the following universally quantified 
		\newline statement is true, 
		$\lnot \forall a \forall b ((f \circ g)(a) = (f \circ g)(b) \implies (a = b))$. Note that 
		the composition of functions $(f \circ g)(x)$ is defined by $f(g(x))$. Thus, we have the 
		equivalent universal quantification 
		$\lnot \forall a \forall b (f(g(a)) = f(g(b)) \implies (a = b))$. In other words, it is not 
		the case that $f$ is injective, by the definition for injective functions. Also, because 
		$f = f$, $\lnot \forall a \forall b (g(a) = g(b) \implies (a = b))$ is a logically 
		equivalent universal quantification. That is, it is not the case that $g$ is injective, by the 
		definition for injective functions. Since the contrapositive follows directly from the negation 
		of the conclusion, it is necessarily the case that if both f and g are injective, then 
		$f \circ g$ is injective.
	\end{proof}

\end{document}
