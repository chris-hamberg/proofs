\documentclass[a4paper, 12pt]{article}
\usepackage[utf8]{inputenc}
\usepackage[english]{babel}
\usepackage{amssymb, amsmath, amsthm}
\theoremstyle{plain}
\newtheorem*{theorem*}{Theorem}
\newtheorem{theorem}{Theorem}

\usepackage{mathtools}
\renewcommand\qedsymbol{$\blacksquare$}

\begin{document}
	
	\begin{theorem*}[2.3.30]
		Let f and $f \circ g$ be injective functions. g is injective.
	\end{theorem*}
	
	\begin{proof}
		By the contrapositive. Suppose that $g$ were not injective. Then by the definition for injective functions we have the following universally quantified statement, with the domain of discourse being the domain of $g$, \newline $\lnot \forall a \forall b ((g(a) = g(b)) \implies (a = b))$. Because $f = f$, this statement is logically equivalent to $\lnot \forall a \forall b ((f(g(a)) = f(g(b))) \implies (a = b))$. By the \newline definition for the compositions of functions we can also draw this equivalence, $\lnot \forall a \forall b ((f \circ g)(a) = (f \circ g)(b)) \implies (a = b))$. That is, it is not the case that $f \circ g$ is injective, by the definition for injective functions. So it follows directly from the negation of the statement "$g$ is injective," that $f \circ g$ is not injective. Thus, if $f$ and $(f \circ g)$ are injective functions, then $g$ is indeed injective.
	\end{proof}

\end{document}
