\documentclass[a4paper, 12pt]{article}
\usepackage[utf8]{inputenc}
\usepackage[english]{babel}
\usepackage{amssymb, amsmath, amsthm}
\theoremstyle{plain}
\newtheorem*{theorem*}{Theorem}
\newtheorem{theorem}{Theorem}

\usepackage{mathtools}
\renewcommand\qedsymbol{$\blacksquare$}

\begin{document}
	
	\begin{theorem*}[2.2.36b]
		Let f be the function $f: A \implies B$. Let S, and T be subsets of A. $f(S \cap T) \subseteq f(S) \cap f(T)$.
	\end{theorem*}
	
	\begin{proof}
		Let $a$ be an element in $A$ such that $f(a) \in f(S \cap T)$. Hence, by the definition for the image of $(S \cap T)$ under the function $f$, $a \in (S \cap T)$. The set intersection is defined as $(a \in S) \land (a \in T)$. Of course \newline $[f(a) \in f(S)] \land [f(a) \in f(T)]$. That is, $f(a) \in [f(S) \cap f(T)]$, and indeed $f(S \cap T) \subseteq f(S) \cap f(T)$.
	\end{proof}

\end{document}
