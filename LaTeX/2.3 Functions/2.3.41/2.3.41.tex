\documentclass[a4paper, 12pt]{article}
\usepackage[utf8]{inputenc}
\usepackage[english]{babel}
\usepackage{amssymb, amsmath, amsthm}
\theoremstyle{plain}
\newtheorem*{theorem*}{Theorem}
\newtheorem{theorem}{Theorem}

\usepackage{mathtools}
\renewcommand\qedsymbol{$\blacksquare$}

\begin{document}
	
	\begin{theorem*}[2.3.41]
		Let f be the function $f: A \implies B$. Let S be a subset of B. 
		$f^{-1}(\overline{S}) = \overline{f^{-1}(S)}$.
	\end{theorem*}
	
	\begin{proof}
		By the definition for the inverse image of $\overline{S}$ under the function $f^{-1}$, 
		we have  $f^{-1}(\overline{S}) \equiv \{a \in A|f(a) \in \overline{S}\}$. Factoring the 
		complementation out from the right-hand side of the equivalence, 
		$f^{-1}(\overline{S}) \equiv \overline{\{a \in A | f(a) \in S}\}$. But this statement is the 
		negation of the formal definition for the inverse image of $S$ under the function $f^{-1}$. 
		In other words, $f^{-1}(\overline{S}) = \overline{f^{-1}(S)}$.
	\end{proof}

\end{document}
