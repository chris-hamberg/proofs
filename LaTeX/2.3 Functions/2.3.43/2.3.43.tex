\documentclass[a4paper, 12pt]{article}
\usepackage[utf8]{inputenc}
\usepackage[english]{babel}
\usepackage{amssymb, amsmath, amsthm}
\theoremstyle{plain}
\newtheorem*{theorem*}{Theorem}
\newtheorem{theorem}{Theorem}

\usepackage{mathtools}
\renewcommand\qedsymbol{$\blacksquare$}
\DeclarePairedDelimiter{\floor}{\lfloor}{\rfloor}
\DeclarePairedDelimiter{\ceil}{\lceil}{\rceil}

\begin{document}
	
	\begin{theorem*}[2.3.43]
		Let x be a real number. $\ceil{x - \frac{1}{2}}$ is the closest integer to x, except when 
		x is midway between two integers, when it is the smaller of these two integers.
	\end{theorem*}
	
	\begin{proof}
		By cases. Let $n$ be the integer such that $n \le x < n+1$ and \newline 
		$\ceil{x - \frac{1}{2}} = \ceil{(n + \epsilon) - \frac{1}{2})}$. $\epsilon$ is the decimal 
		part of $x$.
		\newline \newline $(i)$ If $\epsilon > \frac{1}{2}$, then 
		$\epsilon - \frac{1}{2} > \frac{1}{2} - \frac{1}{2} = 0$. So 
		$\ceil{(n + \epsilon) - \frac{1}{2}} = n + 1.$ \newline \newline $(ii)$ If 
		$\epsilon \le \frac{1}{2}$, then $\epsilon - \frac{1}{2} \le 0$. So 
		$(n - 1) \le [(n + \epsilon) - \frac{1}{2}] < n$, and $\ceil{x - \frac{1}{2}} = n$.
	\end{proof}

\end{document}
