\documentclass[a4paper, 12pt]{article}
\usepackage[utf8]{inputenc}
\usepackage[english]{babel}
\usepackage{amssymb, amsmath, amsthm}
\theoremstyle{plain}
\newtheorem*{theorem*}{Theorem}
\newtheorem{theorem}{Theorem}

\usepackage{mathtools}
\renewcommand\qedsymbol{$\blacksquare$}
\DeclarePairedDelimiter{\floor}{\lfloor}{\rfloor}
\DeclarePairedDelimiter{\ceil}{\lceil}{\rceil}

\begin{document}
	
	\begin{theorem*}[2.3.47a]
		Let x be a real number, and let n be an integer. \newline $x < n \iff \floor{x} < n$.
	\end{theorem*}
	
	\begin{proof}
		$\floor{x} \le x$, by the properties of the floor function. So if $x < n$, then 
		$\floor{x} \le x < n$, and of course $\floor{x} < n$.
		
		Proving the converse, suppose $\floor{x} < n$. Since $\floor{x}$ and $n$ are integers, 
		$\floor{x} + 1 \le n$. Now, by the properties of the floor function, we have the following 
		tautology, $\floor{x} = \floor{x} \iff \floor{x} \le x < \floor{x} + 1$. Since we know 
		that $\floor{x} + 1 \le n$, it must be that $\floor{x} \le x < \floor{x} + 1 \le n$. This 
		statement says that $x < n$.
	\end{proof}

\end{document}
