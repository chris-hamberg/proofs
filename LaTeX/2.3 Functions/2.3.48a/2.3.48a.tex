\documentclass[a4paper, 12pt]{article}
\usepackage[utf8]{inputenc}
\usepackage[english]{babel}
\usepackage{amssymb, amsmath, amsthm}
\theoremstyle{plain}
\newtheorem*{theorem*}{Theorem}
\newtheorem{theorem}{Theorem}

\usepackage{mathtools}
\renewcommand\qedsymbol{$\blacksquare$}
\DeclarePairedDelimiter{\floor}{\lfloor}{\rfloor}
\DeclarePairedDelimiter{\ceil}{\lceil}{\rceil}

\begin{document}
	
	\begin{theorem*}[2.3.48a]
		Let x be a real number, and let n be an integer. \newline $x \le n \iff \ceil{x} \le n$.
	\end{theorem*}
	
	\begin{proof}
		Direct form by the contrapositive. Suppose $\ceil{x} > n$. Since $\ceil{x}$ and $n$ are 
		integers, $\ceil{x} - 1 \ge n$. By the properties of ceiling functions we have the 
		following tautology, $\ceil{x} = \ceil{x} \iff \ceil{x} \ge x > \ceil{x} - 1$. Combining 
		these two inequalities yields $\ceil{x} \ge x > \ceil{x} - 1 \ge n$. This says, $x > n$. 
		Since this statement following from the negation of the direct consequent is itself the 
		negation of the direct hypothesis, $x \le n \implies \ceil{x} \le n$, is true.
		
		Converse form by the contrapositive. Suppose $x > n$. Note that $\ceil{x} \ge x$, by the 
		properties of the ceiling function. So if $x > n$, then $\ceil{x} \ge x > n$, and 
		$\ceil{x} > n$. Since this statement is the negation of the converse hypothesis following 
		directly from negation of the converse consequent, \newline 
		$x \le n \impliedby \ceil{x} \le n$, is true.
		
		Thus proves, the biconditional statement $x \le n \iff \ceil{x} \le n$. 
	\end{proof}

\end{document}
