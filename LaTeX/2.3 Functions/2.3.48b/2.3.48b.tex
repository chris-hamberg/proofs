\documentclass[a4paper, 12pt]{article}
\usepackage[utf8]{inputenc}
\usepackage[english]{babel}
\usepackage{amssymb, amsmath, amsthm}
\theoremstyle{plain}
\newtheorem*{theorem*}{Theorem}
\newtheorem{theorem}{Theorem}

\usepackage{mathtools}
\renewcommand\qedsymbol{$\blacksquare$}
\DeclarePairedDelimiter{\floor}{\lfloor}{\rfloor}
\DeclarePairedDelimiter{\ceil}{\lceil}{\rceil}

\begin{document}
	
	\begin{theorem*}[2.3.48b]
		Let x be a real number, and let n be an integer. \newline $n \le x \iff n \le \floor{x}$.
	\end{theorem*}
	
	\begin{proof}
		By the direct form contrapositive. Suppose $n > \floor{x}$. Since $n$ and $\floor{x}$ are integers, $n \ge \floor{x} + 1$. Now, by the properties of the floor function, we have the following tautology, $\floor{x} = \floor{x} \iff \floor{x} + 1 > x \ge \floor{x}$. Combining these two inequalities yields $n \ge \floor{x} + 1 > x \ge \floor{x}$ This statement says that $n > x$.
		
		Proving the converse form by the contrapositive. Note that $x \ge \floor{x}$, by the properties of the floor function. So if $n > x$, then $n > x \ge \floor{x}$, and of course $n > \floor{x}$.
		
		$\therefore n \le x \iff n \le \floor{x}$
	\end{proof}

\end{document}
