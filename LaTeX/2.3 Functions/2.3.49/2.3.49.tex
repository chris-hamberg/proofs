\documentclass[a4paper, 12pt]{article}
\usepackage[utf8]{inputenc}
\usepackage[english]{babel}
\usepackage{amssymb, amsmath, amsthm}
\theoremstyle{plain}
\newtheorem*{theorem*}{Theorem}
\newtheorem{theorem}{Theorem}

\usepackage{mathtools}
\renewcommand\qedsymbol{$\blacksquare$}
\DeclarePairedDelimiter{\floor}{\lfloor}{\rfloor}
\DeclarePairedDelimiter{\ceil}{\lceil}{\rceil}

\begin{document}
	
	\begin{theorem*}[2.3.49]
		Let n be an integer. If n is even, then $\floor{\frac{n}{2}} = \frac{n}{2}$. \newline If n is odd, then $\floor{\frac{n}{2}} = \frac{(n - 1)}{2}$.
	\end{theorem*}
	
	\begin{proof}
		By cases. \newline \newline $(i)$ Since $n$ is even,  there exists an integer $k$ such that $n = 2k$. \newline $\floor{\frac{n}{2}} = \floor{\frac{2k}{2}} = \floor{k} = k$. Also, $\frac{n}{2} = \frac{2k}{2} = k$. So $k = \floor{\frac{n}{2}} = \frac{n}{2}$.
		\newline
		\newline
		$(ii)$ Since $n$ is odd, there exists an integer $k$ such that $n = 2k + 1$. \newline $\floor{\frac{n}{2}} = \floor{\frac{2k + 1}{2}} = \floor{k + \frac{1}{2}} = k$. Also, $\frac{(n-1)}{2} = \frac{[(2k + 1) - 1]}{2} = \frac{2k}{2} = k$. \newline So, $k = \floor{\frac{n}{2}} = \frac{(n - 1)}{2}$.
	\end{proof}

\end{document}
