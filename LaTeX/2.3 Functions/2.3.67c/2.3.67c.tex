\documentclass[a4paper, 12pt]{article}
\usepackage[utf8]{inputenc}
\usepackage[english]{babel}
\usepackage{amssymb, amsmath, amsthm}
\theoremstyle{plain}
\newtheorem*{theorem*}{Theorem}
\newtheorem{theorem}{Theorem}

\usepackage{mathtools}
\renewcommand\qedsymbol{$\blacksquare$}
\DeclarePairedDelimiter{\floor}{\lfloor}{\rfloor}
\DeclarePairedDelimiter{\ceil}{\lceil}{\rceil}

\begin{document}
	
	\begin{theorem*}[2.3.67c]
		Let A be a set with universal set U. Let $f_{A}$ be the \newline characteristic function 
		$f_{\overline{A}}: U \implies \{0, 1\}$. $f_{\overline{A}}(x) = 1 - f_{A}(x)$.
	\end{theorem*}
	
	\begin{proof}
		Let $x$ be an element in $A$. Then clearly $x \notin \overline{A}$. By the definition for 
		characteristic functions $f_{\overline{A}}(x) = 0$, and $f_{A}(x) = 1$. It follows 
		immediately that $f_{\overline{A}}(x) = 1 - f_{A}(x)$.
		
		Now suppose $(x \notin A) \land (x \in \overline{A})$ By the definition for characteristic 
		functions that is $f_{\overline{A}}(x) = 1$, and $f_{A}(x) = 0$. It follows immediately 
		that $f_{\overline{A}}(x) = 1 - f_{A}(x)$.
	\end{proof}

\end{document}
