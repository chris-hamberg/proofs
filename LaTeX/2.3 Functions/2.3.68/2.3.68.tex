\documentclass[a4paper, 12pt]{article}
\usepackage[utf8]{inputenc}
\usepackage[english]{babel}
\usepackage{amssymb, amsmath, amsthm}
\theoremstyle{plain}
\newtheorem*{theorem*}{Theorem}
\newtheorem{theorem}{Theorem}

\usepackage{mathtools}
\renewcommand\qedsymbol{$\blacksquare$}
\DeclarePairedDelimiter{\floor}{\lfloor}{\rfloor}
\DeclarePairedDelimiter{\ceil}{\lceil}{\rceil}

\begin{document}
	
	\begin{theorem*}[2.3.68]
		Let f be a function $f: A \implies B$, where A and B are finite sets, and $|A| = |B|$. f is injective if and only if f is surjective.
	\end{theorem*}
	
	\begin{proof}
		Direct form by the contrapositive. Suppose the negation of the \newline statement given by the definition for surjective functions, $\exists y \forall x (f(x) \ne y)$. This statement can only be true if either $|A| < |B|$ (contradicting the \newline hypothesis,) or $\exists x \exists y ((f(x) = f(y)) \land (x \ne y))$. Since contradiction is $\bot$ by the law for logical negation, by the identity law for logical disjunction, $f$ is defined as not injective. Thus, if $f$ is injective, then $f$ is surjective.
		
		Converse form by the contrapositive. Suppose the negation of the \newline statement given by the definition for injective functions, \newline $\exists x \exists y ((f(x) = f(y)) \land (x \ne y))$. 
		This statement can only be true if either $|A| > |B|$ (contradicting the hypothesis,) or $\exists y \forall x (f(x) \ne y)$. Since \newline contradiction is $\bot$ by the law for logical negation, by the identity law for logical disjunction, $f$ is defined as not surjective. Thus, if $f$ is surjective, then $f$ is injective.
		
		$\therefore \forall x \forall y ((f(x) = f(y)) \implies (x = y)) \iff \forall y \exists x (f(x) = y)$.
	\end{proof}

\end{document}
