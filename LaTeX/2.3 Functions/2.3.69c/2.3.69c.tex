\documentclass[a4paper, 12pt]{article}
\usepackage[utf8]{inputenc}
\usepackage[english]{babel}
\usepackage{amssymb, amsmath, amsthm}
\theoremstyle{plain}
\newtheorem*{theorem*}{Theorem}
\newtheorem{theorem}{Theorem}

\usepackage{mathtools}
\renewcommand\qedsymbol{$\blacksquare$}
\DeclarePairedDelimiter{\floor}{\lfloor}{\rfloor}
\DeclarePairedDelimiter{\ceil}{\lceil}{\rceil}

\begin{document}

	\begin{theorem*}[2.3.69c]
		Let x and y be real numbers. \newline $\ceil{x} + \ceil{y} - \ceil{x + y} = 0$, or $1$.		
	\end{theorem*}
	
	\begin{proof}
		By cases. There are two possible cases to take into consideration. \newline $(i)$ $x$ or 
		$y$ (or both) are integers in real numbers, or $(ii)$ neither $x$ nor $y$ is an integer.
		\newline
		\newline
		$(i)$ Suppose $x$ or $y$ (or both) are integers in real numbers. Since at least one of 
		these numbers $x$ or $y$ must be an integer, and because addition is commutative, without 
		loss of generality it can be supposed that $y$ is certainly an integer.	Then since $y$ is 
		an integer, the smallest integer greater than or equal to $y$, is $y$. So by the 
		definition for ceiling functions, $\ceil{y} = y$. By that fact, and by Theorem 2.3.46, 
		$\ceil{x} + \ceil{y} - \ceil{x + y} = \ceil{x} + \ceil{y} - \ceil{x} + \ceil{y} = 0$.
		\newline
		\newline
		$(ii)$ Suppose that neither $x$ nor $y$ is an integer. Then let $\epsilon$ and $\sigma$ be 
		real numbers such that $\ceil{x} - x = \epsilon$, and $\ceil{y} - y = \sigma$. By Theorem 
		2.3.44, $\ceil{x} = \floor{x} + 1$, and $\ceil{y} = \floor{y} + 1$. Naturally, 
		$x = \floor{x} + (1 - \epsilon)$, and $y = \floor{y} + (1 - \sigma)$. Thus, 
		$\ceil{x} + \ceil{y} - \ceil{x + y} = 
		(\floor{x}+1) + (\floor{y}+1) - \ceil{\floor{x} + (1 - \epsilon) + \floor{y} + (1 - \sigma)}$. 
		Rearranging these terms according to the usual rules for arithmetic yields 
		$(\floor{x} + \floor{y} + 2) - \ceil{\floor{x} + \floor{y} + [2 - (\epsilon + \sigma)]}$. 
		Now, there are two possible sub cases to consider regarding this expression. Either 
		$(a)$ $\epsilon + \sigma \ge 1$, or $(b)$ $\epsilon + \sigma < 1$.
		\newline
		\newline
		\indent $(a)$ Suppose 
		$\epsilon + \sigma \ge 1$. This means that $1 \ge 2 - (\epsilon + \sigma)$. By  \newline 
		\indent Theorem 2.3.69a we get the following equation, \newline \indent 
		$(\floor{x} + \floor{y} + 2) - \ceil{\floor{x} + \floor{y} + [2 - (\epsilon + \sigma)]} =$ 
		\newline \indent $(\floor{x} + \floor{y} + 2) - (\floor{x} + \floor{y} + 1) = 1$.
		\newline
		\newline
		\indent $(b)$ Suppose $\epsilon + \sigma < 1$. This means that $1 < 2 - (\epsilon + \sigma)$. 
		By \newline \indent Theorem 2.3.69a we get the following equation, \newline \indent 
		$(\floor{x} + \floor{y} + 2) - \ceil{\floor{x} + \floor{y} + [2 - (\epsilon + \sigma)]} =$ 
		\newline \indent $(\floor{x} + \floor{y} + 2) - (\floor{x} + \floor{y} + 2) = 0$.
		\newline
		\newline
		$\therefore \ceil{x} + \ceil{y} - \ceil{x + y} = 0$, or $1$, whenever $x$ and $y$ are real 
		numbers.
	\end{proof}

\end{document}
