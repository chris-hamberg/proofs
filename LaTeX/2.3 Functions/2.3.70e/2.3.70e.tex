\documentclass[a4paper, 12pt]{article}
\usepackage[utf8]{inputenc}
\usepackage[english]{babel}
\usepackage{amssymb, amsmath, amsthm}
\theoremstyle{plain}
\newtheorem*{theorem*}{Theorem}
\newtheorem{theorem}{Theorem}

\usepackage{mathtools}
\renewcommand\qedsymbol{$\blacksquare$}
\DeclarePairedDelimiter{\floor}{\lfloor}{\rfloor}
\DeclarePairedDelimiter{\ceil}{\lceil}{\rceil}

\begin{document}
	
	\begin{theorem*}[2.3.70e]
		Let x, and y be real numbers. 
		$\floor{x} + \floor{y} + \floor{x + y} \le \floor{2x} + \floor{2y}.$
	\end{theorem*}
	
	\begin{proof}
		There are four possible cases that could be considered in this proof, but only two of 
		those cases require consideration. Case $(i)$ demonstrating the minimum possible amount 
		occurring on the right-hand side will establish truth for equality expressed by the 
		theorem. Case $(ii)$ demonstrating the maximum possible amount occurring on the right-hand 
		side will establish truth for inequality expressed by the theorem. In all cases the 
		left-hand side remains relatively constant.
		
		First we establish the necessary preliminary facts. Let $\epsilon$ and $\sigma$ be real 
		numbers such that $x - \floor{x} = \epsilon$, and $y - \floor{y} = \sigma$. Of course, 
		$x - \epsilon = \floor{x}$. By the properties for floor functions we know that 
		$x-\epsilon \le x < (x - \epsilon) + 1$. And, again by the properties for floor functions,
		$2(x-\epsilon) \le 2x < 2[(x - \epsilon) + 1] \implies \floor{2x}$. This means that $(i)$ 
		$\floor{2x} = 2x-2\epsilon$, or (logical) $(ii)$ $\floor{2x} = 2x - 2\epsilon + 2$. 
		Without loss of generality, all of these equations remain true whenever predicated of $y$ 
		and $\sigma$. Also, note that the left-hand side has a constant form, 
		$\floor{x} + \floor{y} + \floor{x + y} = 2(x + y) - 2(\epsilon + \sigma)$.
		\newline
		\newline
		\indent $(i)$ Suppose it were the case that $\floor{2x} = 2x-2\epsilon$, and 
		$\floor{2y} = 2y-2\sigma$. \indent These are the least amounts possible for the right-hand 
		side of the \newline \indent inequality expressed by the theorem. The sum being 
		$2(x + y) - 2(\epsilon + \sigma)$, \indent equal to the left-hand side. In this case 
		$(i)$ the theorem proves true.
		\newline
		\newline
		\indent $(ii)$ Suppose it were the case that $\floor{2x} = 2x - 2\epsilon + 2$, and 
		\newline \indent $\floor{2y} = 2y - 2\sigma + 2$. These are the greatest amounts possible 
		for the \indent right-hand side of the inequality expressed by the theorem. The sum 
		\newline \indent being $2(x + y + 2) - 2(\epsilon + \sigma)$. This is clearly greater than 
		the left-hand \indent side. In this case $(ii)$ the theorem proves true.
		\newline
		\newline
		Since the entire range of all possible values are covered by cases $(i)$ and $(ii)$, and 
		the statement remains true throughout, it is proven that \newline 
		$\floor{x} + \floor{y} + \floor{x + y} \le \floor{2x} + \floor{2y}.$
	\end{proof}

\end{document}
