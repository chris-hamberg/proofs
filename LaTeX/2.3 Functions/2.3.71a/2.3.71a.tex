\documentclass[a4paper, 12pt]{article}
\usepackage[utf8]{inputenc}
\usepackage[english]{babel}
\usepackage{amssymb, amsmath, amsthm}
\theoremstyle{plain}
\newtheorem*{theorem*}{Theorem}
\newtheorem{theorem}{Theorem}

\usepackage{mathtools}
\renewcommand\qedsymbol{$\blacksquare$}
\DeclarePairedDelimiter{\floor}{\lfloor}{\rfloor}
\DeclarePairedDelimiter{\ceil}{\lceil}{\rceil}

\begin{document}
	
	\begin{theorem*}[2.3.71a]
		Let x be a positive real number. $\floor{\sqrt{\floor{x}}} = \floor{\sqrt{x}}$.
	\end{theorem*}
	
	\begin{proof}
		By the properties for floor functions, \newline 
		$\floor{\sqrt{x}} = n \iff n \le \sqrt{x} < n + 1.$ Squaring the inequalities we can 
		determine the value for the floor of $x$. So there are two cases under consideration, 
		\newline $(i)$ $\floor{x} = n^2$, or $(ii)$ $\floor{x} = n^{2} + 2n$.
		\newline
		\newline \indent
		$(i)$ Suppose that $\floor{x} = n^{2}$. It follows that 
		$\floor{\sqrt{\floor{x}}} = \floor{\sqrt{n^2}} = \floor{n}$. \indent Since $n$ is an 
		integer, $n$ is the largest integer less than or equal to $n$. So \indent 
		$\floor{n} = n$, by the definition for floor functions. Because $n = \floor{\sqrt{x}}$, 
		in this \indent case it is proved that $\floor{\sqrt{\floor{x}}} = \floor{\sqrt{x}}$.
		\newline
		\newline \indent
		$(ii)$ Suppose that $\floor{x} = n^{2} + 2n$. Then, \newline \indent 
		$\floor{\sqrt{\floor{x}}} = \floor{\sqrt{n^{2} + 2n}} = 
		\floor{\sqrt{n^2 + 2n} + \sqrt{1} - \sqrt{1}} = \newline \indent 
		\floor{\sqrt{n^2 + 2n + 1} - 1} = \floor{\sqrt{(n+1)^2} + 1} = \floor{(n+1) -1} = n$. 
		Since \indent $n = \floor{\sqrt{x}}$, in this case it is proved that 
		$\floor{\sqrt{\floor{x}}} = \floor{\sqrt{x}}$.
	\end{proof}

\end{document}
