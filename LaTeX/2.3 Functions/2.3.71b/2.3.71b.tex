\documentclass[a4paper, 12pt]{article}
\usepackage[utf8]{inputenc}
\usepackage[english]{babel}
\usepackage{amssymb, amsmath, amsthm}
\theoremstyle{plain}
\newtheorem*{theorem*}{Theorem}
\newtheorem{theorem}{Theorem}

\usepackage{mathtools}
\renewcommand\qedsymbol{$\blacksquare$}
\DeclarePairedDelimiter{\floor}{\lfloor}{\rfloor}
\DeclarePairedDelimiter{\ceil}{\lceil}{\rceil}

\begin{document}
	
	\begin{theorem*}[2.3.71b]
		Let x be a positive real number. $\ceil{\sqrt{\ceil{x}}} = \ceil{\sqrt{x}}$.
	\end{theorem*}
	
	\begin{proof}
		By the properties for floor functions, \newline $\ceil{\sqrt{x}} \iff n-1 < \sqrt{x} \le n$. Squaring the inequalities we can determine the value for the floor of $x$. Thus, there are two cases under consideration $(i)$ $\ceil{x} = n^{2} - 2n$, or $(ii)$ $\ceil{x} = n^{2}$.
		\newline
		\newline
		\indent $(i)$ Suppose that $\ceil{x} = n^{2} - 2n$. It follows that, \newline \indent $\ceil{\sqrt{\ceil{x}}} = \ceil{\sqrt{n^{2} - 2n}} = \ceil{\sqrt{n^{2} - 2n} - \sqrt{1} + \sqrt{1}} = \newline \indent \ceil{\sqrt{n^{2} - 2n - 1} + 1} = \ceil{\sqrt{(n-1)^{2}} + 1} = \ceil{(n-1)+1} = n$. Since \indent $n = \ceil{\sqrt{x}}$, in this case it is proved that $\ceil{\sqrt{\ceil{x}}} = \ceil{\sqrt{x}}$.
		\newline
		\newline
		\indent
		$(ii)$ Suppose that $\ceil{x} = n^{2}$. It follows that $\ceil{\sqrt{\ceil{x}}} = \ceil{\sqrt{n^{2}}} = \ceil{n}$. \indent Since $n$ is an integer, $n$ is the smallest integer that is greater than or \indent equal to $n$. So $\ceil{n} = n$, by the definition for ceiling functions. Because \indent $n = \ceil{\sqrt{x}}$, in this case it is proved that $\ceil{\sqrt{\ceil{x}}} = \ceil{\sqrt{x}}$.
	\end{proof}

\end{document}
