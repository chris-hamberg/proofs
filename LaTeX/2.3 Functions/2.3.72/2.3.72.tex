\documentclass[a4paper, 12pt]{article}
\usepackage[utf8]{inputenc}
\usepackage[english]{babel}
\usepackage{amssymb, amsmath, amsthm}
\theoremstyle{plain}
\newtheorem*{theorem*}{Theorem}
\newtheorem{theorem}{Theorem}

\usepackage{mathtools}
\renewcommand\qedsymbol{$\blacksquare$}
\DeclarePairedDelimiter{\floor}{\lfloor}{\rfloor}
\DeclarePairedDelimiter{\ceil}{\lceil}{\rceil}

\begin{document}
	
	\begin{theorem*}[2.3.72]
		Let x be a real number. 
		$\floor{3x} = \floor{x} + \floor{x + \frac{1}{3}} + \floor{x + \frac{2}{3}}$.
	\end{theorem*}
	
	\begin{proof}
		By cases. Let $\epsilon$ be a real number such that $x - \floor{x} = \epsilon$. Clearly, 
		$3x - \floor{3x} = 3\epsilon$. There are three possible cases that must be proved, 
		\newline $(i)$ $0 \le \epsilon < \frac{1}{3}$, $(ii)$ 
		$\frac{1}{3} \le \epsilon < \frac{2}{3}$, and $(iii)$ $\frac{2}{3} \le \epsilon < 1$.
		\newline
		\newline
		$(i)$ Suppose $0 \le \epsilon < \frac{1}{3}$. Since $\epsilon + \frac{2}{3} < 1$, every 
		term on the right-hand side is equal to $x - \epsilon$. That is $3x - 3\epsilon$. But 
		$\floor{3x} = 3x - 3\epsilon$. Therefore, both sides of the equation are equal in this case.
		\newline
		\newline
		$(ii)$ Suppose $\frac{1}{3} \le \epsilon < \frac{2}{3}$. We know that 
		$\frac{1}{3} + \epsilon < 1$, and $\frac{2}{3} + \epsilon < \frac{4}{3}$. So the first two 
		terms on the right-hand side must be equal to $x - \epsilon$, and the last term equals 
		$(x - \epsilon) + 1$. That is $(3x - 3\epsilon) + 1$. Now, 
		$\floor{3x} = \floor{(3x - 3\epsilon) + 3\epsilon}$. By the inequality, 
		$1 \le 3\epsilon < 2$. This means that $\floor{3x} = (3x - 3\epsilon) + 1$. Hence, both 
		sides of the equation are equal in this case.
		\newline
		\newline
		$(iii)$ Suppose $\frac{2}{3} \le \epsilon < 1$. We know that 
		$1 \le \frac{1}{3} + \epsilon < \frac{1}{3} + 1$, and \newline 
		$\frac{4}{3} \le \frac{2}{3} + \epsilon < \frac{2}{3} + 1$. Clearly the first term in the 
		right-hand side of the equation is equal to $x - \epsilon$ since $\epsilon$ is less than 
		1. The remaining two terms are equal to $(x - \epsilon) + 1$, since 
		$2(\frac{2}{3}) \le 2\epsilon < 2$. Now, 
		$\floor{3x} = \floor{(3x - 3\epsilon) + 3\epsilon)}$, and we know that 
		$2 \le 3\epsilon < 3$, by the inequality. So $\floor{3x} = (3x - 3\epsilon) + 2$, which is 
		exactly the same as the right-hand side of the equation.
	\end{proof}

\end{document}
