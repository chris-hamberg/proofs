\documentclass[a4paper, 12pt]{article}
\usepackage[utf8]{inputenc}
\usepackage[english]{babel}
\usepackage{amssymb, amsmath, amsthm}
\theoremstyle{plain}
\newtheorem*{theorem*}{Theorem}
\newtheorem{theorem}{Theorem}

\usepackage{mathtools}
\renewcommand\qedsymbol{$\blacksquare$}
\DeclarePairedDelimiter{\floor}{\lfloor}{\rfloor}
\DeclarePairedDelimiter{\ceil}{\lceil}{\rceil}

\begin{document}
	
\begin{theorem*}[\textbf{2.4.21b}]
    The summation of natural numbers from 1 to n is $$\frac{n(n+1)}{2}$$
\end{theorem*}

\begin{proof}
    From Theorem 2.4.21a we know that $$\sum_{k=1}^{n} 2k - 1 = n^{2}$$
    This is the same as saying 
    $$n^{2} = \left(-n + \sum_{k=1}^{n} 2k \right) \equiv \left(\sum_{k=1}^{n} 2k \right) = (n^{2} + n) = n(n + 1)$$ 
    We can factor the coefficient $2$ out of the term of summation, 
    $$2 \sum_{k=1}^{n} k = n(n+1)$$
    And of course dividing both sides by $2$ gives 
    $$\sum_{k=1}^{n} k = \frac{n(n+1)}{2}$$ 
    Thus, indeed, the summation of natural numbers 
    from 1 to n is $\frac{n(n+1)}{2}$.
\end{proof}

\end{document}
