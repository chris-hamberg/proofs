\documentclass[a4paper, 12pt]{article}
\usepackage[utf8]{inputenc}
\usepackage[english]{babel}
\usepackage{amssymb, amsmath, amsthm}
\theoremstyle{plain}
\newtheorem*{theorem*}{Theorem}
\newtheorem{theorem}{Theorem}

\usepackage{mathtools}
\renewcommand\qedsymbol{$\blacksquare$}
\DeclarePairedDelimiter{\floor}{\lfloor}{\rfloor}
\DeclarePairedDelimiter{\ceil}{\lceil}{\rceil}

\begin{document}
	
	\begin{theorem*}[2.4.25]
		Let m be a positive integer. The closed form formula for $\sum_{k=0}^{m} \floor{\sqrt{k}}$ is $\floor{\sqrt{m}}[\frac{1}{6}(\floor{\sqrt{m}}-1)(4\floor{\sqrt{m}}+1) + (m - \floor{\sqrt{m}}^{2} + 1)]$.
	\end{theorem*}
	
	\begin{proof}
		By the properties for floor functions, there exists an integer $n = \floor{\sqrt{k}}$ such that $n^{2} \le k < n^{2} + 2n + 1$. This means that each term less than $\floor{\sqrt{m}}$ in the summation occurs exactly $2\floor{\sqrt{k}}+1$ times. Since the last term of summation occurs $(m - \floor{\sqrt{m}}^{2} + 1)$ times, \newline $(\sum_{k=0}^{m} \floor{\sqrt{k}}) - \floor{\sqrt{m}}(m - \floor{\sqrt{m}}^{2} + 1) = \newline  \floor{\sqrt{0}}(2\floor{\sqrt{0}} + 1) + \dots + (\floor{\sqrt{m}}-1)[2(\floor{\sqrt{m}}-1) + 1]$. By the pattern in the terms of that summation the following equation holds, $\sum_{k=0}^{m} \floor{\sqrt{k}} = \newline (\sum_{k=0}^{\floor{\sqrt{m}}-1} k(2k + 1)) + \floor{\sqrt{m}}(m - \floor{\sqrt{m}}^{2} + 1)$. The summation on the right-hand side is the summation of squares, and the summation of integers. Thus, $\sum_{k=0}^{m} \floor{\sqrt{k}} = (2\sum_{k=0}^{\floor{\sqrt{m}}-1} k^{2}) + (\sum_{k=0}^{\floor{\sqrt{m}}-1} k) + [\floor{\sqrt{m}}(m - \floor{\sqrt{m}}^{2} + 1)]$. By Theorem 2.4.22, and by Theorem 2.4.21b, the formula is immediately derived $\sum_{k=0}^{m} \floor{\sqrt{k}} = \frac{2}{6}\floor{\sqrt{m}}(\floor{\sqrt{m}}-1)[2(\floor{\sqrt{m}}-1)+1] + \frac{3}{6}\floor{\sqrt{m}}(\floor{\sqrt{m}}-1) + \floor{\sqrt{m}}(m - \floor{\sqrt{m}}^{2} + 1)$. Factoring $\frac{1}{6}\floor{\sqrt{m}}(\floor{\sqrt{m}}-1)$ out of the first two terms yields $\sum_{k=0}^{m} \floor{\sqrt{k}} = \frac{1}{6}\floor{\sqrt{m}}(\floor{\sqrt{m}}-1)\{2[2(\floor{\sqrt{m}}-1)+1] +3\}+\floor{\sqrt{m}}(m - \floor{\sqrt{m}}^{2} + 1)$. Simplifying that is, $\sum_{k=0}^{m} \floor{\sqrt{k}} = \newline \frac{1}{6}\floor{\sqrt{m}}(\floor{\sqrt{m}}-1)(4\floor{\sqrt{m}} + 1) +\floor{\sqrt{m}}(m - \floor{\sqrt{m}}^{2} + 1)$. Factoring $\floor{\sqrt{m}}$, \newline $\sum_{k=0}^{m} \floor{\sqrt{k}} = \floor{\sqrt{m}}[\frac{1}{6}(\floor{\sqrt{m}}-1)(4\floor{\sqrt{m}}+1) + (m - \floor{\sqrt{m}}^{2} + 1)]$.
	\end{proof}

\end{document}
