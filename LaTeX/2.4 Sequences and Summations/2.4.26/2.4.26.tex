\documentclass[a4paper, 12pt]{article}
\usepackage[utf8]{inputenc}
\usepackage[english]{babel}
\usepackage{amssymb, amsmath, amsthm}
\theoremstyle{plain}
\newtheorem*{theorem*}{Theorem}
\newtheorem{theorem}{Theorem}

\usepackage{mathtools}
\renewcommand\qedsymbol{$\blacksquare$}
\DeclarePairedDelimiter{\floor}{\lfloor}{\rfloor}
\DeclarePairedDelimiter{\ceil}{\lceil}{\rceil}

\begin{document}
	
\begin{theorem*}[\textbf{2.4.26}]
    Let m be a positive integer. The closed form formula for $\sum_{k=0}^{m} \floor{\sqrt[3]{k}}$ 
    is $\floor{\sqrt[3]{m}}
    [\frac{1}{4}(\floor{\sqrt[3]{m}}^{2} - \floor{\sqrt[3]{m}})(3\floor{\sqrt[3]{m}}+1)+
    (m - \floor{\sqrt[3]{m}}^{3} + 1)]$.
\end{theorem*}

\begin{proof}
    By the properties for floor functions there exists an integer $n_k = \floor{\sqrt[3]{k}}$ such 
    that $n_{k}^{3} \le k < n_{k}^{3} + 3n_{k}^{2} + 3n_k + 1$. This means that each value less 
    than $\floor{\sqrt[3]{m}}$ in the terms of summation occurs exactly 
    $3\floor{\sqrt[3]{k}}^{2} + 3\floor{\sqrt[3]{k}} + 1$ times. The maximum value in the terms of 
    summation occurs $(m - \floor{\sqrt[3]{m}}^{3} + 1)$ times. Subtracting those terms consisting 
    of the maximum value produces the sequence
    $$n_0(3n_{0}^{2} + 3n_0 + 1) + 
    n_1(3n_{1}^{2} + 3n_{1} + 1) + 
    \dots +
    n_{\floor{\sqrt[3]{m}} - 1}
    (3n_{\floor{\sqrt[3]{m}}-1}^{2} + 3_{\floor{\sqrt[3]{m}}-1} + 1)$$
    Summarily expressed as
    $$\sum_{n=0}^{\floor{\sqrt[3]{m}}-1} n(3n^2 + 3n + 1)$$
    Thus,
    $$\sum_{k=0}^{m} \floor{\sqrt[3]{k}} = 
    \left(\sum_{n=0}^{\floor{\sqrt[3]{m}}-1} n(3n^{2} + 3n + 1)\right) + 
    \floor{\sqrt[3]{m}}(m - \floor{\sqrt[3]{m}}^{3} + 1)$$ 
    That is, the summation of cubes, squares, and integers
    $$\sum_{k=0}^{m} \floor{\sqrt[3]{k}} =
    \left(3\sum_{n=0}^{\floor{\sqrt[3]{m}}-1} n^{3}\right) + 
    \left(3\sum_{n=0}^{\floor{\sqrt[3]{m}}-1} n^{2}\right) + 
    \left(\sum_{n=0}^{\floor{\sqrt[3]{m}}-1} n\right) +$$
    $$\floor{\sqrt[3]{m}}(m - \floor{\sqrt[3]{m}}^{3} + 1)$$
    By the closed form formula for each individual summation we have, 
    $$\sum_{k=0}^{m} \floor{\sqrt[3]{k}} = 
    \left\{\frac{3}{4}\floor{\sqrt[3]{m}}^{2}(\floor{\sqrt[3]{m}}-1)^{2}\right\} +$$
    $$\left\{
        \frac{2}{4}\floor{\sqrt[3]{m}}(\floor{\sqrt[3]{m}}-1)[2(\floor{\sqrt[3]{m}}-1) +1]
    \right\} +
    \left\{\frac{2}{4}\floor{\sqrt[3]{m}}(\floor{\sqrt[3]{m}}-1)\right\} +$$
    $$\floor{\sqrt[3]{m}}(m - \floor{\sqrt[3]{m}}^{3} + 1)$$
    Algebraic simplification completes the derivation
    $$\sum_{k=0}^{m} \floor{\sqrt[3]{k}} =
    \floor{\sqrt[3]{m}}
    \left[
        \frac{1}{4}(\floor{\sqrt[3]{m}}^{2} - \floor{\sqrt[3]{m}})
        (3\floor{\sqrt[3]{m}}+1)+(m - \floor{\sqrt[3]{m}}^{3} + 1)
    \right]$$
\end{proof}

\end{document}
