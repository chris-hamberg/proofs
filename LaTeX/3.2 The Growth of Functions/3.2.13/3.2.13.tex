\documentclass[a4paper, 12pt]{article}
\usepackage[utf8]{inputenc}
\usepackage[english]{babel}
\usepackage{amssymb, amsmath, amsthm}
\theoremstyle{plain}
\newtheorem*{theorem*}{Theorem}
\newtheorem{theorem}{Theorem}

\usepackage{mathtools}
\renewcommand\qedsymbol{$\blacksquare$}
\DeclarePairedDelimiter{\floor}{\lfloor}{\rfloor}
\DeclarePairedDelimiter{\ceil}{\lceil}{\rceil}

\begin{document}
	
	\begin{theorem*}[3.2.13]
		Let f be the function defined by f(n) = $2^{n}$, and let g be the function defined by g(n) = $3^{n}$. f(n) is $\mathcal{O}(g(n))$, but g(n) is not $\mathcal{O}(f(n))$.
	\end{theorem*}
	
	\begin{proof}
		$f(n)$ is $\mathcal{O}(g(n))$ trivially follows from the fact that $|2^{n}| \le |3^{n}|$, for all $n > 1$. \newline \indent If it were that $g(x) \in \mathcal{O}(f(x))$, then there would exists constant witnesses $C$ and $k$ such that $3^{n} \le C \cdot 2^{n}$, for all $n > k$. This inequality also means that $\left( \frac{3}{2} \right)^{n} \le C$, for all $n > k$. Clearly, no constant $C$ exists such that $\left( \frac{3}{2} \right)^{n} \le C$, for all $n > k$. Thus, $g(n) \notin \mathcal{O}(f(n))$.
	\end{proof}

\end{document}
