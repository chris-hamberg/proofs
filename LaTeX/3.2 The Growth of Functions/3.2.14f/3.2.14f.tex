\documentclass[a4paper, 12pt]{article}
\usepackage[utf8]{inputenc}
\usepackage[english]{babel}
\usepackage{amssymb, amsmath, amsthm}
\theoremstyle{plain}
\newtheorem*{theorem*}{Theorem}
\newtheorem{theorem}{Theorem}

\usepackage{mathtools}
\renewcommand\qedsymbol{$\blacksquare$}
\DeclarePairedDelimiter{\floor}{\lfloor}{\rfloor}
\DeclarePairedDelimiter{\ceil}{\lceil}{\rceil}

\begin{document}
	
	\begin{theorem*}[3.2.14f]
		Let f be the functions defined by f(x) = $x^{3}$, and let g be the function defined by g(x) = $\frac{x^{3}}{2}$. f(x) is $\mathcal{O}(g(x))$.
	\end{theorem*}
	
	\begin{proof}
		If $f(x)$ is $\mathcal{O}(g(x))$, then there exists constant witnesses $C$ and $k$ such that $|f(x)| \le C|g(x)|$, for all $x > k$. Clearly, $x^{3} \le \left( 2 \cdot \frac{x^{3}}{2} \right) = x^{3}$, for all $x > 0$. Thus, $f(x)$ is $\mathcal{O}(g(x))$ with constant witnesses $C = 2$, and $k = 0$.
	\end{proof}

\end{document}
