\documentclass[a4paper, 12pt]{article}
\usepackage[utf8]{inputenc}
\usepackage[english]{babel}
\usepackage{amssymb, amsmath, amsthm}
\theoremstyle{plain}
\newtheorem*{theorem*}{Theorem}
\newtheorem{theorem}{Theorem}

\usepackage{mathtools}
\renewcommand\qedsymbol{$\blacksquare$}
\DeclarePairedDelimiter{\floor}{\lfloor}{\rfloor}
\DeclarePairedDelimiter{\ceil}{\lceil}{\rceil}

\begin{document}
	
	\begin{theorem*}[3.2.19b]
		Let f be the function defined by \newline $f(n) = (n \log n + n^{2})(n^{3} + 2)$. f(n) is $\mathcal{O}(n^{5})$.
	\end{theorem*}
	
	\begin{proof}
		$f(n)$ is the product of functions $(f_1 f_2)(n)$, where $f_1(n) = (n \log n + n^{2})$ and $f_2(n) = (n^{3} + 2)$. \newline \indent Consider $f_1$, which is the sum of functions $(f_1^\prime + f_1^{\prime\prime})$, where $f_1^\prime(n) = n \log n$, and $f_1^{\prime\prime} = n^{2}$. Since $n \log n \le n \cdot n$, for all $n \ge 1$, it follows that $f_1^\prime(n)$ is $\mathcal{O}(n^{2})$ with constant witnesses $C = 1$ and $k = 1$. Clearly, $f_1^{\prime\prime}(n)$ is $\mathcal{O}(n^{2})$ with constant witnesses in $\mathbb{N}$. Now, the bounding function for the sum of functions is the maximum of the bounding functions in the addends for the sum of functions. So $(f_1^\prime + f_1^{\prime\prime})(n)$ is $\mathcal{O}(n^{2})$. Thus, $f_1$ is $\mathcal{O}(n^{2})$.
		\newline \indent We now turn our attention to $f_2$. $f_2$ is a $3^\textsuperscript{rd}$ degree binomial. Since a $k^\textsuperscript{th}$ degree polynomial is $\mathcal{O}(x^{k})$, it follows that $f_2(n)$ is $\mathcal{O}(n^{3})$.
		\newline \indent The bounding function for the product of functions is the product of the bounding functions for each function in the product of functions. So $f(n)$ is $\mathcal{O}(n^{2} \cdot n^{3})$. This means that $f(n)$ is $\mathcal{O}(n^{5})$.
	\end{proof}

\end{document}
