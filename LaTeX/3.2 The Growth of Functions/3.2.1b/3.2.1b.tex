\documentclass[a4paper, 12pt]{article}
\usepackage[utf8]{inputenc}
\usepackage[english]{babel}
\usepackage{amssymb, amsmath, amsthm}
\theoremstyle{plain}
\newtheorem*{theorem*}{Theorem}
\newtheorem{theorem}{Theorem}

\usepackage{mathtools}
\renewcommand\qedsymbol{$\blacksquare$}
\DeclarePairedDelimiter{\floor}{\lfloor}{\rfloor}
\DeclarePairedDelimiter{\ceil}{\lceil}{\rceil}

\begin{document}
	
	\begin{theorem*}[3.2.1b]
		Let f be the function defined by f(x) = 3x + 7. \newline f(x) is $\mathcal{O} (x)$.
	\end{theorem*}
	
	\begin{proof}
		Let $g$ be the function defined by $g(x) = x$. Then $|3x + 7| \le 5|x|$, for all $x > 4$ means that $|f(x)| \le 5|g(x)|$ ,for all $x > 4$. Therefore $f(x)$ is $\mathcal{O}(x)$, with constant witnesses $C = 5$, and $k = 4$, by definition.
	\end{proof}

\end{document}
