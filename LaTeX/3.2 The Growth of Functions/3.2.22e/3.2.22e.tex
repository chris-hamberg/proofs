\documentclass[a4paper, 12pt]{article}
\usepackage[utf8]{inputenc}
\usepackage[english]{babel}
\usepackage{amssymb, amsmath, amsthm}
\theoremstyle{plain}
\newtheorem*{theorem*}{Theorem}
\newtheorem{theorem}{Theorem}

\usepackage{mathtools}
\renewcommand\qedsymbol{$\blacksquare$}
\DeclarePairedDelimiter{\floor}{\lfloor}{\rfloor}
\DeclarePairedDelimiter{\ceil}{\lceil}{\rceil}

\begin{document}
	
	\begin{theorem*}[3.2.22e]
		Let f be the function defined by $f(x) = \floor{x}$. \newline f(x) is $\Theta(x)$.
	\end{theorem*}
	
	\begin{proof}
		By the properties for floor functions $\floor{x} \le x$, for all $x \ge 1$. Thus, $f(x)$ is $\mathcal{O}(x)$ with constant witnesses $C = 1$ and $k = 1$. Also, $\floor{x} \ge \frac{1}{2}x$, for all $x \ge 2$. So $f(x)$ is $\Omega(x)$ with constant witnesses $C = \frac{1}{2}$ and $k = 2$. It follows immediately from the definition that $f(x)$ is $\Theta(x)$.
	\end{proof}

\end{document}
