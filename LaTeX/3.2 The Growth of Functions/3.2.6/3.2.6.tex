\documentclass[a4paper, 12pt]{article}
\usepackage[utf8]{inputenc}
\usepackage[english]{babel}
\usepackage{amssymb, amsmath, amsthm}
\theoremstyle{plain}
\newtheorem*{theorem*}{Theorem}
\newtheorem{theorem}{Theorem}

\usepackage{mathtools}
\renewcommand\qedsymbol{$\blacksquare$}
\DeclarePairedDelimiter{\floor}{\lfloor}{\rfloor}
\DeclarePairedDelimiter{\ceil}{\lceil}{\rceil}

\begin{document}
	
	\begin{theorem*}[3.2.6]
		Let f be the function defined by $f(x) = \frac{x^{3} + 2x}{2x + 1}$. \newline f(x) is $\mathcal{O}(x^{2})$.
	\end{theorem*}
	
	\begin{proof}
		Let $g$ be the function defined by $g(x) = x^{2}$. If $x \ge 1$, then \newline \newline \indent $f(x) = \left(\frac{x^{3} + 2x}{2x + 1}\right) \le \left(\frac{x^{3} + 2x}{2x} = \frac{x^{3}}{2x} + 1 \right) \le \left(\frac{x^{3}}{x} + 1 \right) \le \left( x^{2} + 1 \right)$.
		\newline
		\newline
		Clearly, if $x > 1$, then $x^{2} + 1 \le 2x^{2}$. So $|f(x)| \le 2|g(x)|$, for all $x > 1$. It follows from the definition that $f(x)$ is $\mathcal{O}(x^{2})$ with constant witnesses $C = 2$, and $k = 1$.
	\end{proof}

\end{document}
