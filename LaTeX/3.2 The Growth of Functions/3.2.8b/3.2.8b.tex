\documentclass[a4paper, 12pt]{article}
\usepackage[utf8]{inputenc}
\usepackage[english]{babel}
\usepackage{amssymb, amsmath, amsthm}
\theoremstyle{plain}
\newtheorem*{theorem*}{Theorem}
\newtheorem{theorem}{Theorem}

\usepackage{mathtools}
\renewcommand\qedsymbol{$\blacksquare$}
\DeclarePairedDelimiter{\floor}{\lfloor}{\rfloor}
\DeclarePairedDelimiter{\ceil}{\lceil}{\rceil}

\begin{document}
	
	\begin{theorem*}[3.2.8b]
		Let f be the function defined by f(x) = $3x^{5} + (\log x)^{4}$. \newline f(x) is $\mathcal{O}(x^{5})$.
	\end{theorem*}
	
	\begin{proof}
		Let $g$ be the function defined by $g(x) = x^{5}$. If $x \ge 2$, then
		\newline \newline \indent \indent
		$f(x) = 3x^{5} + (\log x)^{4} \le 3x^{5} + 4x \le 3x^{5} + x^{5} = 4x^{5}$. \newline \newline Therefore $|f(x)| \le 4|g(x)|$, for all $x > 2$, and $f(x)$ is $\mathcal{O}(x^{5})$ with constant witnesses $C = 4$, and $k = 2$.
	\end{proof}

\end{document}
