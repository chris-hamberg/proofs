\documentclass[a4paper, 12pt]{article}
\usepackage[utf8]{inputenc}
\usepackage[english]{babel}
\usepackage{amssymb, amsmath, amsthm}
\theoremstyle{plain}
\newtheorem*{theorem*}{Theorem}
\newtheorem{theorem}{Theorem}

\usepackage{mathtools}
\renewcommand\qedsymbol{$\blacksquare$}
\DeclarePairedDelimiter{\floor}{\lfloor}{\rfloor}
\DeclarePairedDelimiter{\ceil}{\lceil}{\rceil}

\begin{document}
	
	\begin{theorem*}[3.2.9]
		Let f be the function defined by f(x) = $x^{2} + 4x + 17$. \newline f(x) is $\mathcal{O}(x^{3})$, but $x^{3}$ is not $\mathcal{O}(f(x))$.
	\end{theorem*}
	
	\begin{proof}
		Let $g$ be the function defined by $g(x) = x^{3}$. $f(x)$ is $\mathcal{O}(x^{2})$, by the theorem that states that a polynomial of degree $n$ is $\mathcal{O}(x^{n})$. Therefore $f(x)$ is $\mathcal{O}(g(x))$. \newline \indent If $x \ge 2$, then $f(x) \le x^{2} + 4x^{2} + x^{2} = 6x^{2}$. By the definition of big-O, if $g(x)$ is $\mathcal{O}(f(x))$, then $|x^{3}| \le |f(x)| \le 6x^{2}$. That is, $x^{3} \le 6x^{2}$, and $x \le 6$. Clearly it is not the case that $g(x)$ is $\mathcal{O}(f(x))$.
	\end{proof}

\end{document}
