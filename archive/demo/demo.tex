\documentclass[a4paper, 12pt]{article}
\usepackage[utf8]{inputenc}
\usepackage[english]{babel}
\usepackage{amssymb, amsmath, amsthm}
\usepackage{bm}
\usepackage{xcolor}
\theoremstyle{plain}
\newtheorem*{theorem*}{Theorem}
\newtheorem{theorem}{Theorem}
\newtheorem*{lemma*}{Lemma}
\newtheorem{lemma}{Lemma}

\usepackage{mathtools}
\renewcommand\qedsymbol{$\blacksquare$}
\DeclarePairedDelimiter{\ord}{\langle}{\rangle}

\begin{document}


% ============================== Theorem 2.4.26 ================================
\begin{theorem*}[\textbf{2.4.26}]
    Let m be a positive integer such that 
    \bm{$\big \lfloor \sqrt[3] \phi \big \rfloor = \lambda$}.
    The closed form formula for 
    \bm{
        $\Phi = \sum_{\iota=0}^\phi 
                \big \lfloor \sqrt[3] \iota \big \rfloor
    $} 
    is 
    \begin{equation*}
        \bm{
            \frac{
                \lambda ^3 - \lambda ^2
            }
            {4}
            \Bigg[ 3 \lambda + 1 \Bigg]
                + 
            \lambda
            \Bigg[
                \phi - \lambda ^3 + 1
            \Bigg]
        }
    \end{equation*}
\end{theorem*}
\begin{proof}
    Let \bm{$\mathrm{X}$} be the function 
    \bm{$\mathrm{X}: \mathbb{N} \rightarrow \mathbb{N}$}
    such that \bm{$\mathrm{X}[\beta] = 3 \beta ^2 + 3 \beta + 1$}.
    By the associative law for addition from the field axioms,
    \begin{equation*}
        \Theta
            =
        \sum_{\iota=0}^{\lambda - 1}
                \big \lfloor \sqrt[3] \iota \big \rfloor
            =
    \end{equation*}
    \begin{equation*}
        \Bigg(
            \sum_{ \iota = 1 }^{ \mathrm{X} [\beta_0] }
                \big \lfloor \sqrt[3] \iota \big \rfloor_0
        \Bigg)
            +
        \Bigg(
            \sum_{
                \iota = 1 + \mathrm{X}[\beta_0]
            }^{ 
                \mathrm{X}[\beta_0] + \mathrm{X}[\beta_1]
            }
                \big \lfloor \sqrt[3] \iota \big \rfloor_1
        \Bigg)
            +
        \dots
            +
        \Bigg(
            \sum_{
                \iota = 1 + \dots 
                    + 
                \mathrm{X}[\beta_{\lambda - 2}]
            }^{ 
                0 + \dots
                    +
                \mathrm{X}[\beta_{ \lambda - 1 }]
            }
                \big \lfloor \sqrt[3] \iota \big \rfloor_{ \langle \lambda - 1 \rangle }
        \Bigg)
    \end{equation*}
    \bm{$\big \lfloor \sqrt[3] \iota \big \rfloor_\tau$}
    is the unique integer \bm{$\beta_\tau$},
    for \bm{$\tau = 0$} to 
    \bm{$\big \langle \lambda - 1 \big \rangle$}.
    Hence, by Lemma 2.4.4,
    by the partitioning from above, 
    and shifting the index of summation,
    \begin{equation*}
        \Theta
            =
        \Big \langle \beta_0 \mathrm{X} [\beta_0] \Big \rangle
            +
        \Big \langle \beta_1 \mathrm{X} [\beta_1] \Big \rangle
            + 
        \dots 
            +
        \Big \langle 
            \beta_{\langle \lambda - 1 \rangle}
            \mathrm{X} [\beta_{\langle \lambda - 1 \rangle}]
        \Big \rangle
    \end{equation*}
    By the distributive law for real numbers from the field axioms,
    and the commutative law for addition, that series is
    three times the finite summation of cubes,
    plus three times the finite summation of squares,
    plus the finite summation of integers from zero to lambda minus one.
    \begin{equation*}
        \Theta 
            = 
        \sum_{\tau=0}^{\lambda - 1} 3 \tau ^3
            +
        \sum_{\tau=0}^{\lambda - 1} 3 \tau ^2
            +
        \sum_{\tau=0}^{\lambda - 1} \tau
    \end{equation*}
    Lambda occurs exactly  
    \bm{$\big \langle \phi - \lambda^3 + 1 \big \rangle$}
    times in big phi. 
    Thus, by Lemma 2.4.5
    and the distributive law for real numbers from the field axioms,
    \begin{equation*}
        \bm{
            \Phi
                =
            \frac{
                \lambda ^3 - \lambda ^2
            }
            {4}
            \Bigg[ 3 \lambda + 1 \Bigg]
                + 
            \lambda
            \Bigg[
                \phi - \lambda ^3 + 1
            \Bigg]
        }
    \end{equation*}
\color{lightgray} \end{proof}
\pagebreak


% ================================ Lemma 2.4.5 =================================
\begin{lemma*}[\textbf{2.4.5}]
    Let \bm{$\iota$} be a positive integers such that
    \bm{$\big \lfloor \sqrt[3] \iota \big \rfloor = \lambda$}.
    \begin{equation*}
        \bm{
            \Bigg(
                3 \sum_{\phi=0}^{\lambda - 1} \phi ^3
            \Bigg)
                +
            \Bigg(
                3 \sum_{\phi=0}^{\lambda - 1} \phi ^2
            \Bigg)
                +
            \Bigg(
                \sum_{\phi=0}^{\lambda - 1} \phi
            \Bigg)
                \equiv
            \frac{
                \lambda ^3 - \lambda ^2
            }
            {4}
            \Bigg[ 3 \lambda + 1 \Bigg]
        }
    \end{equation*}
\end{lemma*}
\begin{proof}
    Let \bm{$\Phi$} be three times the summation of cubes from zero to lambda minus one.
    Let \bm{$\mathrm{X}$} be three times the summation of squares from zero to lambda minus one.
    Let \bm{$\Omega$} be the summation of integers from zero to lambda minus one.
    By theorems 2.4.22 and 2.4.21b, 
    by the closed formula for the summation of cubes,
    and by shifting the index of summation,
    \begin{equation*}
        \Bigg \{
            \Phi + \mathrm{X} + \Omega
        \Bigg \}
            \equiv
        3
        \Bigg[
            \frac{
                \lambda ^2
                \big[ \lambda - 1 \big] ^2
            }
            {4}
        \Bigg]
            +
        3
        \Bigg[
            \frac{
                \lambda 
                \big[ \lambda - 1 \big]
                \big[ 2 \langle \lambda - 1 \rangle + 1 \big]
            }
            {6}
        \Bigg]
            +
        \Bigg[
            \frac{\lambda \big [ \lambda - 1 ]}
            {2}
        \Bigg]
    \end{equation*} 
    By the multiplicative identity law from the field axioms,
    \begin{equation*}
        \bigg \langle 
            \frac{3}{6}
        \bigg \rangle
             = 
        \bigg \langle
            \frac{2 \cdot 3}{2 \cdot 6} 
        \bigg \rangle 
            = 
        \bigg \langle
            \frac{6}{12} 
        \bigg \rangle 
            = 
        \bigg \langle
            \frac{2}{4}
        \bigg \rangle
    \end{equation*}
    \begin{equation*}
        \bigg \langle 
            \frac{1}{2}
        \bigg \rangle
            =
        \bigg \langle
            \frac{2 \cdot 1}{2 \cdot 2}
        \bigg \rangle
            =
        \bigg \langle 
            \frac{2}{4}
        \bigg \rangle
    \end{equation*}
    Thus, by the identities for \bm{$\mathrm{X}$} and \bm{$\Omega$},
    factoring 
    \bm{$\frac{1}{4} \lambda \big \langle \lambda - 1 \big \rangle$}
    out from the sum of \bm{$\Phi$}, \bm{$\mathrm{X}$}, and \bm{$\Omega$},
    by the distributive laws from real numbers,
    \begin{equation*}
        \Bigg \{
            \Phi + \mathrm{X} + \Omega
        \Bigg \}
                \equiv
            \frac{
                \lambda \big \langle \lambda - 1 \big \rangle
            }
            {4}
            \Bigg[
                3 \lambda \big[ \lambda - 1 \big]
                    +
                2 \bigg \langle 
                    2 \big[ \lambda - 1 \big] + 1
                \bigg \rangle
                    +
                2
            \Bigg]    
    \end{equation*}
    By the distributive law for real numbers, 
    and by the inverse law for addition from the field axioms,
    \begin{equation*}
        \bigg \{
            2 \Big \langle 2 \big[ \lambda - 1 \big] + 1 \Big \rangle + 2
        \bigg \}
            =
        \bigg \{
            4 \big[ \lambda - 1 \big] + 2 + 2
        \bigg \}
            =
        \bigg \{
            4 \lambda - 4 + 4
        \bigg \}
            = 
        \bigg \{
            4 \lambda
        \bigg \}
    \end{equation*}
    Hence, by the identity four lambda,
    \begin{equation*}
        \Bigg \{
            \Phi + \mathrm{X} + \Omega
        \Bigg \}
            \equiv
        \frac{
            \lambda \big \langle \lambda - 1 \big \rangle
        }
        {4}
        \Bigg[
            3 \lambda \big[ \lambda - 1 \big]
                +
            4 \lambda
        \Bigg]
    \end{equation*}
    Factoring out lambda and distributing three,
    by the distributive laws for real numbers,
    \begin{equation*}
        \Bigg\{
            \Phi + \mathrm{X} + \Omega
        \Bigg\}
                \equiv
            \frac{
                \lambda ^2 \big \langle \lambda - 1 \big \rangle
            }
            {4}
            \Bigg[
                3 \lambda - 3 + 4
            \Bigg]
    \end{equation*}
    The proof is complete, 
    by the distributive law for real numbers distributing
    the second power of lambda,
    and the inverse law for addition from the field axioms.
\color{lightgray}\end{proof}
\pagebreak


% =============================== Theorem 1.6.25 ===============================
\begin{theorem*}[\textbf{1.6.25}]
    There does not exist a rational number \bm{$\rho$} such that 
    \bm{$\rho^3 + \rho + 1 = 0$}.
\end{theorem*}
\begin{proof}
    For the purpose of contradiction, 
    assume that there exists a rational number \bm{$\rho$} satisfying the equation \bm{$\rho^3 + \rho + 1 = 0$}. 
    By the definition for rational numbers, 
    there exist integers \bm{$\alpha$} and \bm{$\beta$} (\bm{$\beta$} is nonzero,) such that 
    \begin{equation*}
        \left( 
            \rho ^3 + \rho + 1 
        \right) 
            = 
        \left( 
            \frac{\alpha^3}{\beta^3} + \frac{\alpha}{\beta} + 1 
        \right) 
            = 
        0
    \end{equation*}
    By the additive equality property for equations, that is
    \begin{equation*}
        \frac{\alpha^3}{\beta^3} 
            = 
        \left(
            -1 - \frac{\alpha}{\beta}
        \right)
    \end{equation*}
    It is possible to derive \bm{$\rho^2$} from \bm{$\rho^3$} 
    by multiplying \bm{$\rho^3$} by the multiplicative inverse for \bm{$\rho$}. 
    By the multiplicative equality property for equations,
    \begin{equation*}
        \frac{\alpha^3}{\beta^3} \cdot \frac{\beta}{\alpha}
            =
        \left(
            -1 - \frac{\alpha}{\beta}
        \right) 
            \cdot 
        \frac{\beta}{\alpha} 
            =
        \left\{
            \frac{-\beta}{\alpha} 
                - 
            \frac{\alpha \beta}{\beta \alpha}
        \right\}
    \end{equation*}
    Thus, by the field axioms, \bm{$\rho^2$} is
    \begin{equation*}
        \left\{
            \frac{-\beta}{\alpha} 
                - 
            \frac{\alpha \beta}{\beta \alpha}
        \right\}
            = 
        \left(
            \frac{-\beta - \alpha}{\alpha}
        \right)
            = 
        -1 
            \cdot 
        \left( 
            \frac{\beta + \alpha}{\alpha}
        \right)
    \end{equation*}
    Applying the square root to \bm{$\rho^2$} gives the identity for \bm{$\rho$}
    \begin{equation*}
        \sqrt{ \frac{\alpha^2}{\beta^2} } 
            =
        \sqrt{ -1 \cdot \left( \frac{\beta + \alpha}{\alpha} \right) } 
            =
        i \cdot \sqrt{ \left( \frac{\beta + \alpha}{\alpha} \right) }
    \end{equation*}
    \bm{$\rho$} is imaginary and rational. 
    Thus, the negation of the hypothesis implies a contradiction. 
    In other words, \bm{$\rho$} does not exist.
\color{lightgray} \end{proof}
\pagebreak


% ============================== Theorem 2.3.70e ===============================
\begin{theorem*}[\textbf{2.3.70e}]
    Let \bm{$\lambda$}, and \bm{$\iota$} be real numbers.
    \begin{equation*}
        \bm{
            \big \lfloor \lambda \rfloor 
                + 
            \big \lfloor \iota \rfloor  
                + 
            \big \lfloor \lambda + \iota \rfloor 
                \le 
            \big \lfloor 2 \lambda \rfloor 
                + 
            \big \lfloor 2 \iota \rfloor
        }
    \end{equation*}
\end{theorem*}
\begin{proof}
    By cases.
    There exist real numbers \bm{$\epsilon$} and \bm{$\sigma$} such that
    \bm{$\lambda - \lfloor \lambda \rfloor = \epsilon$}. 
    By the property for floor functions,
    \bm{$\lfloor \lambda \rfloor = \lambda - \epsilon$},
    if and only if
    \begin{equation*}
        \Big \langle \lambda - \epsilon \Big \rangle
            \leq 
        \Big \langle \lambda \Big \rangle
            < 
        \Big \langle \lambda - \epsilon + 1 \Big \rangle
    \end{equation*}
    Without loss of generality with respect to \bm{$\iota$}, 
    by the additive compatibility law from the order axioms,
    there exists an integer 
    \bm{$
        \lfloor \lambda + \iota \rfloor 
            = 
        \big \langle \lambda - \epsilon \big \rangle
            +
        \big \langle \iota - \sigma \big \rangle$},
    if and only if
    \begin{equation*}
        \Big \langle
            \lambda - \epsilon
                + 
            \iota - \sigma
        \Big \rangle
            \leq 
        \Big \langle \lambda + \iota \Big \rangle
            < 
        \Big \langle
            \lambda - \epsilon
                + 
            \iota - \sigma
                + 
            1
        \Big \rangle
    \end{equation*}
    Thus, by the identities for the floor of \bm{$\lambda$},
    the floor of \bm{$\iota$}, 
    and the floor of the sum of \bm{$\lambda$} and \bm{$\iota$},
    we deduce 
    \begin{equation*}
        \big \lfloor \lambda \big \rfloor
            +
        \big \lfloor \iota \big \rfloor
            +
        \big \lfloor \lambda + \iota \big \rfloor
            =
        \Big[
            \big \langle \lambda - \epsilon \big \rangle
                +
            \big \langle \iota - \sigma \big \rangle
                +
            \big \langle \lambda - \epsilon + \iota - \sigma \big \rangle
        \Big]
            =
    \end{equation*}
    \begin{equation*}
        2 \big \langle \lambda + \iota \big \rangle 
            -
        2 \big \langle \epsilon + \sigma \big \rangle
    \end{equation*}
    Now, by multiplicative compatibility, 
    and transitivity from the order axioms,
    \begin{equation*}
        \Big \langle \lambda - \epsilon \Big \rangle
            \leq 
        \Big \langle \lambda \Big \rangle
            < 
        \Big \langle \lambda - \epsilon + 1 \Big \rangle
            \equiv
        \Big \langle 2 \lambda - 2\epsilon \Big \rangle
            \leq 
        \Big \langle 2 \lambda \Big \rangle
            < 
        \Big \langle 2 \lambda - 2\epsilon + 2 \Big \rangle
            \equiv
    \end{equation*}
    \begin{equation*}
        \Big \langle 2 \big[ \lambda - \epsilon \big] \Big \rangle
            \leq
        \Big \langle 2 \lambda \Big \rangle
            \leq
        \Big \langle 2 \big[ \lambda - \epsilon \big] + 1 \Big \rangle
    \end{equation*}
    So \bm{$
        \big \lfloor 2 \lambda \big \rfloor
    $}
    is the integer
    $(i)$ \bm{$
        2 \big \langle \lambda - \epsilon \big \rangle
    $}, 
    or $(ii)$ \bm{$
        2 \big \langle \lambda - \epsilon \big \rangle + 1
    $}, by the properties for the floor function.
    \\ \\
    \bm{$(i)$} Let \bm{$
        \big \lfloor 2 \lambda \big \rfloor 
            = 
        2 \big \langle \lambda - \epsilon \big \rangle
    $}. 
    Without loss of generality with respect to \bm{$\iota$},
    \begin{equation*}
        \Big \{ 
            \big \lfloor 2 \lambda \big \rfloor 
                + 
            \big \lfloor 2 \iota \big \rfloor
        \Big \}
            =
        \Big \{
            2 \big \langle \lambda - \epsilon \big \rangle 
                + 
            2 \big \langle \iota - \sigma \big \rangle
        \Big \}
            =
        \Big \{
            2 \big \langle \lambda + \iota \big \rangle
                - 
            2 \big \langle \epsilon + \sigma \big \rangle
        \Big \}
            =
    \end{equation*}
    \begin{equation*}
        \Big \{
            \big \lfloor \lambda \big \rfloor
                +
            \big \lfloor \iota \big \rfloor
                +
            \big \lfloor \lambda + \iota \big \rfloor
        \Big \}
    \end{equation*}
    \bm{$(ii)$} Let \bm{$
    \big \lfloor 2 \lambda \big \rfloor
        = 
    2 \big \langle \lambda - \epsilon \big \rangle + 1
    $}.
    Without loss of generality with respect to \bm{$\iota$},
    \begin{equation*}
        \Big \{
            \big \lfloor 2 \lambda \big \rfloor 
                + 
            \big \lfloor 2 \iota \big \rfloor
        \Big \}
            =
        \Big \{
            2 \big \langle \lambda - \epsilon \big \rangle + 1 
                + 
            2 \big \langle \iota - \sigma \big \rangle + 1
        \Big \}
            =
        \Big \{
            2 \big \langle \lambda + \iota + 1 \big \rangle 
                - 
            2 \big \langle \epsilon + \sigma \big \rangle
        \Big \}
    \end{equation*}
    \begin{equation*}
            >
        \Big\{
            \big \lfloor \lambda \big \rfloor
                +
            \big \lfloor \iota \big \rfloor
                +
            \big \lfloor \lambda + \iota \big \rfloor
        \Big\}
    \end{equation*}
    $\therefore \text{\space} \bm{
        \big \lfloor \lambda \rfloor 
            + 
        \big \lfloor \iota \rfloor  
            + 
        \big \lfloor \lambda + \iota \rfloor 
            \le 
        \big \lfloor 2 \lambda \rfloor 
            + 
        \big \lfloor 2 \iota \rfloor 
    }$.
\color{lightgray} \end{proof}
\pagebreak


% =============================== Theorem 1.6.8 ================================
\begin{theorem*}[\textbf{1.6.8}]
    If \bm{$\eta$} is a perfect square, 
    then \bm{$\eta + 2$} is not a perfect square.
\end{theorem*}
\begin{proof}
    Let \bm{$\eta$} be a perfect square. 
    Assume \bm{$\eta + 2$} is a perfect square for the purpose of contradiction. 
    By the definition of perfect square, 
    \bm{$\sqrt{\eta}$} has to be an integer, 
    and by our assumption there exists an integer \bm{$\zeta$} such that 
    \bm{$\zeta^2 = \eta + 2$}. 
    So the equivalence 
    \bm{$\zeta^2 - \big \langle \sqrt{\eta} \big \rangle ^2 = 2$} must be the difference of squares 
    \bm{$ \big \langle \zeta + \sqrt{\eta} \big \rangle 
    \big \langle \zeta - \sqrt{\eta} \big \rangle = 2$}. 
    Since integers are closed on addition and subtraction, 
    it follows that the factors of \bm{$2$}, 
    \bm{$\big \langle \zeta + \sqrt{\eta} \big \rangle$} and 
    \bm{$\big \langle \zeta - \sqrt{\eta} \big \rangle$}, 
    have to be integers. 
    Because \bm{$2$} is prime, those integer factors can only be elements in the set
    \bm{$\{-2, -1, 1, 2\}$}. Thus, there are exactly two possibilities:
    \\ \\ \indent \indent \bm{$(i)$} $\zeta ^2 - \big \langle \sqrt{\eta} \big \rangle ^2 = 
                            \big \langle 2 \big \rangle \big \langle 1 \big \rangle$,
    \\ \indent \indent or \bm{$(ii)$} $\zeta ^2 - \big \langle \sqrt{\eta} \big \rangle ^2 = 
                            \big \langle -1 \big \rangle \big \langle -2 \big \rangle.$
    \\ \\ In case \bm{$(i)$}, without loss of generality, 
    we have a system of linear equations in two variables \bm{$\zeta$} and \bm{$\sqrt{\eta}$}: 
    \begin{equation*}
        \zeta + \sqrt{\eta} = 2
    \end{equation*}
    \begin{equation*}
        \zeta - \sqrt{\eta} = 1
    \end{equation*}
    The matrix of coefficients 
    $\bm{\Psi =} \left[\begin{smallmatrix}
        \bm{1} & \bm{1} \\
        \bm{1} & \bm{-1} 
    \end{smallmatrix}\right]$, 
    the inverse for which is
    $\bm{\Psi^{-1} =} \left[\begin{smallmatrix}
        \bm{0.5} & \bm{0.5} \\
        \bm{0.5} & \bm{-0.5}
    \end{smallmatrix}\right]$. 
    The product of \bm{$\Psi^{-1}$} and the matrix of solutions yields \bm{$\zeta = 1.5$}, 
    which is not in \bm{$\mathbb{Z}$}; 
    contradicting the assumption that \bm{$\zeta^2$} was a perfect square.
    \\ \\ 
    In case \bm{$(ii)$}, 
    we are presented with a similar system of linear equations. 
    The only difference in this system compared to \bm{$(i)$} is the matrix of solutions 
    $\bm{\Phi =} \left[\begin{smallmatrix}
        \bm{-1} \\
        \bm{-2}
    \end{smallmatrix}\right]$. 
    \bm{$\Psi^{-1}\Phi$} yields \bm{$\zeta = -1.5$}, 
    which is not in \bm{$\mathbb{Z}$}, 
    a contradiction. 
    Thus, the assumption that \bm{$\zeta^2$} was a perfect square must be false in this case, as well.
    \\ \\
    Since the assumption proves false in all possible cases, 
    it is not possbile that both \bm{$\eta + 2$}, and \bm{$\eta$} are perfect squares.
\color{lightgray} \end{proof}
\pagebreak


% ============================== Theorem 2.3.71b ===============================
\begin{theorem*}[\textbf{2.3.71b}]
    Let \bm{$\lambda$} be a positive real number.
    \begin{equation*}
        \bm{
            \bigg \lceil 
                \sqrt{ \strut \big \lceil \lambda \big \rceil }
            \bigg \rceil 
                = 
            \bigg \lceil \sqrt{ \strut \lambda } \bigg \rceil
        }
    \end{equation*}
\end{theorem*}
\begin{proof}
    By the properties for ceiling functions,
    there exists an integer
    \bm{$\big \lceil \sqrt{ \lceil \lambda \rceil} \big \rceil$}
    such that
    \bm{$
        \big \lceil \sqrt{ \lambda } \big \rceil
            =
        \big \lceil \sqrt{ \lceil \lambda \rceil} \big \rceil$
    },
    if and only if
    \begin{equation*}
        \bigg \langle
            \bigg \lceil
                \sqrt{ \strut \big \lceil \lambda \big \rceil }
            \bigg \rceil
                -
            1
        \bigg \rangle
            <
        \bigg \langle
            \sqrt{ \strut \lambda }
        \bigg \rangle
            \leq
        \bigg \langle
            \bigg \lceil
                \sqrt{ \strut \big \lceil \lambda \big \rceil }
            \bigg \rceil
        \bigg \rangle
    \end{equation*}
    By the multiplicative compatibility law from the order axioms, that is
    \begin{equation*}
        \bigg \langle
            \bigg \lceil
                \sqrt{ \strut \big \lceil \lambda \big \rceil }
            \bigg \rceil
                -
            1
        \bigg \rangle
            ^2
            <
        \bigg \langle
            \lambda
        \bigg \rangle
            \leq
        \bigg \langle
            \bigg \lceil
                \sqrt{ \strut \big \lceil \lambda \big \rceil }
            \bigg \rceil
        \bigg \rangle
            ^2
    \end{equation*}
    \bm{$\big \lceil \lambda \big \rceil$} is the smallest integer that is greater than or equal to \bm{$\lambda$},
    so by the definition of the ceiling function, 
    \bm{$\lambda \leq \big \lceil \lambda \big \rceil$}.
    Also, 
    \bm{$\big \lceil \sqrt{ \lceil \lambda \rceil} \big \rceil ^2$}
    is an integer by the definition of floor functions,
    since integers are closed under multiplication.
    Hence,
    \bm{$
        \big \lceil \lambda \big \rceil
            \leq
        \big \lceil \sqrt{ \lceil \lambda \rceil} \big \rceil ^2
    $}.
    So by the transitivity law from the order axioms,
    \begin{equation*}
        \bigg \langle
            \bigg \lceil
                \sqrt{ \strut \big \lceil \lambda \big \rceil }
            \bigg \rceil
                -
            1
        \bigg \rangle
            ^2
            <
        \bigg \langle
            \lambda
        \bigg \rangle
            \leq
        \bigg \langle
            \big \lceil \lambda \big \rceil
        \bigg \rangle
            \leq
        \bigg \langle
            \bigg \lceil
                \sqrt{ \strut \big \lceil \lambda \big \rceil }
            \bigg \rceil
        \bigg \rangle
            ^2
            \equiv
    \end{equation*}
    \begin{equation*}
        \bigg \langle
            \bigg \lceil
                \sqrt{ \strut \big \lceil \lambda \big \rceil }
            \bigg \rceil
                -
            1
        \bigg \rangle
            ^2
            <
        \bigg \langle
            \big \lceil \lambda \big \rceil
        \bigg \rangle
            \leq
        \bigg \langle
            \bigg \lceil
                \sqrt{ \strut \big \lceil \lambda \big \rceil }
            \bigg \rceil
        \bigg \rangle
            ^2
    \end{equation*}
    By the multiplicative compatibility law from the order axioms,
    the following is an equivalent statement,
    \begin{equation*}
        \bigg \langle
            \bigg \lceil
                \sqrt{ \strut \big \lceil \lambda \big \rceil }
            \bigg \rceil
                -
            1
        \bigg \rangle
            <
        \bigg \langle
            \sqrt{ \strut \big \lceil \lambda \big \rceil }
        \bigg \rangle
            \leq
        \bigg \langle
            \bigg \lceil
                \sqrt{ \strut \big \lceil \lambda \big \rceil }
            \bigg \rceil
        \bigg \rangle
    \end{equation*}
    $\therefore \text{\space} \bm{
        \bigg \lceil 
            \sqrt{ \strut \big \lceil \lambda \big \rceil }
        \bigg \rceil 
            = 
        \bigg \lceil \sqrt{ \strut \lambda } \bigg \rceil
    }$, by the properties of ceiling functions.
\color{lightgray} \end{proof}
\pagebreak


% ============================== Theorem 2.3.67b ===============================
\begin{theorem*}[\textbf{2.3.67b}]
    Suppose that \bm{$\mathrm{A}$}, and \bm{$\Lambda$} are sets with universal set \bm{$\Omega$}. 
    Let \bm{$\lambda_{\mathrm{A} \cup \Lambda}$} be the characteristic function 
    \bm{$\lambda_{\mathrm{A} \cup \Lambda} : \Omega \rightarrow \{0, 1\}$},
    let \bm{$\lambda_{\mathrm{A}}$} be the characteristic function 
    \bm{$\lambda_{\mathrm{A}} : \Omega \rightarrow \{0, 1\}$},
    and let \bm{$\lambda_{\Lambda}$} be the characteristic function 
    \bm{$\lambda_{\Lambda}: \Omega \rightarrow \{0, 1\}$}.
    \begin{equation*}
        \bm{
            \lambda_{\mathrm{A} \cup \Lambda} \big[ \iota \big] 
                = 
            \lambda_{\mathrm{A}} \big[ \iota \big] 
                + 
            \lambda_{\Lambda} \big[ \iota \big] 
                - 
            \lambda_{\mathrm{A}} \big[ \iota \big]
                \times 
            \lambda_{\Lambda} \big[ \iota \big]
        }
    \end{equation*}
\end{theorem*}
\begin{proof}
    By cases. There are two major cases under consideration.
    \\ \\
    \bm{$(i)$} Assume \bm{$\iota$} is not an element in \bm{$\mathrm{A} \cup \Lambda$}. 
    Note that, by the definition for characteristic functions, 
    \bm{$\lambda_{\mathrm{A} \cup \Lambda}\big[ \iota \big] = 0$}.
    Now, by the definition for set union 
    \bm{$\iota$} is in neither \bm{$\mathrm{A}$} nor \bm{$\Lambda$}.
    So \bm{$\lambda_{\mathrm{A}} \big[ \iota \big] = 0$}, 
    and \bm{$\lambda_{\Lambda} \big[ \iota \big] = 0$}. 
    Hence,
    \begin{equation*}
        \bigg \langle
            \lambda_{\mathrm{A}} \big[ \iota \big] 
                + 
            \lambda_{\Lambda} \big[ \iota \big] 
                - 
            \lambda_{\mathrm{A}} \big[ \iota \big] 
                \times 
            \lambda_{\Lambda} \big[ \iota \big] 
        \bigg \rangle
            =
        \bigg \langle 
            0 + 0 - 0 \times 0 
        \bigg \rangle
            =
        \bigg \langle 
            0
        \bigg \rangle
    \end{equation*}
    $\therefore \text{\space} \bm{
        \lambda_{\mathrm{A} \cup \Lambda} \big[ \iota \big] 
            = 
        \lambda_{\mathrm{A}} \big[ \iota \big] 
            + 
        \lambda_{\Lambda} \big[ \iota \big] 
            - 
        \lambda_{\mathrm{A}} \big[ \iota \big]
            \times 
        \lambda_{\Lambda} \big[ \iota \big]
    }$.
    \\ \\
    \bm{$(ii)$} Assume \bm{$\iota$} is an element in \bm{$\mathrm{A} \cup \Lambda$}. 
    By the definition for characteristic functions 
    \bm{$\lambda_{\mathrm{A} \cup \Lambda} \big[ \iota \big] = 1$}. 
    Also, by the definition for set union 
    \begin{equation*}        
        \Big \langle \iota \in \mathrm{A} \Big \rangle 
            \lor 
        \Big \langle \iota \in \Lambda \Big \rangle
    \end{equation*}
    There are three subcases. 
    \\ \\
    \bm{$(a)$} Suppose \bm{$\iota$} is an element in \bm{$\mathrm{A}$}, but not in \bm{$\Lambda$}.
    By the definition for characteristic functions 
    \bm{$\lambda_{\mathrm{A}} \big[ \iota \big] = 1$}, 
    and \bm{$\lambda_{\Lambda} \big[ \iota \big] = 0$}. 
    Thus,
    \begin{equation*}
        \bigg \langle
            \lambda_{\mathrm{A}} \big[ \iota \big] 
                + 
            \lambda_{\Lambda} \big[ \iota \big] 
                - 
            \lambda_{\mathrm{A}} \big[ \iota \big] 
                \times 
            \lambda_{\Lambda} \big[ \iota \big] 
        \bigg \rangle
            =
        \bigg \langle 
            1 + 0 - 1 \times 0 
        \bigg \rangle
            =
        \bigg \langle 
            1
        \bigg \rangle
    \end{equation*} 
    \bm{$(b)$} Suppose \bm{$\iota$} is not an element in \bm{$\mathrm{A}$}, 
    but is an element in \bm{$\Lambda$}. 
    Without loss of generality this case has the same result as case $(a)$.
    \\ \\
    \bm{$(c)$} Suppose \bm{$\iota$} is in the intersection of \bm{$\mathrm{A}$} and \bm{$\Lambda$}.
    \begin{equation*}
        \bigg \langle
            \lambda_{\mathrm{A}} \big[ \iota \big] 
                + 
            \lambda_{\Lambda} \big[ \iota \big] 
                - 
            \lambda_{\mathrm{A}} \big[ \iota \big] 
                \times 
            \lambda_{\Lambda} \big[ \iota \big] 
        \bigg \rangle
            =
        \bigg \langle 
            1 + 1 - 1 \times 1 
        \bigg \rangle
            =
        \bigg \langle 
            1
        \bigg \rangle
    \end{equation*}
    $\therefore \text{\space} \bm{
        \lambda_{\mathrm{A} \cup \Lambda} \big[ \iota \big] 
            = 
        \lambda_{\mathrm{A}} \big[ \iota \big] 
            + 
        \lambda_{\Lambda} \big[ \iota \big] 
            - 
        \lambda_{\mathrm{A}} \big[ \iota \big]
            \times 
        \lambda_{\Lambda} \big[ \iota \big]
    }$.
\color{lightgray} \end{proof}
\pagebreak


% =============================== Theorem 2.2.24 ===============================
\begin{theorem*}[\textbf{2.2.24}] \color{black}
    Let \bm{$\mathrm{A}$}, \bm{$\Lambda$}, and \bm{$\Delta$} be sets. 
    \begin{equation*}
        \bm{
            \Big \langle \mathrm{A} - \Lambda \Big \rangle - \Delta 
                = 
            \Big \langle \mathrm{A} - \Delta \Big \rangle 
                - 
            \Big \langle \Lambda - \Delta \Big \rangle
        }
    \end{equation*}
\end{theorem*}
\begin{proof} \color{black}
    Let \bm{$\lambda$} be an element in 
    \bm{$
    \big \langle \mathrm{A} - \Lambda \big \rangle - \Delta 
    $}. 
    By the definition for set difference,
    \begin{equation*} %\color{gray}
        \bigg[
            \Big \langle \lambda \in \mathrm{A} \Big \rangle
                \land
            \Big \langle \lambda \notin \Lambda \Big \rangle
        \bigg]
            \land
        \Big \langle \lambda \notin \Delta \Big \rangle
    \end{equation*}
    Note that, by the indentity law for logical disjunction, 
    \bm{$
    \lambda \notin \Lambda
        \equiv 
    \lambda \notin \Lambda
        \lor 
    \bot
    $}.
    And since
    \bm{$
    \lambda \in \Delta
        \equiv 
    \bot
    $},
    by definition,
    it follows that
    \begin{equation*} %\color{gray}
        \Big \langle \lambda \notin \Lambda \Big \rangle
            \equiv
        \bigg[
            \Big \langle \lambda \notin \Lambda \Big \rangle
                \lor
            \Big \langle \bot \Big \rangle
        \bigg]
            \equiv
        \bigg[
            \Big \langle \lambda \notin \Lambda \Big \rangle
                \lor
            \Big \langle \lambda \in \Delta \Big \rangle
        \bigg]
    \end{equation*}
    Moreover, by the double negation law of logic, and by DeMorgans laws,
    \begin{equation*} %\color{gray}
        \Big \langle \lambda \notin \Lambda \Big \rangle
            \equiv
        \lnot \Bigg\{
            \lnot \bigg[
                \Big \langle \lambda \notin \Lambda \Big \rangle
                    \lor
                \Big \langle \lambda \in \Delta \Big \rangle
            \bigg]
        \Bigg\}
            \equiv
        \lnot \bigg[
            \Big \langle \lambda \in \Lambda \Big \rangle
                \land
            \Big \langle \lambda \notin \Delta \Big \rangle
        \bigg]
    \end{equation*}
    Thus, the proposition
    \bm{$
    \lambda \in
    \big \langle \mathrm{A} - \Lambda \big \rangle - \Delta 
    $},
    is equivalent to
    \begin{equation*} %\color{gray}
        \Bigg\{
        \Big \langle \lambda \in \mathrm{A} \Big \rangle
            \land
        \lnot \bigg[
            \Big \langle \lambda \in \Lambda \Big \rangle
                \land
            \Big \langle \lambda \notin \Delta \Big \rangle
        \bigg]
        \Bigg\}
            \land
        \Big \langle \lambda \notin \Delta \Big \rangle
    \end{equation*}
    By law of commutativity (and association) for logical conjunction, 
    that is
    \begin{equation*} %\color{gray}
        \bigg[
            \Big \langle \lambda \in \mathrm{A} \Big \rangle
                \land
            \Big \langle \lambda \notin \Delta \Big \rangle
        \bigg]
            \land
        \lnot \bigg[
            \Big \langle \lambda \in \Lambda \Big \rangle
                \land
            \Big \langle \lambda \notin \Delta \Big \rangle
        \bigg]
    \end{equation*}
    By the definitions for the difference of sets, set complementation,
    and the intersection of sets,
    \begin{equation*} %\color{gray}
        \bigg[
        \Big \langle \mathrm{A} - \Lambda \Big \rangle - \Delta
        \bigg]
            \equiv
        \bigg[
        \Big \langle \mathrm{A} - \Delta \Big \rangle
            \cap
        \overline{
            \Big \langle \Lambda - \Delta \Big \rangle
        }
        \bigg]
    \end{equation*}
    $\therefore \text{\space} \bm{
    \big \langle \mathrm{A} - \Lambda \big \rangle - \Delta 
        = 
    \big \langle \mathrm{A} - \Delta \big \rangle 
        - 
    \big \langle \Lambda - \Delta \big \rangle
    }$, by Thereom 2.2.19

\color{lightgray} \end{proof}
\pagebreak


% ============================== Theorem 2.4.22 ================================
\begin{theorem*}[\textbf{2.4.22}]
    The sum of squares from \bm{$1$} to \bm{$\lambda$} is 
    \begin{equation*}
        \bm{
            \frac{ 
                \lambda 
                \langle \lambda + 1 \rangle 
                \langle 2 \lambda + 1 \rangle 
            }
            {6}
        }
    \end{equation*}
\end{theorem*}
\begin{proof}
    The formula for the summation of squares 
    from \bm{$1$} to \bm{$\lambda$} 
    can be derived from the cube of \bm{$\lambda$}.
    It is trivial that 
    \bm{$\lambda ^3 = \lambda ^3 - \big \langle 1 - 1 \big \rangle ^3$}. 
    By this identity for \bm{$\lambda$},
    the cube of \bm{$\lambda$} is the telescopic summation given by theorem 2.4.19, 
    \begin{equation*}
        \lambda ^3 
            = 
        \sum_{\iota=1}^\lambda \iota ^3 - \big \langle \iota - 1 \big \rangle ^3    
    \end{equation*}
    The expansion for 
    \bm{$\big \langle \iota - 1 \big \rangle ^3$} 
    is 
    \bm{$\iota ^3 - 3 \iota ^2 + 3 \iota - 1$}, 
    by the Binomial Theorem. 
    Thus, 
    by the inverse law for addition from the field axioms,
    yielding the algebraic identity
    \begin{equation*}
        \iota ^3 - \big \langle \iota - 1 \big \rangle ^3 
            = 
        3 \iota ^2 - 3 \iota + 1
    \end{equation*}
    Hence, 
    \bm{$\lambda ^3 = \sum_{\iota=1}^\lambda 3 \iota ^2 - 3 \iota + 1$}.
    By the commutative law for addition from the field axioms, 
    and by the distributive law for real numbers,
    that is
    \begin{equation*}
        \lambda ^3 
            = 
        \left( 3 \sum_{\iota=1}^\lambda \iota ^2 \right) 
            - 
        \left( 3 \sum_{\iota=1}^\lambda \iota \right) 
            + 
        \left( \sum_{\iota=1}^\lambda 1 \right)
    \end{equation*}
    Note that \bm{$
        \sum_{\iota=1} ^\lambda 1 
            = 
        \lambda \big \langle 1 \big \rangle
    $}. 
    And by Theorem 2.4.21b, 
    \bm{$
        \sum_{\iota=1}^\lambda \iota 
            = 
        \frac{ \lambda \big \langle \lambda + 1 \big \rangle }{2}
    $}. 
    Thus, 
    by those identities, 
    and by the inverse law for addition from the field axioms
    \begin{equation*}
        \lambda ^3 
            + 
        3 \frac{ \lambda \big \langle \lambda + 1 \big \rangle }
        {2} 
            - 
        \lambda 
            = 
        3 \sum_{\iota=1}^\lambda \iota ^2
    \end{equation*}
    Eliminating the coefficient \bm{$3$} from the right-hand side, 
    by the inverse law for multiplication from the field axioms, 
    gives us the sum of squares in terms of an equation, 
    \begin{equation*}
        \frac{1}{3}
        \bigg[
            \lambda ^3 
                + 
            3 \frac{ \lambda \big \langle \lambda + 1 \big \rangle }
            {2} 
                - 
            \lambda
        \bigg]
            = 
        \sum_{\iota=1}^\lambda \iota ^2
    \end{equation*}
    By Lemma 2.4.1, that is
    \begin{equation*}
        \bm{
            \sum_{\iota=1}^\lambda \iota ^2 
                = 
            \frac{ 
                \lambda \big \langle \lambda + 1 \big \rangle
                \big \langle 2 \lambda + 1 \big \rangle 
            }
            {6}
        }
    \end{equation*}
\color{lightgray} \end{proof}
\pagebreak


% ============================== Theorem 2.4.19 ================================
\begin{theorem*}[\textbf{2.4.19}]
    Let \bm{$\{\lambda_\zeta\}$} be a sequence of real numbers.
    \begin{equation*}
        \bm{
            \sum_{\iota=1}^\zeta \big \langle 
                \lambda_\iota - \lambda_{\iota-1}
            \big \rangle
                = 
            \lambda_\zeta - \lambda_0
        }
    \end{equation*}
\end{theorem*}
\begin{proof}
    \begin{equation*}
        \sum_{\iota=1}^\zeta \big \langle
            \lambda_\iota - \lambda_{\iota-1}
        \big \rangle
            =
        \big \langle \lambda_\zeta - \lambda_{\zeta-1} \big \rangle 
            + 
        \big \langle \lambda_{\zeta-1} - \lambda_{\zeta-2} \big \rangle 
            + 
        \dots 
            + 
        \big \langle \lambda_1 - \lambda_0 \big \rangle
    \end{equation*}
    By associativity for addition from the field axioms for real numbers, that is 
    \begin{equation*}
        \lambda_\zeta 
            + 
        \big \langle -\lambda_{\zeta-1} + \lambda_{\zeta-1} \big \rangle
            + 
        \big \langle -\lambda_{\zeta-2} + \lambda_{\zeta-2} \big \rangle
        %    + 
        %\big \langle -\lambda_{\zeta-3} + \lambda_{\zeta-3} \big \rangle
            + 
        \dots 
            + 
        \big \langle -\lambda_1 + \lambda_1 \big \rangle
            + 
        -\lambda_0
    \end{equation*}
    The inner terms cancel out 
    by the inverse law for addition from the field axioms. 
    $\therefore$
    \begin{equation*}
        \bm{
            \sum_{\iota=1}^\zeta \big \langle 
                \lambda_\iota - \lambda_{\iota-1}
            \big \rangle
                = 
            \lambda_\zeta - \lambda_0
        }
    \end{equation*} 
\color{lightgray} \end{proof}
\pagebreak


\end{document}