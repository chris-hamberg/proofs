\documentclass[preview]{standalone}
\usepackage{amssymb, amsthm}
\usepackage{mathtools}
\usepackage{bm}


\newtheorem{theorem}{Theorem}
\renewcommand\qedsymbol{$\blacksquare$}


\begin{document}


\begin{theorem} %[\textbf{1606}]
    The product of two odd numbers is odd.
\end{theorem}

\begin{proof}
    Suppose that \bm{$\mu$} and \bm{$\zeta$} are odd numbers. 
    By the definition for odd numbers, 
    there exist integers \bm{$\sigma$} and \bm{$\epsilon$} such that 
    \bm{$\mu = 2\sigma + 1$} and \bm{$\zeta = 2\epsilon + 1$}. 
    Thus, the product of odd numbers \bm{$\mu \zeta$} is,
    \begin{equation*}
        \mu \zeta 
            =
        \Big[ 
            \big \langle 2 \sigma + 1 \big \rangle 
            \big \langle 2 \epsilon + 1 \big \rangle
        \Big]
            = 
        \Big[% \langle
            2 \sigma 2 \epsilon + 2 \sigma + 2 \epsilon + 1 
        \Big]% \rangle
            = 
        \Big[
            2 \big \langle 
                \sigma \epsilon + \sigma + \epsilon 
            \big \rangle 
                + 1
        \Big]
    \end{equation*}     
    The factor 
    \bm{$\big \langle \sigma \epsilon + \sigma + \epsilon \big \rangle$} 
    is an integer because \bm{$\sigma$} and \bm{$\epsilon$} are integers by definition, 
    and integers are closed on addition. 
    Therefore, \bm{$\mu \zeta$} is odd by the definition for odd numbers.
\end{proof}


\end{document}