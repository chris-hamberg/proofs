\documentclass[preview]{standalone}
\usepackage{amssymb, amsthm}
\usepackage{mathtools}
\usepackage{bm}


\newtheorem{theorem}{Theorem}
\renewcommand\qedsymbol{$\blacksquare$}


\begin{document}


\begin{theorem} %[\textbf{1613}]
    If \bm{$\chi$} is an irrational number, 
    then \bm{$\frac{1}{\chi}$} is irrational.
\end{theorem}

\begin{proof}
    By the contrapositive. 
    Suppose that \bm{$\frac{1}{\chi}$} is a rational number. 
    By the definition for rational numbers, 
    there exist integers \bm{$\alpha$} and \bm{$\gamma$} such that 
    \bm{$\frac{1}{\chi} = \frac{\alpha}{\gamma}$}. 
    Note that \bm{$\alpha$} is nonzero (because \bm{$\frac{1}{\chi}$} is nonzero.) 
    By the multiplicative property of equality for equations,
    \begin{equation*}
        \Bigg\{
            \left(
                \chi \cdot \frac{1}{\chi}
            \right) 
                = 
            \left(
                \chi \cdot \frac{\alpha}{\gamma}
            \right)
        \Bigg\} 
            \equiv
        \Bigg\{
            \left(
                \frac{\chi}{\chi} \cdot \frac{\gamma}{\alpha}
            \right) 
                = 
            \left(
                \frac{\chi \alpha}{\gamma} \cdot \frac{\gamma}{\alpha}
            \right)
        \Bigg\} 
            \equiv
        \Bigg\{
            \frac{\gamma}{\alpha} 
                = 
            \chi
        \Bigg\}
    \end{equation*}
    \bm{$\frac{\gamma}{\alpha} = \chi$} is rational, 
    by definition. 
    Thus, if \bm{$\frac{1}{\chi}$} is rational, 
    then \bm{$\chi$} is rational.
\end{proof}


\end{document}