\documentclass[preview]{standalone}
\usepackage{amssymb, amsthm}
\usepackage{mathtools}
\usepackage{bm}


\newtheorem{theorem}{Theorem}
\renewcommand\qedsymbol{$\blacksquare$}


\begin{document}


\begin{theorem} %[\textbf{1614}]
    If \bm{$\chi$} is a rational number and 
    \bm{$\chi \neq 0$}, 
    then \bm{$\frac{1}{\chi}$} is rational.
\end{theorem}

\begin{proof}
    Let \bm{$\alpha$} and \bm{$\gamma$} be nonzero integers. 
    \bm{$\chi = \frac{\alpha}{\gamma}$}, 
    by the definition for rational numbers. 
    By the multiplicative property of equality for equations
    \begin{equation*}
        \left\{
            \left(
                \frac{1}{\chi} \cdot \chi
            \right) 
                = 
            \left(
                \frac{1}{\chi} \cdot \frac{\alpha}{\gamma}
            \right)
        \right\} 
            \equiv 
        \left\{
            \left(
                \frac{\chi}{\chi} \cdot \frac{\gamma}{\alpha}
            \right) 
                = 
            \left(
                \frac{\alpha}{\chi\gamma} \cdot \frac{\gamma}{\alpha}
            \right)
        \right\}
            \equiv 
        \left\{
            \frac{\gamma}{\alpha} 
                = 
            \frac{1}{\chi}
        \right\}
    \end{equation*}
    \bm{$\frac{\gamma}{\alpha} = \frac{1}{\chi}$} is rational, by definition. 
    Thus, if \bm{$\chi$} is a rational number and \bm{$\chi \neq 0$}, 
    then \bm{$\frac{1}{\chi}$} is rational.
\end{proof}


\end{document}