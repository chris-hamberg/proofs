\documentclass[preview]{standalone}
\usepackage{amssymb, amsthm}
\usepackage{mathtools}
\usepackage{bm}


\newtheorem{theorem}{Theorem}
\renewcommand\qedsymbol{$\blacksquare$}


\begin{document}


\begin{theorem} %[\textbf{1617}]
    Let \bm{$\zeta$} be an integer. 
    If \bm{$\zeta^3 + 5$} is odd, 
    then \bm{$\zeta$} is even.
\end{theorem}

\begin{proof}
    By the contrapositive. 
    Suppose that \bm{$\zeta$} were odd. 
    By the definition for odd numbers, 
    there exists an integer \bm{$\gamma$} such that 
    \bm{$\zeta = 2 \gamma + 1$}. 
    By the Binomial Theorem,
    \begin{equation*}
        \Bigg \{
            \big \langle 2 \gamma + 1 \big \rangle ^3 + 5
        \Bigg \}
            = 
        \Bigg \{
            5 
                + 
            \sum\limits_{\iota=0} ^3 {3 \choose \iota} 
                2 \gamma ^{\langle 3 - \iota \rangle}
        \Bigg \}
            = 
        \Bigg \{
            2 
            \big \langle 4 \gamma ^3 - 6 \gamma ^2 + 3 \gamma + 3 \big \rangle
        \Bigg \}
    \end{equation*}
    The factor 
    \bm{$\big \langle 4 \gamma ^3 - 6 \gamma ^2 + 3 \gamma + 3 \big \rangle$} 
    is an integer because integers are closed on addition and multiplication. 
    Thus, \bm{$\zeta^3 + 5$} is even, by definition.
\end{proof}


\end{document}