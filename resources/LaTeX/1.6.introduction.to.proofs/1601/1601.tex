\documentclass[preview]{standalone}
\usepackage{amssymb, amsthm}
\usepackage{mathtools}
\usepackage{bm}
\usepackage{xcolor}


\newtheorem{theorem}{Theorem}
\renewcommand\qedsymbol{$\blacksquare$}


\begin{document}
    
\begin{theorem} %[\textbf{1601}]
    Let \bm{$\chi$} and \bm{$\zeta$} be integers. 
    If \bm{$\chi$} and \bm{$\zeta$} are odd, 
    then \bm{$\chi + \zeta$} is even.
\end{theorem}

\begin{proof}
    By the definition for odd numbers, 
    there exists integers \bm{$\mu$} and \bm{$\nu$} such that 
    \bm{$\chi = 2\mu + 1$} and \bm{$\zeta = 2\nu + 1$}. 
    Hence,
    \begin{equation*}
        \chi + \zeta 
            = 
        \Big[
            \big \langle 2 \mu + 1 \big \rangle 
                + 
            \big \langle 2 \nu + 1 \big \rangle
        \Big]
            = 
        \Big[
            2 \big \langle \mu + \nu + 1 \big \rangle
        \Big]
    \end{equation*} 
    Integers are closed under addition. 
    Thus, the factor 
    \bm{$\big \langle \mu + \nu + 1 \big \rangle$} is an integer. 
    It follows that \bm{$\chi + \zeta$} is even, 
    by the definition for even numbers.
\color{lightgray} \end{proof}

\end{document}