\documentclass[preview]{standalone}
\usepackage{amssymb, amsthm}
\usepackage{mathtools}
\usepackage{bm}


\newtheorem{theorem}{Theorem}
\renewcommand\qedsymbol{$\blacksquare$}


\begin{document}

\begin{theorem} %[\textbf{1602}]
    Let \bm{$\chi$} and \bm{$\zeta$} be integers. 
    If \bm{$\chi$} and \bm{$\zeta$} are even, 
    then \bm{$\chi \textbf{ + } \zeta$} is even.
\end{theorem}

\begin{proof}
    By the definition for even numbers, 
    there exist integers \bm{$\mu$} and \bm{$\nu$} such that 
    \bm{$2\mu \textbf{ = } \chi$} and \bm{$2\nu \textbf{ = } \zeta$}. 
    Hence,
    \begin{equation*}
        \chi + \zeta 
            = 
        \Big[
            \big \langle 2\mu \big \rangle
                + 
            \big \langle 2\nu \big \rangle
        \Big]
            =
        \Big[ 
            2 \big \langle \mu + \nu \big \rangle
        \Big]
    \end{equation*} 
    Integers are closed under addition. 
    Thus, the factor 
    \bm{$\big \langle \mu \textbf{ + } \nu \big \rangle$} 
    is an integer. 
    It follows that \bm{$\chi \textbf{ + } \zeta$} is even, 
    by the definition for even numbers.
\end{proof}

\end{document}