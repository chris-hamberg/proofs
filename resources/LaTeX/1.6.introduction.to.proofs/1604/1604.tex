\documentclass[preview]{standalone}
\usepackage{amssymb, amsmath, amsthm}
\usepackage{mathtools}
\usepackage{bm}
\usepackage{xcolor}


\newtheorem{theorem}{Theorem}
\renewcommand\qedsymbol{$\blacksquare$}


\begin{document}


\begin{theorem} %[\textbf{1604}]
    The additive inverse of an even number is an even number.
\end{theorem}

\begin{proof}
    Let \bm{$\chi$} be an even number. 
    There exists an integer \bm{$\eta$} such that 
    \bm{$\chi = 2\eta$}, 
    by the definition for even numbers. 
    The additive inverse for \bm{$\chi$} is,
    \begin{equation*}
        -1 \big \langle \chi \big \rangle
            = 
        -1 \big \langle 2 \eta \big \rangle
    \end{equation*} 
    By commutativity of multiplication that is,
    \begin{equation*}
        -1 \big \langle 2 \eta \big \rangle
            = 
        2 \big \langle - \eta \big \rangle
    \end{equation*}
    Since integers are closed under multiplication, 
    the factor \bm{$\big \langle -\eta \big \rangle$} is an integer. 
    It follows that the additive inverse of \bm{$\chi$} is an even number, 
    by the definition for even numbers.
\color{lightgray} \end{proof}


\end{document}