\documentclass[preview]{standalone}
\usepackage{amssymb, amsthm}
\usepackage{mathtools}
\usepackage{bm}


\newtheorem{theorem}{Theorem}
\renewcommand\qedsymbol{$\blacksquare$}


\begin{document}


\begin{theorem} %[\textbf{1608}]
    If \bm{$\eta$} is a perfect square, 
    then \bm{$\eta \textbf{ + } 2$} is not a perfect square.
\end{theorem}

\begin{proof}
    Let \bm{$\eta$} be a perfect square. 
    Assume \bm{$\eta \textbf{ + } 2$} is a perfect square for the purpose of contradiction. 
    By the definition of perfect square, 
    \bm{$\sqrt{\eta}$} has to be an integer, 
    and by our assumption there exists an integer \bm{$\zeta$} such that 
    \bm{$\zeta^2 \textbf{ = } \eta \textbf{ + } 2$}. 
    So the equivalence 
    \bm{$\zeta^2 \textbf{ - } \big \langle \sqrt{\eta} \big \rangle ^2 \textbf{ = } 2$} 
    must be the difference of squares 
    \begin{equation*}
        \big \langle \zeta + \sqrt{\eta} \big \rangle 
        \big \langle \zeta - \sqrt{\eta} \big \rangle 
            =
        2
    \end{equation*}
    Since integers are closed on addition and subtraction, 
    it follows that the factors of \bm{$2$}, 
    \bm{$\big \langle \zeta \textbf{ + } \sqrt{\eta} \big \rangle$} and 
    \bm{$\big \langle \zeta \textbf{ - } \sqrt{\eta} \big \rangle$}, 
    have to be integers. 
    Because \bm{$2$} is prime, those integer factors can only be elements in the set
    \bm{$\{-2, -1, 1, 2\}$}. Thus, there are exactly two possibilities:
    \\ \\ \indent \indent \bm{$(i)$} $\zeta ^2 \textbf{ - } \big \langle \sqrt{\eta} \big \rangle ^2 \textbf{ = } 
                            \big \langle 2 \big \rangle \big \langle 1 \big \rangle$,
    \\ \indent \indent or \bm{$(ii)$} $\zeta ^2 \textbf{ - } \big \langle \sqrt{\eta} \big \rangle ^2 \textbf{ = } 
                            \big \langle -1 \big \rangle \big \langle -2 \big \rangle.$
    \\ \\ In case \bm{$(i)$}, without loss of generality, 
    we have a system of linear equations in two variables \bm{$\zeta$} and \bm{$\sqrt{\eta}$}:
    \begin{equation*}
        \zeta + \sqrt{\eta} = 2
    \end{equation*}
    \begin{equation*}
        \zeta - \sqrt{\eta} = 1
    \end{equation*}
    The matrix of coefficients 
    $\bm{\Psi \textbf{ = }} \left[\begin{smallmatrix}
        \bm{1} & \bm{1} \\
        \bm{1} & \bm{-1} 
    \end{smallmatrix}\right]$, 
    the inverse for which is
    $\bm{\Psi^{-1} \textbf{ = }} \left[\begin{smallmatrix}
        \bm{0.5} & \bm{0.5} \\
        \bm{0.5} & \bm{-0.5}
    \end{smallmatrix}\right]$. 
    The product of \bm{$\Psi^{-1}$} and the matrix of solutions yields \bm{$\zeta \textbf{ = } 1.5$}, 
    which is not in \bm{$\mathbb{Z}$}; 
    contradicting the assumption that \bm{$\zeta^2$} was a perfect square.
    \\ \\ 
    In case \bm{$(ii)$}, 
    we are presented with a similar system of linear equations. 
    The only difference in this system compared to \bm{$(i)$} is the matrix of solutions 
    $\bm{\Phi \textbf{ = }} \left[\begin{smallmatrix}
        \bm{-1} \\
        \bm{-2}
    \end{smallmatrix}\right]$. 
    \bm{$\Psi^{-1}\Phi$} yields \bm{$\zeta \textbf{ = } -1.5$}, 
    which is not in \bm{$\mathbb{Z}$}, 
    a contradiction. 
    Thus, the assumption that \bm{$\zeta^2$} was a perfect square must be false in this case, as well.
    \\ \\
    Since the assumption proves false in all possible cases, 
    it is not possbile that both \bm{$\eta \textbf{ + } 2$}, and \bm{$\eta$} are perfect squares.
\end{proof}


\end{document}