\documentclass[preview]{standalone}
\usepackage{amssymb, amsthm}
\usepackage{mathtools}
\usepackage{bm}
\usepackage{xcolor}


\newtheorem{theorem}{Theorem}
\renewcommand\qedsymbol{$\blacksquare$}


\begin{document}


\begin{theorem} %[\textbf{1612}]
    The product of a nonzero rational number and an irrational number is 
    irrational.
\end{theorem}

\begin{proof}
    For the purpose of contradiction, assume the negation of the hypothesis; 
    the proposition 
        $$\lnot p : \emph{the product of a nonzero rational number and an 
        irrational number}$$
        $\indent \indent \indent \emph{is rational}$
        \\ \\
    Let \bm{$\alpha$}, \bm{$\beta$}, \bm{$\gamma$}, and \bm{$\delta$} be integers such that 
    \bm{$\alpha \neq 0$}, and let \bm{$\chi$} be an irrational number. 
    Then the proposition \bm{$\lnot p$} states 
    \begin{equation*}
        \Bigg(
            \frac{\alpha}{\beta} \space \cdot \space \chi 
        \Bigg)
            = 
        \Bigg(
            \frac{\gamma}{\delta}
        \Bigg)
    \end{equation*}    
    By the multiplicative equality property for equations, that is
    \begin{equation*}
        \Bigg(
            \chi
        \Bigg) 
            = 
        \Bigg(
            \frac{\gamma}{\delta} \space \cdot \space \frac{\beta}{\alpha} 
        \Bigg)
            =
        \Bigg( 
            \frac{\gamma\beta}{\delta\alpha}
        \Bigg)
    \end{equation*}
    By Theorem 9 (the closure property for multplication on rational numbers,) 
    \bm{$\chi$} is rational. 
    Thus, \bm{$\lnot p$} implies \bm{$\chi$} is rational and irrational.
\color{lightgray} \end{proof}
\

\end{document}