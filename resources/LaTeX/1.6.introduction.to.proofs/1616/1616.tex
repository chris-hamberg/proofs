\documentclass[preview]{standalone}
\usepackage{amssymb, amsthm}
\usepackage{mathtools}
\usepackage{bm}
\usepackage{xcolor}


\newtheorem{theorem}{Theorem}
\renewcommand\qedsymbol{$\blacksquare$}


\begin{document}


\begin{theorem} %[\textbf{1616}]
    Let \bm{$\mu$} and \bm{$\zeta$} be integers. 
    If the product \bm{$\mu\zeta$} is even, 
    then \bm{$\mu$} is even or \bm{$\zeta$} is even.
\end{theorem}

\begin{proof}
    For the purpose of contraposition, suppose the negation of the consequent 
    \bm{$q$}
        $$\lnot q : \mu \text{ is odd and } \zeta \text{ is odd.}$$ 
    By definition, 
    there exist integers \bm{$\sigma$} and \bm{$\epsilon$} such that 
    \bm{$\mu = 2\sigma + 1$} and \bm{$\zeta = 2\epsilon + 1$}. 
    Thus, 
    \begin{equation*}
        \mu\zeta 
            =
        \Big[ 
            \big \langle 2 \sigma + 1 \big \rangle
            \big \langle 2 \epsilon + 1 \big \rangle 
        \Big]
            =
        \Big[ 
            2 
            \big \langle \sigma\epsilon + \sigma + \epsilon \big \rangle 
                + 
            1
        \Big]
    \end{equation*}
    The factor 
    \bm{$\big \langle \sigma \epsilon + \sigma + \epsilon \big \rangle$} 
    is an integer, 
    because integers are closed under addition and multiplication. 
    Thus, the product \bm{$\mu \zeta$} is odd, by definition.
\color{lightgray} \end{proof}
\

\end{document}