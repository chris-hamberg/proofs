\documentclass[preview]{standalone}
\usepackage{amssymb, amsthm}
\usepackage{mathtools}
\usepackage{bm}
\usepackage{xcolor}


\newtheorem{theorem}{Theorem}
\renewcommand\qedsymbol{$\blacksquare$}


\begin{document}


\begin{theorem} %[\textbf{1618}]
    Let \bm{$\gamma$} be an integer. 
    If \bm{$3 \gamma + 2$} is even, 
    then \bm{$\gamma$} is even.
\end{theorem}

\begin{proof}
    By the contrapositive. 
    Suppose \bm{$\gamma$} were odd. 
    By the definition of odd numbers,
    there exist an integer \bm{$\mu$} such that 
    \bm{$\gamma = 2 \mu + 1$}. 
    Thus,
    \begin{equation*}
        \Big \langle
            3 \big[ 2 \mu + 1 \big] + 2 
        \Big \rangle
            = 
        \Big \langle
            6 \mu + 5
        \Big \rangle
            =
        \Big \langle
            6 \mu + 4 + 1 
        \Big \rangle
            =
        \Big \langle
            2 \big[ 3 \mu + 2 \big] + 1
        \Big \rangle
    \end{equation*} 
    The factor \bm{$\big[ 3 \mu + 2 \big]$} is an integer, 
    since integers are closed on addition and multiplcation. 
    Thus, \bm{$3\gamma + 2$} is odd, by definition.
\color{lightgray} \end{proof}



\end{document}