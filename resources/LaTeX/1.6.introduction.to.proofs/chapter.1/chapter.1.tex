\documentclass[preview]{standalone}


\begin{document}

\section{Theorems}
\begin{figure}[h!]
    \centering
    \includegraphics[width=9.5cm]{../resources/jpg/1.6.introduction.to.proofs/euclid.jpg}
    \caption*{\emph{Euclid.}}
\end{figure} 

% ============================== 0001 Theorem 1601 ================================
\subsection[The sum of two odd integers is even.]{\color{section} Theorem 1 \color{black} : the sum of two odd integers is even.}
\input{../resources/LaTeX/1.6.introduction.to.proofs/0001_1601/0001_1601.tex}
\pagebreak


% ============================== 0002 Theorem 1602 =================================
\subsection[The sum of two even integers is even.]{\color{section} Theorem 2 \color{black} : the sum of two even integers is even.}
\documentclass[preview]{standalone}
\usepackage{amssymb, amsthm}
\usepackage{mathtools}
\usepackage{bm}


\newtheorem{theorem}{Theorem}
\renewcommand\qedsymbol{$\blacksquare$}


\begin{document}

\begin{theorem} %[\textbf{1602}]
    Let \bm{$\chi$} and \bm{$\zeta$} be integers. 
    If \bm{$\chi$} and \bm{$\zeta$} are even, 
    then \bm{$\chi + \zeta$} is even.
\end{theorem}

\begin{proof}
    By the definition for even numbers, 
    there exist integers \bm{$\mu$} and \bm{$\nu$} such that 
    \bm{$2\mu = \chi$} and \bm{$2\nu = \zeta$}. 
    Hence,
    \begin{equation*}
        \chi + \zeta 
            = 
        \Big[
            \big \langle 2\mu \big \rangle
                + 
            \big \langle 2\nu \big \rangle
        \Big]
            =
        \Big[ 
            2 \big \langle \mu + \nu \big \rangle
        \Big]
    \end{equation*} 
    Integers are closed under addition. 
    Thus, the factor 
    \bm{$\big \langle \mu + \nu \big \rangle$} 
    is an integer. 
    It follows that \bm{$\chi + \zeta$} is even, 
    by the definition for even numbers.
\end{proof}


\end{document}
\sep


% ============================== 0003 Theorem 1603 =================================
\subsection[The square of an even number is even.]{\color{section} Theorem 3 \color{black} : the square of an even number is even.}
\input{../resources/LaTeX/1.6.introduction.to.proofs/0003_1603/0003_1603.tex}
\sep


% ============================== 0004 Theorem 1604 =================================
\subsection[The additive inverse of an even number.]{\color{section} Theorem 4 \color{black} : the additive inverse of an even number.}
\documentclass[preview]{standalone}
\usepackage{amssymb, amsmath, amsthm}
\usepackage{mathtools}
\usepackage{bm}


\newtheorem{theorem}{Theorem}
\renewcommand\qedsymbol{$\blacksquare$}


\begin{document}


\begin{theorem} %[\textbf{1604}]
    The additive inverse of an even number is an even number.
\end{theorem}

\begin{proof}
    Let \bm{$\chi$} be an even number. 
    There exists an integer \bm{$\eta$} such that 
    \bm{$\chi = 2\eta$}, 
    by the definition for even numbers. 
    The additive inverse for \bm{$\chi$} is,
    \begin{equation*}
        -1 \big \langle \chi \big \rangle
            = 
        -1 \big \langle 2 \eta \big \rangle
    \end{equation*} 
    By commutativity of multiplication that is,
    \begin{equation*}
        -1 \big \langle 2 \eta \big \rangle
            = 
        2 \big \langle - \eta \big \rangle
    \end{equation*}
    Since integers are closed under multiplication, 
    the factor \bm{$\big \langle - \eta \big \rangle$} is an integer. 
    It follows that the additive inverse of \bm{$\chi$} is an even number, 
    by the definition for even numbers.
\end{proof}


\end{document} 
\pagebreak


% ============================== 0005 Theorem 1605 =================================
\subsection[A special parity.]{\color{section} Theorem 5 \color{black} : a special parity.}
\input{../resources/LaTeX/1.6.introduction.to.proofs/0005_1605/0005_1605.tex}
\begin{center}
    \includegraphics[width=6cm]{../resources/jpg/1.6.introduction.to.proofs/olympics.jpg}
\end{center}


% ============================== 0006 Theorem 1606 =================================
\subsection[The product of two odd numbers is odd.]{\color{section} Theorem 6 \color{black} : the product of two odd numbers is odd.}
\input{../resources/LaTeX/1.6.introduction.to.proofs/0006_1606/0006_1606.tex}
\pagebreak


% ============================== 0007 Theorem 1608 =================================
\subsection[Two plus a perfect square is not perfect.]{\color{section} Theorem 7 \color{black} : two plus a perfect square is not perfect.}
\input{../resources/LaTeX/1.6.introduction.to.proofs/0007_1608/0007_1608.tex}
\vspace{1.5\baselineskip}
\sep
\pagebreak


% ============================== 0008 Theorem 1609 =================================
\subsection[A sum of irrational and rational numbers.]{\color{section} Theorem 8 \color{black} : a sum of irrational and rational numbers.}
\documentclass[preview]{standalone}
\usepackage{amssymb, amsthm}
\usepackage{mathtools}
\usepackage{bm}


\newtheorem{theorem}{Theorem}
\renewcommand\qedsymbol{$\blacksquare$}


\begin{document}


\begin{theorem} %[\textbf{1609}]
    The sum of an irrational number and a rational number is irrational.
\end{theorem}

\begin{proof}
    By contradiction. Suppose that \bm{$\mu$} and \bm{$\zeta$} are rational numbers, 
    and let \bm{$\chi$} be an irrational number. 
    For the purpose of contradiction, 
    assume the negation of the hypothesis. 
    That is, the proposition 
        $$\lnot p: \emph{the sum of an irrational number and a rational number 
        is rational.}$$
    Hence, \bm{$\chi \textbf{ + } \mu \textbf{ = } \zeta$}, 
    by the assumption \bm{$\lnot p$}. 
    Thus, 
    \bm{$\chi \textbf{ = } \zeta \textbf{ + } \big \langle \textbf{ - } \mu \big \rangle$}, 
    by the additive equality property for equations. 
    But rational numbers are closed under addition by the closure property for rational numbers. 
    So \bm{$\lnot p$} implies \bm{$\chi$} is rational, 
    and \bm{$\chi$} is irrational; a contradiction.
\end{proof}


\end{document}
\begin{figure}[h!]
    \centering
    \includegraphics[width=12.5cm]{../resources/jpg/1.6.introduction.to.proofs/plato-republic.jpg}
    \caption*{\emph{Plato and Aristotle.}}
\end{figure}


% ============================== 0009 Theorem 1610 ================================
\subsection[The product of two rational numbers.]{\color{section} Theorem 9 \color{black} : the product of two rational numbers.}
\input{../resources/LaTeX/1.6.introduction.to.proofs/0009_1610/0009_1610.tex}
\pagebreak


% ============================== 0010 Theorem 1612 ================================
\subsection[An irrational times a rational number.]{\color{section} Theorem 10 \color{black} : an irrational times a rational number.}
\input{../resources/LaTeX/1.6.introduction.to.proofs/0010_1612/0010_1612.tex}
\sep


% ============================== 0011 Theorem 1613 ================================
\subsection[An irrational multiplicative inverse.]{\color{section} Theorem 11 \color{black} : an irrational multiplicative inverse.}
\documentclass[preview]{standalone}
\usepackage{amssymb, amsthm}
\usepackage{mathtools}
\usepackage{bm}


\newtheorem{theorem}{Theorem}
\renewcommand\qedsymbol{$\blacksquare$}


\begin{document}


\begin{theorem} %[\textbf{1613}]
    If \bm{$\chi$} is an irrational number, 
    then \bm{$\frac{1}{\chi}$} is irrational.
\end{theorem}

\begin{proof}
    By the contrapositive. 
    Suppose that \bm{$\frac{1}{\chi}$} is a rational number. 
    By the definition for rational numbers, 
    there exist integers \bm{$\alpha$} and \bm{$\gamma$} such that 
    \bm{$\frac{1}{\chi} = \frac{\alpha}{\gamma}$}. 
    Note that \bm{$\alpha$} is nonzero (because \bm{$\frac{1}{\chi}$} is nonzero.) 
    By the multiplicative property of equality for equations,
    \begin{equation*}
        \Bigg\{
            \left(
                \chi \cdot \frac{1}{\chi}
            \right) 
                = 
            \left(
                \chi \cdot \frac{\alpha}{\gamma}
            \right)
        \Bigg\} 
            \equiv
        \Bigg\{
            \left(
                \frac{\chi}{\chi} \cdot \frac{\gamma}{\alpha}
            \right) 
                = 
            \left(
                \frac{\chi \alpha}{\gamma} \cdot \frac{\gamma}{\alpha}
            \right)
        \Bigg\} 
            \equiv
        \Bigg\{
            \frac{\gamma}{\alpha} 
                = 
            \chi
        \Bigg\}
    \end{equation*}
    \bm{$\frac{\gamma}{\alpha} = \chi$} is rational, 
    by definition. 
    Thus, if \bm{$\frac{1}{\chi}$} is rational, 
    then \bm{$\chi$} is rational.
\end{proof}


\end{document}
\pagebreak


% ============================== 0012 Theorem 1614 ================================
\subsection[A rational multiplicative inverse.]{\color{section} Theorem 12 \color{black} : a rational multiplicative inverse.}
\input{../resources/LaTeX/1.6.introduction.to.proofs/0012_1614/0012_1614.tex}
\begin{figure}[h!]
    \centering
    \includegraphics[width=13cm]{../resources/jpg/1.6.introduction.to.proofs/pythagoras.jpg}
    \caption*{\emph{Pythagoras.}}
\end{figure}


% ============================== 0013 Theorem 1615 ================================
\subsection[A corollary from additive compatibility.]{\color{section} Theorem 13 \color{black} : a corollary from additive compatibility.}
\documentclass[preview]{standalone}
\usepackage{amssymb, amsthm}
\usepackage{mathtools}
\usepackage{bm}


\newtheorem{theorem}{Theorem}
\renewcommand\qedsymbol{$\blacksquare$}


\begin{document}


\begin{theorem} %[\textbf{1615}]
    Let \bm{$\chi$} and \bm{$\zeta$} be real numbers. 
    If \bm{$\chi + \zeta \ge 2$}, 
    then 
    \bm{$
        \big \langle \chi \ge 1 \big \rangle 
            \lor
        \big \langle \zeta \ge 1 \big \rangle
    $}.
\end{theorem}

\begin{proof}
    By the contrapositive.
    Suppose the negation of the consequent: 
    \begin{equation*}
        \big \langle \chi < 1 \big \rangle 
            \land 
        \big \langle \zeta < 1 \big \rangle    
    \end{equation*}    
    By additive compatibility,
    \begin{equation*}
        \Big \langle \chi + \zeta \Big \rangle 
            < 
        \Big \langle 1 + 1 \Big \rangle 
            = 
        \Big \langle 
            2
        \Big \rangle
    \end{equation*}
    This is the logical negation of the direct hypothesis. Thus concludes the proof.
\end{proof}


\end{document}
\pagebreak


% ============================== 0014 Theorem 1616 ================================
\subsection[Divisors of an even number.]{\color{section} Theorem 14 \color{black} : divisors of an even number.}
\input{../resources/LaTeX/1.6.introduction.to.proofs/0014_1616/0014_1616.tex}
%\sep
\begin{figure}[h!]
    \centering
    \includegraphics[width=8.5cm]{../resources/jpg/1.6.introduction.to.proofs/plato.jpg}
    \caption*{\emph{Plato.}}
\end{figure}
\pagebreak


% ============================== 0015 Theorem 1617 ================================
\subsection[\texorpdfstring{Odd integers of the form $\zeta ^3 + 5$.}
    {Odd integers of the form zeta cubed + 5.}
    ]{
        \color{section} Theorem 15 \color{black} : odd integers of the form \bm{$\zeta ^3 + 5$}.
    }
\documentclass[preview]{standalone}
\usepackage{amssymb, amsthm}
\usepackage{mathtools}
\usepackage{bm}


\newtheorem{theorem}{Theorem}
\renewcommand\qedsymbol{$\blacksquare$}


\begin{document}


\begin{theorem} %[\textbf{1617}]
    Let \bm{$\zeta$} be an integer. 
    If \bm{$\zeta^3 + 5$} is odd, 
    then \bm{$\zeta$} is even.
\end{theorem}

\begin{proof}
    By the contrapositive. 
    Suppose that \bm{$\zeta$} were odd. 
    By the definition for odd numbers, 
    there exists an integer \bm{$\gamma$} such that 
    \bm{$\zeta = 2 \gamma + 1$}. 
    By the Binomial Theorem,
    \begin{equation*}
        \Bigg \{
            \big \langle 2 \gamma + 1 \big \rangle ^3 + 5
        \Bigg \}
            = 
        \Bigg \{
            5 
                + 
            \sum\limits_{\iota=0} ^3 {3\choose\iota} 
                2 \gamma ^{\langle 3 - \iota \rangle}
        \Bigg \}
            = 
        \Bigg \{
            2 
            \big \langle 4 \gamma ^3 - 6 \gamma ^2 + 3 \gamma + 3 \big \rangle
        \Bigg \}
    \end{equation*}
    The factor 
    \bm{$\big \langle 4 \gamma ^3 - 6 \gamma ^2 + 3 \gamma + 3 \big \rangle$} 
    is an integer because integers are closed on addition and multiplication. 
    Thus, \bm{$\zeta^3 + 5$} is even, by definition.
\end{proof}


\end{document}
\begin{figure}[h!]
    \centering
    \includegraphics[width=13.25cm]{../resources/jpg/1.6.introduction.to.proofs/socrates.jpg}
    \caption*{\emph{Socrates with hemlock.}}
\end{figure}


% ============================== 0016 Theorem 1618 ================================
\subsection[\texorpdfstring{Even numbers of the form $3 \gamma + 2$}
        {Even numbers of the form $3 gamma + 2$}
    ]{
        \color{section} Theorem 16 \color{black} : even numbers of the form \bm{$3 \gamma + 2$}.
    }
\input{../resources/LaTeX/1.6.introduction.to.proofs/0016_1618/0016_1618.tex}
\pagebreak


% ============================== 0017 Theorem 1625 ================================
\subsection[\texorpdfstring{$\rho$ does not exist.}
        {Rho does not exist.}
    ]{
        \color{section} Theorem 17 \color{black} : $\bm{\rho}$ does not exist.
    }
\input{../resources/LaTeX/1.6.introduction.to.proofs/0017_1625/0017_1625.tex}
\sep
\begin{figure}[h!]
    \centering
    \includegraphics[width=10cm]{../resources/jpg/1.6.introduction.to.proofs/know_thyself.jpg}
    \caption*{\emph{Know thyself.}}
\end{figure} 
\pagebreak
\thispagestyle{empty}


\end{document}