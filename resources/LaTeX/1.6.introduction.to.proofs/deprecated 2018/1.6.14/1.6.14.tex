\documentclass[a4paper, 12pt]{article}
\usepackage[utf8]{inputenc}
\usepackage[english]{babel}
\usepackage{amssymb, amsmath, amsthm}
\theoremstyle{plain}
\newtheorem*{theorem*}{Theorem}
\newtheorem{theorem}{Theorem}

\usepackage{mathtools}
\renewcommand\qedsymbol{$\blacksquare$}

\begin{document}
	
	\begin{theorem*}[1.6.14]
		If x is a rational number and x $\neq$ 0, then $\frac{1}{x}$ is rational.
	\end{theorem*}
	
	\begin{proof}
		It is trivial to express $x$ as $x = \frac{x}{1}$. Since $x$ is rational, by the definition of 
		ration numbers there exist integers $a$ and $b$ such that $\frac{x}{1} = \frac{a}{b}$. By 
		equivalence we have $\frac{b}{a} = \frac{1}{x}$, so $\frac{1}{x}$ is rational by definition 
		whenever $x$ is a nonzero rational number.
	\end{proof}

\end{document}
