\documentclass[a4paper, 12pt]{article}
\usepackage[utf8]{inputenc}
\usepackage[english]{babel}
\usepackage{amssymb, amsmath, amsthm}
\theoremstyle{plain}
\newtheorem*{theorem*}{Theorem}
\newtheorem{theorem}{Theorem}

\usepackage{mathtools}
\renewcommand\qedsymbol{$\blacksquare$}

\begin{document}
	
	\begin{theorem*}[1.6.17]
		Let n be an integer. If $n^{3}$ + 5 is odd, then n is even.
	\end{theorem*}
	
	\begin{proof}
		By the contrapositive. Suppose that $n$ is odd. By definition there exists and integer $k$ such 
		that $n = 2k + 1$. By the Binomial Theorem, 
		$(2k + 1)^{3} + 5 = 5 + \sum\limits_{i=0}^3 {3 \choose i} 2k^{(3-i)} = 2(4k^{3} - 6k^{2} + 3k + 3)$. 
		That is an integer factor with a coefficient of 2, even by definition.
	\end{proof}

\end{document}
