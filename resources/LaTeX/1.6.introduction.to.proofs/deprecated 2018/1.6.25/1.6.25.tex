\documentclass[a4paper, 12pt]{article}
\usepackage[utf8]{inputenc}
\usepackage[english]{babel}
\usepackage{amssymb, amsmath, amsthm}
\theoremstyle{plain}
\newtheorem*{theorem*}{Theorem}
\newtheorem{theorem}{Theorem}

\usepackage{mathtools}
\renewcommand\qedsymbol{$\blacksquare$}

\begin{document}
	
	\begin{theorem*}[1.6.25]
		There does not exist a rational number r such that \newline $r^{3}$ + r + 1 = 0.
	\end{theorem*}
	
	\begin{proof}
		By contradiction. Assume that there exists a rational number $r$ \newline satisfying the 
		equation $r^{3} + r + 1 = 0$. By definition there exist integers $a$ and $b$ 
		($b$ is nonzero,) such that 
		$\frac{a^{3}}{b^{3}} + \frac{a}{b} + 1 = a^{3} + ab^{2} + b^{3} = 0$. 
		Clearly $a^{3} = -(ab^{2} + b^{3})$ and $b^{3} = -(a^{3} + ab^{2})$. So we have 
		$-(ab^{2} + b^{3}) + ab^{2} - (a^{3} + ab^{2}) = 0$. Simplifying we find that 
		$-a^{3} -ab^{2} - b^{3} = a^{3} + ab^{2} + b^{3}$. This can only happen when $b = 0$, but 
		$b = 0$ is a contradiction because $b$ is a divisor in $r$.
	\end{proof}

\end{document}
