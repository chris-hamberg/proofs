\documentclass[preview]{standalone}
\usepackage{amssymb, amsthm}
\usepackage{mathtools}
\usepackage{bm}


\newtheorem{lemma}{Lemma}
\renewcommand\qedsymbol{$\blacksquare$}


\begin{document}


\begin{lemma}%[\textbf{2201}]
    Let \bm{$\mathrm{A}$}, and \bm{$\Lambda$} be sets.
    \begin{equation*}
        \bm{
            \mathrm{A} \oplus \Lambda
                =
            \Big \langle \mathrm{A} \cup \Lambda \Big \rangle
                \cap
            \Big \langle 
                \overline{\mathrm{A}} 
                    \cup 
                \overline{\Lambda} 
            \Big \rangle
        }
    \end{equation*}
\end{lemma}
\begin{proof}
    By Theorem 52,
    \bm{$
        \big \langle \mathrm{A} \oplus \Lambda \big \rangle
            \equiv
        \big \langle \mathrm{A} \cup \Lambda \big \rangle
            -
        \big \langle \mathrm{A} \cap \Lambda \big \rangle
    $}, 
    and by Theorem 45, that is
    \bm{$
        \big \langle \mathrm{A} \cup \Lambda \big \rangle
            \cap
        \big \langle \overline{\mathrm{A} \cap \Lambda} \big \rangle
    $}.
    By Demorgans law for the complement of set intersection,
    \begin{equation*}
        \Big \langle \mathrm{A} \cup \Lambda \Big \rangle
            \cap
        \Big \langle \overline{\mathrm{A} \cap \Lambda} \Big \rangle
            \equiv
        \Big \langle \mathrm{A} \cup \Lambda \Big \rangle
            \cap
        \Big \langle \overline{\mathrm{A}} \cup \overline{\Lambda} \Big \rangle
    \end{equation*}
    $\therefore \text{\space} \bm{
    \mathrm{A} \oplus \Lambda
        =
    \big \langle \mathrm{A} \cup \Lambda \big \rangle
        \cap
    \big \langle 
        \overline{\mathrm{A}} 
            \cup 
        \overline{\Lambda} 
    \big \rangle}$
\end{proof}


\end{document}