\documentclass[preview]{standalone}
\usepackage{amssymb, amsthm}
\usepackage{mathtools}
\usepackage{bm}


\newtheorem{lemma}{Lemma}
\renewcommand\qedsymbol{$\blacksquare$}


\begin{document}


\begin{lemma} %[\textbf{2205}]
    Let \bm{$\Gamma$}, \bm{$\Pi$}, and \bm{$\Xi$} be sets.
    \begin{equation*}
        \bm{
            \overline{
                \Big \langle \Gamma \cup \Pi \Big \rangle
                    \cap
                \Big \langle \overline{\Gamma} \cup \overline{\Pi} \Big \rangle
            }
                \cap
            \Xi
                    =
            \Big \langle \overline{\Gamma} \cap \overline{\Pi} \cap \Xi \Big \rangle
                \cup
            \Big \langle \Gamma \cap \Pi \cap \Xi \Big \rangle
        }
    \end{equation*}
\end{lemma}
\begin{proof}
    By DeMorgans Law for sets, 
    the expression occurring in the left-hand side of the equation in the lemma is
    \begin{equation*}
        \bigg[
            \Big \langle \overline{\Gamma \cup \Pi} \Big \rangle
                \cup
            \Big \langle \overline{\overline{\Gamma} \cup \overline{\Pi}} \Big \rangle
        \bigg]
            \cap
        \Xi
    \end{equation*}
    Which, again by DeMorgans law for sets, and by the complementation law for sets, is equivalent to
    \begin{equation*}
        \bigg[
            \Big \langle \overline{\Gamma} \cap \overline{\Pi} \Big \rangle
                \cup
            \Big \langle \Gamma \cap \Pi \Big \rangle
        \bigg]
            \cap
        \Xi
    \end{equation*}
    By the distributive law for set intersection over set union,
    and by associative law for the intersection of sets, that is
    \begin{equation*}
        \Big \langle \overline{\Gamma} \cap \overline{\Pi} \cap \Xi \Big \rangle
            \cup
        \Big \langle \Gamma \cap \Pi \cap \Xi \Big \rangle
    \end{equation*}
    $\therefore \bm{
        \overline{
            \big \langle \Gamma \cup \Pi \big \rangle
                \cap
            \big \langle \overline{\Gamma} \cup \overline{\Pi} \big \rangle
        }
            \cap
        \Xi
            =
        \big \langle \overline{\Gamma} \cap \overline{\Pi} \cap \Xi \big \rangle
            \cup
        \big \langle \Gamma \cap \Pi \cap \Xi \big \rangle
    }$.
\end{proof}


\end{document}