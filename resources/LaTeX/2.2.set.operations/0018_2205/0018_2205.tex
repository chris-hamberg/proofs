\documentclass[preview]{standalone}
\usepackage{amssymb, amsthm}
\usepackage{mathtools}
\usepackage{bm}


\newtheorem{theorem}{Theorem}
\renewcommand\qedsymbol{$\blacksquare$}


\begin{document}


\begin{theorem} %[\textbf{2205}] \color{black}
    Let \bm{$\Lambda$} be a subset of \space \bm{$\Omega$}.
    \bm{$\overline{\overline{\Lambda}} = \Lambda$}.
\end{theorem}
\begin{proof} \color{black}
    Suppose there exists an element \bm{$\chi$} such that \bm{$\chi$} is a member of 
    \bm{$\overline{\overline{\Lambda}}$}. 
    By the definition for set complementation, 
    and by the defintion for set membership, 
    that is 
    \begin{equation*}
        \Big \langle \chi \in \overline{\overline{\Lambda}} \Big \rangle 
            \equiv
        \Big \langle \chi \notin \overline{\Lambda} \Big \rangle 
            \equiv
        \lnot \Big \langle \chi \in \overline{\Lambda} \Big \rangle 
            \equiv
        \lnot \Big \langle \chi \notin \Lambda \Big \rangle 
            \equiv
        \lnot \Big \langle \lnot \big \langle \chi \in \Lambda \big \rangle \Big \rangle
    \end{equation*}
    By the logical law of double negation, $\bm{\chi \in \Lambda}$.
    Since logical equivalence is biconditional by definition, 
    this sequence of equivalencies proves both, that 
    \begin{equation*}
        \Big \langle 
            \overline{\overline{\Lambda}}
                \subseteq 
            \Lambda
        \Big \rangle 
            \land 
        \Big \langle 
            \Lambda
                \subseteq 
            \overline{\overline{\Lambda}}
        \Big \rangle
    \end{equation*}
    $\therefore \text{\space} \bm{\overline{\overline{\Lambda}} = \Lambda$}; 
    the complementation law for sets.
\end{proof}


\end{document}