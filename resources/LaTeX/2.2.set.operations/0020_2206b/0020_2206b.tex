\documentclass[preview]{standalone}
\usepackage{amssymb, amsthm}
\usepackage{mathtools}
\usepackage{bm}


\newtheorem*{theorem*}{Theorem}
\renewcommand\qedsymbol{$\blacksquare$}


\begin{document}


\begin{theorem} %[\textbf{2206b}]
    Let \bm{$\Xi$} be a set with universal set \bm{$\Omega$}. 
    The set identity for \bm{$\Xi$} is \bm{$\Xi \cap \Omega = \Xi$}.
\end{theorem}
\begin{proof}
    Suppose there exists an element \bm{$\zeta$} such that \bm{$\zeta$} is a member of \bm{$\Xi \cap \Omega$}. 
    By the definition for set intersection, that is
    \begin{equation*}
        \Big \langle \zeta \in \Xi \Big \rangle 
            \land 
        \Big \langle \zeta \in \Omega \Big \rangle
    \end{equation*}
    The logical identity for the statement \bm{$\zeta \in \Omega$} is trivially \bm{$\top$}, 
    because \bm{$\Omega$} is the universe. 
    Thus, by that identity, and by the identity law for logical conjunction, 
    \begin{equation*}
        \Bigg\{
            \Big \langle \zeta \in \Xi \Big \rangle 
                \land 
            \Big \langle \zeta \in \Omega \Big \rangle 
        \Bigg\} 
            \equiv
        \Bigg\{
            \Big \langle \zeta \in \Xi \Big \rangle 
                \land 
            \Big \langle \top \Big \rangle 
                \equiv 
            \Big \langle \zeta \in \Xi \Big \rangle
        \Bigg\}
            \equiv
    \end{equation*}
    \begin{equation*}
        \Bigg\{
            \Big \langle \zeta \in \Xi \Big \rangle 
                \land 
            \Big \langle \zeta \in \Omega \Big \rangle
                \equiv
            \Big \langle \zeta \in \Xi \Big \rangle
        \Bigg\}
    \end{equation*}
    $\therefore$ by the definition for the intersection of sets,
    the set identity for \bm{$\Xi$} is \bm{$\Xi \cap \Omega = \Xi$}.
\end{proof}


\end{document}