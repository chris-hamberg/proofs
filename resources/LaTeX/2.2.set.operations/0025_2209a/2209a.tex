\documentclass[preview]{standalone}
\usepackage{amssymb, amsthm}
\usepackage{mathtools}
\usepackage{bm}


\newtheorem{theorem}{Theorem}
\renewcommand\qedsymbol{$\blacksquare$}


\begin{document}


\begin{theorem}[\textbf{2209a}]
    Let \bm{$\Psi$} be a set with universal set \bm{$\Omega$}.
    \bm{$\Psi \cup \overline{\Psi} = \Omega$}.
\end{theorem}
\begin{proof}
    Let \bm{$\sigma$} be an element in \bm{$\Psi \cup \overline{\Psi}$}. 
    By the definition for set union,  
    \begin{equation*}
        \Big \langle \sigma \in \Psi \Big \rangle 
            \lor 
        \Big \langle \sigma \in \overline{\Psi} \Big \rangle
    \end{equation*}
    The right-hand side of this disjunction is equivalent to \bm{$\sigma \in \Omega - \Psi$}, 
    by the definition for set complementation. 
    By Theorem 45, \bm{$\sigma \in \Omega \cap \overline{\Psi}$}, 
    which is defined as 
    \bm{$\big \langle \sigma \in \Omega \big \rangle 
        \land 
    \big \langle \sigma \notin \Psi \big \rangle$},
    by the definitions for set intersection and set complementation. 
    Thus, the original disjunction is the same as 
    \begin{equation*}
        \Big \langle \sigma \in \Psi \Big \rangle 
            \lor 
        \Bigg[
            \Big \langle \sigma \in \Omega \Big \rangle 
                \land 
            \Big \langle \sigma \notin \Psi \Big \rangle
        \Bigg]
    \end{equation*}
    We must distribute the left-hand side of this disjunction over the conjunction occurring in the right-hand side. 
    We get 
    \begin{equation*}
        \Bigg[
            \Big \langle \sigma \in \Psi \Big \rangle 
                \lor 
            \Big \langle \sigma \in \Omega \Big \rangle
        \Bigg] 
            \land 
        \Bigg[
            \Big \langle \sigma \in \Psi \Big \rangle 
                \lor 
            \Big \langle \sigma \notin \Psi \Big \rangle
        \Bigg]
    \end{equation*}
    By the logical law of negation, 
    the identity for the right-hand side of this conjunction is \bm{$\top$}. 
    The left-hand side of this conjunction is dominated by \bm{$\Omega$}, 
    according to Theorem 21. 
    Therefore, 
    the statement \bm{$\sigma \in \Psi \cup \overline{\Psi}$} can be equivalently stated as 
    \bm{$\big \langle \sigma \in \Omega \big \rangle \land \top$}; 
    the logical identity for which is \bm{$\sigma \in \Omega$}. 
    The converse trivially follows from the fact of logical equivalence. 
    Thus, 
    proves the set complement law for the union of sets, 
    \bm{$\Psi \cup \overline{\Psi} = \Omega$}.
\end{proof}


\end{document}