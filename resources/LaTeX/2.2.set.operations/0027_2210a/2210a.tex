\documentclass[preview]{standalone}
\usepackage{amssymb, amsthm}
\usepackage{mathtools}
\usepackage{bm}


\newtheorem{theorem}{Theorem}
\renewcommand\qedsymbol{$\blacksquare$}


\begin{document}


\begin{theorem}[\textbf{2210a}]
    Let \bm{$\Xi$} be a set. 
    \bm{$\Xi - \varnothing = \Xi$}.
\end{theorem}
\begin{proof}
    Suppose there exists an element \bm{$\zeta$} such that \bm{$\zeta$} is a member of 
    \bm{$\Xi - \varnothing$}.
    By the definition for set difference, that is
    \begin{equation*}
        \Big \langle \zeta \in \Xi \Big \rangle
            \land 
        \Big \langle \zeta \notin \varnothing \Big \rangle
    \end{equation*}
    It is trivial that the logical identity for the statement 
    \bm{$\zeta \notin \varnothing$} is \bm{$\top$},
    since the empty set contains no members. 
    Thus, by that identity, 
    and by the identity law for logical conjunction,
    \begin{equation*}
        \Bigg\{
            \Big \langle \zeta \in \Xi \Big \rangle
                \land 
            \Big \langle \zeta \notin \varnothing \Big \rangle
        \Bigg\}
            \equiv
        \Bigg\{
            \Big \langle \zeta \in \Xi \Big \rangle
                \land 
            \Big \langle \top \Big \rangle
                \equiv
            \Big \langle \zeta \in \Xi \Big \rangle
        \Bigg\}
            \equiv
    \end{equation*}
    \begin{equation*}
        \Bigg\{
            \Big \langle \zeta \in \Xi \Big \rangle
                \land 
            \Big \langle \zeta \notin \varnothing \Big \rangle
                \equiv
            \Big \langle \zeta \in \Xi \Big \rangle
        \Bigg\}
    \end{equation*}
    $\therefore$ by the definition for set difference,
    \bm{$\Xi - \varnothing = \Xi$}.
\end{proof}


\end{document}