\documentclass[preview]{standalone}
\usepackage{amssymb, amsthm}
\usepackage{mathtools}
\usepackage{bm}


\newtheorem{theorem}{Theorem}
\renewcommand\qedsymbol{$\blacksquare$}


\begin{document}


\begin{theorem}[\textbf{2210b}]
    Let \bm{$\Xi$} be a set. 
    \bm{$\varnothing - \Xi = \varnothing$}.
\end{theorem}
\begin{proof}
    Let \bm{$\zeta$} be an element in \bm{$\varnothing - \Xi$}. 
    By the definition for set difference,
    \begin{equation*}
        \Big \langle \zeta \in \varnothing \Big \rangle
            \land 
        \Big \langle \zeta \notin \Xi \Big \rangle
    \end{equation*}
    The logical identity for the statement \bm{$\zeta \in \varnothing$} is trivially \bm{$\bot$}, 
    since the empty set contains no members. 
    Thus, by that identity, and by the domination law for logical conjunction,
    \begin{equation*}
        \Bigg\{
            \Big \langle \zeta \in \varnothing \Big \rangle
                \land 
            \Big \langle \zeta \notin \Xi \Big \rangle
        \Bigg\}
            \equiv
        \Bigg\{
            \Big \langle \bot \Big \rangle
                \land 
            \Big \langle \zeta \notin \Xi \Big \rangle 
                \equiv 
            \Big \langle \bot \Big \rangle
        \Bigg\}
            \equiv
    \end{equation*}
    \begin{equation*}
        \Bigg\{
            \Big \langle \zeta \in \varnothing \Big \rangle
                \land 
            \Big \langle \zeta \notin \Xi \Big \rangle
                \equiv
            \Big \langle \zeta \in \varnothing \Big \rangle
        \Bigg\}
    \end{equation*}
    $\therefore$ by the definition for set difference,  
    \bm{$\varnothing - \Xi = \varnothing$}
\end{proof}


\end{document}