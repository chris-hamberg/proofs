\documentclass[preview]{standalone}
\usepackage{amssymb, amsthm}
\usepackage{mathtools}
\usepackage{bm}


\newtheorem{theorem}{Theorem}
\renewcommand\qedsymbol{$\blacksquare$}


\begin{document}


\begin{theorem}[\textbf{2211a}]
    Let \bm{$\mathrm{A}$} and \bm{$\Lambda$} be sets. 
    The union of \bm{$\mathrm{A}$} and \bm{$\Lambda$} is commutative.
\end{theorem}
\begin{proof}
    Let \bm{$\lambda$} be an element in \bm{$\mathrm{A} \cup \Lambda$}. 
    By the definition for set union,
    \begin{equation*}
        \Big \langle \lambda \in \mathrm{A} \Big \rangle 
            \lor 
        \Big \langle \lambda \in \Lambda \Big \rangle
    \end{equation*} 
    Because logical disjunction is commutative, 
    that is 
    \begin{equation*}
        \Bigg[
            \Big \langle \lambda \in \mathrm{A} \Big \rangle
                \lor 
            \Big \langle \lambda \in \Lambda \Big \rangle
        \Bigg]
            \equiv 
        \Bigg[
            \Big \langle \lambda \in \Lambda \Big \rangle
                \lor 
            \Big \langle \lambda \in \mathrm{A} \Big \rangle
        \Bigg]
    \end{equation*} 
    $\therefore \text{\space} 
    \bm{\mathrm{A} \cup \Lambda = \Lambda \cup \mathrm{A}$}, 
    and the union of \bm{$\mathrm{A}$} and \bm{$\Lambda$} is indeed commutative.
\end{proof}


\end{document}