\documentclass[preview]{standalone}
\usepackage{amssymb, amsthm}
\usepackage{mathtools}
\usepackage{bm}


\newtheorem{theorem}{Theorem}
\renewcommand\qedsymbol{$\blacksquare$}


\begin{document}


\begin{theorem}[\textbf{2218b}]
    Let \bm{$\mathrm{A}$}, \bm{$\Lambda$} and \bm{$\Delta$} be sets. 
    \bm{$
    \big \langle \mathrm{A} \cap \Lambda \cap \Delta \big \rangle
        \subseteq 
    \big \langle \mathrm{A} \cap \Lambda \big \rangle
    $}.
\end{theorem}
\begin{proof}
    Let \bm{$\lambda$} be an element in \bm{$\mathrm{A} \cap \Lambda \cap \Delta$}. 
    By the definition for set intersection, that is
    \begin{equation*}
        \Big \langle \lambda \in \mathrm{A} \Big \rangle
            \land 
        \Big \langle \lambda \in \Lambda \Big \rangle
            \land 
        \Big \langle \lambda \in \Delta \Big \rangle
    \end{equation*} 
    By the simplification rule of inference,
    \begin{equation*}
        \Big \langle \lambda \in \mathrm{A} \Big \rangle
            \land 
        \Big \langle \lambda \in \Lambda \Big \rangle
            \land 
        \Big \langle \lambda \in \Delta \Big \rangle
            \rightarrow
        \Big \langle \lambda \in \mathrm{A} \Big \rangle
            \land 
        \Big \langle \lambda \in \Lambda \Big \rangle
    \end{equation*} 
    $\therefore \text{\space} \bm{
    \big \langle \mathrm{A} \cap \Lambda \cap \Delta \big \rangle
        \subseteq 
    \big \langle \mathrm{A} \cap \Lambda \big \rangle
    }$,
    by the defintions for set intersection and subsets.
\end{proof}


\end{document}