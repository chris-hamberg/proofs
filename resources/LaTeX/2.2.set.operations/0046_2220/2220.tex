\documentclass[preview]{standalone}
\usepackage{amssymb, amsthm}
\usepackage{mathtools}
\usepackage{bm}


\newtheorem{theorem}{Theorem}
\renewcommand\qedsymbol{$\blacksquare$}


\begin{document}


\begin{theorem}[\textbf{2220}]
    Let \bm{$\mathrm{A}$}, and \bm{$\Lambda$} be sets. 
    \bm{$
    \big \langle \mathrm{A} \cap \Lambda \big \rangle 
        \cup 
    \big \langle \mathrm{A} \cap \overline{\Lambda} \big \rangle 
        = 
    \mathrm{A}
    $}.
\end{theorem}
\begin{proof}
    Let \bm{$\lambda$} be an element in 
    \bm{$
    \big \langle \mathrm{A} \cap \Lambda \big \rangle 
        \cup 
    \big \langle \mathrm{A} \cap \overline{\Lambda} \big \rangle
    $}.
    By the definitions for the union of sets, and set intersection, that is,
    \begin{equation*}
        \bigg[
            \Big \langle \lambda \in \mathrm{A} \Big \rangle
                \land
            \Big \langle \lambda \in \Lambda \Big \rangle
        \bigg]
            \lor
        \bigg[
            \Big \langle \lambda \in \mathrm{A} \Big \rangle
                \land
            \Big \langle \lambda \in \overline{\Lambda} \Big \rangle
        \bigg]
    \end{equation*} 
    By the law of distribution for logical conjunction over disjunction,
    we can factor out the term \bm{$\lambda \in \mathrm{A}$}. 
    Hence, the following statement is equivalent,
    \begin{equation*}
        \Big \langle \lambda \in \mathrm{A} \Big \rangle
            \land
        \bigg[
            \Big \langle \lambda \in \Lambda \Big \rangle
                \lor
            \Big \langle \lambda \in \overline{\Lambda} \Big \rangle
        \bigg]
    \end{equation*}
    \bm{$
    \lambda \in \Lambda 
        \lor 
    \lambda \in \overline{\Lambda} 
        \equiv 
    \top
    $},
    by the negation laws of logic.
    Thus, by the identity law for logical conjunction,
    \begin{equation*}
        \Bigg\{
            \bigg[
                \Big \langle \lambda \in \mathrm{A} \Big \rangle
                    \land
                \Big \langle \lambda \in \Lambda \Big \rangle
            \bigg]
                \lor
            \bigg[
                \Big \langle \lambda \in \mathrm{A} \Big \rangle
                    \land
                \Big \langle \lambda \in \overline{\Lambda} \Big \rangle
            \bigg]
        \Bigg\}
            \equiv
        \Bigg\{
            \Big \langle \lambda \in \mathrm{A} \Big \rangle
                \land
            \Big \langle \top \Big \rangle
        \Bigg\}
            \equiv
    \end{equation*}
    \begin{equation*}
        \Big \langle \lambda \in \mathrm{A} \Big \rangle
    \end{equation*}
    $
    \therefore \text{\space} \bm{
    \big \langle \mathrm{A} \cap \Lambda \big \rangle 
        \cup 
    \big \langle \mathrm{A} \cap \overline{\Lambda} \big \rangle
        = 
    \mathrm{A}
    }$.
\end{proof}


\end{document}