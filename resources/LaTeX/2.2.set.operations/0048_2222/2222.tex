\documentclass[preview]{standalone}
\usepackage{amssymb, amsthm}
\usepackage{mathtools}
\usepackage{bm}


\newtheorem{theorem}{Theorem}
\renewcommand\qedsymbol{$\blacksquare$}


\begin{document}


\begin{theorem}[\textbf{2222}]
    Let \bm{$\mathrm{A}$}, \bm{$\Lambda$}, and \bm{$\Delta$} be sets. 
    \bm{$
    \mathrm{A} 
        \cap 
    \big \langle \Lambda \cap \Delta \big \rangle 
        = 
    \big \langle \mathrm{A} \cap \Lambda \big \rangle
        \cap 
    \Delta
    $}, 
    such that set intersection is associative.
\end{theorem}
\begin{proof}
    Let \bm{$\lambda$} be an element in 
    \bm{$
    \mathrm{A} 
        \cap 
    \big \langle \Lambda \cap \Delta \big \rangle
    $}. 
    By the definition for the intersection of sets, that is 
    \begin{equation*}
        \Big \langle \lambda \in \mathrm{A} \Big \rangle
            \land
        \bigg[
            \Big \langle \lambda \in \Lambda \Big \rangle
                \land
            \Big \langle \lambda \in \Delta \Big \rangle
        \bigg]
    \end{equation*}
    It trivially follows from the associative law for logical conjunction that
    \begin{equation*}
        \Big \langle \lambda \in \mathrm{A} \Big \rangle
            \land
        \bigg[
            \Big \langle \lambda \in \Lambda \Big \rangle
                \land
            \Big \langle \lambda \in \Delta \Big \rangle
        \bigg]
            \equiv
        \bigg[
            \Big \langle \lambda \in \mathrm{A} \Big \rangle
                \land
            \Big \langle \lambda \in \Lambda \Big \rangle
        \bigg]
            \land
        \Big \langle \lambda \in \Delta \Big \rangle
    \end{equation*}
    $
    \therefore \text{\space} \bm{
    \mathrm{A} 
        \cap 
    \big \langle \Lambda \cap \Delta \big \rangle 
        = 
    \big \langle \mathrm{A} \cap \Lambda \big \rangle
        \cap 
    \Delta
    }$,
    such that set intersection is associative,
    by the definition for the intersection of sets.
\end{proof}


\end{document}