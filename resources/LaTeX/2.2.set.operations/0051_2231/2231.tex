\documentclass[preview]{standalone}
\usepackage{amssymb, amsthm}
\usepackage{mathtools}
\usepackage{bm}


\newtheorem{theorem}{Theorem}
\renewcommand\qedsymbol{$\blacksquare$}


\begin{document}


\begin{theorem}[\textbf{2231}]
    Let \bm{$\mathrm{A}$}, and \bm{$\Lambda$} be subsets of a universal set \bm{$\Omega$}. 
    \begin{equation*}
        \bm{
            \mathrm{A} \subseteq \Lambda
                \text{ if and only if } 
            \overline{\Lambda} \subseteq \overline{\mathrm{A}}
            }
    \end{equation*}
\end{theorem}
\begin{proof}
    The proposition \bm{$\mathrm{A} \subseteq \Lambda$} is defined by the universal quantification% statement 
    \begin{equation*}
        \forall \lambda \Big \langle \lambda \in \mathrm{A} \rightarrow \lambda \in \Lambda \Big \rangle
    \end{equation*}
    Where \bm{$\lambda$} is an element in the domain of discourse \bm{$\Omega$}.
    It is a tautology that the truth value for the predicate is equivalent to its contrapositive. 
    Thus,  
    \begin{equation*}
        \forall \lambda \Big \langle \lambda \notin \Lambda \rightarrow \lambda \notin \mathrm{A} \Big \rangle
    \end{equation*} 
    By the definition for set complementation, and subsets
    \bm{$\lambda \in \overline{\Lambda} \subseteq \overline{\mathrm{A}}$} 
    $\therefore$ \bm{$\mathrm{A} \subseteq \Lambda$} \textbf{\emph{ iff }} \bm{$\overline{\Lambda} \subseteq \overline{\mathrm{A}}$}.
\end{proof}


\end{document}