\documentclass[preview]{standalone}
\usepackage{amssymb, amsthm}
\usepackage{mathtools}
\usepackage{bm}


\newtheorem{theorem}{Theorem}
\renewcommand\qedsymbol{$\blacksquare$}


\begin{document}


\begin{theorem}[\textbf{2235}]
    Let \bm{$\mathrm{A}$}, and \bm{$\Lambda$} be sets. 
    \bm{$
    \mathrm{A} \oplus \Lambda 
        = 
    \big \langle \mathrm{A} \cup \Lambda \big \rangle 
        - 
    \big \langle \mathrm{A} \cap \Lambda \big \rangle
    $}.
\end{theorem}
\begin{proof}
    Let \bm{$\lambda$} be an element in \bm{$\mathrm{A} \oplus \Lambda$}. 
    By the definition for the symmetric difference of sets,
    \begin{equation*}
        \bigg[
            \Big \langle \lambda \in \mathrm{A} \Big \rangle 
                \land 
            \Big \langle \lambda \notin \Lambda \Big \rangle
        \bigg] 
            \lor 
        \bigg[
            \Big \langle \lambda \notin \mathrm{A} \Big \rangle
                \land 
            \Big \langle \lambda \in \Lambda \Big \rangle
        \bigg]
    \end{equation*}
    Distributing the right-hand side over the left-hand side,
    by the distributive laws of logic, that is
    \begin{equation*}
        \Bigg\{
            \Big \langle \lambda \in \mathrm{A} \Big \rangle
                \lor
            \bigg[
                \Big \langle \lambda \notin \mathrm{A} \Big \rangle
                    \land 
                \Big \langle \lambda \in \Lambda \Big \rangle
            \bigg]
        \Bigg\}
            \land
        \Bigg\{
            \Big \langle \lambda \notin \Lambda \Big \rangle
                \lor
            \bigg[
                \Big \langle \lambda \notin \mathrm{A} \Big \rangle
                    \land 
                \Big \langle \lambda \in \Lambda \Big \rangle
            \bigg]
        \Bigg\}
    \end{equation*}
    Again, by the distributive law for logical disjunction over conjunction,
    and by the associative law for logical conjunction, we have
    \begin{equation*}
        \bigg[
            \Big \langle \lambda \in \mathrm{A} \Big \rangle
                \lor
            \Big \langle \lambda \notin \mathrm{A} \Big \rangle
        \bigg]
            \land
        \bigg[
            \Big \langle \lambda \in \mathrm{A} \Big \rangle
                \lor 
            \Big \langle \lambda \in \Lambda \Big \rangle
        \bigg]
            \land
        \bigg[
            \Big \langle \lambda \notin \Lambda \Big \rangle
                \lor
            \Big \langle \lambda \notin \mathrm{A} \Big \rangle
        \bigg]
            \land
    \end{equation*}
    \begin{equation*}
        \bigg[
            \Big \langle \lambda \notin \Lambda \Big \rangle
                \lor 
            \Big \langle \lambda \in \Lambda \Big \rangle
        \bigg]
    \end{equation*}
    The following identity is given by the negation laws of logic,
    \begin{equation*}
        \big \langle \top \big \rangle
            \land
        \bigg[
            \Big \langle \lambda \in \mathrm{A} \Big \rangle
                \lor 
            \Big \langle \lambda \in \Lambda \Big \rangle
        \bigg]
            \land
        \bigg[
            \Big \langle \lambda \notin \Lambda \Big \rangle
                \lor
            \Big \langle \lambda \notin \mathrm{A} \Big \rangle
        \bigg]
            \land
        \big \langle \top \big \rangle
    \end{equation*}
    By DeMorgans laws,
    and by the identity law for logical conjunction, 
    that is
    \begin{equation*}
        \bigg[
            \Big \langle \lambda \in \mathrm{A} \Big \rangle
                \lor 
            \Big \langle \lambda \in \Lambda \Big \rangle
        \bigg]
            \land
        \lnot \bigg[
            \Big \langle \lambda \in \Lambda \Big \rangle
                \land
            \Big \langle \lambda \in \mathrm{A} \Big \rangle
        \bigg]
    \end{equation*}
    Which, by the definitions for set union, set intersection,
    and set membership, is equivalent to
    \begin{equation*}
        \Bigg\{
            \lambda \in \Big \langle \mathrm{A} \cup \Lambda \Big \rangle
                \land
            \lnot \bigg[ 
                \lambda \in \Big \langle \Lambda \cap \mathrm{A} \Big \rangle
            \bigg]
        \Bigg\}
            \equiv
        \Bigg\{
            \lambda \in \Big \langle \mathrm{A} \cup \Lambda \Big \rangle
                \land
            \lambda \notin \Big \langle \Lambda \cap \mathrm{A} \Big \rangle
        \Bigg\}
    \end{equation*}
    $\therefore$ by the definition for set difference,
    \bm{$
    \mathrm{A} \oplus \Lambda 
        = 
    \big \langle \mathrm{A} \cup \Lambda \big \rangle 
        - 
    \big \langle \mathrm{A} \cap \Lambda \big \rangle
    $}.
\end{proof}


\end{document}