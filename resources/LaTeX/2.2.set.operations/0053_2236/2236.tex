\documentclass[preview]{standalone}
\usepackage{amssymb, amsthm}
\usepackage{mathtools}
\usepackage{bm}


\newtheorem{theorem}{Theorem}
\renewcommand\qedsymbol{$\blacksquare$}


\begin{document}


\begin{theorem}[\textbf{2236}]
    Let \bm{$\Gamma$}, and \bm{$\Xi$} be sets. 
    \begin{equation*}
        \bm{
            \Gamma \oplus \Xi 
                = 
            \Big \langle \Gamma - \Xi \Big \rangle
                \cup 
            \Big \langle \Xi - \Gamma \Big \rangle
        }
    \end{equation*}
\end{theorem}
\begin{proof}
    Suppose there exists an element \bm{$\zeta$} such that \bm{$\zeta$} is a member of \bm{$\Gamma \oplus \Xi$}. 
    By the definition for symmetric difference,
    \begin{equation*}
        \bigg[
            \Big \langle \zeta \in \Gamma \Big \rangle 
                \land 
            \Big \langle \zeta \notin \Xi \Big \rangle
        \bigg] 
            \lor 
        \bigg[
            \Big \langle \zeta \notin \Gamma \Big \rangle
                \land 
            \Big \langle \zeta \in \Xi \Big \rangle
        \bigg]   
    \end{equation*}
    Because logical conjunction is associative, this statement is equivalent to 
    \begin{equation*}
        \bigg[
            \Big \langle \zeta \in \Gamma \Big \rangle 
                \land 
            \Big \langle \zeta \notin \Xi \Big \rangle
        \bigg] 
            \lor 
        \bigg[
            \Big \langle \zeta \in \Xi \Big \rangle
                \land 
            \Big \langle \zeta \notin \Gamma \Big \rangle
        \bigg]   
    \end{equation*}
    $\therefore \text{\space} \bm{
        \Gamma \oplus \Xi 
        = 
    \big \langle \Gamma - \Xi \big \rangle
        \cup 
    \big \langle \Xi - \Gamma \big \rangle
    }$, by the defintions for set union and the difference of sets.
\end{proof}


\end{document}