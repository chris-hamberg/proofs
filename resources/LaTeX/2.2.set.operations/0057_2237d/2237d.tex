\documentclass[preview]{standalone}
\usepackage{amssymb, amsthm}
\usepackage{mathtools}
\usepackage{bm}


\newtheorem{theorem}{Theorem}
\renewcommand\qedsymbol{$\blacksquare$}


\begin{document}


\begin{theorem}[\textbf{2237d}]
    Let \bm{$\Xi$} be a subset of a universal set \bm{$\Omega$}. 
    \begin{equation*}
        \bm{\Xi \oplus \overline{\Xi} = \Omega}
    \end{equation*}
\end{theorem}
\begin{proof}
    By Theorem 52, 
    \bm{$
    \Xi \oplus \overline{\Xi} 
        = 
    \big \langle \Xi \cup \overline{\Xi} \big \rangle
        - 
    \big \langle \Xi \cap \overline{\Xi} \big \rangle
    $}. 
    By the set complement laws that is \bm{$\Omega - \varnothing$}. 
    Rather, \bm{$\Omega \cap \overline{\bm{\varnothing}}$}, 
    by Theorem 45. 
    Since \bm{$\overline{\bm{\varnothing}} = \Omega$}, that is 
    \bm{$\Omega \cap \Omega$}. Which is \bm{$\Omega$}, 
    by the idempotent law for set intersection.
    Thus, \bm{$\Xi \oplus \overline{\Xi} = \Omega$}.
\end{proof}


\end{document}