\documentclass[a4paper, 12pt]{article}
\usepackage[utf8]{inputenc}
\usepackage[english]{babel}
\usepackage{amssymb, amsmath, amsthm}
\theoremstyle{plain}
\newtheorem*{theorem*}{Theorem}
\newtheorem{theorem}{Theorem}

\usepackage{mathtools}
\renewcommand\qedsymbol{$\blacksquare$}

\begin{document}
	
	\begin{theorem*}[2.2.16d]
		Let A and B be sets. $A \cap (B - A) = \emptyset$.
	\end{theorem*}
	
	\begin{proof}
		Let $x$ be an element in $A \cap (B - A)$. Note that 
		$(B - A) \equiv (B \cap \overline{A})$. Because set intersection is associative we can 
		drop the parentheses, giving us $A \cap B \cap \overline{A}$. This is logically defined 
		as $(x \in A) \land (x \in B) \land (x \notin A) \equiv \bot$. Because this statement is 
		false $\forall x$ in the domain, $A \cap (B - A)$ is empty.
	\end{proof}

\end{document}
