\documentclass[a4paper, 12pt]{article}
\usepackage[utf8]{inputenc}
\usepackage[english]{babel}
\usepackage{amssymb, amsmath, amsthm}
\theoremstyle{plain}
\newtheorem*{theorem*}{Theorem}
\newtheorem{theorem}{Theorem}

\usepackage{mathtools}
\renewcommand\qedsymbol{$\blacksquare$}

\begin{document}
	
	\begin{theorem*}[2.2.16e]
		Let A and B be sets. $A \cup (B - A) = A \cup B$.
	\end{theorem*}
	
	\begin{proof}
		Let $x$ be an element in $A \cup (B - A)$. Note that 
		$(B - A) \equiv (B \cap \overline{A})$. So by definition we have 
		$(x \in A) \lor [(x \in B) \land (x \notin A)]$. Distributing logical disjunction over 
		logical conjunction yields \newline 
		$[(x \in A) \lor (x \in B)] \land [(x \in A) \lor (x \notin A)]$. Which by logical negation 
		and by logical identity reduces to $(x \in A) \lor (x \in B)$, that is the very definition for 
		$A \cup B$.
		
		Suppose the converse case in which  $x$ is an element of $A \cup B$. That is, of course as 
		already stated, defined as $(x \in A) \lor (x \in B)$. Note the fact that the conjunction of 
		this proposition with another true proposition is true. Let $p$ be that proposition, 
		$(x \in A)$. Then $p \lor \lnot p \equiv (x \in A) \lor (x \notin A) \equiv T$, and thus we 
		can make the following statement 
		$[(x \in A) \lor (x \in B)] \land [(x \in A) \lor (x \notin A)]$, which holds. Factoring the 
		term $(x \in A)$ out on the logical operators gives the form 
		$(x \in A) \lor [(x \in B) \land (x \notin A)]$. This statement is the definition for 
		$A \cup (B - A)$.
		
		Since $A \cup (B - A) \subseteq A \cup B$ and 
		$A \cup B \subseteq A \cup (B - A)$, 
		\newline 
		$A \cup (B - A) = A \cup B$ by definition.
	\end{proof}

\end{document}
