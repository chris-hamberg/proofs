\documentclass[a4paper, 12pt]{article}
\usepackage[utf8]{inputenc}
\usepackage[english]{babel}
\usepackage{amssymb, amsmath, amsthm}
\theoremstyle{plain}
\newtheorem*{theorem*}{Theorem}
\newtheorem{theorem}{Theorem}

\usepackage{mathtools}
\renewcommand\qedsymbol{$\blacksquare$}

\begin{document}
	
	\begin{theorem*}[2.2.18a]
		Let A, B, and C be sets. $(A \cup B) \subseteq (A \cup B \cup C)$.
	\end{theorem*}
	
	\begin{proof}
		Let $x$ be an element in $(A \cup B)$. We have $(x \in A) \lor (x \in B)$, by definition. 
		Let this definition statement be represented by $p$. Trivially, the fact of $p$ would be 
		unaffected were $p$ disjunct any proposition $q$. Supposing such a $q$ existed, we would have 
		$p \lor q \equiv T$ by logical domination (because the hypothetical supposition assumes 
		$p \equiv T$.) Let $q$ be the proposition $(x \in C)$. Then 
		$p \lor q \equiv (x \in A) \lor (x \in B) \lor (x \in C)$ is the well formed statement 
		defining the superset under interrogation. Since we know that $x$ is in the union of $A$ and 
		$B$ by the hypothesis, and because $p \lor q \equiv T$ means that $x$ is in the union of 
		$A$, $B$, and $C$, it immediately follows that \newline $(A \cup B) \subseteq (A \cup B \cup C)$.
	\end{proof}

\end{document}
