\documentclass[a4paper, 12pt]{article}
\usepackage[utf8]{inputenc}
\usepackage[english]{babel}
\usepackage{amssymb, amsmath, amsthm}
\theoremstyle{plain}
\newtheorem*{theorem*}{Theorem}
\newtheorem{theorem}{Theorem}

\usepackage{mathtools}
\renewcommand\qedsymbol{$\blacksquare$}

\begin{document}
	
	\begin{theorem*}[2.2.18c]
		Let A, B, and C be sets. $(A - B) - C \subseteq (A - C)$.
	\end{theorem*}
	
	\begin{proof}
		Let $x$ be an element in $(A - B) - C$. Note that $A - B \equiv A \cap \overline{B}$ and 
		$(A \cap \overline{B}) - C \equiv (A \cap \overline{B}) \cap \overline{C}$. By the 
		associative laws and the commutative laws for the intersection of sets we have, 
		$x \in (A \cap \overline{C}) \cap \overline{B}$. By definition that is 
		$[(x \in A) \land (x \in \overline{C})] \land (x \notin B)$. Or rather, 
		$x \in (A - C) \land (x \notin B)$. By logical identity $x \in (A - C)$. Since 
		$x \in [(A - B) - C] \implies x \in (A - C)$, $(A - B) - C \subseteq (A - C)$.
	\end{proof}

\end{document}
