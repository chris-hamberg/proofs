\documentclass[a4paper, 12pt]{article}
\usepackage[utf8]{inputenc}
\usepackage[english]{babel}
\usepackage{amssymb, amsmath, amsthm}
\theoremstyle{plain}
\newtheorem*{theorem*}{Theorem}
\newtheorem{theorem}{Theorem}

\usepackage{mathtools}
\renewcommand\qedsymbol{$\blacksquare$}

\begin{document}
	
	\begin{theorem*}[2.2.19]
		Let A, and B be sets. $A - B = A \cap \overline{B}$.
	\end{theorem*}
	
	\begin{proof}
		Let $x$ be an element in $A - B$. By the definition for set difference, 
		$(x \in A) \land (x \notin B)$. By the definition for set complementation this is the 
		same as $(x \in A) \land (x \in \overline{B})$. Which is exactly the definition for 
		$x \in (A \cap \overline{B})$.
		
		Proving the converse trivially follows by reversing our steps in the direct form. Suppose 
		there exists an element $x$ such that $x \in (A \cap \overline{B})$. By the definition for 
		set intersection we have $(x \in A) \land (x \in \overline{B})$. By the definition for set 
		complementation we arrive at the definition for set difference \newline 
		$(x \in A) \land (x \notin B)$. Therefore $x \in (A - B)$.
		
		Since $(A - B) \subseteq (A \cap \overline{B})$ and $(A \cap \overline{B}) \subseteq (A - B)$, 
		by the definition for set equality we have $(A - B) = (A \cap \overline{B})$.
	\end{proof}

\end{document}
