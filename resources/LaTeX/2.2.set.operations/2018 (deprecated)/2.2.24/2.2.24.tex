\documentclass[a4paper, 12pt]{article}
\usepackage[utf8]{inputenc}
\usepackage[english]{babel}
\usepackage{amssymb, amsmath, amsthm}
\theoremstyle{plain}
\newtheorem*{theorem*}{Theorem}
\newtheorem{theorem}{Theorem}

\usepackage{mathtools}
\renewcommand\qedsymbol{$\blacksquare$}

\begin{document}
	
	\begin{theorem*}[2.2.24]
		Let A, B, and C be sets. $(A - B) - C = (A - C) - (B - C)$.
	\end{theorem*}
	
	\begin{proof}
		Let $x$ be an element in $(A - B) - C$. By the definition for set difference we have 
		$[(x \in A) \land (x \notin B)] \land (x \notin C)$.  Now, the logical identity for the 
		proposition $(x \notin B)$ is $(x \notin B) \lor \bot \equiv (x \notin B)$, and since 
		\newline $(x \in C) \equiv \bot$ by reason of the hypothesis, it necessarily follows 
		from logical identity that $(x \notin B) \equiv [(x \notin B) \lor (x \in C)]$. Thus, 
		by these facts, the law of logical commutativity, and the law of logical association we 
		make the logically equivalent statement with respect to the definition given by the first 
		expression, $[(x \in A) \land (x \notin C)] \land [(x \notin B) \lor (x \in C)]$. Since 
		the right-hand side of the conjunction is true and in its proper logical identity, if it 
		were double negated, it would remain intact by the law of double negation. Thus applying 
		first a negation by DeMorgans law, and second a negation directly on the propositional 
		statement, we have 
		$[(x \in A) \land (x \notin C)] \land \lnot [(x \in B) \land (x \notin C)]$. By the 
		definition for set difference and set complementation $x$ is an element in 
		$(A - C) \cap \overline{(B - C)}$. Which is equivalent to the expression 
		$(A - C) - (B - C)$, by Theorem 2.2.19.
		
		To prove the converse case, let $x$ be an element in $(A - C) - (B - C)$. By Theorem 2.2.19  
		$(A - C) \cap \overline{(B - C)}$ is an equivalent expression. This expression containing 
		the element $x$ is defined by 
		\newline $[(x \in A) \land (x \notin C)] \land \lnot [(x \in B) \land (x \notin C)]$. 
		Applying DeMorgans law to the statement on the right-hand side of the conjunction we get 
		\newline $[(x \in A) \land (x \notin C)] \land [(x \notin B) \lor (x \in C)]$. Note that as 
		demonstrated in the first paragraph we have the following identity 
		$[(x \notin B) \lor (x \in C)] \equiv (x \notin B)$. Thus, by that fact, the law of logical 
		commutativity, and by the law of logical association it follows that the identity of our 
		statement is \newline $[(x \in A) \land (x \notin B)] \land (x \notin C)$. This is the 
		definition for $x \in [(A - B) - C]$.
		
		Since $(A - B) - C \subseteq (A - C) - (B - C)$ and 
		\newline $(A - C) - (B - C) \subseteq (A - B) - C$, it follows immediately from the 
		definition for set equality that $(A - B) - C = (A - C) - (B - C)$
	\end{proof}

\end{document}
