\documentclass[a4paper, 12pt]{article}
\usepackage[utf8]{inputenc}
\usepackage[english]{babel}
\usepackage{amssymb, amsmath, amsthm}
\theoremstyle{plain}
\newtheorem*{theorem*}{Theorem}
\newtheorem{theorem}{Theorem}

\usepackage{mathtools}
\renewcommand\qedsymbol{$\blacksquare$}

\begin{document}
	
	\begin{theorem*}[2.2.31]
		Let A, and B be subsets of a universal set U. \newline 
		$A \subseteq B \iff \overline{B} \subseteq \overline{A}$.
	\end{theorem*}
	
	\begin{proof}
		The proposition $A \subseteq B$ is equivalent to the universally quantified statement 
		$\forall x ( x \in A \implies x \in B)$. It is tautological that the propositional 
		function in this statement is logically equivalent to its contrapositive form 
		(satisfying the biconditional requirement.) That is,  
		$\forall x (x \notin B \implies x \notin A)$ is the logically equivalent statement. 
		By the definition for set complementation that is 
		$\forall x (x \in \overline{B} \implies x \in \overline{A})$. By universal generalization 
		\newline $A \subseteq B \iff \overline{B} \subseteq \overline{A}$.
	\end{proof}

\end{document}
