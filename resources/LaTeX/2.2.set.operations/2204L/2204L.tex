\documentclass[preview]{standalone}
\usepackage{amssymb, amsthm}
\usepackage{mathtools}
\usepackage{bm}
\usepackage{xcolor}


\newtheorem*{lemma*}{Lemma}
\renewcommand\qedsymbol{$\blacksquare$}


\begin{document}


\begin{lemma*}[\textbf{2204}] \color{black}
    Let \bm{$\Gamma$}, \bm{$\Pi$}, and \bm{$\Xi$} be sets.
    \begin{equation*}
    \bm{
        \Big \langle \Gamma \cup \Pi \Big \rangle
            \cap
        \Big \langle \overline{\Gamma} \cup \overline{\Pi} \Big \rangle
            \cap
        \overline{\Xi}
            =
        \Big \langle \Pi \cap \overline{\Gamma} \cap \overline{\Xi} \Big \rangle
            \cup
        \Big \langle \Gamma \cap \overline{\Pi} \cap \overline{\Xi} \Big \rangle
        }
    \end{equation*}
\end{lemma*}
\begin{proof} \color{black}
    Distributing the term \bm{$\Gamma \cup \Pi$} over the union of 
    \bm{$\overline{\Gamma}$} and \bm{$\overline{\Pi}$},
    by the law of distribution for the intersection of sets over union,
    in the left-hand side of the equation expressed by the lemma is
    \begin{equation*}
        \bigg[
            \Big \langle \Gamma \cup \Pi \Big \rangle
                \cap
            \overline{\Gamma}
        \bigg]
            \cup
        \bigg[
            \Big \langle \Gamma \cup \Pi \Big \rangle
                \cap
            \overline{\Pi}
        \bigg]
            \cap
        \overline{\Xi}
    \end{equation*}
    Again, by the law of distribution for intersection over set union,
    \begin{equation*}
        \equiv
        \bigg[
            \Big \langle \Gamma \cap \overline{\Gamma} \Big \rangle
                \cup
            \Big \langle \Pi \cap \overline{\Gamma} \Big \rangle
        \bigg]
            \cup
        \bigg[
            \Big \langle \Gamma \cap \overline{\Pi} \Big \rangle
                \cup
            \Big \langle \Pi \cap \overline{\Pi} \Big \rangle
        \bigg]
            \cap
        \overline{\Xi}
    \end{equation*}
    \bm{$\Gamma \cap \overline{\Gamma}$} and \bm{$\Pi \cap \overline{\Pi}$}
    are both empty, by the domination law for set intersection. And any set,
    union the empty set, is itself, by the identity law for set union. Thus,
    by association, what is left is
    \begin{equation*}       
        \bigg[
            \Big \langle \Pi \cap \overline{\Gamma} \Big \rangle
                \cup
            \Big \langle \Gamma \cap \overline{\Pi} \Big \rangle
        \bigg]
            \cap
        \overline{\Xi}
    \end{equation*}
    $\therefore \bm{
        \big \langle \Gamma \cup \Pi \big \rangle
            \cap
        \big \langle \overline{\Gamma} \cup \overline{\Pi} \big \rangle
            \cap
        \overline{\Xi}
            =
        \big \langle \Pi \cap \overline{\Gamma} \cap \overline{\Xi} \big \rangle
            \cup
        \big \langle \Gamma \cap \overline{\Pi} \cap \overline{\Xi} \big \rangle
    }$,
    by the law of distribution for set intersection over set union,
    and by association for the intersection of sets.
    
\color{lightgray} \end{proof}

\end{document}