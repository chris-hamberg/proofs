\documentclass[preview]{standalone}
\usepackage{amssymb, amsthm}
\usepackage{mathtools}
\usepackage{bm}
\usepackage{xcolor}


\newtheorem*{theorem*}{Theorem}
\renewcommand\qedsymbol{$\blacksquare$}


\begin{document}


\begin{theorem*}[\textbf{2206a}] \color{black}
    Let \bm{$\Xi$} be a set. 
    The set identity for \bm{$\Xi$} is \bm{$\Xi \cup \varnothing = \Xi$}.
\end{theorem*}
\begin{proof} \color{black}
    Suppose there exists an element \bm{$\zeta$} such that \bm{$\zeta$} is a member of 
    \bm{$\Xi \cup \varnothing$}. 
    By the definition of set union, that is 
    \begin{equation*}
        \Big \langle \zeta \in \Xi \Big \rangle 
            \lor 
        \Big \langle \zeta \in \varnothing \Big \rangle
    \end{equation*}
    The logical identity for the statement \bm{$\zeta \in \varnothing$} is trivially \bm{$\bot$}, 
    because the empty set contains no members. 
    Thus, by that identity, and by the identity law for logical disjunction,
    \begin{equation*} 
        \Bigg\{
            \Big \langle \zeta \in \Xi \Big \rangle 
                \lor 
            \Big \langle \zeta \in \varnothing \Big \rangle
        \Bigg\} 
            \equiv
        \Bigg\{
            \Big \langle \zeta \in \Xi \Big \rangle 
                \lor
            \Big \langle \bot \Big \rangle 
                \equiv
            \Big \langle \zeta \in \Xi \Big \rangle
        \Bigg\}
            \equiv
    \end{equation*}
    \begin{equation*} 
        \Bigg\{
            \Big \langle \zeta \in \Xi \Big \rangle 
                \lor 
            \Big \langle \zeta \in \varnothing \Big \rangle
                \equiv
            \Big \langle \zeta \in \Xi \Big \rangle
        \Bigg\}
    \end{equation*}
    $\therefore$ 
    by the definition for set union,
    the set identity for \bm{$\Xi$} is \bm{$\Xi \cup \varnothing = \Xi$}.
\color{lightgray} \end{proof}

\end{document}