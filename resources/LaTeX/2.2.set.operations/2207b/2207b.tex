\documentclass[preview]{standalone}
\usepackage{amssymb, amsthm}
\usepackage{mathtools}
\usepackage{bm}
\usepackage{xcolor}


\newtheorem*{theorem*}{Theorem}
\renewcommand\qedsymbol{$\blacksquare$}


\begin{document}


\begin{theorem*}[\textbf{2207b}] \color{black}
    Let \bm{$\Xi$} be a set. 
    The empty set dominates set intersection such that 
    \bm{$\Xi \cap \varnothing = \varnothing$}.
\end{theorem*}
\begin{proof} \color{black}
    Let \bm{$\zeta$} be an element in \bm{$\Xi \cap \varnothing$}. 
    By the definition for set intersection, that is 
    \begin{equation*}
        \Big \langle \zeta \in \Xi \Big \rangle 
            \land 
        \Big \langle \zeta \in \varnothing \Big \rangle
    \end{equation*}
    The logical identity for the statement \bm{$\zeta \in \varnothing$} is trivially \bm{$\bot$}, 
    since the empty set contains no members. 
    Thus, by that identity, 
    and by the domination law for logical conjunction,
    \begin{equation*}
        \Bigg\{
            \Big \langle \zeta \in \Xi \Big \rangle 
                \land 
            \Big \langle \zeta \in \varnothing \Big \rangle
        \Bigg\}
            \equiv
        \Bigg\{
            \Big \langle \zeta \in \Xi \Big \rangle 
                \land 
            \Big \langle \bot \Big \rangle
                \equiv
            \Big \langle \zeta \in \varnothing \Big \rangle
        \Bigg\}
            \equiv
    \end{equation*}
    \begin{equation*}
        \Bigg\{
            \Big \langle \zeta \in \Xi \Big \rangle 
                \land 
            \Big \langle \zeta \in \varnothing \Big \rangle
                \equiv
            \Big \langle \zeta \in \varnothing \Big \rangle
        \Bigg\}
    \end{equation*}
    $\therefore$ by the definition for the intersection of sets,
    the empty set dominates set intersection such that 
    \bm{$\Xi \cap \varnothing = \varnothing$}.
\color{lightgray} \end{proof}

\end{document}