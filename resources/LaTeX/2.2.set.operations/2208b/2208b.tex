\documentclass[preview]{standalone}
\usepackage{amssymb, amsthm}
\usepackage{mathtools}
\usepackage{bm}
\usepackage{xcolor}


\newtheorem*{theorem*}{Theorem}
\renewcommand\qedsymbol{$\blacksquare$}


\begin{document}


\begin{theorem*}[\textbf{2208b}] \color{black}
    Let \bm{$\Lambda$} be a set.
    \bm{$\Lambda$} is idempotent such that 
    \bm{$\Lambda \cap \Lambda = \Lambda$}.
\end{theorem*}
\begin{proof} \color{black}
    Let \bm{$\mu$} be an element in \bm{$\Lambda \cap \Lambda$}. 
    By the definition for set intersection,
    \begin{equation*}
        \Big \langle \mu \in \Lambda \Big \rangle 
            \land 
        \Big \langle \mu \in \Lambda \Big \rangle
    \end{equation*}
    Thus, by the idempotent law for logical conjunction,
    \begin{equation*}
        \Big \langle \mu \in \Lambda \Big \rangle 
            \land 
        \Big \langle \mu \in \Lambda \big \rangle 
            \equiv 
        \Big \langle \mu \in \Lambda \Big \rangle
    \end{equation*}
    $\therefore$ by the definition for the intersection of sets,
    \bm{$\Lambda$} is idempotent such that 
    \bm{$\Lambda \cap \Lambda = \Lambda$}.
\color{lightgray} \end{proof}

\end{document}