\documentclass[preview]{standalone}
\usepackage{amssymb, amsthm}
\usepackage{mathtools}
\usepackage{bm}
\usepackage{xcolor}


\newtheorem*{theorem*}{Theorem}
\renewcommand\qedsymbol{$\blacksquare$}


\begin{document}


\begin{theorem*}[\textbf{2209b}] \color{black}
    Let \bm{$\Xi$} be a set. 
    \bm{$\Xi \cap \overline{\Xi} = \varnothing$}.
\end{theorem*}
\begin{proof} \color{black}
    Let \bm{$\zeta$} be an element in \bm{$\Xi \cap \overline{\Xi}$}. 
    By the definition for the intersection of sets, that is
    \begin{equation*}
        \Big \langle \zeta \in \Xi \Big \rangle 
            \land 
        \Big \langle \zeta \in \overline{\Xi} \Big \rangle
    \end{equation*} 
    According to the definitions for set complementation and set membership, 
    and by the negation law of logic, that is
    \begin{equation*}
        \Bigg\{
            \Big \langle \zeta \in \Xi \Big \rangle 
                \land 
            \Big \langle \zeta \in \overline{\Xi} \Big \rangle
        \Bigg\}
            \equiv
        \Bigg\{
            \Big \langle \zeta \in \Xi \Big \rangle 
                \land 
            \lnot \Big \langle \zeta \in \Xi \Big \rangle
                \equiv
            \Big \langle \bot \Big \rangle
        \Bigg\}
    \end{equation*}
    \bm{$\big \langle \bot \big \rangle$} is trivially the logical identity for 
    the statement \bm{$\zeta \in \varnothing$}, 
    since the empty set contains no members. 
    Thus, 
    by that identity, 
    and following from the series of equivalencies from above,
    \begin{equation*}
        \Big \langle \zeta \in \Xi \Big \rangle 
            \land 
        \Big \langle \zeta \in \overline{\Xi} \Big \rangle 
            \equiv
        \Big \langle \zeta \in \varnothing \Big \rangle
    \end{equation*}
    $\therefore$ the complement law for sets, 
    \bm{$\Xi \cap \overline{\Xi} = \varnothing$}, 
    follows immediately from the definition for the intersection of sets.
\color{lightgray} \end{proof}

\end{document}