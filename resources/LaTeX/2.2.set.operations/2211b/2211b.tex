\documentclass[preview]{standalone}
\usepackage{amssymb, amsthm}
\usepackage{mathtools}
\usepackage{bm}
\usepackage{xcolor}


\newtheorem*{theorem*}{Theorem}
\renewcommand\qedsymbol{$\blacksquare$}


\begin{document}


\begin{theorem*}[\textbf{2211b}] \color{black}
    Let \bm{$\mathrm{A}$} and \bm{$\Lambda$} be sets. 
    The intersection of \bm{$\mathrm{A}$} and \bm{$\Lambda$} is commutative.
\end{theorem*}
\begin{proof} \color{black}
    Let \bm{$\lambda$} be an element in \bm{$\mathrm{A} \cap \Lambda$}. 
    By the definition for set intersection, 
    \begin{equation*}
        \Big \langle \lambda \in \mathrm{A} \Big \rangle 
            \land 
        \Big \langle \lambda \in \Lambda \Big \rangle
    \end{equation*}
    Because logical conjunction is commutative, that is
    \begin{equation*}
        \Bigg[
            \Big \langle \lambda \in \mathrm{A} \Big \rangle 
                \land 
            \Big \langle \lambda \in \Lambda \Big \rangle
        \Bigg]
            \equiv
        \Bigg[
            \Big \langle \lambda \in \Lambda \Big \rangle 
                \land 
            \Big \langle \lambda \in \mathrm{A} \Big \rangle
        \Bigg]
    \end{equation*}
    $\therefore \text{\space} 
    \bm{\mathrm{A} \cap \Lambda = \Lambda \cap \mathrm{A}}$, 
    and indeed the intersection of \bm{$\mathrm{A}$} and \bm{$\Lambda$} is commutative.
\color{lightgray} \end{proof}

\end{document}