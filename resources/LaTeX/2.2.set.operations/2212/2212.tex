\documentclass[preview]{standalone}
\usepackage{amssymb, amsthm}
\usepackage{mathtools}
\usepackage{bm}
\usepackage{xcolor}


\newtheorem*{theorem*}{Theorem}
\renewcommand\qedsymbol{$\blacksquare$}


\begin{document}


\begin{theorem*}[\textbf{2212}] \color{black}
    Let \bm{$\mathrm{A}$} and \bm{$\Lambda$} be sets. 
    \bm{$
    \mathrm{A} 
        \cup 
    \big \langle \mathrm{A} \cap \Lambda \big \rangle 
        = 
    \mathrm{A}$}.
\end{theorem*}
\begin{proof} \color{black}
    Let \bm{$\lambda$} be an element in 
    \bm{$
    \mathrm{A} 
        \cup 
    \big \langle \mathrm{A} \cap \Lambda \big \rangle
    $}. 
    By the definitions for the union of sets and the intersection of sets, that is
    \begin{equation*}
        \Big \langle \lambda \in \mathrm{A} \Big \rangle 
            \lor 
        \bigg[
            \Big \langle \lambda \in \mathrm{A} \Big \rangle 
                \land 
            \Big \langle \lambda \in \Lambda \Big \rangle
        \bigg]
    \end{equation*}
    It follows immediately from the laws of logical absorption that
    \begin{equation*}
        \Bigg\{
            \Big \langle \lambda \in \mathrm{A} \Big \rangle 
                \lor 
            \bigg[
                \Big \langle \lambda \in \mathrm{A} \Big \rangle 
                    \land 
                \Big \langle \lambda \in \Lambda \Big \rangle
            \bigg]
        \Bigg\}
            \equiv 
        \bigg \langle \lambda \in \mathrm{A} \bigg \rangle 
    \end{equation*}
    $\therefore$ by the definitions for set union and set intersection, 
    \bm{$
    \mathrm{A} 
        \cup 
    \big \langle \mathrm{A} \cap \Lambda \big \rangle 
        = 
    \mathrm{A}$}; 
    the absorption law for set union over intersection.
\color{lightgray} \end{proof}

\end{document}