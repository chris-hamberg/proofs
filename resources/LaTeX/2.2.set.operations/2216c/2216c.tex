\documentclass[preview]{standalone}
\usepackage{amssymb, amsthm}
\usepackage{mathtools}
\usepackage{bm}
\usepackage{xcolor}


\newtheorem*{theorem*}{Theorem}
\renewcommand\qedsymbol{$\blacksquare$}


\begin{document}


\begin{theorem*}[\textbf{2216c}] \color{black}
    Let \bm{$\mathrm{A}$} and \bm{$\Lambda$} be sets. 
    \bm{$
    \big \langle \mathrm{A} - \Lambda \big \rangle 
        \subseteq 
    \mathrm{A}
    $}.
\end{theorem*}
\begin{proof} \color{black}
    Let \bm{$\lambda$} be an element in \bm{$\mathrm{A} - \Lambda$}. 
    By the definition for set difference,
    \begin{equation*}
        \Big \langle \lambda \in \mathrm{A} \Big \rangle
            \land
        \Big \langle \lambda \notin \Lambda \Big \rangle   
    \end{equation*}
    It trivially follows from the simplification rule of inference that
    \begin{equation*}
        \bigg[
            \Big \langle \lambda \in \mathrm{A} \Big \rangle
                \land
            \Big \langle \lambda \notin \Lambda \Big \rangle
        \bigg]
            \rightarrow
        \Big \langle \lambda \in \mathrm{A} \Big \rangle
    \end{equation*}
    $\therefore$ 
    \bm{$
    \big \langle \mathrm{A} - \Lambda \big \rangle 
        \subseteq 
    \mathrm{A}
    $},
    by the definition for subsets.
\color{lightgray} \end{proof}

\end{document}