\documentclass[preview]{standalone}
\usepackage{amssymb, amsthm}
\usepackage{mathtools}
\usepackage{bm}
\usepackage{xcolor}


\newtheorem*{theorem*}{Theorem}
\renewcommand\qedsymbol{$\blacksquare$}


\begin{document}


\begin{theorem*}[\textbf{2216d}] \color{black}
    Let \bm{$\mathrm{A}$} and \bm{$\Lambda$} be sets. 
    \bm{$
    \mathrm{A} 
        \cap 
    \big \langle \Lambda - \mathrm{A} \big \rangle 
        = 
    \varnothing
    $}.
\end{theorem*}
\begin{proof} \color{black}
    Let \bm{$\lambda$} be an element in 
    \bm{$\mathrm{A} \cap \big \langle \Lambda - \mathrm{A} \big \rangle$}. 
    By the definitions for set difference, 
    and set intersection, that is
    \begin{equation*}
       \Big \langle \lambda \in \mathrm{A} \Big \rangle
            \land
        \bigg[
            \Big \langle \lambda \in \Lambda \Big \rangle
                \land
            \Big \langle \lambda \notin \mathrm{A} \Big \rangle
        \bigg] 
    \end{equation*}
    Since logical conjunction is associative, 
    the logical identity for this statement is \bm{$\bot$}, 
    by the negation law for logical conjunction, 
    and by the domination law for logical conjunction. 
    It is trivial that the logical identity for the statement
    \bm{$\lambda \in \varnothing$} is \bm{$\bot$}, 
    since the empty set contains no members. 
    Thus,
    \begin{equation*}
        \Big \langle \lambda \in \mathrm{A} \Big \rangle
             \land
         \bigg[
             \Big \langle \lambda \in \Lambda \Big \rangle
                 \land
             \Big \langle \lambda \notin \mathrm{A} \Big \rangle
         \bigg] 
            \equiv
        \Big \langle \lambda \in \varnothing \Big \rangle
     \end{equation*}
    $\therefore$ 
    by the definitions for set difference and intersection,
    \bm{$
    \mathrm{A} 
        \cap 
    \big \langle \Lambda - \mathrm{A} \big \rangle 
        = 
    \varnothing
    $}.
\color{lightgray} \end{proof}

\end{document}