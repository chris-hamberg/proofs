\documentclass[preview]{standalone}
\usepackage{amssymb, amsthm}
\usepackage{mathtools}
\usepackage{bm}
\usepackage{xcolor}


\newtheorem*{theorem*}{Theorem}
\renewcommand\qedsymbol{$\blacksquare$}


\begin{document}


\begin{theorem*}[\textbf{2218a}] \color{black}
    Let \bm{$\mathrm{A}$}, \bm{$\Lambda$}, and \bm{$\Delta$} be sets. 
    \bm{$
    \big \langle \mathrm{A} \cup \Lambda \big \rangle
        \subseteq 
    \big \langle \mathrm{A} \cup \Lambda \cup \Delta \big \rangle
    $}.
\end{theorem*}
\begin{proof} \color{black}
    Let \bm{$\lambda$} be an element in \bm{$\mathrm{A} \cup \Lambda$}. 
    By the defintion for the union of sets, that is
    \begin{equation*}
        \Big \langle \lambda \in \mathrm{A} \Big \rangle
            \lor 
        \Big \langle \lambda \in \Lambda \Big \rangle
    \end{equation*}
    Let this statement be represented by the propositional variable \bm{$p$}. 
    By the addition rule of inference, \bm{$p$} implies \bm{$p \lor q$}, 
    for any propositional variable \bm{$q$}. 
    Let \bm{$q$} be the statement \bm{$\lambda \in \Delta$}. 
    Thus,
    \begin{equation*}
        \Big \langle \lambda \in \mathrm{A} \Big \rangle
            \lor 
        \Big \langle \lambda \in \Lambda \Big \rangle
            \rightarrow
        \Big \langle \lambda \in \mathrm{A} \Big \rangle
            \lor 
        \Big \langle \lambda \in \Lambda \Big \rangle
            \lor
        \Big \langle \lambda \in \Delta \Big \rangle
    \end{equation*}
    $\therefore \text{\space} \bm{
    \big \langle \mathrm{A} \cup \Lambda \big \rangle
        \subseteq 
    \big \langle \mathrm{A} \cup \Lambda \cup \Delta \big \rangle
    }
    $,
    by the definitions for set union and subsets.
\color{lightgray} \end{proof}

\end{document}