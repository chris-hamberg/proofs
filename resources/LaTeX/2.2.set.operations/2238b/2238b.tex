\documentclass[preview]{standalone}
\usepackage{amssymb, amsthm}
\usepackage{mathtools}
\usepackage{bm}
\usepackage{xcolor}


\newtheorem*{theorem*}{Theorem}
\renewcommand\qedsymbol{$\blacksquare$}


\begin{document}


\begin{theorem*}[\textbf{2238b}] \color{black}
    Let \bm{$\Gamma$}, and \bm{$\Xi$} be sets.
    \bm{$
    \big \langle \Gamma \oplus \Xi \big \rangle
        \oplus 
    \Xi 
        = 
    \Gamma$}.
\end{theorem*}
\begin{proof} \color{black}
    By Theorem 2235, 
    \begin{equation*}
        \Big \langle \Gamma \oplus \Xi \Big \rangle 
            \oplus 
        \Xi 
            = 
        \bigg[
            \Big \langle \Gamma \oplus \Xi \Big \rangle 
                \cup 
            \Xi
        \bigg] 
            - 
        \bigg[
            \Big \langle \Gamma \oplus \Xi \Big \rangle
                \cap 
            \Xi
        \bigg]
    \end{equation*}
    By Lemma 2201, that is
    \begin{equation*}
        \Bigg\{
            \bigg[
                \Big \langle \Gamma \cup \Xi \Big \rangle
                    \cap
                \Big \langle
                    \overline{\Gamma} 
                        \cup 
                    \overline{\Xi} 
                \Big \rangle
            \bigg]
                \cup
            \Xi
        \Bigg\}
            -
        \Bigg\{
            \bigg[
                \Big \langle \Gamma \cup \Xi \Big \rangle
                    \cap
                \Big \langle
                    \overline{\Gamma} 
                        \cup 
                    \overline{\Xi} 
                \Big \rangle
            \bigg]
                \cap
            \Xi
        \Bigg\}
    \end{equation*}
    Since set intersection is associative, 
    by the associative law for the intersection of sets, 
    the identities for the terms in the difference are given immediately
    by Lemma 2202, and Lemma 2203. Thus, 
    \begin{equation*}
        \Big \langle \Gamma \oplus \Xi \Big \rangle 
            \oplus 
        \Xi
            =
        \Big \langle \Gamma \cup \Xi \Big \rangle
            -
        \Big \langle 
            \Xi
                \cap 
            \overline{\Gamma} 
        \Big \rangle
    \end{equation*}
    By Theorem 2219, 
    by DeMorgans law for the complement of intersections, 
    and by the complementation law for sets, 
    that is
    \begin{equation*}
        \bigg[
            \Big \langle \Gamma \cup \Xi \Big \rangle
                -
            \Big \langle 
                \Xi
                    \cap 
                \overline{\Gamma} 
            \Big \rangle
        \bigg]
            \equiv
        \bigg[
            \Big \langle \Gamma \cup \Xi \Big \rangle
                \cap
            \Big \langle \overline{
                \Xi
                    \cap 
                \overline{\Gamma} 
            } \Big \rangle
        \bigg]
            \equiv
        \bigg[
            \Big \langle \Gamma \cup \Xi \Big \rangle
                \cap
            \Big \langle 
                \overline{\Xi}
                    \cup 
                \Gamma
            \Big \rangle
        \bigg]
    \end{equation*}
    \bm{$\Gamma$} can be factored out,
    by the distribution law for set union over intersection.
    \begin{equation*}
        \Big \langle \Gamma \oplus \Xi \Big \rangle
            \oplus 
        \Xi
            \equiv
        \Gamma 
            \cup
        \Big \langle \Xi \cap \overline{\Xi} \Big \rangle
    \end{equation*}
    \bm{$\Xi \cap \overline{\Xi}$} is empty, 
    by the complement law for set intersection. 
    And \bm{$\Gamma$} union the empty set is \bm{$\Gamma$},
    by the identity law for set union $\therefore \bm{
    \big \langle \Gamma \oplus \Xi \big \rangle
        \oplus 
    \Xi 
        = 
    \Gamma}$.
\color{lightgray} \end{proof}

\end{document}