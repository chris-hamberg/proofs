\documentclass[preview]{standalone}
\usepackage{amssymb, amsthm}
\usepackage{mathtools}
\usepackage{bm}
\usepackage{xcolor}


\newtheorem*{theorem*}{Theorem}
\renewcommand\qedsymbol{$\blacksquare$}


\begin{document}


\begin{theorem*}[\textbf{2240}]
    Let \bm{$\Gamma$}, \bm{$\Pi$}, and \bm{$\Xi$} be sets. 
    The symmetric difference for sets is associative such that 
    \begin{equation*}
        \bm{
            \Big \langle \Gamma \oplus \Pi \Big \rangle
                \oplus 
            \Xi 
                = 
            \Gamma 
                \oplus 
            \Big \langle \Pi \oplus \Xi \Big \rangle
        }
    \end{equation*}
\end{theorem*}
\begin{proof}
    Let \bm{$\zeta$} be an element in 
    \bm{$
        \big \langle \Gamma \oplus \Pi \big \rangle
            \oplus 
        \Xi 
    $}. 
    By the definition for the symmetric 
    difference of sets, \bm{$\zeta$} is in
    \begin{equation*}
        \bigg[
            \Big \langle \Gamma \oplus \Pi \Big \rangle
                \cap
            \overline{\Xi}
        \bigg]
            \cup
        \bigg[
            \Big \langle \overline{
                \Gamma \oplus \Pi
            } \Big \rangle 
                \cap
            \Xi
        \bigg]
    \end{equation*}
    By Lemma 2201, \bm{$\zeta$} is an element of
    \begin{equation*}
        \Bigg\{
            \bigg[
                \Big \langle \Gamma \cup \Pi \Big \rangle
                    \cap
                \Big \langle \overline{\Gamma} \cup \overline{\Pi} \Big \rangle
            \bigg]
                \cap
            \overline{\Xi}
        \Bigg\}
            \cup
        \Bigg\{
            \bigg[ \overline{
                \Big \langle \Gamma \cup \Pi \Big \rangle
                \cap
            \Big \langle \overline{\Gamma} \cup \overline{\Pi} \Big \rangle
            } \bigg]
                \cap
            \Xi
        \Bigg\}
    \end{equation*}
    Each superset on either side of this union is described either by Lemma 2204,
    or by Lemma 2205. Thus, by Lemma 2204, and 2205, \bm{$\zeta$} is in
    \begin{equation*}
        \Big \langle \Pi \cap \overline{\Gamma} \cap \overline{\Xi} \Big \rangle
            \cup
        \Big \langle \Gamma \cap \overline{\Pi} \cap \overline{\Xi} \Big \rangle
            \cup
        \Big \langle \overline{\Gamma} \cap \overline{\Pi} \cap \Xi \Big \rangle
            \cup
        \Big \langle \Gamma \cap \Pi \cap \Xi \Big \rangle
            \equiv
        \Delta
    \end{equation*}
    Now, suppose it were the case that \bm{$\zeta$} were an element in
    \bm{$
        \Gamma 
            \oplus 
    \big \langle \Pi \oplus \Xi \big \rangle
    $}. Because set intersection and set union are commutative,
    from the definition for the symmetric difference of sets,
    \bm{$\zeta$} would have to be in
    \begin{equation*}
        \bigg[
            \Big \langle \Pi \oplus \Xi \Big \rangle
                \cap
            \overline{\Gamma}
        \bigg]
            \cup
        \bigg[
            \Big \langle \overline{
                \Pi \oplus \Xi
            } \Big \rangle
                \cap
            \Gamma 
        \bigg]
    \end{equation*}
    By Lemma 2201, \bm{$\zeta$} is an element of
    \begin{equation*}
        \Bigg\{
            \bigg[
                \Big \langle \Pi \cup \Xi \Big \rangle
                    \cap
                \Big \langle \overline{\Pi} \cup \overline{\Xi} \Big \rangle
            \bigg]
                \cap
            \overline{\Gamma}
        \Bigg\}
            \cup
        \Bigg\{
            \bigg[ \overline{
                \Big \langle \Pi \cup \Xi \Big \rangle
                \cap
            \Big \langle \overline{\Pi} \cup \overline{\Xi} \Big \rangle
            } \bigg]
                \cap
            \Gamma
        \Bigg\}
    \end{equation*}
    And by Lemma 2204, and Lemma 2205, \bm{$\zeta$} is an element in
    \begin{equation*}
        \Big \langle \Xi \cap \overline{\Pi} \cap \overline{\Gamma} \Big \rangle
            \cup
        \Big \langle \Pi \cap \overline{\Xi} \cap \overline{\Gamma} \Big \rangle
            \cup
        \Big \langle \overline{\Pi} \cap \overline{\Xi} \cap \Gamma \Big \rangle
            \cup
        \Big \langle \Pi \cap \Xi \cap \Gamma \Big \rangle
            \equiv
        \Delta
    \end{equation*}
    Because \bm{$\zeta$} is in \bm{$\Delta$} whenever \bm{$\zeta$} is in
    \bm{$\big \langle \Gamma \oplus \Pi \big \rangle \oplus \Xi$}, 
    and because \bm{$\zeta$} is in \bm{$\Delta$} whenever \bm{$\zeta$} is in
    \bm{$\Gamma \oplus \big \langle \Pi \oplus \Xi \big \rangle$}, 
    it follows immediately that the symmetric difference for sets is associative such that 
    \bm{$
        \big \langle \Gamma \oplus \Pi \big \rangle
            \oplus 
        \Xi 
            = 
        \Gamma 
            \oplus 
        \big \langle \Pi \oplus \Xi \big \rangle
    $}.
\color{lightgray} \end{proof}

\end{document}