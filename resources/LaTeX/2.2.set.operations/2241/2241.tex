\documentclass[preview]{standalone}
\usepackage{amssymb, amsthm}
\usepackage{mathtools}
\usepackage{bm}
\usepackage{xcolor}


\newtheorem*{theorem*}{Theorem}
\renewcommand\qedsymbol{$\blacksquare$}


\begin{document}


\begin{theorem*}[\textbf{2241}]
    Let \bm{$\Gamma$}, \bm{$\Pi$}, and \bm{$\Xi$} be sets.
    \begin{equation*}
        \bm{ \text{If }
            \Gamma \oplus \Xi 
                = 
            \Pi \oplus \Xi
                \text{, then } 
            \Gamma = \Pi
        }    
    \end{equation*}
\end{theorem*}
\begin{proof} 
By contraposition. Note that the statement 
\bm{$
    \Gamma \oplus \Xi 
        = 
    \Pi \oplus \Xi
$}
is by definition 
\begin{equation*}
    \Big \langle \Gamma \cap \overline{\Xi} \Big \rangle 
        \cup 
    \Big \langle \overline{\Gamma} \cap \Xi \Big \rangle 
        \equiv
    \Big \langle \Pi \cap \overline{\Xi} \Big \rangle 
        \cup 
    \Big \langle \overline{\Pi} \cap \Xi \Big \rangle
\end{equation*}
Assume there exists an element \bm{$\zeta$} 
such that \bm{$\zeta \in \Gamma$} and \bm{$\zeta \notin \Pi$}. 
Thus, \bm{$\Gamma \not \subseteq \Pi$},
the negation of the consequent, by the definiton of subsets. 
By that hypothesis, 
\bm{$\zeta$} has to be in 
\bm{$\Gamma \cap \overline{\Xi}$} 
and cannot be in 
\bm{$\overline{\Gamma} \cap \Xi$}. 
This means that \bm{$\zeta$} is not in \bm{$\Xi$}. 
Neither can \bm{$\zeta$} be in 
\bm{$\Pi \cap \overline{\Xi}$}. 
And since \bm{$\zeta \notin \Xi$}, 
\bm{$\zeta$} cannot be in \bm{$\overline{\Pi} \cap \Xi$}. 
So \bm{$\zeta$} is in \bm{$\Gamma \oplus \Xi$} 
but not \bm{$\Pi \oplus \Xi$}. 
Therefore, 
\bm{$\Gamma \oplus \Xi \not \subseteq \Pi \oplus \Xi$},
by the definition of subsets.
The implication,
\begin{equation*}
    \Big \langle \Pi \not \subseteq \Gamma \Big \rangle
        \rightarrow
    \bigg[
        \Big \langle \Pi \oplus \Xi \Big \rangle
            \not \subseteq 
        \Big \langle \Gamma \oplus \Xi \Big \rangle
    \bigg]
\end{equation*} 
follows without loss of generality
$\therefore \text{\space} \bm{
    \Gamma \neq \Pi}$ 
implies 
\bm{$\Gamma \oplus \Xi \neq \Pi \oplus \Xi
}$.
\color{lightgray} \end{proof}

\end{document}