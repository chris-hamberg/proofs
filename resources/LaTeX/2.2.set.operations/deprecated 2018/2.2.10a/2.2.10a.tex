\documentclass[a4paper, 12pt]{article}
\usepackage[utf8]{inputenc}
\usepackage[english]{babel}
\usepackage{amssymb, amsmath, amsthm}
\theoremstyle{plain}
\newtheorem*{theorem*}{Theorem}
\newtheorem{theorem}{Theorem}

\usepackage{mathtools}
\renewcommand\qedsymbol{$\blacksquare$}

\begin{document}
	
	\begin{theorem*}[2.2.10a]
		Let A be a set. $A - \emptyset = A$.
	\end{theorem*}
	
	\begin{proof}
		Let $x$ be an element in $A - \emptyset \equiv A \cap \overline{\emptyset}$. By definition we have, 
		\newline $(x \in A) \land (x \notin \emptyset)$. Because by supposition 
		$\exists x (x \in (A - \emptyset))$, we know that the statement $(x \notin \emptyset)$ must be true. 
		Therefore, by logical identity, the statement $x \in (A - \emptyset)$ is defined as $x \in A$. 
		So $A - \emptyset = A.$
	\end{proof}

\end{document}
