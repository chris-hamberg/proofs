\documentclass[a4paper, 12pt]{article}
\usepackage[utf8]{inputenc}
\usepackage[english]{babel}
\usepackage{amssymb, amsmath, amsthm}
\theoremstyle{plain}
\newtheorem*{theorem*}{Theorem}
\newtheorem{theorem}{Theorem}

\usepackage{mathtools}
\renewcommand\qedsymbol{$\blacksquare$}

\begin{document}
	
	\begin{theorem*}[2.2.10b]
		Let A be a set. $\emptyset - A = \emptyset$.
	\end{theorem*}
	
	\begin{proof}
		Let $x$ be an element in $\emptyset - A \equiv \emptyset \cap \overline{A}$. Because this expression 
		is defined as $(x \in \emptyset) \land (x \notin A)$ the supposition 
		$\exists x (x \in (\emptyset - A))$ immediately contradicts $x \in \emptyset$. Meaning that no such 
		$x$ could possibly exist. It follows that $(\emptyset - A)$ must be empty. Hence, 
		$\emptyset - A = \emptyset$
	\end{proof}

\end{document}
