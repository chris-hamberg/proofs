\documentclass[a4paper, 12pt]{article}
\usepackage[utf8]{inputenc}
\usepackage[english]{babel}
\usepackage{amssymb, amsmath, amsthm}
\theoremstyle{plain}
\newtheorem*{theorem*}{Theorem}
\newtheorem{theorem}{Theorem}

\usepackage{mathtools}
\renewcommand\qedsymbol{$\blacksquare$}

\begin{document}
	
	\begin{theorem*}[2.2.20]
		Let A, and B be sets. $(A \cap B) \cup (A \cap \overline{B}) = A$.
	\end{theorem*}
	
	\begin{proof}
		Let $x$ be an element in $(A \cap B) \cup (A \cap \overline{B})$. By definition then, we 
		have $[(x \in A) \land (x \in B)] \lor [(x \in A) \land (x \notin B)]$. By the logical 
		law of distribution for conjunction over disjunction we can factor the term $(x \in A)$ 
		out on the conjunction. Thus, $(x \in A) \land [(x \in B) \lor (x \notin B)]$. By the 
		logical law of negation we have $(x \in A) \land T$ which is equivalent to $x \in A$ by 
		the identity law of logic.
		
		Proving the converse, let $x$ be an element in $A$. That is, $(x \in A)$. By the logical 
		law of identity, $(x \in A) \land T \equiv (x \in A)$. Since $(x \in B) \lor (x \notin B)$ 
		is true by the logical law of negation, we can, by logical equivalence, construct the 
		following statement while retaining the logical identity for the statement $(x \in A)$; that 
		is $(x \in A) \land [(x \in B) \lor (x \notin B)]$. By the logical law of distribution for 
		conjunction over disjunction we have the definition of our original expression, 
		$[(x \in A) \land (x \in B)] \lor [(x \in A) \land (x \notin B)]$. Hence, 
		$x \in [(A \cap B) \cup (A \cap \overline{B})]$.
		
		Because $(A \cap B) \cup (A \cap \overline{B}) \subseteq A$ and 
		$A \subseteq  (A \cap B) \cup (A \cap \overline{B})$ it follows from the definition of set 
		equivalence that $(A \cap B) \cup (A \cap \overline{B}) = A$.
	\end{proof}

\end{document}
