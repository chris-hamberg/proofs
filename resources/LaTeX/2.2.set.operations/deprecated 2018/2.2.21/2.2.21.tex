\documentclass[a4paper, 12pt]{article}
\usepackage[utf8]{inputenc}
\usepackage[english]{babel}
\usepackage{amssymb, amsmath, amsthm}
\theoremstyle{plain}
\newtheorem*{theorem*}{Theorem}
\newtheorem{theorem}{Theorem}

\usepackage{mathtools}
\renewcommand\qedsymbol{$\blacksquare$}

\begin{document}
	
	\begin{theorem*}[2.2.21]
		Let A, B, and C be sets. $A \cup (B \cup C) = (A \cup B) \cup C$, such that set union is 
		associative.
	\end{theorem*}
	
	\begin{proof}
		Let $x$ be an element in $A \cup (B \cup C)$. The logical definition being \newline 
		$(x \in A) \lor [(x \in B) \lor (x \in C)]$. It trivially follows from the associative 
		law for logical disjunction that $[(x \in A) \lor (x \in B)] \lor (x \in C)$. Hence, $x$ 
		is an element of $(A \cup B) \cup C$.
		
		In the converse case, let $x$ be an element of $(A \cup B) \cup C$. The logical definition 
		being $[(x \in A) \lor (x \in B)] \lor (x \in C)$. It trivially follows from the associative 
		law for logical disjunction that $(x \in A) \lor [(x \in B) \lor (x \in C)]$. Hence, $x$ is 
		an element of $A \cup (B \cup C) $.
		
		Since $A \cup (B \cup C) \subseteq (A \cup B) \cup C$ and 
		$(A \cup B) \cup C \subseteq A \cup (B \cup C)$ it follows immediately from the definition 
		for set equality that \newline $A \cup (B \cup C) = (A \cup B) \cup C$. Thus, the union of 
		three sets is indeed associative.
	\end{proof}

\end{document}
