\documentclass[a4paper, 12pt]{article}
\usepackage[utf8]{inputenc}
\usepackage[english]{babel}
\usepackage{amssymb, amsmath, amsthm}
\theoremstyle{plain}
\newtheorem*{theorem*}{Theorem}
\newtheorem{theorem}{Theorem}

\usepackage{mathtools}
\renewcommand\qedsymbol{$\blacksquare$}

\begin{document}
	
	\begin{theorem*}[2.2.35]
		Let A, and B be sets. $A \oplus B = (A \cup B) - (A \cap B)$.
	\end{theorem*}
	
	\begin{proof}
		Let $x$ be an element in $A \oplus B$. This is logically defined as \newline 
		$[(x \in A) \land (x \notin B)] \lor [(x \notin A) \land (x \in B)]$. By the definition 
		for set difference $x$ is an element in $(A - B) \cup (B - A)$ which by Theorem 2.2.19 
		can be expressed as 
		$(A \cap \overline{B}) \cup (B \cap \overline{A})$. By Theorem 2.2.23, which proves that 
		set unions are distributive over set intersections, the following expression is equivalent 
		$[A \cup (B \cap \overline{A})] \cap [\overline{B} \cup (B \cap \overline{A})]$. Again, 
		by Theorem 2.2.23, we have 
		$[(A \cup B) \cap (A \cup \overline{A})] \cap [(\overline{B} \cup B) \cap (\overline{B} 
			\cup \overline{A})]$. 
		Let the logical definition for this expression be represented in two propositional variables 
		$p \land q$ such that: \newline \indent 
		$p \equiv \{[(x \in A) \lor (x \in B)] \land [(x \in A) \lor (x \notin A)]\} 
		\newline \indent q \equiv \{[(x \notin B) \lor (x \in B)] \land 
		[(x \notin B) \lor (x \notin A)]\}$. 
		\newline By the logical law of identity $p \equiv (x \in A) \lor (x \in B)$, and 
		\newline $q \equiv (x \notin B) \lor (x \notin A)$. So we have 
		$[(x \in A) \lor (x \in B)] \land [(x \notin B) \lor (x \notin A)]$. Applying DeMorgans law 
		to the right-hand side of the conjunction we get \newline 
		$[(x \in A) \lor (x \in B)] \land \lnot [(x \in B) \land (x \in A)]$. Then by definition, 
		$x$ is an element in $(A \cup B) \cap \overline{(A \cap B)}$, and according to Theorem 
		2.2.19 $x$ is an element in $(A \cup B) - (A \cap B)$.
		
		Proving the converse case is trivial. Let $x$ be an element in \newline 
		$(A \cup B) - (A \cap B)$. By Theorem 2.2.19 $x$ is an element in 
		$(A \cup B) \cap \overline{(A \cap B)}$. By definition we have, 
		$[(x \in A) \lor (x \in B)] \land \lnot [(x \in B) \land (x \in A)]$. By DeMorgans law, 
		$[(x \in A) \lor (x \in B)] \land [(x \notin B) \lor (x \notin A)]$. Let this logical formula 
		be represented in two propositional variables $p \land q$ such that 
		$p \equiv (x \in A) \lor (x \in B)$ and $q \equiv (x \notin B) \lor (x \notin A)$. By the 
		logical law of identity $p \land T \equiv T$, and $q \land T \equiv T$. By the negation law 
		of logic, $(x \notin B) \lor (x \in B) \equiv T$, and $(x \in A) \lor (x \notin A) \equiv T$. 
		Therefore, \newline \indent 
		$p \equiv \{[(x \in A) \lor (x \in B)] \land [(x \in A) \lor (x \notin A)]\} 
		\newline \indent q \equiv \{[(x \notin B) \lor (x \in B)] \land 
		[(x \notin B) \lor (x \notin A)]\}$. 
		\newline By definition, $x$ is an element in 
		$[(A \cup B) \cap (A \cup \overline{A})] \cap 
		[(\overline{B} \cup B) \cap (\overline{B} \cup \overline{A})]$. By Theorem 2.2.23, 
		factoring $A$ out of the left-hand side of the intersection, and factoring $\overline{B}$ 
		out of the right-hand side of the intersection, the following expression is equivalent 
		$[A \cup (B \cap \overline{A})] \cap [\overline{B} \cup (B \cap \overline{A})]$. Again, by 
		Theorem 2.2.23, factoring $(B \cap \overline{A})$ out of the intersection we have the following 
		equivalent expression $(A \cap \overline{B}) \cup (B \cap \overline{A})$. Which, by 
		Theorem 2.2.19 is equivalently stated as $(A - B) \cup (B - A)$, defined by 
		$[(x \in A) \land (x \notin B)] \lor [(x \notin A) \land (x \in B)]$. But this is the formal 
		definition for the symmetric difference of sets, so $x$ must be an element in $A \oplus B$.
		
		Since $A \oplus B \subseteq (A \cup B) - (A \cap B)$ and 
		$(A \cup B) - (A \cap B) \subseteq A \oplus B$, it follows immediately that 
		$A \oplus B = (A \cup B) - (A \cap B)$.
	\end{proof}

\end{document}
