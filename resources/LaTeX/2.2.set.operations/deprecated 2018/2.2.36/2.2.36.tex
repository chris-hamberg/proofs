\documentclass[a4paper, 12pt]{article}
\usepackage[utf8]{inputenc}
\usepackage[english]{babel}
\usepackage{amssymb, amsmath, amsthm}
\theoremstyle{plain}
\newtheorem*{theorem*}{Theorem}
\newtheorem{theorem}{Theorem}

\usepackage{mathtools}
\renewcommand\qedsymbol{$\blacksquare$}

\begin{document}
	
	\begin{theorem*}[2.2.36]
		Let A, and B be sets. $A \oplus B = (A - B) \cup (B - A)$
	\end{theorem*}
	
	\begin{proof}
		Let $x$ be an element in $A \oplus B$. Then by the definition for symmetric 
		\newline difference $[(x \in A) \land (x \notin B)] \lor [(x \notin A) \land (x \in B)]$. 
		Because logical \newline conjunction is associative, the statement is equivalent to 
		\newline $[(x \in A) \land (x \notin B)] \lor [(x \in B) \land (x \notin A)]$. According 
		to the definition for set difference, and by the definition for set union, it follows 
		that $x$ is an element in $(A - B) \cup (B - A)$.
		
		Proving the converse, suppose $x$ were an element in $(A - B) \cup (B - A)$. The logical 
		definition being $[(x \in A) \land (x \notin B)] \lor [(x \in B) \land (x \notin A)]$. 
		By the associative law for logical conjunction the following statement is equivalent 
		$[(x \in A) \land (x \notin B)] \lor [(x \notin A) \land (x \in B)]$. Since this is the 
		definition for symmetric difference, $x$ is an element in $A \oplus B$.
		
		Since $A \oplus B \subseteq (A - B) \cup (B - A)$ and 
		$(A - B) \cup (B - A) \subseteq A \oplus B$ it immediately follows from the definition of 
		set equality that \newline $A \oplus B = (A - B) \cup (B - A)$.
	\end{proof}

\end{document}
