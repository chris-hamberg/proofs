\documentclass[a4paper, 12pt]{article}
\usepackage[utf8]{inputenc}
\usepackage[english]{babel}
\usepackage{amssymb, amsmath, amsthm}
\theoremstyle{plain}
\newtheorem*{theorem*}{Theorem}
\newtheorem{theorem}{Theorem}

\usepackage{mathtools}
\renewcommand\qedsymbol{$\blacksquare$}

\begin{document}
	
	\begin{theorem*}[2.2.37b]
		Let A be a subset of the universal set U. $A \oplus \emptyset = A$.
	\end{theorem*}
	
	\begin{proof}
		By Theorem 2.2.35, $A \oplus \emptyset = (A \cup \emptyset) - (A \cap \emptyset)$. By set 
		identity, and by set domination, that is $A - \emptyset$, which by Theorem 2.2.19 means 
		$A \cap \overline{\emptyset}$. Because $\overline{\emptyset} = U$, we have by set identity 
		that $A \oplus \emptyset = A$.
	\end{proof}

\end{document}
