\documentclass[a4paper, 12pt]{article}
\usepackage[utf8]{inputenc}
\usepackage[english]{babel}
\usepackage{amssymb, amsmath, amsthm}
\theoremstyle{plain}
\newtheorem*{theorem*}{Theorem}
\newtheorem{theorem}{Theorem}

\usepackage{mathtools}
\renewcommand\qedsymbol{$\blacksquare$}

\begin{document}
	
	\begin{theorem*}[2.2.5]
		Let A be a subset of U. $\overline{\overline{A}} = A$.
	\end{theorem*}
	
	\begin{proof}
		Suppose $x$ is an element in $\overline{\overline{A}}$. By the definition of set complementation 
		$x \in \lnot \overline{A}$, and of course by the same reasoning $x \in \lnot ( \lnot A)$. By the 
		logical law of double negation $x \in A$. Thus it follows directly that 
		$\overline{\overline{A}} = A$.
	\end{proof}

\end{document}
