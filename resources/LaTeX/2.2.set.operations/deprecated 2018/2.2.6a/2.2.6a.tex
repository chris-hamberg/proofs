\documentclass[a4paper, 12pt]{article}
\usepackage[utf8]{inputenc}
\usepackage[english]{babel}
\usepackage{amssymb, amsmath, amsthm}
\theoremstyle{plain}
\newtheorem*{theorem*}{Theorem}
\newtheorem{theorem}{Theorem}

\usepackage{mathtools}
\renewcommand\qedsymbol{$\blacksquare$}

\begin{document}
	
	\begin{theorem*}[2.2.6a]
		Let A be a set. The set identity for A is $A \cup \emptyset = A$.
	\end{theorem*}
	
	\begin{proof}
		Let $x$ be an element in $A \cup \emptyset$. By the definition of set union \newline 
		$(x \in A) \lor (x \in \emptyset)$. But $x \in \emptyset$ is $\bot$ because $\emptyset$ is 
		empty. Therefore $x$ must be in $A$. It follows directly that $A \cup \emptyset = A$. Thus 
		proves the set identity law for set union.
	\end{proof}

\end{document}
