\documentclass[preview]{standalone}
\usepackage{amssymb, amsthm}
\usepackage{mathtools}
\usepackage{bm}


\newtheorem{lemma}{Lemma}
\renewcommand\qedsymbol{$\blacksquare$}


\begin{document}


\begin{lemma}[\textbf{2303}]
    Let \bm{$\lambda$} be a real number, 
    such that \bm{$\big \lfloor \lambda \big \rfloor + \epsilon = \lambda$}.
    \begin{equation*}
        \bm{
            \big \lfloor 3 \lambda \big \rfloor 
                =
            \big \langle 
                3 \lambda - 3 \epsilon 
            \big \rangle
                +
            \big \lfloor 3 \epsilon \big \rfloor 
        }
    \end{equation*}
\end{lemma}

\begin{proof}
    Given the real number \bm{$\lambda$}, 
    by the properties for floor functions
    there exists a real number \bm{$\epsilon$}
    and an integer \bm{$\lambda - \epsilon$} such that 
    \bm{$\lfloor \lambda \rfloor = \lambda - \epsilon$}.
    Thus, \bm{$\lambda = \lambda - \epsilon + \epsilon$}.
    By the identity \bm{$\lambda$},
    \begin{equation*}
        \Big \lfloor 3 \lambda \Big \rfloor 
            =
        \Big \lfloor 
                3 
                \big \langle 
                    \lambda - \epsilon + \epsilon 
                \big \rangle 
        \Big \rfloor
            =
        \Big \lfloor 
            3 \lambda - 3 \epsilon + 3 \epsilon 
        \Big \rfloor
    \end{equation*}
    \bm{$3 \lambda - 3 \epsilon$} is an integer 
    since \bm{$\lambda - \epsilon$} is an integer, 
    and integers are closed under multiplication.
    Thus, by Lemma 2301
    \begin{equation*} 
        \Big \lfloor 
                3 \lambda - 3 \epsilon + 3 \epsilon 
        \Big \rfloor
            =
        \Big \langle 3 \lambda - 3 \epsilon \Big \rangle
            + 
        \Big \lfloor            
            3 \epsilon 
        \Big \rfloor
    \end{equation*}
    $\therefore \text{\space} \bm{
        \big \lfloor 3 \lambda \big \rfloor 
            =
        \big \langle 
            3 \lambda - 3 \epsilon 
        \big \rangle
            +
        \big \lfloor 3 \epsilon \big \rfloor
    }$.
\end{proof}


\end{document}