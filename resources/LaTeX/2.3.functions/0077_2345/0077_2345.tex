\documentclass[preview]{standalone}
\usepackage{amssymb, amsthm}
\usepackage{mathtools}
\usepackage{bm}


\newtheorem{theorem}{Theorem}
\renewcommand\qedsymbol{$\blacksquare$}


\begin{document}


\begin{theorem}[\textbf{2345}]
    Let \bm{$\lambda$} be a real number. 
    \begin{equation*}
        \bm{
            \big \langle \lambda - 1 \big \rangle
                < 
            \big \langle \lfloor \lambda \rfloor \big \rangle
                \leq 
            \big \langle \lambda \big \rangle
                \leq
            \big \langle \lceil \lambda \rceil \big \rangle
                <
            \big \langle \lambda + 1 \big \rangle
        }
    \end{equation*}
\end{theorem}

\begin{proof}
    Let \bm{$\lambda - \lfloor \lambda \rfloor = \sigma$}, 
    and let \bm{$\lceil \lambda \rceil - \lambda = \epsilon$}.
    By the additive and multiplicative compatibility laws from the order axioms,
    and by the properties for floor functions,
    there exists an integer \bm{$\lfloor \lambda \rfloor = \zeta$} such that
    \begin{equation*}
        \Big \langle \zeta \Big \rangle 
            \leq 
        \Big \langle \lambda \Big \rangle
            <
        \Big \langle \zeta + 1 \Big \rangle
            \equiv
    \end{equation*}
    \begin{equation*}
        \Big \langle \lfloor \lambda \rfloor \Big \rangle
            \leq 
        \Big \langle \lfloor \lambda \rfloor + \sigma \Big \rangle 
            < 
        \Big \langle \lfloor \lambda \rfloor + 1 \Big \rangle
            \equiv
    \end{equation*}
    \begin{equation*}
        \bigg[
            \Big \langle 0 \Big \rangle 
                \leq 
            \Big \langle \sigma \Big \rangle 
                < 
            \Big \langle 1 \Big \rangle
        \bigg]
            \equiv
        \bigg[
            \Big \langle -1 \Big \rangle
                <
            \Big \langle - \sigma \Big \rangle
                \leq
            \Big \langle 0 \Big \rangle
        \bigg]
    \end{equation*}
    Also, by the properties for ceiling functions,
    there exists an integer \bm{$\lceil \lambda \rceil = \xi$}.
    From which, by similar reasoning as to that of above,
    we can derive
    \begin{equation*}
        \Big \langle 0 \Big \rangle
            \leq
        \Big \langle \epsilon \Big \rangle
            <
        \Big \langle 1 \Big \rangle
    \end{equation*}
    Thus, combining both results by transitivity from the order axioms
    \begin{equation*}
        \Big \langle -1 \Big \rangle
            <
        \Big \langle - \sigma \Big \rangle
            \leq
        \Big \langle 0 \Big \rangle
            \leq
        \Big \langle \epsilon \Big \rangle
            <
        \Big \langle 1 \Big \rangle
    \end{equation*}
    Again, by the additive compatibility law from the order axioms, 
    and by the identities for 
    \bm{$\lfloor \lambda \rfloor$} and \bm{$\lceil \lambda \rceil$},
    that is
    \begin{equation*}
        \Big \langle \lambda - 1 \Big \rangle
            <
        \Big \langle \lambda - \sigma \Big \rangle
            \leq
        \Big \langle \lambda + 0 \Big \rangle
            \leq
        \Big \langle \lambda + \epsilon \Big \rangle
            <
        \Big \langle \lambda + 1 \Big \rangle
            \equiv
    \end{equation*}
    \begin{equation*}
        \Big \langle \lambda - 1 \Big \rangle
            <
        \Big \langle \lfloor \lambda \rfloor \Big \rangle
            \leq
        \Big \langle \lambda \Big \rangle
            \leq
        \Big \langle \lceil \lambda \rceil \Big \rangle
            <
        \Big \langle \lambda + 1 \Big \rangle
    \end{equation*}
\end{proof}


\end{document}