\documentclass[preview]{standalone}
\usepackage{amssymb, amsthm}
\usepackage{mathtools}
\usepackage{bm}


\newtheorem{theorem}{Theorem}
\renewcommand\qedsymbol{$\blacksquare$}


\begin{document}


\begin{theorem}[\textbf{2366}]
    Let \bm{$\sigma$} be the invertible function 
    \bm{$\sigma : \Theta \rightarrow \Omega$}, 
    and let \bm{$\phi$} be the invertible function 
    \raggedright \bm{$\phi : \Phi \rightarrow \Theta$}. 
    \begin{equation*}
        \text{The inverse of the composition }
        \bm{\sigma \circ \phi}
        \text{ is given by }
    \end{equation*}
    \begin{equation*}
        \bm{
            \big \langle \sigma \circ \phi \big \rangle ^{-1} 
                = \phi ^{-1} \circ \sigma ^{-1}
            }
    \end{equation*}
\end{theorem}

\begin{proof}
    By Theorem 2329a and Theorem 2329b, and by the definition for bijective functions, 
    \bm{$\sigma \circ \phi$} is invertible. 
    Thus, 
    \begin{equation*}
        \big \langle \sigma \circ \phi \big \rangle ^{-1} 
            \circ 
        \big \langle \sigma \circ \phi \big \rangle 
            = 
        \iota_{\alpha}
    \end{equation*}
    What remains to be determined is whether 
    \bm{$
        \big \langle \phi ^{-1} \circ \sigma ^{-1} \big \rangle
            \circ 
        \big \langle \sigma \circ \phi \big \rangle
            = 
        \iota_{\alpha}$
    }. 
    Let \bm{$\alpha$} be an element in the domain \bm{$\Phi$} such that 
    \begin{equation*}
        \big[
            \big \langle \phi ^{-1} \circ \sigma ^{-1} \big \rangle 
                \circ 
            \big \langle \sigma \circ \phi \big \rangle
        \big] 
            \big[ \alpha \big] 
            = 
        \alpha
    \end{equation*}
    By the definition for the composition of functions, 
    that is 
    \begin{equation*}
        \phi ^{-1} \text{\space} \big[ 
            \sigma ^{-1} \text{\space} \big[ 
                \sigma \text{\space} \big[ 
                    \phi \text{\space} \big[
                        \alpha \text{\space}
                    \big]
                \big]
            \big]
        \big]
             = 
        \alpha
    \end{equation*}
    Clearly, 
    \bm{$
        \big \langle \phi ^{-1} \circ \sigma ^{-1} \big \rangle
            \circ 
        \big \langle \sigma \circ \phi \big \rangle
            = 
        \iota_{\alpha}$}
    \\ \\
    $\therefore \text{ \emph{the inverse of the composition} } \bm{
        \sigma \circ \phi 
        \text{ \emph{is given by} }
        \big \langle \sigma \circ \phi \big \rangle ^{-1} 
            = 
        \phi^{-1} \circ \sigma^{-1}
    }$.
\end{proof}


\end{document}