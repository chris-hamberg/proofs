\documentclass[preview]{standalone}
\usepackage{amssymb, amsthm}
\usepackage{mathtools}
\usepackage{bm}


\newtheorem{theorem}{Theorem}
\renewcommand\qedsymbol{$\blacksquare$}


\begin{document}


\begin{theorem}[\textbf{2371a}]
    Let \bm{$\lambda$} be a positive real number. 
    \begin{equation*}
        \bm{
            \bigg \lfloor 
                \sqrt{ \strut \big \lfloor \lambda \big \rfloor } 
            \bigg \rfloor 
                = 
            \bigg \lfloor \sqrt{ \strut \lambda} \bigg \rfloor
        }
    \end{equation*}
\end{theorem}


\begin{proof}
    By the properties for floor functions,
    there exists an integer 
    \bm{$
        \big \lfloor \sqrt{ 
            \lfloor \lambda \rfloor 
        } \big \rfloor
    $} such that 
    \bm{$
        \big \lfloor \sqrt{ \lambda} \big \rfloor
            = 
        \big \lfloor \sqrt{ 
            \lfloor \lambda \rfloor 
        } \big \rfloor 
    $},
    if and only if
    \begin{equation*}
        \bigg \langle 
            \bigg \lfloor 
                \sqrt{ 
                    \big \lfloor \lambda \big \rfloor
                }
            \bigg \rfloor
        \bigg \rangle
            \leq
        \bigg \langle
            \sqrt{ \strut \lambda }
        \bigg \rangle
            <
        \bigg \langle 
            \bigg \lfloor 
                \sqrt{ 
                    \big \lfloor \lambda \big \rfloor
                }
            \bigg \rfloor
                +
            1
        \bigg \rangle
    \end{equation*}
    By the multiplicative compatibility law from the order axioms, that is
    \begin{equation*}
        \bigg \langle 
            \bigg \lfloor 
                \sqrt{ 
                    \big \lfloor \lambda \big \rfloor
                }
            \bigg \rfloor
        \bigg \rangle
            ^2
            \leq
        \bigg \langle
            \lambda
        \bigg \rangle
            <
        \bigg \langle 
            \bigg \lfloor 
                \sqrt{ 
                    \big \lfloor \lambda \big \rfloor
                }
            \bigg \rfloor
                +
            1
        \bigg \rangle
        ^2
    \end{equation*}
    \bm{$\big \lfloor \lambda \big \rfloor$} 
    is the largest integer that is less than or equal \bm{$\lambda$},
    so by the definition of the floor function, 
    \bm{$\big \lfloor \lambda \big \rfloor \leq \lambda$}. 
    Also,
    \bm{$\big \lfloor \sqrt{ \lfloor \lambda \rfloor } \big \rfloor ^2$}
    is an integer by the defintion of floor functions,
    since integers are closed under multiplication. 
    Hence,
    \bm{$
        \big \lfloor \sqrt{ \lfloor \lambda \rfloor } \big \rfloor ^2
            \leq
        \big \lfloor \lambda \big \rfloor
    $}.
    So by the transitivity law from the order axioms,
    \begin{equation*}
        \bigg \langle 
            \bigg \lfloor 
                \sqrt{ 
                    \big \lfloor \lambda \big \rfloor
                }
            \bigg \rfloor
        \bigg \rangle
            ^2
            \leq
        \bigg \langle
            \big \lfloor \lambda \big \rfloor
        \bigg \rangle
            \leq
        \bigg \langle
            \lambda
        \bigg \rangle
            <
        \bigg \langle 
            \bigg \lfloor 
                \sqrt{ 
                    \big \lfloor \lambda \big \rfloor
                }
            \bigg \rfloor
                +
            1
        \bigg \rangle
            ^2
            \equiv
    \end{equation*}
    \begin{equation*}
        \bigg \langle 
            \bigg \lfloor 
                \sqrt{ 
                    \big \lfloor \lambda \big \rfloor
                }
            \bigg \rfloor
        \bigg \rangle
            ^2
            \leq
        \bigg \langle
            \big \lfloor \lambda \big \rfloor
        \bigg \rangle
            <
        \bigg \langle 
            \bigg \lfloor 
                \sqrt{ 
                    \big \lfloor \lambda \big \rfloor
                }
            \bigg \rfloor
                +
            1
        \bigg \rangle
            ^2
    \end{equation*}
    By the multiplicative compatibility law from the order axioms,
    the following is an equivalent statement,
    \begin{equation*}
        \bigg \langle 
            \bigg \lfloor 
                \sqrt{ 
                    \big \lfloor \lambda \big \rfloor
                }
            \bigg \rfloor
        \bigg \rangle
            \leq
        \bigg \langle
            \sqrt{ \strut \big \lfloor \lambda \big \rfloor }
        \bigg \rangle
            <
        \bigg \langle 
            \bigg \lfloor 
                \sqrt{ 
                    \big \lfloor \lambda \big \rfloor
                }
            \bigg \rfloor
                +
            1
        \bigg \rangle
    \end{equation*}
    $\therefore \text{\space} \bm{
        \bigg \lfloor 
            \sqrt{ \strut \big \lfloor \lambda \big \rfloor } 
        \bigg \rfloor 
            = 
        \bigg \lfloor \sqrt{ \strut \lambda} \bigg \rfloor
    }$, by the properties of floor functions.
\end{proof}


\end{document}