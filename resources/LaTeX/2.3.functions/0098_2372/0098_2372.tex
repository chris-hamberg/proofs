\documentclass[preview]{standalone}
\usepackage{amssymb, amsthm}
\usepackage{mathtools}
\usepackage{bm}


\newtheorem{theorem}{Theorem}
\renewcommand\qedsymbol{$\blacksquare$}


\begin{document}


\begin{theorem}[\textbf{2372}]
    Let \bm{$\lambda$} be a real number, such that
    \bm{$\big \lfloor \lambda \big \rfloor + \epsilon = \lambda$}.
    \begin{equation*}
        \bm{
            \bigg \lfloor 3 \lambda \bigg \rfloor
                = 
            \bigg \lfloor \lambda \bigg \rfloor
                + 
            \bigg \lfloor \lambda + \frac{1}{3} \bigg \rfloor
                + 
            \bigg \lfloor \lambda + \frac{2}{3} \bigg \rfloor
        }
    \end{equation*}
\end{theorem}

\begin{proof}
    By cases. By Lemma 2304, it is sufficent to prove
    \begin{equation*}
        \bigg \lfloor 3 \epsilon \bigg \rfloor
            =
        \Bigg \langle 
            \mathrm{A} 
                = 
            \bigg \lfloor \epsilon \bigg \rfloor 
        \Bigg \rangle
            +
        \Bigg \langle
            \Lambda
                = 
            \bigg \lfloor \epsilon + \frac{1}{3} \bigg \rfloor 
        \Bigg \rangle
            +
        \Bigg \langle
            \Delta
                = 
            \bigg \lfloor \epsilon + \frac{2}{3} \bigg \rfloor
        \Bigg \rangle
    \end{equation*}
    Let \bm{$p$} be the propostion: \bm{$\Delta \geq \Lambda \geq \mathrm{A} \geq 0$}.
    The proof for which is trivial.
    There are three cases:
    \\ \\
    \indent \indent \bm{$(i)$} \space \space
    $
        \Big \langle 0 \Big \rangle 
            \le 
        \Big \langle \epsilon \Big \rangle
            < 
        \Big \langle \frac{1}{3} \Big \rangle
    $
    \\ \\
    \indent \indent \bm{$(ii)$} \space
    $
        \Big \langle \frac{1}{3} \Big \rangle 
            \le 
        \Big \langle \epsilon \Big \rangle
            < 
        \Big \langle \frac{2}{3} \Big \rangle
    $ 
    \\ \\
    \indent \indent \bm{$(iii)$}
    $ 
        \Big \langle \frac{2}{3} \Big \rangle 
            \le 
        \Big \langle \epsilon \Big \rangle 
            < 
        \Big \langle 1 \Big \rangle
    $
    \\ \\
    \bm{$(i)$}
    Since \bm{$p$}, 
    \bm{$\Delta$} is sufficient for inferring \bm{$\mathrm{A}$}, 
    and \bm{$\Lambda$} in this case.
    By the additive compatibility law from the order axioms, 
    and by the law of transitivity from the order axioms,
    \begin{equation*}
        \bigg[
            \bigg \langle \frac{0}{3} + \frac{2}{3} \bigg \rangle
                \leq 
            \bigg \langle \epsilon + \frac{2}{3} \bigg \rangle 
                < 
            \bigg \langle \frac{1}{3} + \frac{2}{3} \bigg \rangle
        \bigg]
            \equiv
        \bigg[
            \bigg \langle 0 \bigg \rangle
                \leq 
            \bigg \langle \epsilon + \frac{2}{3} \bigg \rangle 
                < 
            \bigg \langle 1 \bigg \rangle
        \bigg]
    \end{equation*}
    Thus, \bm{$\Delta = 0$}, by the properties of floor functions. 
    Hence, \bm{$\mathrm{A} + \Lambda + \Delta = 0$}, by \bm{$p$}. 
    Also, by the multiplicative compatibility law from the order axioms,
    and by the properties of floor functions,
    \begin{equation*}
        \bigg[
            \bigg \langle 3 \cdot 0 \bigg \rangle
                \leq
            \bigg \langle 3 \cdot \epsilon \bigg \rangle
                <
            \bigg \langle 3 \cdot \frac{1}{3} \bigg \rangle
        \bigg]
            \equiv
        \bigg[
            \big \lfloor 3 \epsilon \big \rfloor
                =
            0
        \bigg]
    \end{equation*}
    $\therefore \text{\space} \bm{
        \big \lfloor 3 \epsilon \big \rfloor
            =
        \mathrm{A} + \Lambda + \Delta
    }$, by the identity \bm{$0$}, in this case.
    \\ \\
    \bm{$(ii)$}
    Since \bm{$p$}, \bm{$\Lambda$} is sufficent for inferring \bm{$\mathrm{A}$} in this case.
    By the additive compatibility law from the order axioms,
    and by the law of transitivity,
    \begin{equation*}
        \bigg[
            \bigg \langle \frac{1}{3} + \frac{1}{3} \bigg \rangle
                \leq 
            \bigg \langle \epsilon + \frac{1}{3} \bigg \rangle 
                < 
            \bigg \langle \frac{2}{3} + \frac{1}{3} \bigg \rangle
        \bigg]
            \equiv
        \bigg[
            \bigg \langle 0 \bigg \rangle
                \leq 
            \bigg \langle \epsilon + \frac{1}{3} \bigg \rangle 
                < 
            \bigg \langle 1 \bigg \rangle
        \bigg]
    \end{equation*}
    Thus, \bm{$\Lambda = 0$}, by the properties of floor functions.
    Hence, \bm{$\mathrm{A} + \Lambda = 0$}, by \bm{$p$}. 
    Now, for \bm{$\Delta$}, 
    by the additive compatibility law from the order axioms,
    and by the law of transitivity from the order axioms,
    \pagebreak
    \begin{equation*}
        \bigg[
            \bigg \langle \frac{1}{3} + \frac{2}{3} \bigg \rangle
                \leq 
            \bigg \langle \epsilon + \frac{2}{3} \bigg \rangle 
                < 
            \bigg \langle \frac{2}{3} + \frac{2}{3} \bigg \rangle
        \bigg]
            \equiv
        \bigg[
            \bigg \langle 1 \bigg \rangle
                \leq 
            \bigg \langle \epsilon + \frac{2}{3} \bigg \rangle 
                < 
            \bigg \langle 1 + 1 \bigg \rangle
        \bigg]
    \end{equation*}
    Thus, \bm{$\Delta = 1$}, by the properties of floor functions. 
    Hence, \bm{$\mathrm{A} + \Lambda + \Delta = 1$}.
    Also, by the multiplicative compatibility law from the order axioms,
    and by the properties of floor functions,
    \begin{equation*}
        \bigg[
            \bigg \langle 3 \cdot \frac{1}{3} \bigg \rangle
                \leq
            \bigg \langle 3 \cdot \epsilon \bigg \rangle
                <
            \bigg \langle 3 \cdot \frac{2}{3} \bigg \rangle
        \bigg]
            \equiv
    \end{equation*}
    \begin{equation*}
        \bigg[
            \Big \langle 1 \Big \rangle
                \leq
            \Big \langle 3 \epsilon \Big \rangle
                <
            \Big \langle 1 + 1 \Big \rangle
        \bigg]
            \equiv
        \bigg[
            \big \lfloor 3 \epsilon \big \rfloor
                =
            1
        \bigg]
    \end{equation*}
    $\therefore \text{\space} \bm{
        \lfloor 3 \epsilon \rfloor
            =
        \mathrm{A} + \Lambda + \Delta
    }$, by the identity \bm{$1$}, in this case.
    \\ \\
    \bm{$(iii)$} 
    \bm{$\mathrm{A} = 0$} can be inferred from the law of transitivity from the order axioms,
    and by the properties of floor functions, in this case.
    For \bm{$\Lambda$},
    by the additive compatibility law from the order axioms,
    and by the law of transitivity from the order axioms,
    \begin{equation*}
        \bigg[
            \bigg \langle \frac{2}{3} + \frac{1}{3} \bigg \rangle
                \leq 
            \bigg \langle \epsilon + \frac{1}{3} \bigg \rangle 
                < 
            \bigg \langle \frac{3}{3} + \frac{1}{3} \bigg \rangle
        \bigg]
            \equiv
        \bigg[
            \bigg \langle 1 \bigg \rangle
                \leq 
            \bigg \langle \epsilon + \frac{1}{3} \bigg \rangle 
                < 
            \bigg \langle 1 + 1 \bigg \rangle
        \bigg]
    \end{equation*}
    Thus, \bm{$\Lambda = 1$}, 
    by the properties of floor functions.
    For \bm{$\Delta$},
    by the additive compatibility law from the order axioms,
    and by the law of transitivity from the order axioms,
    \begin{equation*}
        \bigg[
            \bigg \langle \frac{2}{3} + \frac{2}{3} \bigg \rangle
                \leq 
            \bigg \langle \epsilon + \frac{2}{3} \bigg \rangle 
                < 
            \bigg \langle \frac{3}{3} + \frac{2}{3} \bigg \rangle
        \bigg]
            \equiv
        \bigg[
            \bigg \langle 1 \bigg \rangle
                \leq 
            \bigg \langle \epsilon + \frac{2}{3} \bigg \rangle 
                < 
            \bigg \langle 1 + 1 \bigg \rangle
        \bigg]
    \end{equation*}
    Thus, \bm{$\Delta = 1$}, 
    by the properties of floor functions.
    Hence, \bm{$\mathrm{A} + \Lambda + \Delta = 2$}.
    Also, by the multiplicative compatibility law from the order axioms,
    and by the properties of floor functions,
    \begin{equation*}
        \bigg[
            \bigg \langle 3 \cdot \frac{2}{3} \bigg \rangle
                \leq
            \bigg \langle 3 \cdot \epsilon \bigg \rangle
                <
            \bigg \langle 3 \cdot 1 \bigg \rangle
        \bigg]
            \equiv
    \end{equation*}
    \begin{equation*}
        \bigg[
            \bigg \langle 2 \bigg \rangle
                \leq
            \bigg \langle 3 \epsilon \bigg \rangle
                <
            \bigg \langle 2 + 1 \bigg \rangle
        \bigg]
            \equiv
        \bigg[
            \big \lfloor 3 \epsilon \big \rfloor
                =
            2
        \bigg]
    \end{equation*}
    $\therefore \text{\space} \bm{
        \lfloor 3 \epsilon \rfloor
            =
        \mathrm{A} + \Lambda + \Delta
    }$, by the identity \bm{$2$}, in this case.
    \\ \\
    This completes the proof.
\end{proof}


\end{document}