\documentclass[a4paper, 12pt]{article}
\usepackage[utf8]{inputenc}
\usepackage[english]{babel}
\usepackage{amssymb, amsmath, amsthm}
\theoremstyle{plain}
\newtheorem*{theorem*}{Theorem}
\newtheorem{theorem}{Theorem}

\usepackage{mathtools}
\renewcommand\qedsymbol{$\blacksquare$}

\begin{document}
	
	\begin{theorem*}[2.3.29b]
		Let f be a function $f: B \implies C$, and let g be a function $g: A \implies B$. If both f 
		and g are surjective, then $f \circ g$ is surjective.
	\end{theorem*}
	
	\begin{proof}
		Let $C$ be the domain of discourse. By the hypothesis, and by the definition for surjective 
		functions, the following universally quantified \newline statement must be true, 
		$\forall c \exists b (f(b) = c)$. Note that $g(x)$ is in the domain of $f$, for every $x$ 
		in the domain of $g$, by the definition of $g$. It immediately follows from the general 
		definition for functions that $\forall c \exists a (f(g(a)) = c)$ must be a logically 
		equivalent universal quantification. Since the composition of \newline functions 
		$(f \circ g)(x)$ is defined by $f(g(x))$, it follows directly from the \newline hypothesis 
		that $f \circ g$ is surjective, by the definition for surjective functions.  
	\end{proof}

\end{document}
