\documentclass[a4paper, 12pt]{article}
\usepackage[utf8]{inputenc}
\usepackage[english]{babel}
\usepackage{amssymb, amsmath, amsthm}
\theoremstyle{plain}
\newtheorem*{theorem*}{Theorem}
\newtheorem{theorem}{Theorem}

\usepackage{mathtools}
\renewcommand\qedsymbol{$\blacksquare$}
\DeclarePairedDelimiter{\floor}{\lfloor}{\rfloor}
\DeclarePairedDelimiter{\ceil}{\lceil}{\rceil}

\begin{document}
	
	\begin{theorem*}[2.3.45]
		Let x be a real number. \newline $(x - 1) < \floor{x} \le x \le \ceil{x} < (x + 1)$.
	\end{theorem*}
	
	\begin{proof}
		Notice that $\epsilon = x - \floor{x}$, so $(0 \le \epsilon < 1)$. It is important to note 
		too, that multiplying this inequality by $-1$ on every side yields 
		$(0 \ge -\epsilon > -1) = (-1 < -\epsilon \le 0)$. Finally, note that 
		$\sigma = \ceil{x} - x$, so $(0 \le \sigma < 1)$. But these inequalities together state 
		that $-1 < -\epsilon \le 0 \le \sigma < 1$. Since this inequality is true, by adding $x$ to 
		every side we find that the following statement is also true: 
		$(x - 1) < \floor{x} \le x \le \ceil{x} < (x + 1)$.
	\end{proof}

\end{document}
