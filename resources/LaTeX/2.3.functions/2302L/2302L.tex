\documentclass[preview]{standalone}
\usepackage{amssymb, amsthm}
\usepackage{mathtools}
\usepackage{bm}
\usepackage{xcolor}


\newtheorem*{lemma*}{Lemma}
\renewcommand\qedsymbol{$\blacksquare$}


\begin{document}


\begin{lemma*}[\textbf{2302}]
    Let \bm{$\lambda$} be a real number, 
    such that \bm{$\big \lfloor \lambda \big \rfloor + \epsilon = \lambda$}.
    \begin{equation*}
        \bm{
            \Bigg\{
                \bigg \lfloor \lambda \bigg \rfloor 
                    + 
                \bigg \lfloor \lambda + \frac{1}{3} \bigg \rfloor 
                    + 
                \bigg \lfloor \lambda + \frac{2}{3} \bigg \rfloor
            \Bigg\}
                =
            \Bigg\{
                3 \lambda - 3 \epsilon 
                    +
                \bigg \lfloor \epsilon \bigg \rfloor 
                    +
                \bigg \lfloor \epsilon + \frac{1}{3} \bigg \rfloor
                    +
                \bigg \lfloor \epsilon + \frac{2}{3} \bigg \rfloor            
            \Bigg\}
        }
    \end{equation*}
\end{lemma*}

\begin{proof}
    Given the real number \bm{$\lambda$}, 
    by the properties for floor functions
    there exists a real number \bm{$\epsilon$}
    and an integer \bm{$\lambda - \epsilon$} such that 
    \bm{$\lfloor \lambda \rfloor = \lambda - \epsilon$}.
    Thus, \bm{$\lambda = \lambda - \epsilon + \epsilon$}.
    By the identity \bm{$\lambda$},
    \begin{equation*}
        \bigg \lfloor \lambda \bigg \rfloor 
            + 
        \bigg \lfloor \lambda + \frac{1}{3} \bigg \rfloor 
            + 
        \bigg \lfloor \lambda + \frac{2}{3} \bigg \rfloor
            =
    \end{equation*}
    \begin{equation*}
        \bigg \lfloor \lambda - \epsilon + \epsilon \bigg \rfloor 
            + 
        \bigg \lfloor \lambda - \epsilon + \epsilon + \frac{1}{3} \bigg \rfloor 
            + 
        \bigg \lfloor \lambda - \epsilon + \epsilon + \frac{2}{3} \bigg \rfloor
    \end{equation*}
    Since \bm{$\lambda - \epsilon$} is an integer, by Lemma 2301 that is
    \begin{equation*}
        \bigg \langle \lambda - \epsilon \bigg \rangle + \bigg \lfloor \epsilon \bigg \rfloor 
            + 
        \bigg \langle \lambda - \epsilon \bigg \rangle + \bigg \lfloor \epsilon + \frac{1}{3} \bigg \rfloor 
            + 
        \bigg \langle \lambda - \epsilon \bigg \rangle + \bigg \lfloor \epsilon + \frac{2}{3} \bigg \rfloor
            =
    \end{equation*}
    \begin{equation*}
        \bigg \langle 3 \lambda - 3 \epsilon \bigg \rangle
            + 
        \bigg \lfloor \epsilon \bigg \rfloor 
            + 
        \bigg \lfloor \epsilon + \frac{1}{3} \bigg \rfloor 
            + 
        \bigg \lfloor \epsilon + \frac{2}{3} \bigg \rfloor
    \end{equation*}
    This completes the proof.
\color{lightgray} \end{proof}

\end{document}