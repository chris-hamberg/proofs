\documentclass[preview]{standalone}
\usepackage{amssymb, amsthm}
\usepackage{mathtools}
\usepackage{bm}
\usepackage{xcolor}


\newtheorem*{lemma*}{Lemma}
\renewcommand\qedsymbol{$\blacksquare$}


\begin{document}


\begin{lemma*}[\textbf{2304}]
    Let \bm{$\lambda$} be a real number, 
    such that \bm{$\big \lfloor \lambda \big \rfloor + \epsilon = \lambda$}.
    \begin{equation*}
        \bm{
            \bigg \lfloor 3 \lambda \bigg \rfloor
                =
            \bigg \lfloor \lambda \bigg \rfloor 
                + 
            \bigg \lfloor \lambda + \frac{1}{3} \bigg \rfloor 
                + 
            \bigg \lfloor \lambda + \frac{2}{3} \bigg \rfloor
        }
    \end{equation*}
    \begin{equation*}
        \textbf{if and only if}
    \end{equation*}
    \begin{equation*}
        \bm{
            \bigg \lfloor 3 \epsilon \bigg \rfloor
                =
            \bigg \lfloor \epsilon \bigg \rfloor 
                + 
            \bigg \lfloor \epsilon + \frac{1}{3} \bigg \rfloor 
                + 
            \bigg \lfloor \epsilon + \frac{2}{3} \bigg \rfloor
        }
    \end{equation*}
\end{lemma*}

\begin{proof}
    By Lemma 2302, and 2303, the left-hand side of the equivalence is
    \begin{equation*}
        \Bigg\{
            \bigg \langle 
                3 \lambda - 3 \epsilon 
            \bigg \rangle
                +
            \bigg \lfloor 3 \epsilon \bigg \rfloor
        \Bigg\}
            =
        \Bigg\{
            \bigg \langle 3 \lambda - 3 \epsilon \bigg \rangle
                + 
            \bigg \lfloor \epsilon \bigg \rfloor 
                + 
            \bigg \lfloor \epsilon + \frac{1}{3} \bigg \rfloor 
                + 
            \bigg \lfloor \epsilon + \frac{2}{3} \bigg \rfloor
        \Bigg\}
    \end{equation*}
    The right-hand side of the equivalence follows immediately 
    from the inverse law of addition.
\color{lightgray} \end{proof}

\end{document}