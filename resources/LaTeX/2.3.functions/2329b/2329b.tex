\documentclass[preview]{standalone}
\usepackage{amssymb, amsthm}
\usepackage{mathtools}
\usepackage{bm}
\usepackage{xcolor}


\newtheorem*{theorem*}{Theorem}
\renewcommand\qedsymbol{$\blacksquare$}


\begin{document}


\begin{theorem*}[\textbf{2329b}]
    Let \bm{$\delta$} be a function 
    \bm{$\delta: \Delta \rightarrow \mathrm{A}$}, 
    and let \bm{$\gamma$} be a function 
    \bm{$\gamma : \Lambda \rightarrow \Delta$}. 
    If both \bm{$\delta$} and \bm{$\gamma$} are surjective, 
    then 
    \bm{$\big \langle \delta \circ \gamma \big \rangle$} 
    is surjective.
\end{theorem*}

\begin{proof}
    By the contrapositive. 
    Suppose it were not the case that
    \bm{$\big \langle \delta \circ \gamma \big \rangle$} 
    were surjective. 
    By the definition for surjective functions,
    with domain of discourse \bm{$\lambda \in \Lambda$} 
    and \bm{$\alpha \in \mathrm{A}$}, 
    that is
    \begin{equation*}
        \lnot \forall \alpha \exists \lambda \Big \langle
            \big \langle \delta \circ \gamma \big \rangle [\lambda] 
                = 
            \alpha
        \Big \rangle
    \end{equation*}
    The composition of functions 
    \bm{$\big \langle \delta \circ \gamma \big \rangle [\lambda]$} 
    is defined as \bm{$\delta \big[ \gamma(\lambda) \big]$}.
    Thus, we have the equivalent universal quantification
    \begin{equation*}
        \lnot \forall \alpha \exists \lambda \Big \langle
            \delta \big[ \gamma (\lambda) \big] = \alpha \Big \rangle
    \end{equation*}
    In other words, it follows from the negation of the direct consequent 
    that it is not the case that \bm{$\delta$} is surjective, 
    by the definition for surjective functions.
    This is sufficient to prove the logical negation of the direct hypothesis.
    Thus, if both \bm{$\delta$} and \bm{$\gamma$} are surjective, 
    then \bm{$\big \langle \delta \circ \gamma \big \rangle$} is surjective.
\color{lightgray} \end{proof}

\end{document}