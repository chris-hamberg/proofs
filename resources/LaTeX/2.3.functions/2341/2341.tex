\documentclass[preview]{standalone}
\usepackage{amssymb, amsthm}
\usepackage{mathtools}
\usepackage{bm}
\usepackage{xcolor}


\newtheorem*{theorem*}{Theorem}
\renewcommand\qedsymbol{$\blacksquare$}


\begin{document}


\begin{theorem*}[\textbf{2341}]
    Let \bm{$\lambda$} be the function 
    \bm{$\lambda : \Delta \rightarrow \mathrm{A}$}. 
    Let \bm{$\Lambda$} be a subset of \bm{$\mathrm{A}$}. 
    \begin{equation*}
        \bm{
        \lambda ^{-1} \big[ \overline{\Lambda} \big] 
            = 
        \overline{ \lambda ^{-1} \big[ \Lambda \big]}
        }
    \end{equation*}
\end{theorem*}

\begin{proof}
    Let \bm{$\alpha$} be an element in 
    \bm{$\lambda ^{-1} \big[ \overline{\Lambda} \big]$}.
    By \bm{$\lambda ^{-1}$} inverse, by the definition of set complement,
    by \bm{$\lambda$} inverse, and again by the defintion of set complement,
    \begin{equation*}
        \Big \langle
            \alpha \in \lambda ^{-1} \big[ \overline{\Lambda} \big]
        \Big \rangle
            \equiv
        \Big \langle
            \lambda \big[ \alpha \big] \in \overline{\Lambda}
        \Big \rangle
            \equiv
        \Big \langle
            \lambda \big[ \alpha \big] \notin \Lambda
        \Big \rangle
            \equiv
    \end{equation*}
    \begin{equation*}
        \Big \langle 
            \alpha 
                \notin 
            \lambda ^{-1} \big[ \Lambda \big]
        \Big \rangle
            \equiv
        \Big \langle
            \alpha 
                \in
            \overline{
                \lambda ^{-1} \big[ \Lambda \big]
            }
        \Big \rangle
    \end{equation*}
    $\therefore \text{\space} \bm{
        \lambda ^{-1} \big[ \overline{\Lambda} \big] 
            = 
        \overline{ \lambda ^{-1} \big[ \Lambda \big]}
    }$.
\color{lightgray} \end{proof}

\end{document}