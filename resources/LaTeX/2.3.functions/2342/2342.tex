\documentclass[preview]{standalone}
\usepackage{amssymb, amsthm}
\usepackage{mathtools}
\usepackage{bm}
\usepackage{xcolor}


\newtheorem*{theorem*}{Theorem}
\renewcommand\qedsymbol{$\blacksquare$}


\begin{document}


\begin{theorem*}[\textbf{2342}]
    Let \bm{$\zeta$} be a real number. 
    \bm{$\lfloor \zeta + \frac{1}{2} \rfloor$}
    is the closest integer to \bm{$\zeta$}, 
    except when \bm{$\zeta$} is midway between two integers, 
    when it is the larger of these two integers.
\end{theorem*}

\begin{proof}
    By cases. By the properties of floor functions, 
    there exists an integer \bm{$\lambda$} such that 
    \begin{equation*}
        \Big \langle \lambda \Big \rangle
            \le 
        \Big \langle \zeta \Big \rangle
            < 
        \Big \langle \lambda + 1 \Big \rangle
    \end{equation*}
    and 
    \bm{$\zeta - \lfloor \zeta \rfloor = \epsilon$}.
    \\ \\
    \bm{$(i)$} Suppose the case in which \bm{$\zeta$} is midway between two integers, 
    or is closest to the larger of two integers \bm{$\lambda$} and 
    \bm{$\big \langle \lambda + 1 \big \rangle$}. 
    Then, the inequality \bm{$\frac{1}{2} \leq \epsilon < 1$} must be true. Thus, 
    \begin{equation*}
        \bigg[
            \Big \langle
                \lambda + \frac{1}{2}
            \Big \rangle
                \leq 
            \Big \langle
                \lambda + \epsilon
            \Big \rangle
                <
            \Big \langle
                \lambda + 1
            \Big \rangle
        \bigg]
            \equiv
        \bigg[
            \Big \langle    
                \lambda + \frac{1}{2}
            \Big \rangle
                \leq
            \Big \langle
                \zeta
            \Big \rangle
                <
            \Big \langle
                \lambda + 1
            \Big \rangle
        \bigg]
            \equiv
    \end{equation*}
    \begin{equation*}
        \bigg[
            \Big \langle 
                \lambda + 1
            \Big \rangle
                \leq
            \Big \langle 
                \zeta + \frac{1}{2}
            \Big \rangle
                <
            \Big \langle
                \lambda + 1 + \frac{1}{2}
            \Big \rangle
        \bigg]
    \end{equation*}
    Since 
    \bm{$
    \big \langle \lambda + 1 + \frac{1}{2} \big \rangle 
        < 
    \big \langle 
        \lambda + 1 + 1
    \big \rangle
    $}, 
    the integer \bm{$\lfloor \zeta + \frac{1}{2} \rfloor$} is 
    \bm{$\big \langle \lambda + 1 \big \rangle$}, 
    by the properties for floor functions,
    and by the law of transitivity from the order axioms.
    \\ \\
    \bm{$(ii)$} Suppose the case in which \bm{$\zeta$} is closest to the integer \bm{$\lambda$}. 
    Then the inequality \bm{$0 \leq \epsilon < \frac{1}{2}$}, 
    must be true. Thus, 
    \begin{equation*}
        \bigg[
            \Big \langle
                \lambda + 0
            \Big \rangle
                \leq
            \Big \langle
                \lambda + \epsilon
            \Big \rangle
                <
            \Big \langle
                \lambda + \frac{1}{2}
            \Big \rangle
        \bigg]
            \equiv
    \end{equation*}
    \begin{equation*}
        \bigg[
            \Big \langle
                \lambda
            \Big \rangle
                \leq
            \Big \langle
                \zeta
            \Big \rangle
                <
            \Big \langle
                \lambda + \frac{1}{2}
            \Big \rangle
        \bigg]
            \equiv
        \bigg[
            \Big \langle
                \lambda + \frac{1}{2}
            \Big \rangle
                \leq
            \Big \langle
                \zeta + \frac{1}{2}
            \Big \rangle
                <
            \Big \langle
                \lambda + 1
            \Big \rangle
        \bigg]
    \end{equation*}
    Since 
    \bm{$
    \big \langle \lambda \big \rangle 
        \leq 
    \big \langle \lambda + \frac{1}{2} \big \rangle
    $},
    the integer \bm{$\lfloor \zeta + \frac{1}{2} \rfloor$} is 
    \bm{$\big \langle \lambda \big \rangle$}, 
    by the properties for floor functions,
    and by the law of transitivity from the order axioms.
    $\therefore$
    \bm{$\lfloor \zeta + \frac{1}{2} \rfloor$} 
    is the closest integer to \bm{$\zeta$}, 
    except when \bm{$\zeta$} is midway between two integers, 
    when it is the larger of these two integers.
\color{lightgray} \end{proof}

\end{document}