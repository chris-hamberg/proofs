\documentclass[preview]{standalone}
\usepackage{amssymb, amsthm}
\usepackage{mathtools}
\usepackage{bm}
\usepackage{xcolor}


\newtheorem*{theorem*}{Theorem}
\renewcommand\qedsymbol{$\blacksquare$}


\begin{document}


\begin{theorem*}[\textbf{2367d}]
    Suppose \bm{$\mathrm{A}$}, and \bm{$\Lambda$} are sets with universal set \bm{$\Omega$}. 
    Let \bm{$\lambda_{\mathrm{A} \oplus \Lambda}$} be the characteristic function 
    \bm{$\lambda_{\mathrm{A} \oplus \Lambda}: \Omega \rightarrow \{0, 1\}$},
    let \bm{$\lambda_{\mathrm{A}}$} be the characteristic function 
    \bm{$\lambda_{\mathrm{A}} : \Omega \rightarrow \{0, 1\}$}, 
    and let \bm{$\lambda_{\Lambda}$} be the characteristic function  
    \bm{$\lambda_{\Lambda}: \Omega \rightarrow \{0, 1\}$}. 
    \begin{equation*}
        \bm{
            \lambda_{\mathrm{A} \oplus \Lambda} \big[ \iota \big]
                = 
            \lambda_{\mathrm{A}} \big[ \iota \big] 
                + 
            \lambda_{\Lambda} \big[ \iota \big] 
                - 
            2 \Big[ \lambda_{\mathrm{A}} \big[ \iota \big] \lambda_{\Lambda} \big[ \iota \big] \Big]
        }
    \end{equation*}
\end{theorem*}

\begin{proof}
    By cases. There are two cases under consideration. 
    Either $(i)$ \bm{$\iota$} is an element in \bm{$\mathrm{A} \oplus \Lambda$}, 
    or $(ii)$ \bm{$\iota$} is not an element in \bm{$\mathrm{A} \oplus \Lambda$}.
    \\ \\
    \bm{$(i)$} Assume \bm{$\iota$} is an element in \bm{$\mathrm{A} \oplus \Lambda$}. 
    By the defintion for the symmetric difference of sets, that is
    \begin{equation*}
        \bigg[
            \Big \langle \iota \in \mathrm{A} \Big \rangle 
                \land 
            \Big \langle \iota \notin \Lambda \Big \rangle
        \bigg] 
            \lor 
        \bigg[
            \Big \langle \iota \notin \mathrm{A} \Big \rangle 
                \land 
            \Big \langle \iota \in \Lambda \Big \rangle
        \bigg]    
    \end{equation*}
    Without loss of generality, assume 
    \bm{
        $\big \langle \iota \in \mathrm{A} \big \rangle 
            \land 
        \big \langle \iota \notin \Lambda \big \rangle
    $}.
    By the definition for characteristic functions,
    \begin{equation*}
        \bigg[
            \Big \langle \lambda_{\mathrm{A} \oplus \Lambda} \big[ \iota \big] = 1 \Big \rangle
                \land 
            \Big \langle \lambda_{\mathrm{A}} \big[ \iota \big] = 1 \Big \rangle
                \land 
            \Big \langle \lambda_{\Lambda} \big[ \iota \big] = 0 \Big \rangle
        \bigg]
            \rightarrow
    \end{equation*}
    \begin{equation*}
        \lambda_{\mathrm{A} \oplus \Lambda} \big[ \iota \big]
            =
        \lambda_{\mathrm{A}} \big[ \iota \big] 
            + 
        \lambda_{\Lambda} \big[ \iota \big] 
            - 
        2 \Big[ \lambda_{\mathrm{A}} \big[ \iota \big] \lambda_{\Lambda} \big[ \iota \big] \Big]
            =
        \Big \langle 1 + 0 - 2 [ 1 ] [ 0 ] \Big \rangle
            =
        \Big \langle
            1
        \Big \rangle 
    \end{equation*}
    \bm{$(ii)$} Assume \bm{$\iota$} is not an element in \bm{$\mathrm{A} \oplus \Lambda$}. 
    There are two subcases. 
    $(a)$ \bm{$\iota$} is in the intersection of \bm{$\mathrm{A}$} and \bm{$\Lambda$}, 
    xor $(b)$ \bm{$\iota$} is in \bm{$\Omega$} minus \bm{$\mathrm{A} \cup \Lambda$}.
    \\ \\
    \bm{$(a)$} Assume \bm{$\iota$} is in the intersection of \bm{$\mathrm{A}$} and \bm{$\Lambda$}.
    Thus, by the definition for characteristic functions,
    \begin{equation*}
        \bigg[
            \Big \langle \lambda_{\mathrm{A} \oplus \Lambda} \big[ \iota \big] = 0 \Big \rangle
                \land 
            \Big \langle \lambda_{\mathrm{A}} \big[ \iota \big] = 1 \Big \rangle
                \land 
            \Big \langle \lambda_{\Lambda} \big[ \iota \big] = 1 \Big \rangle
        \bigg]
            \rightarrow
    \end{equation*}
    \begin{equation*}
        \lambda_{\mathrm{A} \oplus \Lambda} \big[ \iota \big] 
            =
        \lambda_{\mathrm{A}} \big[ \iota \big] 
            + 
        \lambda_{\Lambda} \big[ \iota \big] 
            - 
        2 \Big[ \lambda_{\mathrm{A}} \big[ \iota \big] \lambda_{\Lambda} \big[ \iota \big] \Big]
            = 
        \Big \langle 1 + 1 - 2[1][1] \Big \rangle
            = 
        \Big \langle 0 \Big \rangle
    \end{equation*}
    \bm{$(b)$} Assume \bm{$\iota$} is in \bm{$\Omega$} minus \bm{$\mathrm{A} \cup \Lambda$}.
    By the definition for characteristic functions, 
    \begin{equation*}
        \bigg[
            \Big \langle \lambda_{\mathrm{A} \oplus \Lambda} \big[ \iota \big] = 0 \Big \rangle
                \land 
            \Big \langle \lambda_{\mathrm{A}} \big[ \iota \big] = 0 \Big \rangle
                \land 
            \Big \langle \lambda_{\Lambda} \big[ \iota \big] = 0 \Big \rangle
        \bigg]
            \rightarrow
    \end{equation*}
    \begin{equation*}
        \lambda_{\mathrm{A} \oplus \Lambda} \big[ \iota \big] 
            =
        \lambda_{\mathrm{A}} \big[ \iota \big] 
            + 
        \lambda_{\Lambda} \big[ \iota \big] 
            - 
        2 \Big[ \lambda_{\mathrm{A}} \big[ \iota \big] \lambda_{\Lambda} \big[ \iota \big] \Big]
            = 
        \Big \langle 0 + 0 - 2[ 0 ] [ 0 ] \Big \rangle
            = 
        \Big \langle 0 \Big \rangle
    \end{equation*}
    This completes the proof.
\color{lightgray} \end{proof}

\end{document}