\documentclass[preview]{standalone}
\usepackage{amssymb, amsthm}
\usepackage{mathtools}
\usepackage{bm}
\usepackage{xcolor}


\newtheorem*{theorem*}{Theorem}
\renewcommand\qedsymbol{$\blacksquare$}


\begin{document}


\begin{theorem*}[\textbf{2370a}]
    Let \bm{$\lambda$} be a real number. 
    \begin{equation*}
        \bm{
            \Big \lfloor \lceil \lambda \rceil \Big \rfloor 
                = 
            \lceil \lambda \rceil
        }
    \end{equation*}
\end{theorem*}

\begin{proof}
    By the properties of ceiling functions,
    there exists an integer \bm{$\iota$} such that
    \bm{$\lceil \lambda \rceil = \iota$}, 
    and by the identity \bm{$\iota$},
    \begin{equation*}
        \bigg \langle \lceil \lambda \rceil = \iota \bigg \rangle
            \equiv
        \bigg \langle 
            \Big \lfloor \lceil \lambda \rceil \Big \rfloor
                =
            \big \lfloor \iota \big \rfloor
        \bigg \rangle
    \end{equation*}
    \bm{$\iota$} is the largest integer that is less than or equal to \bm{$\iota$}.
    Therefore, by the definition for the floor function
    \bm{$\lfloor \iota \rfloor = \iota$}.
    Hence,
    \begin{equation*}
        \bigg \langle \lceil \lambda \rceil = \iota \bigg \rangle
            \land
        \bigg \langle 
            \Big \lfloor \lceil \lambda \rceil \Big \rfloor
                =
            \big \lfloor \iota \big \rfloor
                =
            \iota
        \bigg \rangle
    \end{equation*}
    $\therefore \text{\space} \bm{
        \Big \lfloor \lceil \lambda \rceil \Big \rfloor 
            = 
        \lceil \lambda \rceil
    }$, by the identity \bm{$\iota$}.
\color{lightgray} \end{proof}

\end{document}