\documentclass[preview]{standalone}
\usepackage{amssymb, amsthm}
\usepackage{mathtools}
\usepackage{bm}
\usepackage{xcolor}


\newtheorem*{theorem*}{Theorem}
\renewcommand\qedsymbol{$\blacksquare$}


\begin{document}


\begin{theorem*}[\textbf{2371b}]
    Let \bm{$\lambda$} be a positive real number.
    \begin{equation*}
        \bm{
            \bigg \lceil 
                \sqrt{ \strut \big \lceil \lambda \big \rceil }
            \bigg \rceil 
                = 
            \bigg \lceil \sqrt{ \strut \lambda } \bigg \rceil
        }
    \end{equation*}
\end{theorem*}

\begin{proof}
    By the properties for ceiling functions,
    there exists an integer
    \bm{$\big \lceil \sqrt{ \lceil \lambda \rceil} \big \rceil$}
    such that
    \bm{$
        \big \lceil \sqrt{ \lambda } \big \rceil
            =
        \big \lceil \sqrt{ \lceil \lambda \rceil} \big \rceil$
    },
    if and only if
    \begin{equation*}
        \bigg \langle
            \bigg \lceil
                \sqrt{ \strut \big \lceil \lambda \big \rceil }
            \bigg \rceil
                -
            1
        \bigg \rangle
            <
        \bigg \langle
            \sqrt{ \strut \lambda }
        \bigg \rangle
            \leq
        \bigg \langle
            \bigg \lceil
                \sqrt{ \strut \big \lceil \lambda \big \rceil }
            \bigg \rceil
        \bigg \rangle
    \end{equation*}
    By the multiplicative compatibility law from the order axioms, that is
    \begin{equation*}
        \bigg \langle
            \bigg \lceil
                \sqrt{ \strut \big \lceil \lambda \big \rceil }
            \bigg \rceil
                -
            1
        \bigg \rangle
            ^2
            <
        \bigg \langle
            \lambda
        \bigg \rangle
            \leq
        \bigg \langle
            \bigg \lceil
                \sqrt{ \strut \big \lceil \lambda \big \rceil }
            \bigg \rceil
        \bigg \rangle
            ^2
    \end{equation*}
    \bm{$\big \lceil \lambda \big \rceil$} is the smallest integer that is greater than or equal to \bm{$\lambda$},
    so by the definition of the ceiling function, 
    \bm{$\lambda \leq \big \lceil \lambda \big \rceil$}.
    Also, 
    \bm{$\big \lceil \sqrt{ \lceil \lambda \rceil} \big \rceil ^2$}
    is an integer by the definition of floor functions,
    since integers are closed under multiplication.
    Hence,
    \bm{$
        \big \lceil \lambda \big \rceil
            \leq
        \big \lceil \sqrt{ \lceil \lambda \rceil} \big \rceil ^2
    $}.
    So by the transitivity law from the order axioms,
    \begin{equation*}
        \bigg \langle
            \bigg \lceil
                \sqrt{ \strut \big \lceil \lambda \big \rceil }
            \bigg \rceil
                -
            1
        \bigg \rangle
            ^2
            <
        \bigg \langle
            \lambda
        \bigg \rangle
            \leq
        \bigg \langle
            \big \lceil \lambda \big \rceil
        \bigg \rangle
            \leq
        \bigg \langle
            \bigg \lceil
                \sqrt{ \strut \big \lceil \lambda \big \rceil }
            \bigg \rceil
        \bigg \rangle
            ^2
            \equiv
    \end{equation*}
    \begin{equation*}
        \bigg \langle
            \bigg \lceil
                \sqrt{ \strut \big \lceil \lambda \big \rceil }
            \bigg \rceil
                -
            1
        \bigg \rangle
            ^2
            <
        \bigg \langle
            \big \lceil \lambda \big \rceil
        \bigg \rangle
            \leq
        \bigg \langle
            \bigg \lceil
                \sqrt{ \strut \big \lceil \lambda \big \rceil }
            \bigg \rceil
        \bigg \rangle
            ^2
    \end{equation*}
    By the multiplicative compatibility law from the order axioms,
    the following is an equivalent statement,
    \begin{equation*}
        \bigg \langle
            \bigg \lceil
                \sqrt{ \strut \big \lceil \lambda \big \rceil }
            \bigg \rceil
                -
            1
        \bigg \rangle
            <
        \bigg \langle
            \sqrt{ \strut \big \lceil \lambda \big \rceil }
        \bigg \rangle
            \leq
        \bigg \langle
            \bigg \lceil
                \sqrt{ \strut \big \lceil \lambda \big \rceil }
            \bigg \rceil
        \bigg \rangle
    \end{equation*}
    $\therefore \text{\space} \bm{
        \bigg \lceil 
            \sqrt{ \strut \big \lceil \lambda \big \rceil }
        \bigg \rceil 
            = 
        \bigg \lceil \sqrt{ \strut \lambda } \bigg \rceil
    }$, by the properties of ceiling functions.
\color{lightgray} \end{proof}

\end{document}