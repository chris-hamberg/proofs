\documentclass[a4paper, 12pt]{article}
\usepackage[utf8]{inputenc}
\usepackage[english]{babel}
\usepackage{amssymb, amsmath, amsthm}
\theoremstyle{plain}
\newtheorem*{theorem*}{Theorem}
\newtheorem{theorem}{Theorem}

\usepackage{mathtools}
\renewcommand\qedsymbol{$\blacksquare$}

\begin{document}
	
	\begin{theorem*}[2.3.40b]
		Let f be the function $f: A \implies B$. Let S, and T be subsets of B. 
		$f^{-1}(S \cap T) = f^{-1}(S) \cap f^{-1}(T)$.
	\end{theorem*}
	
	\begin{proof}
		By the definition for the inverse image of the set $(S \cap T)$ under the function 
		$f^{-1}$, we have $f^{-1}(S \cap T) = \{a \in A | f(a) \in (S \cap T)\}$. Then 
		\newline equivalently, 
		$f^{-1}(S \cap T) \equiv \{a \in A | f(a) \in S\} \cap \{a \in A | f(a) \in T\}$. This 
		is the formal definition for $f^{-1}(S \cap T) = f^{-1}(S) \cap f^{-1}(T)$.
	\end{proof}

\end{document}
