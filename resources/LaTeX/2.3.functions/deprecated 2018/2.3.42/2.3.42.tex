\documentclass[a4paper, 12pt]{article}
\usepackage[utf8]{inputenc}
\usepackage[english]{babel}
\usepackage{amssymb, amsmath, amsthm}
\theoremstyle{plain}
\newtheorem*{theorem*}{Theorem}
\newtheorem{theorem}{Theorem}

\usepackage{mathtools}
\renewcommand\qedsymbol{$\blacksquare$}
\DeclarePairedDelimiter{\floor}{\lfloor}{\rfloor}
\begin{document}
	
	\begin{theorem*}[2.3.42]
		Let x be a real number. $\floor{x + \frac{1}{2}}$ is the closest integer to x, except when 
		x is midway between two integers, when it is the larger of these two integers.
	\end{theorem*}
	
	\begin{proof}
		By cases. Let $n$ be the integer such that $n \le x < n+1$ and \newline 
		$\floor{x + \frac{1}{2}} = \floor{n + \epsilon + \frac{1}{2}}$. $\epsilon$ is the decimal 
		part of $x$. \newline \newline $(i)$ If $\epsilon \ge \frac{1}{2}$, then 
		$\epsilon + \frac{1}{2} \ge \frac{1}{2} + \frac{1}{2}$. That is, if 
		$\epsilon + \frac{1}{2} \ge 1$, then \newline  
		$\floor{x + \frac{1}{2}} \ge \floor {(x - \epsilon) + 1} = n + 1$.
		\newline \newline $(ii)$ If $\epsilon < \frac{1}{2}$, then $\epsilon + \frac{1}{2} < 1$. 
		That is, if $\epsilon + \frac{1}{2} < 1$, then \newline 
		$\floor{x + \frac{1}{2}} = \floor{n + (\epsilon + \frac{1}{2})} = n$.
	\end{proof}

\end{document}
