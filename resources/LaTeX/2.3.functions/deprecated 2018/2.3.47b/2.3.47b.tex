\documentclass[a4paper, 12pt]{article}
\usepackage[utf8]{inputenc}
\usepackage[english]{babel}
\usepackage{amssymb, amsmath, amsthm}
\theoremstyle{plain}
\newtheorem*{theorem*}{Theorem}
\newtheorem{theorem}{Theorem}

\usepackage{mathtools}
\renewcommand\qedsymbol{$\blacksquare$}
\DeclarePairedDelimiter{\floor}{\lfloor}{\rfloor}
\DeclarePairedDelimiter{\ceil}{\lceil}{\rceil}

\begin{document}
	
	\begin{theorem*}[2.3.47b]
		Let x be a real number, and let n be an integer. \newline $n < x \iff n < \ceil{x}$.
	\end{theorem*}
	
	\begin{proof}
		$x \le \ceil{x}$, by the properties of the ceiling function. So if $n < x$, then 
		$n < x \le \ceil{x}$, and $n < \ceil{x}$.
		
		Proving the converse, suppose $n < \ceil{x}$. Since $n$ and $\ceil{x}$ are integers, 
		$n \le \ceil{x} - 1$. By the properties of ceiling functions we have the following 
		tautology, $\ceil{x} = \ceil{x} \iff \ceil{x} - 1 < x \le \ceil{x}$. Combining these 
		two inequalities yields $n \le \ceil{x} - 1 < x \le \ceil{x}$. Thus, $n < x$.
	\end{proof}

\end{document}
