\documentclass[a4paper, 12pt]{article}
\usepackage[utf8]{inputenc}
\usepackage[english]{babel}
\usepackage{amssymb, amsmath, amsthm}
\theoremstyle{plain}
\newtheorem*{theorem*}{Theorem}
\newtheorem{theorem}{Theorem}

\usepackage{mathtools}
\renewcommand\qedsymbol{$\blacksquare$}
\DeclarePairedDelimiter{\floor}{\lfloor}{\rfloor}
\DeclarePairedDelimiter{\ceil}{\lceil}{\rceil}

\begin{document}
	
	\begin{theorem*}[2.3.67a]
		Let A, and B be sets with universal set U. Let $f_{A \cap B}$ be the characteristic 
		function $f_{A \cap B}: U \implies \{0, 1\}$. Let $f_{A}$ be the characteristic function 
		$f_{A}: U \implies \{0, 1\}$. Let $f_{B}$ be the characteristic function \newline 
		$f_{B}: U \implies \{0, 1\}$. $f_{A \cap B}(x) = f_{A}(x) \times f_{B}(x)$.
	\end{theorem*}
	
	\begin{proof}
		Let $x$ be an element in $A \cap B$. By the definition for characteristic functions, 
		$f_{A \cap B}(x) = 1$. Since the definition for set intersection says that \newline 
		$(x \in A) \land (x \in B)$, we know by the definition for characteristic functions that 
		$f_{A}(x) = f_{B}(x) = 1$. Thus, it follows immediately by the multiplicative identity law 
		from the field axioms that $f_{A \cap B}(x) = f_{A}(x) \times f_{B}(x)$.
		
		Suppose it were not the case that $x$ were an element in $A \cap B$. That is, 
		$x \notin (A \cap B) \equiv [(x \notin A) \lor (x \notin B)]$, by DeMorgans law. By the 
		definition for characteristic functions, $f_{A \cap B}(x) = 0$. Also, again by the definition 
		for characteristic functions we know that ($f_{A}(x) = 0) \lor (f_{B}(x) = 0)$. Without loss 
		of generality we can suppose $f_{A}(x) = 0$. It follows immediately from the multiplicative 
		property of zero that $f_{A \cap B}(x) = 0 \times f_{B}(x) = 0$. Thus, 
		$f_{A \cap B}(x) = f_{A}(x) \times f_{B}(x)$.
	\end{proof}

\end{document}
