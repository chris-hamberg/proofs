\documentclass[a4paper, 12pt]{article}
\usepackage[utf8]{inputenc}
\usepackage[english]{babel}
\usepackage{amssymb, amsmath, amsthm}
\theoremstyle{plain}
\newtheorem*{theorem*}{Theorem}
\newtheorem{theorem}{Theorem}

\usepackage{mathtools}
\renewcommand\qedsymbol{$\blacksquare$}
\DeclarePairedDelimiter{\floor}{\lfloor}{\rfloor}
\DeclarePairedDelimiter{\ceil}{\lceil}{\rceil}

\begin{document}
	
	\begin{theorem*}[2.3.67d]
		Let A, and B be sets with universal set U. Let $f_{A \oplus B}$ be the characteristic 
		function $f_{A \oplus B}: U \implies \{0, 1\}$. Let $f_{A}$ be the characteristic function 
		$f_{A}: U \implies \{0, 1\}$. Let $f_{B}$ be the characteristic function \newline 
		$f_{B}: U \implies \{0, 1\}$. $f_{A \oplus B}(x) = f_{A}(x) + f_{B}(x) - 2f_{A}(x)f_{B}(x)$.
	\end{theorem*}
	
	\begin{proof}
		There are two major cases to consider, each consisting of two sub cases. The major cases are 
		where $x$ is an element in $A \oplus B$, and the negation of that statement.
		\newline
		\newline
		$(i)$ Let $x$ be an element in $A \oplus B$. By the definition for \newline characteristic 
		functions, $f_{A \oplus B}(x) = 1$. Since the definition for set symmetric difference says 
		$[(x \in A) \land (x \notin B)] \lor [(x \notin A) \land (x \in B)]$, there are two sub cases 
		that need to be taken under consideration. \newline \newline \indent $(a)$ Suppose 
		$(x \in A) \land (x \notin B)$. By the definition for characteristic \indent functions 
		$f_{A}(x) = 1$ and $f_{B}(x) = 0$. This means that \newline \indent 
		$f_{A}(x) + f_{B}(x) - 2f_{A}(x)f_{B}(x) = 1 + 0 - 2(1)(0)$ = 1. 
		\newline \newline \indent $(b)$ Suppose $(x \notin A) \land (x \in B)$. Without loss of 
		generality we arrive \indent at the same result as that of case $(a)$.
		\newline
		\newline
		Thus, if $x$ is an element in $A \oplus B$, 
		$f_{A \oplus B}(x) = f_{A}(x) + f_{B}(x) - 2f_{A}(x)f_{B}(x)$.
		\newline
		\newline
		$(ii)$ Suppose it were not the case that $x$ were an element in $A \oplus B$. Then $(c)$ $x$ 
		must either be an element in the intersection of $A$ and $B$, or $(d)$ $x$ must be in the 
		universe minus $A \cup B$.
		\newline
		\newline \indent $(c)$ Suppose $x \in (A \cap B)$. By the definition for characteristic \newline 
		\indent functions $f_{A \oplus B}(x) = 0$, $f_{A}(x) = 1$ and $f_{B}(x) = 1$. Thus, \newline 
		\indent $f_{A}(x) + f_{B}(x) - 2f_{A}(x)f_{B}(x) = 1 + 1 - 2(1)(1) = 0$.
		\newline \newline \indent $(d)$ Suppose $x \in [U - (A \cup B)$. In this case, by the definition 
		for \newline \indent characteristic functions, $f_{A \oplus B}(x) = 0$, $f_{A}(x) = 0$ and 
		$f_{B}(x) = 0$. So, \indent $f_{A}(x) + f_{B}(x) - 2f_{A}(x)f_{B}(x) = 0 + 0 - 2(0)(0) = 0$.
		\newline
		\newline
		Thus, if $x$ is not an element in $A \oplus B$, 
		$f_{A \oplus B}(x) = f_{A}(x) + f_{B}(x) - 2f_{A}(x)f_{B}(x)$ is still a true statement; 
		concludes the proof.
	\end{proof}

\end{document}
