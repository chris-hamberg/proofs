\documentclass[preview]{standalone}
\usepackage{amssymb, amsthm}
\usepackage{mathtools}
\usepackage{bm}


\newtheorem{lemma}{Lemma}
\renewcommand\qedsymbol{$\blacksquare$}


\begin{document}


\begin{lemma}[\textbf{2401}]
    Let \bm{$\lambda$} be a positive integer.
    \begin{equation*}
        \bm{
            \frac{1}{3} 
            \bigg[ 
                \lambda^3 
                    + 
                3 \bigg \langle 
                    \frac{ \lambda [ \lambda + 1 ] }{2} 
                \bigg \rangle
                    - 
                    \lambda 
            \bigg] 
                =
            \bigg[
                \frac{
                    \lambda
                    \langle \lambda + 1 \rangle 
                    \langle 2 \lambda + 1 \rangle}
                    {6}
            \bigg]
        }
    \end{equation*}
\end{lemma}

\begin{proof}
    By the distributive laws for real numbers, 
    and by the associative law for multiplication,
    \begin{equation*}
        3 \bigg \langle \frac{ \lambda [ \lambda + 1 ] }{2} \bigg \rangle 
            = 
        \bigg \langle \frac{3 \lambda ^2 + 3 \lambda }{2} \bigg \rangle
    \end{equation*}
    Since the rational number \bm{$\frac{2}{2} = 1$} 
    by the inverse law for multiplication, 
    by the multiplicative identity law from the field axioms,
    (and by the identity established above,)
    \begin{equation*} 
            \bigg[ 
                \lambda ^3 
                    + 
                3 \bigg \langle \frac{ \lambda [ \lambda + 1 ] }{2} \bigg \rangle
                    - 
                \lambda 
            \bigg]
            =
            \bigg[
                \frac{ 2 \lambda ^3 }{2} \
                    + 
                \frac{ 3 \lambda ^2 + 3 \lambda }{2} 
                    - 
                \frac{ 2 \lambda }{2}
            \bigg]
    \end{equation*}
    By the distributive law for real numbers
    \begin{equation*}
            \frac{1}{3}
            \bigg[
                \frac{ 2 \lambda ^3 }{2} 
                    + 
                \frac{ 3 \lambda ^2 + 3 \lambda }{2} 
                    - 
                \frac{ 2 \lambda }{2}
            \bigg]
                =
            \bigg[
                \frac{ 2 \lambda ^3 + 3 \lambda ^2 + 3 \lambda - 2 \lambda }
                {6}
            \bigg]
    \end{equation*}
    Repeated factoring, 
    by the distributive laws for real numbers,
    completes the proof.
    \begin{equation*}
        \bigg[
            \frac{ 2 \lambda ^3 + 3 \lambda ^2 + 3 \lambda - 2 \lambda }{6}
        \bigg]
            =
        \bigg[
            \frac{ \lambda \langle 2 \lambda ^2 + 2 \lambda + \lambda + 1 \rangle }{6}
        \bigg]
            =
        \bigg[
            \frac{ \lambda \langle 2 \lambda [\lambda + 1] + [\lambda + 1] \rangle }{6}
        \bigg]
    \end{equation*}
    \begin{equation*}
            =
        \bm{
            \bigg[
                \frac{ \lambda \langle \lambda + 1 \rangle \langle 2 \lambda + 1 \rangle }{6}
            \bigg]
        }
    \end{equation*}
\end{proof}


\end{document}