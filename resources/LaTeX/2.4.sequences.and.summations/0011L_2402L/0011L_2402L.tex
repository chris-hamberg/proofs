\documentclass[preview]{standalone}
\usepackage{amssymb, amsthm}
\usepackage{mathtools}
\usepackage{bm}


\newtheorem{lemma}{Lemma}
\renewcommand\qedsymbol{$\blacksquare$}


\begin{document}


\begin{lemma}[\textbf{2402}]
    Let \bm{$\iota$} be a positive integer such that
    \bm{$\big \lfloor \sqrt{\iota} \big \rfloor = \lambda$}.
    \begin{equation*}
        \bm{
            \Bigg( 2\sum_{\phi=0}^{\lambda - 1} \phi ^2 \Bigg)
                + 
            \Bigg( \sum_{\phi=0}^{\lambda - 1} \phi \Bigg)
                \equiv
            \frac{
                \lambda
                \big \langle \lambda - 1 \big \rangle
            }
            {6}
            \Bigg[
                4 \lambda + 1
            \Bigg]
        }
    \end{equation*}
\end{lemma}

\begin{proof}
    Let \bm{$\Phi$} be two times the sum of squares from zero to \bm{$\lambda$} minus one.
    Let \bm{$\mathrm{X}$} be the sum of integers from zero to \bm{$\lambda$} minus one.
    By theorems 103, and 102, and by shifting the index of summation,
    \begin{equation*}
        \Bigg\{
            \Phi + \mathrm{X}
        \Bigg\}
            \equiv
        2
        \Bigg[
            \frac{ 
                \lambda [ \lambda - 1 ]
                [ 2 \langle \lambda - 1 \rangle + 1 ] 
            }
            {6}
        \Bigg]
            +
        \Bigg[
            \frac{\lambda [ \lambda - 1 ] }
            {2}
        \Bigg]
    \end{equation*}
    Since the rational number \bm{$\frac{3}{3} = 1$}
    by the inverse law for multiplication,
    by the multiplicative identity law from the field axioms
    \begin{equation*}
        \bigg \{ \mathrm{X} \bigg\}   
            \equiv
        \frac{
            3
            \lambda [ \lambda - 1 ]
        }
        {6}
    \end{equation*}
    Thus, 
    by the identity \bm{$\mathrm{X}$}, 
    factoring
    \bm{$\frac{1}{6} \lambda \big \langle \lambda - 1 \big \rangle$}
    out from the sum of \bm{$\Phi$} and \bm{$\mathrm{X}$}, 
    by the distributive laws for real numbers,
    \begin{equation*}
        \Bigg\{
            \Phi
                +
            \mathrm{X}
        \Bigg\}
            \equiv
        \frac{
            \lambda
            \big \langle \lambda - 1 \big \rangle
        }
        {6}
        \Bigg[
            2 \bigg \langle 2 [ \lambda - 1 ] + 1 \bigg \rangle
                +
            3
        \Bigg]
    \end{equation*}
    By the distributive laws for real numbers, 
    and by the associative law for addition from the field axioms,
    \begin{equation*}
        \bigg \{
            2 \Big \langle 2 [ \lambda - 1 ] + 1 \Big \rangle + 3
        \bigg \}
            =
        \bigg \{
            4 [ \lambda - 1 \big ] + 5
        \bigg \}
            =
        \bigg \{
            4 \lambda + 1
        \bigg \}
    \end{equation*}
    \bm{$\therefore$}
    \begin{equation*}
        \bm{
            \Bigg \{
                \Phi
                    +
                \mathrm{X}
            \Bigg \}
                \equiv
            \frac{
                \lambda
                \big \langle \lambda - 1 \big \rangle
            }
            {6}
            \Bigg[
                4 \lambda + 1
            \Bigg]
        }
    \end{equation*}
\end{proof}


\end{document}