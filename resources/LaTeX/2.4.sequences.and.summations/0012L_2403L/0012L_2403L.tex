\documentclass[preview]{standalone}
\usepackage{amssymb, amsthm}
\usepackage{mathtools}
\usepackage{bm}


\newtheorem{lemma}{Lemma}
\renewcommand\qedsymbol{$\blacksquare$}


\begin{document}


\begin{lemma}[\textbf{2403}]
    Let \bm{$\iota$} be a natural number such that
    \bm{$\big \lfloor \sqrt \iota \big \rfloor = \lambda$}.
    \begin{equation*}
        \bm{
            \sum_{\iota=1}^{2 \lambda + 1} 
                    \big \lfloor \sqrt \iota \big \rfloor
                =
            \lambda \big[ 2 \lambda + 1 \big]
        }
    \end{equation*}
\end{lemma}

\begin{proof}
    It is trivial that
    \begin{equation*}
        \sum_{\iota=1}^{2 \lambda + 1} 
                1 = 2 \lambda + 1
    \end{equation*}
    By the inverse law for multiplication from the field axioms,
    by the distributive law for real numbers, and by the identity \bm{$\lambda$},
    that is
    \begin{equation*}
        \Bigg\{
            \lambda \sum_{\iota=1}^{2 \lambda + 1} 1 = \lambda \big[ 2 \lambda + 1 \big]
        \Bigg\}
                \equiv
        \Bigg\{
            \sum_{\iota=1}^{2 \lambda + 1} \lambda = \lambda \big[ 2 \lambda + 1 \big]
        \Bigg\}
            \equiv
        \Bigg\{
            \sum_{\iota=1}^{2 \lambda + 1} \big \lfloor \sqrt \iota \big \rfloor
                = 
            \lambda \big[ 2 \lambda + 1 \big]
        \Bigg\}
    \end{equation*}
\end{proof}


\end{document}