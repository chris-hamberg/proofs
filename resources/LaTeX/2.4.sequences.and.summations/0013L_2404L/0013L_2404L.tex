\documentclass[preview]{standalone}
\usepackage{amssymb, amsthm}
\usepackage{mathtools}
\usepackage{bm}


\newtheorem{lemma}{Lemma}
\renewcommand\qedsymbol{$\blacksquare$}


\begin{document}


\begin{lemma}[\textbf{2404}]
    Let \bm{$\iota$} be a natural number such that
    \bm{$\big \lfloor \sqrt[3] \iota \big \rfloor = \lambda$}.
    \begin{equation*}
        \bm{
            \sum_{\iota=1}^{ 3 \lambda ^2 + 3 \lambda + 1 } 
                    \big \lfloor \sqrt[3] \iota \big \rfloor
                =
            \lambda \big [ 3 \lambda ^2 + 3 \lambda + 1 \big ]
        }
    \end{equation*}
\end{lemma}

\begin{proof}
    Similiar to Lemma 12, it is trivial that
    \begin{equation*}
        \sum_{\iota=1}^{ 3 \lambda ^2 + 3 \lambda + 1 } 
                    1
                =
            3 \lambda ^2 + 3 \lambda + 1
    \end{equation*}
    By the inverse law for multiplication from the field axioms,
    by the distributive law for real numbers, 
    and by the identity lambda, that is
    \begin{equation*}
        \Bigg \{
            \lambda \sum_{\iota=1}^{ 3 \lambda ^2 + 3 \lambda + 1 } 
                        1
                    =
            \lambda \big [ 3 \lambda ^2 + 3 \lambda + 1 \big ]
        \Bigg \}
            \equiv
        \Bigg \{
            \sum_{\iota=1}^{ 3 \lambda ^2 + 3 \lambda + 1 } 
                        \lambda
                    =
            \lambda \big [ 3 \lambda ^2 + 3 \lambda + 1 \big ]
        \Bigg \}
            \equiv
    \end{equation*}
    \begin{equation*}
        \Bigg \{
            \sum_{\iota=1}^{ 3 \lambda ^2 + 3 \lambda + 1 } 
                        \big \lfloor \sqrt[3] \iota \big \rfloor
                    =
            \lambda \big [ 3 \lambda ^2 + 3 \lambda + 1 \big ]
        \Bigg \}
    \end{equation*}
\end{proof}


\end{document}