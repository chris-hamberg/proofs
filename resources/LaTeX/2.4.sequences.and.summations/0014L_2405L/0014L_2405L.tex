\documentclass[preview]{standalone}
\usepackage{amssymb, amsthm}
\usepackage{mathtools}
\usepackage{bm}


\newtheorem{lemma}{Lemma}
\renewcommand\qedsymbol{$\blacksquare$}


\begin{document}


\begin{lemma}[\textbf{2405}]
    Let \bm{$\iota$} be a positive integers such that
    \bm{$\big \lfloor \sqrt[3] \iota \big \rfloor = \lambda$}.
    \begin{equation*}
        \bm{
            \Bigg(
                3 \sum_{\phi=0}^{\lambda - 1} \phi ^3
            \Bigg)
                +
            \Bigg(
                3 \sum_{\phi=0}^{\lambda - 1} \phi ^2
            \Bigg)
                +
            \Bigg(
                \sum_{\phi=0}^{\lambda - 1} \phi
            \Bigg)
                \equiv
            \frac{
                \lambda ^3 - \lambda ^2
            }
            {4}
            \Bigg[ 3 \lambda + 1 \Bigg]
        }
    \end{equation*}
\end{lemma}

\begin{proof}
    Let \bm{$\Phi$} be three times the summation of cubes from zero to lambda minus one.
    Let \bm{$\mathrm{X}$} be three times the summation of squares from zero to lambda minus one.
    Let \bm{$\Omega$} be the summation of integers from zero to lambda minus one.
    By theorems 2422 and 2421b, 
    by the closed formula for the summation of cubes,
    and by shifting the index of summation,
    \begin{equation*}
        \Bigg \{
            \Phi + \mathrm{X} + \Omega
        \Bigg \}
            \equiv
        3
        \Bigg[
            \frac{
                \lambda ^2
                \big[ \lambda - 1 \big] ^2
            }
            {4}
        \Bigg]
            +
        3
        \Bigg[
            \frac{
                \lambda 
                \big[ \lambda - 1 \big]
                \big[ 2 \langle \lambda - 1 \rangle + 1 \big]
            }
            {6}
        \Bigg]
            +
        \Bigg[
            \frac{\lambda \big [ \lambda - 1 ]}
            {2}
        \Bigg]
    \end{equation*} 
    By the multiplicative identity law from the field axioms,
    \begin{equation*}
        \bigg \langle 
            \frac{3}{6}
        \bigg \rangle
             = 
        \bigg \langle
            \frac{2 \cdot 3}{2 \cdot 6} 
        \bigg \rangle 
            = 
        \bigg \langle
            \frac{6}{12} 
        \bigg \rangle 
            = 
        \bigg \langle
            \frac{2}{4}
        \bigg \rangle
    \end{equation*}
    \begin{equation*}
        \bigg \langle 
            \frac{1}{2}
        \bigg \rangle
            =
        \bigg \langle
            \frac{2 \cdot 1}{2 \cdot 2}
        \bigg \rangle
            =
        \bigg \langle 
            \frac{2}{4}
        \bigg \rangle
    \end{equation*}
    Thus, by the identities for \bm{$\mathrm{X}$} and \bm{$\Omega$},
    factoring 
    \bm{$\frac{1}{4} \lambda \big \langle \lambda - 1 \big \rangle$}
    out from the sum of \bm{$\Phi$}, \bm{$\mathrm{X}$}, and \bm{$\Omega$},
    by the distributive laws from real numbers,
    \begin{equation*}
        \Bigg \{
            \Phi + \mathrm{X} + \Omega
        \Bigg \}
                \equiv
            \frac{
                \lambda \big \langle \lambda - 1 \big \rangle
            }
            {4}
            \Bigg[
                3 \lambda \big[ \lambda - 1 \big]
                    +
                2 \bigg \langle 
                    2 \big[ \lambda - 1 \big] + 1
                \bigg \rangle
                    +
                2
            \Bigg]    
    \end{equation*}
    By the distributive law for real numbers, 
    and by the inverse law for addition from the field axioms,
    \begin{equation*}
        \bigg \{
            2 \Big \langle 2 \big[ \lambda - 1 \big] + 1 \Big \rangle + 2
        \bigg \}
            =
        \bigg \{
            4 \big[ \lambda - 1 \big] + 2 + 2
        \bigg \}
            =
        \bigg \{
            4 \lambda - 4 + 4
        \bigg \}
            = 
        \bigg \{
            4 \lambda
        \bigg \}
    \end{equation*}
    Hence, by the identity four lambda,
    \begin{equation*}
        \Bigg \{
            \Phi + \mathrm{X} + \Omega
        \Bigg \}
            \equiv
        \frac{
            \lambda \big \langle \lambda - 1 \big \rangle
        }
        {4}
        \Bigg[
            3 \lambda \big[ \lambda - 1 \big]
                +
            4 \lambda
        \Bigg]
    \end{equation*}
    Factoring out lambda and distributing three,
    by the distributive laws for real numbers,
    \begin{equation*}
        \Bigg\{
            \Phi + \mathrm{X} + \Omega
        \Bigg\}
                \equiv
            \frac{
                \lambda ^2 \big \langle \lambda - 1 \big \rangle
            }
            {4}
            \Bigg[
                3 \lambda - 3 + 4
            \Bigg]
    \end{equation*}
    The proof is complete, 
    by the distributive law for real numbers distributing
    the second power of lambda,
    and the inverse law for addition from the field axioms.
\end{proof}


\end{document}