\documentclass[preview]{standalone}
\usepackage{amssymb, amsthm}
\usepackage{mathtools}
\usepackage{bm}


\newtheorem{theorem}{Theorem}
\renewcommand\qedsymbol{$\blacksquare$}


\begin{document}


\begin{theorem}[\textbf{2421a}]
    The summation of odd numbers from \bm{$1$} to \bm{$\phi$} is 
    \bm{$\phi^{2}$}.
\end{theorem}

\begin{proof}
    There exists an integer \bm{$\lambda$}, by the definition of odd numbers, such that
    the summation of odd numbers from \bm{$1$} to \bm{$\phi$} is given by, 
    \begin{equation*}
        \sum_{\lambda=1}^\phi 
                2 \lambda - 1
    \end{equation*}
    The identity for \bm{$2 \lambda - 1$} is the difference of squares 
    \bm{$\lambda ^2 - \big \langle \lambda - 1 \big \rangle ^2$}. 
    This identity can be demonstrated by the statement 
    \begin{equation*}
        \bigg \langle \lambda ^2 - [ \lambda - 1 ] ^2 \bigg \rangle
            = 
        \bigg \langle 
            \bigg[ 
                \lambda  + \langle \lambda - 1 \rangle 
            \bigg]
            \bigg[
                \lambda - \langle \lambda - 1 \rangle 
            \bigg] 
        \bigg \rangle
            =
    \end{equation*}
    \begin{equation*} 
        \bigg \langle 
            \bigg[
                2 \lambda - 1
            \bigg]
            \bigg[
                \lambda + \langle - \lambda + 1 \rangle
            \bigg]
        \bigg \rangle
            = 
        \bigg \langle [ 2 \lambda - 1 ] 1 \bigg \rangle
    \end{equation*}
    So the summation of odd numbers from \bm{$1$} to \bm{$\phi$} 
    is the telescoping summation
    \begin{equation*}
        \sum_{\lambda=1}^\phi 
                \lambda ^2 - \langle \lambda - 1 \rangle ^2
    \end{equation*}
    By Theorem 99, that is \bm{$\phi ^2 - 0 ^2 = \phi ^2$}. 
    Thus, 
    \begin{equation*}
        \sum_{\lambda=1}^\phi 2 \lambda - 1 = \phi^2
    \end{equation*}
    and indeed the summation of odd numbers from \bm{$1$} to \bm{$\phi$} 
    is \bm{$\phi ^2$}.
\end{proof}


\end{document}