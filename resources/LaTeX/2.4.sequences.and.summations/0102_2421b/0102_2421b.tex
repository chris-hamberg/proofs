\documentclass[preview]{standalone}
\usepackage{amssymb, amsthm}
\usepackage{mathtools}
\usepackage{bm}


\newtheorem{theorem}{Theorem}
\renewcommand\qedsymbol{$\blacksquare$}


\begin{document}


\begin{theorem}[\textbf{2421b}]
    The summation of natural numbers from \bm{$1$} to \bm{$\lambda$} 
    is 
    \begin{equation*}
        \bm{
            \frac{ \lambda [ \lambda + 1 ]}{2}
        }
    \end{equation*}
\end{theorem}

\begin{proof}
    It is possible to derive the closed formula for the summation of natural numbers
    from \bm{$1$} to \bm{$\lambda$} 
    from the summation of odd numbers from \bm{$1$} to \bm{$\lambda$}.
    By the definition for odd numbers, 
    an integer \bm{$\phi$} exists such that,
    by Theorem 2.4.21a
    \begin{equation*}
        \sum_{\phi=1}^\lambda 2 \phi - 1 
            = 
        \lambda ^2
    \end{equation*}
    By the associative law for addition from the field axioms,
    \begin{equation*}
        \bigg \langle \sum_{\phi=1}^\lambda 2 \phi - 1 \bigg \rangle
            \equiv
        \bigg \langle 
            \sum_{\phi=1}^\lambda 2 \phi 
                +
            \sum_{\phi=1}^\lambda - 1
        \bigg \rangle
            \equiv
        \bigg \langle 
            -\lambda 
                +
            \sum_{\phi=1}^\lambda 2 \phi 
        \bigg \rangle
    \end{equation*}
    Thus, by that identity, and by the inverse law for addition from the field axioms,
    \begin{equation*}
        \Bigg\{
            \bigg \langle \lambda ^2 \bigg \rangle
                = 
            \bigg \langle 
                -\lambda 
                    + 
                \sum_{\phi=1}^\lambda 2 \phi 
            \bigg \rangle
        \Bigg\}
            \equiv 
        \Bigg\{
            \bigg \langle \sum_{\phi=1}^\lambda 2 \phi \bigg \rangle 
                = 
            \bigg \langle \lambda ^2 + \lambda \bigg \rangle 
                = 
            \lambda \bigg \langle \lambda + 1 \bigg \rangle
        \Bigg\}
    \end{equation*}
    By the distributive laws for real numbers from the field axioms, that is
    \begin{equation*}
        2 \sum_{\phi=1}^\lambda \phi 
            = 
        \lambda \Big \langle \lambda + 1 \Big \rangle
    \end{equation*} 
    And by the inverse law for multiplication from the field axioms,
    \begin{equation*}
        \bm{
            \sum_{\phi=1}^\lambda \phi 
                = 
            \frac{ \lambda [ \lambda + 1 ] }{2}
        }
    \end{equation*}
\end{proof}


\end{document}