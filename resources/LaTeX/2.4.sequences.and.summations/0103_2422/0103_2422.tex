\documentclass[preview]{standalone}
\usepackage{amssymb, amsthm}
\usepackage{mathtools}
\usepackage{bm}


\newtheorem{theorem}{Theorem}
\renewcommand\qedsymbol{$\blacksquare$}


\begin{document}


\begin{theorem}[\textbf{2422}]
    The sum of squares from \bm{$1$} to \bm{$\lambda$} is 
    \begin{equation*}
        \bm{
            \frac{ 
                \lambda 
                \langle \lambda + 1 \rangle 
                \langle 2 \lambda + 1 \rangle 
            }
            {6}
        }
    \end{equation*}
\end{theorem}

\begin{proof}
    The formula for the summation of squares 
    from \bm{$1$} to \bm{$\lambda$} 
    can be derived from the cube of \bm{$\lambda$}.
    It is trivial that 
    \bm{$\lambda ^3 = \lambda ^3 - \big \langle 1 - 1 \big \rangle ^3$}. 
    By this identity for \bm{$\lambda$},
    the cube of \bm{$\lambda$} is the telescopic summation given by Theorem 2419, 
    \begin{equation*}
        \lambda ^3 
            = 
        \sum_{\iota=1}^\lambda \iota ^3 - \big \langle \iota - 1 \big \rangle ^3    
    \end{equation*}
    The expansion for 
    \bm{$\big \langle \iota - 1 \big \rangle ^3$} 
    is 
    \bm{$\iota ^3 - 3 \iota ^2 + 3 \iota - 1$}, 
    by the Binomial Theorem. 
    Thus, 
    by the inverse law for addition from the field axioms,
    yielding the algebraic identity
    \begin{equation*}
        \iota ^3 - \big \langle \iota - 1 \big \rangle ^3 
            = 
        3 \iota ^2 - 3 \iota + 1
    \end{equation*}
    Hence, 
    \bm{$\lambda ^3 = \sum_{\iota=1}^\lambda 3 \iota ^2 - 3 \iota + 1$}.
    By the commutative law for addition from the field axioms, 
    and by the distributive law for real numbers,
    that is
    \begin{equation*}
        \lambda ^3 
            = 
        \left( 3 \sum_{\iota=1}^\lambda \iota ^2 \right) 
            - 
        \left( 3 \sum_{\iota=1}^\lambda \iota \right) 
            + 
        \left( \sum_{\iota=1}^\lambda 1 \right)
    \end{equation*}
    Note that \bm{$
        \sum_{\iota=1} ^\lambda 1 
            = 
        \lambda \big \langle 1 \big \rangle
    $}. 
    And by Theorem 2421b, 
    \bm{$
        \sum_{\iota=1}^\lambda \iota 
            = 
        \frac{ \lambda \big \langle \lambda + 1 \big \rangle }{2}
    $}. 
    Thus, 
    by those identities, 
    and by the inverse law for addition from the field axioms
    \begin{equation*}
        \lambda ^3 
            + 
        3 \frac{ \lambda \big \langle \lambda + 1 \big \rangle }
        {2} 
            - 
        \lambda 
            = 
        3 \sum_{\iota=1}^\lambda \iota ^2
    \end{equation*}
    Eliminating the coefficient \bm{$3$} from the right-hand side, 
    by the inverse law for multiplication from the field axioms, 
    gives us the sum of squares in terms of an equation, 
    \begin{equation*}
        \frac{1}{3}
        \bigg[
            \lambda ^3 
                + 
            3 \frac{ \lambda \big \langle \lambda + 1 \big \rangle }
            {2} 
                - 
            \lambda
        \bigg]
            = 
        \sum_{\iota=1}^\lambda \iota ^2
    \end{equation*}
    By Lemma 2401, that is
    \begin{equation*}
        \bm{
            \sum_{\iota=1}^\lambda \iota ^2 
                = 
            \frac{ 
                \lambda \big \langle \lambda + 1 \big \rangle
                \big \langle 2 \lambda + 1 \big \rangle 
            }
            {6}
        }
    \end{equation*}
\end{proof}


\end{document}