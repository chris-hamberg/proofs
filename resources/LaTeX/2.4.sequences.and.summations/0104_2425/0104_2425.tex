\documentclass[preview]{standalone}
\usepackage{amssymb, amsthm}
\usepackage{mathtools}
\usepackage{bm}


\newtheorem{theorem}{Theorem}
\renewcommand\qedsymbol{$\blacksquare$}


\begin{document}


\begin{theorem}[\textbf{2425}]
    
    \raggedright Let \bm{$\phi$} be a positive integer such that
    \raggedright \bm{$\lfloor \sqrt \phi \rfloor = \lambda$}.
    \raggedright The closed form formula for 
    \raggedright \bm{$
        \Phi
            = 
        \sum_{\iota=0}^\phi 
                \big \lfloor \sqrt \iota \big \rfloor
    $} 
    is 
    \begin{equation*}
        \bm{
            \frac{
                \lambda
                \big \langle \lambda - 1 \big \rangle
            }
            {6}
            \Bigg[
                4 \lambda + 1
            \Bigg]
                +
            \lambda
            \Bigg[
                \phi
                    -
                \lambda ^2
                    +
                1
            \Bigg]
        }
    \end{equation*}
\end{theorem}

\begin{proof}
    By the associative law for addition from the field axioms,
    \begin{equation*}
        \Theta
            = 
        \sum_{\iota=0}^{\lambda - 1} 
                \big \lfloor \sqrt \iota \big \rfloor 
            =
    \end{equation*}
    \begin{equation*}
        \Bigg (
            \sum_{\iota=1}^{ 2 \beta_0 + 1 } 
                    \big \lfloor \sqrt \iota \big \rfloor_0 
        \Bigg )
            +
        \Bigg (
            \sum_{\iota=1 + 2 \beta_0 + 1}^{ 2 \beta_0 + 1 + 2 \beta_1 + 1 } 
                    \big \lfloor \sqrt \iota \big \rfloor_1 
        \Bigg )
            +
        \dots
            +
        \Bigg (
            \sum_{\iota=1 +  \dots + 2 \beta_{\langle \lambda - 2 \rangle} + 1 }
                ^{0 + \dots + 2 \beta_{\langle \lambda - 1 \rangle} + 1} 
                    \big \lfloor \sqrt \iota \big \rfloor_{
                        \langle \lambda - 1 \rangle
                    } 
        \Bigg )
    \end{equation*}
    \bm{$\big \lfloor \sqrt \iota \big \rfloor_\tau$}
    is the unique integer \bm{$\beta_\tau$}, 
    for \bm{$\tau = 0$} to \bm{$\big \langle \lambda - 1 \big \rangle$}.
    Hence, by Lemma 2403 and the distributive law for real numbers from the field axioms, 
    by the partitioning from above, and shifting the index of summation,
    \begin{equation*}
        \Theta
            =
        \Big \langle 2 \beta_{0} ^2 + \beta_0 \Big \rangle
            +
        \Big \langle 2 \beta_{1} ^2 + \beta_1 \Big \rangle
            +
        \dots
            +
        \Big \langle 
            2 \beta_{[ \lambda - 1 ]} ^2
                +
            \beta_{[ \lambda - 1 ]}
        \Big \rangle
    \end{equation*}
    By the commutative law for addition from the field axioms,
    that series is two times the finite summation of squares, 
    plus the finite summation of integers from zero to lambda minus one.
    \begin{equation*}
        \Theta
            = 
        \sum_{\tau=0}^{ \lambda - 1 } 2 \tau ^2 
            +
        \sum_{\tau=0}^{ \lambda - 1 } \tau
    \end{equation*}
    Lambda occurs exactly 
    \bm{$
        \big \langle 
            \phi - \lambda ^2 + 1
        \big \rangle
    $} 
    times in big phi.
    Thus, by Lemma 2402 and the distributive law for real numbers from the field axioms,
    \begin{equation*}
        \bm{
            \Phi
                =
            \frac{
                \lambda
                \big \langle \lambda - 1 \big \rangle
            }
            {6}
            \Bigg[
                4 \lambda + 1
            \Bigg]
                +
            \lambda
            \Bigg[
                \phi
                    -
                \lambda ^2
                    +
                1
            \Bigg]
        }
    \end{equation*}
\end{proof}


\end{document}