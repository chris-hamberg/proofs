\documentclass[preview]{standalone}
\usepackage{amssymb, amsthm}
\usepackage{mathtools}
\usepackage{bm}


\newtheorem{theorem}{Theorem}
\renewcommand\qedsymbol{$\blacksquare$}


\begin{document}


\begin{theorem}[\textbf{2426}]
    \raggedright Let \bm{$\phi$} be a positive integer such that 
    \raggedright \bm{$\big \lfloor \sqrt[3] \phi \big \rfloor = \lambda$}.
    \raggedright The closed form formula for 
    \raggedright \bm{
        $\Phi = \sum_{\iota=0}^\phi 
                \big \lfloor \sqrt[3] \iota \big \rfloor
    $} 
    is 
    \begin{equation*}
        \bm{
            \frac{
                \lambda ^3 - \lambda ^2
            }
            {4}
            \Bigg[ 3 \lambda + 1 \Bigg]
                + 
            \lambda
            \Bigg[
                \phi - \lambda ^3 + 1
            \Bigg]
        }
    \end{equation*}
\end{theorem}

\begin{proof}
    Let \bm{$\mathrm{X}$} be the function 
    \bm{$\mathrm{X}: \mathbb{N} \rightarrow \mathbb{N}$}
    such that \bm{$\mathrm{X}[\beta] = 3 \beta ^2 + 3 \beta + 1$}.
    By the associative law for addition from the field axioms,
    \begin{equation*}
        \Theta
            =
        \sum_{\iota=0}^{\lambda - 1}
                \big \lfloor \sqrt[3] \iota \big \rfloor
            =
    \end{equation*}
    \begin{equation*}
        \Bigg(
            \sum_{ \iota = 1 }^{ \mathrm{X} [\beta_0] }
                \big \lfloor \sqrt[3] \iota \big \rfloor_0
        \Bigg)
            +
        \Bigg(
            \sum_{
                \iota = 1 + \mathrm{X}[\beta_0]
            }^{ 
                \mathrm{X}[\beta_0] + \mathrm{X}[\beta_1]
            }
                \big \lfloor \sqrt[3] \iota \big \rfloor_1
        \Bigg)
            +
        \dots
            +
        \Bigg(
            \sum_{
                \iota = 1 + \dots 
                    + 
                \mathrm{X}[\beta_{\lambda - 2}]
            }^{ 
                0 + \dots
                    +
                \mathrm{X}[\beta_{ \lambda - 1 }]
            }
                \big \lfloor \sqrt[3] \iota \big \rfloor_{ \langle \lambda - 1 \rangle }
        \Bigg)
    \end{equation*}
    \bm{$\big \lfloor \sqrt[3] \iota \big \rfloor_\tau$}
    is the unique integer \bm{$\beta_\tau$},
    for \bm{$\tau = 0$} to 
    \bm{$\big \langle \lambda - 1 \big \rangle$}.
    Hence, by Lemma 2404,
    by the partitioning from above, 
    and shifting the index of summation,
    \begin{equation*}
        \Theta
            =
        \Big \langle \beta_0 \mathrm{X} [\beta_0] \Big \rangle
            +
        \Big \langle \beta_1 \mathrm{X} [\beta_1] \Big \rangle
            + 
        \dots 
            +
        \Big \langle 
            \beta_{\langle \lambda - 1 \rangle}
            \mathrm{X} [\beta_{\langle \lambda - 1 \rangle}]
        \Big \rangle
    \end{equation*}
    By the distributive law for real numbers from the field axioms,
    and the commutative law for addition, that series is
    three times the finite summation of cubes,
    plus three times the finite summation of squares,
    plus the finite summation of integers from zero to lambda minus one.
    \begin{equation*}
        \Theta 
            = 
        \sum_{\tau=0}^{\lambda - 1} 3 \tau ^3
            +
        \sum_{\tau=0}^{\lambda - 1} 3 \tau ^2
            +
        \sum_{\tau=0}^{\lambda - 1} \tau
    \end{equation*}
    Lambda occurs exactly  
    \bm{$\big \langle \phi - \lambda^3 + 1 \big \rangle$}
    times in big phi. 
    Thus, by Lemma 2405
    and the distributive law for real numbers from the field axioms,
    \begin{equation*}
        \bm{
            \Phi
                =
            \frac{
                \lambda ^3 - \lambda ^2
            }
            {4}
            \Bigg[ 3 \lambda + 1 \Bigg]
                + 
            \lambda
            \Bigg[
                \phi - \lambda ^3 + 1
            \Bigg]
        }
    \end{equation*}
\end{proof}


\end{document}