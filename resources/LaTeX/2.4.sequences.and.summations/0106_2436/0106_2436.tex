\documentclass[preview]{standalone}
\usepackage{amssymb, amsthm}
\usepackage{mathtools}
\usepackage{bm}


\newtheorem{theorem}{Theorem}
\renewcommand\qedsymbol{$\blacksquare$}


\begin{document}


\begin{theorem}[\textbf{2436}]
    A subset of a countable set is countable.
\end{theorem}

\begin{proof}
    Let \bm{$\mathrm{A}$} and \bm{$\Lambda$} be sets such that 
    \bm{$\mathrm{A}$} is a subset of the countable set \bm{$\Lambda$}. 
    By the definition for countability,
    the cardinality of \bm{$\Lambda$} is less than or equal to \bm{$\aleph_0$}. 
    By the definition for subsets, 
    the cardinality of \bm{$\mathrm{A}$}
    is less than or equal to \bm{$\Lambda$}.
    Hence,
    the cardinality of \bm{$\mathrm{A}$} is less than or equal to \bm{$\aleph_0$}
    \bm{$\therefore$} the subset of a countable set is countable.
\end{proof}


\end{document}