\documentclass[preview]{standalone}
\usepackage{amssymb, amsthm}
\usepackage{mathtools}
\usepackage{bm}


\newtheorem{theorem}{Theorem}
\renewcommand\qedsymbol{$\blacksquare$}


\begin{document}


\begin{theorem}[\textbf{2442}]
    The cardinality of $\mathbb{Z^{+}} \times \mathbb{Z^{+}}$ is aleph null.
\end{theorem}

\begin{proof}
    $\mathbb{Z^{+}} \times \mathbb{Z^{+}}$ is defined as 
    $\{ \langle x, y \rangle | (x \in \mathbb{Z^{+}}) \land (y \in \mathbb{Z^{+}})\}$. 
    Since $x$ and $y$ are positive integers, for every ordered pair $\langle x, y \rangle$ in 
    $\mathbb{Z^{+}} \times \mathbb{Z^{+}}$, $\langle x, y \rangle$ exists if and only if the rational number 
    $\frac{x}{y}$ exists. Thus, $\frac{x}{y}$ exists, and all elements in 
    $\mathbb{Z^{+}} \times \mathbb{Z^{+}}$ can be represented by the two dimensional list
    $$\langle 1,1 \rangle \iff \frac{1}{1}, \langle 1,2 \rangle \iff \frac{1}{2}, \langle 1,3 \rangle \iff \frac{1}{3}, \dots$$
    $$\langle 2,1 \rangle \iff \frac{2}{1}, \langle 2,2 \rangle \iff \frac{2}{2}, \langle 2,3 \rangle \iff \frac{2}{3}, \dots$$
    $$\langle 3,1 \rangle \iff \frac{3}{1}, \langle 3,2 \rangle \iff \frac{3}{2}, \langle 3,3 \rangle \iff \frac{3}{3}, \dots$$
    $$\vdots$$
    The hypotheses in the biconditional converse statements for each list entry are the list 
    elements in the proof for the countability of rational numbers. That means 
    $\mathbb{Z^{+}} \times \mathbb{Z^{+}}$ is countable if and only if the rational numbers are 
    countable. We know the rational numbers are countable. Therefore the cardinality of 
    $\mathbb{Z^{+}} \times \mathbb{Z^{+}}$ is $\aleph_0$.
\end{proof}


\end{document}