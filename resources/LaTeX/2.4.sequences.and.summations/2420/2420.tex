\documentclass[preview]{standalone}
\usepackage{amssymb, amsthm}
\usepackage{mathtools}
\usepackage{bm}
\usepackage{xcolor}


\newtheorem*{theorem*}{Theorem}
\renewcommand\qedsymbol{$\blacksquare$}


\begin{document}


\begin{theorem*}[\textbf{2420}]
    \begin{equation*}
        \bm{
            \sum_{\epsilon=1}^\lambda 
                    \bigg \langle 
                        \frac{1}{\epsilon [ \epsilon + 1 ]}
                    \bigg \rangle
                = 
            \frac{\lambda}{\lambda + 1}
        }
    \end{equation*}
\end{theorem*}

\begin{proof}
    The identity \bm{$\Big \langle \frac{1}{\epsilon [ \epsilon + 1 ]} \Big \rangle$} is 
    \bm{$\Big \langle \frac{1}{\epsilon} - \frac{1}{[ \epsilon + 1 ]} \Big \rangle$}. 
    This can be demonstrated by the equation
    \begin{equation*}
        \epsilon \bigg \langle \frac{1}{\epsilon} - \frac{1}{ [ \epsilon + 1 ] } \bigg \rangle 
            = 
        \bigg \langle \frac{\epsilon + 1}{\epsilon + 1} - \frac{\epsilon}{\epsilon + 1} \bigg \rangle
            = 
        \bigg \langle \frac{\epsilon + 1 - \epsilon}{\epsilon + 1} \bigg \rangle
            = 
        \bigg \langle \frac{1}{\epsilon + 1} \bigg \rangle
    \end{equation*}
    Dividing both sides of this equation by \bm{$\epsilon$},
    by the inverse law for multiplication from the field axioms,
    gives the desired identity such that
    \begin{equation*}
        \bigg \langle 
        \sum_{\epsilon=1}^\lambda 
                %\bigg \langle 
                    \frac{1}{\epsilon [ \epsilon + 1 ]}
                \bigg \rangle 
            = 
        \bigg \langle 
        \sum_{\epsilon=1}^\lambda 
                %\bigg \langle 
                    \frac{1}{\epsilon} - \frac{1}{\epsilon + 1} 
                \bigg \rangle
    \end{equation*}
    The sequence for which is the telescopic summation
    \begin{equation*}
        \bigg \langle \frac{1}{\lambda} - \frac{1}{\lambda + 1} \bigg \rangle
            + 
        \bigg \langle \frac{1}{\lambda - 1} - \frac{1}{\lambda} \bigg \rangle 
            + 
        \bigg \langle \frac{1}{\lambda - 2} - \frac{1}{\lambda - 1} \bigg \rangle
            + 
        \dots 
            + 
        \bigg \langle \frac{1}{1} - \frac{1}{2} \bigg \rangle
    \end{equation*}
    Thus, by Theorem 2419
    \begin{equation*}
        \bm{
            \bigg \langle
            \sum_{\epsilon=1}^\lambda
                    %\bigg \langle
                        \frac{1}{ \epsilon [ \epsilon + 1 ] } 
                    \bigg \rangle
                = 
            \bigg \langle -\frac{1}{\lambda + 1} + \frac{1}{1} \bigg \rangle 
                = 
            \bigg \langle \frac{ [ -1 ] + [ \lambda + 1 ] }{\lambda + 1} \bigg \rangle 
                = 
            \frac{\lambda}{\lambda+1}
        }
    \end{equation*}
\color{lightgray} \end{proof}

\end{document}