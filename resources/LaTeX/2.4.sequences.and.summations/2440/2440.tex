\documentclass[preview]{standalone}
\usepackage{amssymb, amsthm}
\usepackage{mathtools}
\usepackage{bm}
\usepackage{xcolor}


\newtheorem*{theorem*}{Theorem}
\renewcommand\qedsymbol{$\blacksquare$}


\begin{document}


\begin{theorem*}[\textbf{2440}]
    The union of two countable sets is countable.
\end{theorem*}

\begin{proof}
    By cases. Let \bm{$\mathrm{A}$}, and \bm{$\Lambda$} be countable sets. 
    There are three possible cases. 
    $(i)$ \bm{$\mathrm{A}$} and \bm{$\Lambda$} are finite, 
    $(ii)$ exclusively \bm{$\mathrm{A}$} or \bm{$\Lambda$} is finite and the other is countably infinite, 
    $(iii)$ \bm{$\mathrm{A}$} and \bm{$\Lambda$} are both countably infinite.
    \\ \\
    \bm{$(i)$} Assume \bm{$\mathrm{A}$} and \bm{$\Lambda$} are finite. 
    There exist natural numbers \bm{$\lambda$}, and \bm{$\iota$} such that 
    \bm{$\big |\mathrm{A} \big | = \lambda$} and 
    \bm{$\big | \Lambda \big | = \iota$}. 
    The maximum cardinality for \bm{$\mathrm{A} \cup \Lambda$} 
    occurs when the intersection of \bm{$\mathrm{A}$} and \bm{$\Lambda$} is 
    the empty set. The cardinality for such a union is \bm{$\lambda + \iota$}. 
    \bm{$\lambda + \iota$} is a natural number by the closure property for addition on integers. 
    Thus, \bm{$\lambda + \iota$} is less than \bm{$\aleph_0$}. 
    By the definition for countably finite sets, \bm{$\mathrm{A} \cup \Lambda$} is countable. 
    \\ \\
    \bm{$(ii)$} Without loss of generality assume \bm{$\mathrm{A}$} is finite with cardinality \bm{$\lambda$}, 
    and \bm{$\Lambda$} is countably infinite. 
    A finite sequence 
    \bm{$\big \{ \alpha_\iota \big \}$}
    containing all members of \bm{$\mathrm{A}$},
    and an infinite sequence 
    \bm{$\big \{ \beta_{ \mathbb{N} } \big \}$}
    containing all members of \bm{$\Lambda$}
    exists.
    For the union of \bm{$\mathrm{A}$} and \bm{$\Lambda$} 
    there exists a sequence \bm{$ \big \{ \delta \big \} $} such that
    \begin{equation*}
        \{ 
            \delta_\mathbb{N} 
        \} 
            = 
        \{\alpha_0, \alpha_1, \dots, \alpha_\lambda, 
        \beta_{\lambda+1}, \beta_{\lambda +2}, \beta_{\lambda + 3}, \dots \}
    \end{equation*}
    Clearly this is countably infinite by the definition for countability 
    since \bm{$\lambda + \chi$} is a natural number 
    for all \bm{$\chi$} in natural numbers, 
    by the closure property for addition on natural numbers. 
    \\ \\
    \bm{$(iii)$} assume both \bm{$\mathrm{A}$} and \bm{$\Lambda$} are countable infinite sets. 
    Since each set cardinality is \bm{$\aleph_0$}, 
    an infinite sequence 
    \bm{$\big \{ \alpha_\mathbb{N} \big \}$}
    containing all members of \bm{$\mathrm{A}$},
    and an infinite sequence 
    \bm{$\big \{ \beta_{ \mathbb{N} } \big \}$}
    containing all members of \bm{$\Lambda$}
    exist.
    For the union of \bm{$\mathrm{A}$} and \bm{$\Lambda$}
    there exists a infinite sequence \bm{$\big \{ \delta \big \} $} such that
    \begin{equation*}
        \big \{ \delta_\mathbb{N} \big \} 
            = 
        \big \{
            \alpha_{0_0}, \beta_{0_1}, 
            \alpha_{1_2}, \beta_{1_3}, 
            \alpha_{2_4}, \beta_{2_5}, 
            \dots
        \big \}    
    \end{equation*}
    Thus a bijection 
    exists between \bm{$\mathbb{N}$} and the union of \bm{$\mathrm{A}$} and \bm{$\Lambda$}, 
    and that union is countable by the definition for countability.
\color{lightgray} \end{proof}

\end{document}