\documentclass[preview]{standalone}
\usepackage{amssymb, amsthm}
\usepackage{mathtools}
\usepackage{bm}
\usepackage{xcolor}


\newtheorem*{theorem*}{Theorem}
\renewcommand\qedsymbol{$\blacksquare$}


\begin{document}


\begin{theorem*}[\textbf{2443}]
    The set of all finite bit strings is countable.
\end{theorem*}

\begin{proof}
    Let $\{a_{n-1}\}$ be the sequence of bits for any finite bit string $a($base-$2)$ of length $n$. 
    The unique base-$2$ expansion for $\{a_{n-1}\}$ is the integer
    $$a(\text{base-}10) = \sum_{i = 0}^{n-1} a_{i}2^{i}$$
    Also, this integer can be converted to the unique base-$2$ bit string for \\ $a($base-$10)$ by 
    $$a(\text{base-}2) = 
    \sum_{i = 0}^{n-1}\left[a(\text{base-}10)(\text{mod } 2^{i+1})\right]10^{i}$$ 
    Since an invertible function exists between each finite bit string and some positive integer, 
    there exists, a one-to-one correspondence between $\mathbb{Z}$ and the set of all 
    finite bit strings. Thus, the cardinality for the set of all finite bit strings is $\aleph_0$,
    and the set of all finite bit strings is countable, by definition.
\color{lightgray} \end{proof}

\end{document}