\documentclass[preview]{standalone}
\usepackage{amssymb, amsthm}
\usepackage{mathtools}
\usepackage{bm}
\usepackage{xcolor}
\usepackage{standalone}


%\newtheorem*{theorem*}{Theorem}
%\newtheorem*{lemma*}{Lemma}
%\renewcommand\qedsymbol{$\blacksquare$}


\begin{document}


\section{Theorems}


% ============================== 0099 Theorem 2419 ================================
\subsection[Telescopic summations.]{
    \color{section} Theorem 99. \color{black} Telescopic summations.
    }
\documentclass[preview]{standalone}
\usepackage{amssymb, amsthm}
\usepackage{mathtools}
\usepackage{bm}


\newtheorem{theorem}{Theorem}
\renewcommand\qedsymbol{$\blacksquare$}


\begin{document}


\begin{theorem}[\textbf{2419}]
    Let \bm{$\{\lambda_\zeta\}$} be a sequence of real numbers.
    \begin{equation*}
        \bm{
            \sum_{\iota=1}^\zeta \big \langle 
                \lambda_\iota - \lambda_{\iota-1}
            \big \rangle
                = 
            \lambda_\zeta - \lambda_0
        }
    \end{equation*}
\end{theorem}

\begin{proof}
    \begin{equation*}
        \sum_{\iota=1}^\zeta \big \langle
            \lambda_\iota - \lambda_{\iota-1}
        \big \rangle
            =
        \big \langle \lambda_\zeta - \lambda_{\zeta-1} \big \rangle 
            + 
        \big \langle \lambda_{\zeta-1} - \lambda_{\zeta-2} \big \rangle 
            + 
        \dots 
            + 
        \big \langle \lambda_1 - \lambda_0 \big \rangle
    \end{equation*}
    By associativity for addition from the field axioms for real numbers, that is 
    \begin{equation*}
        \lambda_\zeta 
            + 
        \big \langle -\lambda_{\zeta-1} + \lambda_{\zeta-1} \big \rangle
            + 
        \big \langle -\lambda_{\zeta-2} + \lambda_{\zeta-2} \big \rangle
        %    + 
        %\big \langle -\lambda_{\zeta-3} + \lambda_{\zeta-3} \big \rangle
            + 
        \dots 
            + 
        \big \langle -\lambda_1 + \lambda_1 \big \rangle
            + 
        -\lambda_0
    \end{equation*}
    The inner terms cancel out 
    by the inverse law for addition from the field axioms 
    $\therefore$
    \begin{equation*}
        \bm{
            \sum_{\iota=1}^\zeta \big \langle 
                \lambda_\iota - \lambda_{\iota-1}
            \big \rangle
                = 
            \lambda_\zeta - \lambda_0
        }
    \end{equation*} 
\end{proof}


\end{document}
\sep
\pagebreak


% ============================== 0100 Theorem 2420 ================================
\subsection[Lambda divided by lambda plus one is telescopic.]{
    \color{section} Theorem 100. \color{black} Lambda over lambda plus 1 is telescopic
}
\vspace{-0.8cm}
\documentclass[preview]{standalone}
\usepackage{amssymb, amsthm}
\usepackage{mathtools}
\usepackage{bm}


\newtheorem{theorem}{Theorem}
\renewcommand\qedsymbol{$\blacksquare$}


\begin{document}


\begin{theorem}[\textbf{2420}]
    \begin{equation*}
        \bm{
            \sum_{\epsilon=1}^\lambda 
                    \bigg \langle 
                        \frac{1}{\epsilon [ \epsilon + 1 ]}
                    \bigg \rangle
                = 
            \frac{\lambda}{\lambda + 1}
        }
    \end{equation*}
\end{theorem}

\begin{proof}
    The identity \bm{$\Big \langle \frac{1}{\epsilon [ \epsilon + 1 ]} \Big \rangle$} is 
    \bm{$\Big \langle \frac{1}{\epsilon} - \frac{1}{[ \epsilon + 1 ]} \Big \rangle$}. 
    This can be demonstrated by the equation
    \begin{equation*}
        \epsilon \bigg \langle \frac{1}{\epsilon} - \frac{1}{ [ \epsilon + 1 ] } \bigg \rangle 
            = 
        \bigg \langle \frac{\epsilon + 1}{\epsilon + 1} - \frac{\epsilon}{\epsilon + 1} \bigg \rangle
            = 
        \bigg \langle \frac{\epsilon + 1 - \epsilon}{\epsilon + 1} \bigg \rangle
            = 
        \bigg \langle \frac{1}{\epsilon + 1} \bigg \rangle
    \end{equation*}
    Dividing both sides of this equation by \bm{$\epsilon$},
    by the inverse law for multiplication from the field axioms,
    gives the desired identity such that
    \begin{equation*}
        \bigg \langle 
        \sum_{\epsilon=1}^\lambda 
                %\bigg \langle 
                    \frac{1}{\epsilon [ \epsilon + 1 ]}
                \bigg \rangle 
            = 
        \bigg \langle 
        \sum_{\epsilon=1}^\lambda 
                %\bigg \langle 
                    \frac{1}{\epsilon} - \frac{1}{\epsilon + 1} 
                \bigg \rangle
    \end{equation*}
    The sequence for which is the telescopic summation
    \begin{equation*}
        \bigg \langle \frac{1}{\lambda} - \frac{1}{\lambda + 1} \bigg \rangle
            + 
        \bigg \langle \frac{1}{\lambda - 1} - \frac{1}{\lambda} \bigg \rangle 
            + 
        \bigg \langle \frac{1}{\lambda - 2} - \frac{1}{\lambda - 1} \bigg \rangle
            + 
        \dots 
            + 
        \bigg \langle \frac{1}{1} - \frac{1}{2} \bigg \rangle
    \end{equation*}
    Thus, by Theorem 99
    \begin{equation*}
        \bm{
            \bigg \langle
            \sum_{\epsilon=1}^\lambda
                    %\bigg \langle
                        \frac{1}{ \epsilon [ \epsilon + 1 ] } 
                    \bigg \rangle
                = 
            \bigg \langle -\frac{1}{\lambda + 1} + \frac{1}{1} \bigg \rangle 
                = 
            \bigg \langle \frac{ [ -1 ] + [ \lambda + 1 ] }{\lambda + 1} \bigg \rangle 
                = 
            \frac{\lambda}{\lambda+1}
        }
    \end{equation*}
\end{proof}


\end{document}
\sep
\pagebreak


% ============================= 0101 Theorem 2421a ================================
\subsection[The summation of odd numbers is a square.]{
    \color{section} Theorem 101. \color{black} The summation of odd numbers.
}
\input{../resources/LaTeX/2.4.sequences.and.summations/0101_2421a/0101_2421a.tex}
\sep
\pagebreak

% ============================= 0102 Theorem 2421b ================================
\subsection[The summation of natural numbers.]{
    \color{section} Theorem 102. \color{black} The summation of natural numbers.
}
\documentclass[preview]{standalone}
\usepackage{amssymb, amsthm}
\usepackage{mathtools}
\usepackage{bm}


\newtheorem{theorem}{Theorem}
\renewcommand\qedsymbol{$\blacksquare$}


\begin{document}


\begin{theorem}[\textbf{2421b}]
    The summation of natural numbers from \bm{$1$} to \bm{$\lambda$} 
    is 
    \begin{equation*}
        \bm{
            \frac{ \lambda [ \lambda + 1 ]}{2}
        }
    \end{equation*}
\end{theorem}

\begin{proof}
    It is possible to derive the closed formula for the summation of natural numbers
    from \bm{$1$} to \bm{$\lambda$} 
    from the summation of odd numbers from \bm{$1$} to \bm{$\lambda$}.
    By the definition for odd numbers, 
    an integer \bm{$\phi$} exists such that,
    by Theorem 2.4.21a
    \begin{equation*}
        \sum_{\phi=1}^\lambda 2 \phi - 1 
            = 
        \lambda ^2
    \end{equation*}
    By the associative law for addition from the field axioms,
    \begin{equation*}
        \bigg \langle \sum_{\phi=1}^\lambda 2 \phi - 1 \bigg \rangle
            \equiv
        \bigg \langle 
            \sum_{\phi=1}^\lambda 2 \phi 
                +
            \sum_{\phi=1}^\lambda - 1
        \bigg \rangle
            \equiv
        \bigg \langle 
            -\lambda 
                +
            \sum_{\phi=1}^\lambda 2 \phi 
        \bigg \rangle
    \end{equation*}
    Thus, by that identity, and by the inverse law for addition from the field axioms,
    \begin{equation*}
        \Bigg\{
            \bigg \langle \lambda ^2 \bigg \rangle
                = 
            \bigg \langle 
                -\lambda 
                    + 
                \sum_{\phi=1}^\lambda 2 \phi 
            \bigg \rangle
        \Bigg\}
            \equiv 
        \Bigg\{
            \bigg \langle \sum_{\phi=1}^\lambda 2 \phi \bigg \rangle 
                = 
            \bigg \langle \lambda ^2 + \lambda \bigg \rangle 
                = 
            \lambda \bigg \langle \lambda + 1 \bigg \rangle
        \Bigg\}
    \end{equation*}
    By the distributive laws for real numbers from the field axioms, that is
    \begin{equation*}
        2 \sum_{\phi=1}^\lambda \phi 
            = 
        \lambda \Big \langle \lambda + 1 \Big \rangle
    \end{equation*} 
    And by the inverse law for multiplication from the field axioms,
    \begin{equation*}
        \bm{
            \sum_{\phi=1}^\lambda \phi 
                = 
            \frac{ \lambda [ \lambda + 1 ] }{2}
        }
    \end{equation*}
\end{proof}


\end{document}
\sep
\pagebreak


% ============================== 0103 Theorem 2422 ================================
\subsection[The summation of squares.]{
    \color{section} Theorem 103. \color{black} The summation of squares.
}
\input{../resources/LaTeX/2.4.sequences.and.summations/0103_2422/0103_2422.tex}
\sep
\pagebreak


% ============================== 0104 Theorem 2425 ================================
\subsection[The summation of floors of square roots.]{
    \color{section} Theorem 104. \color{black} The summation of floors of square roots.
}
\documentclass[preview]{standalone}
\usepackage{amssymb, amsthm}
\usepackage{mathtools}
\usepackage{bm}


\newtheorem{theorem}{Theorem}
\renewcommand\qedsymbol{$\blacksquare$}


\begin{document}


\begin{theorem}[\textbf{2425}]
    
    \raggedright Let \bm{$\phi$} be a positive integer such that
    \raggedright \bm{$\lfloor \sqrt \phi \rfloor = \lambda$}.
    \raggedright The closed form formula for 
    \raggedright \bm{$
        \Phi
            = 
        \sum_{\iota=0}^\phi 
                \big \lfloor \sqrt \iota \big \rfloor
    $} 
    is 
    \begin{equation*}
        \bm{
            \frac{
                \lambda
                \big \langle \lambda - 1 \big \rangle
            }
            {6}
            \Bigg[
                4 \lambda + 1
            \Bigg]
                +
            \lambda
            \Bigg[
                \phi
                    -
                \lambda ^2
                    +
                1
            \Bigg]
        }
    \end{equation*}
\end{theorem}

\begin{proof}
    By the associative law for addition from the field axioms,
    \begin{equation*}
        \Theta
            = 
        \sum_{\iota=0}^{\lambda - 1} 
                \big \lfloor \sqrt \iota \big \rfloor 
            =
    \end{equation*}
    \begin{equation*}
        \Bigg (
            \sum_{\iota=1}^{ 2 \beta_0 + 1 } 
                    \big \lfloor \sqrt \iota \big \rfloor_0 
        \Bigg )
            +
        \Bigg (
            \sum_{\iota=1 + 2 \beta_0 + 1}^{ 2 \beta_0 + 1 + 2 \beta_1 + 1 } 
                    \big \lfloor \sqrt \iota \big \rfloor_1 
        \Bigg )
            +
        \dots
            +
        \Bigg (
            \sum_{\iota=1 +  \dots + 2 \beta_{\langle \lambda - 2 \rangle} + 1 }
                ^{0 + \dots + 2 \beta_{\langle \lambda - 1 \rangle} + 1} 
                    \big \lfloor \sqrt \iota \big \rfloor_{
                        \langle \lambda - 1 \rangle
                    } 
        \Bigg )
    \end{equation*}
    \bm{$\big \lfloor \sqrt \iota \big \rfloor_\tau$}
    is the unique integer \bm{$\beta_\tau$}, 
    for \bm{$\tau = 0$} to \bm{$\big \langle \lambda - 1 \big \rangle$}.
    Hence, by Lemma 2403 and the distributive law for real numbers from the field axioms, 
    by the partitioning from above, and shifting the index of summation,
    \begin{equation*}
        \Theta
            =
        \Big \langle 2 \beta_{0} ^2 + \beta_0 \Big \rangle
            +
        \Big \langle 2 \beta_{1} ^2 + \beta_1 \Big \rangle
            +
        \dots
            +
        \Big \langle 
            2 \beta_{[ \lambda - 1 ]} ^2
                +
            \beta_{[ \lambda - 1 ]}
        \Big \rangle
    \end{equation*}
    By the commutative law for addition from the field axioms,
    that series is two times the finite summation of squares, 
    plus the finite summation of integers from zero to lambda minus one.
    \begin{equation*}
        \Theta
            = 
        \sum_{\tau=0}^{ \lambda - 1 } 2 \tau ^2 
            +
        \sum_{\tau=0}^{ \lambda - 1 } \tau
    \end{equation*}
    Lambda occurs exactly 
    \bm{$
        \big \langle 
            \phi - \lambda ^2 + 1
        \big \rangle
    $} 
    times in big phi.
    Thus, by Lemma 2402 and the distributive law for real numbers from the field axioms,
    \begin{equation*}
        \bm{
            \Phi
                =
            \frac{
                \lambda
                \big \langle \lambda - 1 \big \rangle
            }
            {6}
            \Bigg[
                4 \lambda + 1
            \Bigg]
                +
            \lambda
            \Bigg[
                \phi
                    -
                \lambda ^2
                    +
                1
            \Bigg]
        }
    \end{equation*}
\end{proof}


\end{document}
\sep
\pagebreak

% ============================== 0105 Theorem 2426 ================================
\subsection[The summation of floors of cube roots.]{
    \color{section} Theorem 105. \color{black} The summation of floors of cube roots.
}
\documentclass[preview]{standalone}
\usepackage{amssymb, amsthm}
\usepackage{mathtools}
\usepackage{bm}


\newtheorem{theorem}{Theorem}
\renewcommand\qedsymbol{$\blacksquare$}


\begin{document}


\begin{theorem}[\textbf{2426}]
    \raggedright Let \bm{$\phi$} be a positive integer such that 
    \raggedright \bm{$\big \lfloor \sqrt[3] \phi \big \rfloor = \lambda$}.
    \raggedright The closed form formula for 
    \raggedright \bm{
        $\Phi = \sum_{\iota=0}^\phi 
                \big \lfloor \sqrt[3] \iota \big \rfloor
    $} 
    is 
    \begin{equation*}
        \bm{
            \frac{
                \lambda ^3 - \lambda ^2
            }
            {4}
            \Bigg[ 3 \lambda + 1 \Bigg]
                + 
            \lambda
            \Bigg[
                \phi - \lambda ^3 + 1
            \Bigg]
        }
    \end{equation*}
\end{theorem}

\begin{proof}
    Let \bm{$\mathrm{X}$} be the function 
    \bm{$\mathrm{X}: \mathbb{N} \rightarrow \mathbb{N}$}
    such that \bm{$\mathrm{X}[\beta] = 3 \beta ^2 + 3 \beta + 1$}.
    By the associative law for addition from the field axioms,
    \begin{equation*}
        \Theta
            =
        \sum_{\iota=0}^{\lambda - 1}
                \big \lfloor \sqrt[3] \iota \big \rfloor
            =
    \end{equation*}
    \begin{equation*}
        \Bigg(
            \sum_{ \iota = 1 }^{ \mathrm{X} [\beta_0] }
                \big \lfloor \sqrt[3] \iota \big \rfloor_0
        \Bigg)
            +
        \Bigg(
            \sum_{
                \iota = 1 + \mathrm{X}[\beta_0]
            }^{ 
                \mathrm{X}[\beta_0] + \mathrm{X}[\beta_1]
            }
                \big \lfloor \sqrt[3] \iota \big \rfloor_1
        \Bigg)
            +
        \dots
            +
        \Bigg(
            \sum_{
                \iota = 1 + \dots 
                    + 
                \mathrm{X}[\beta_{\lambda - 2}]
            }^{ 
                0 + \dots
                    +
                \mathrm{X}[\beta_{ \lambda - 1 }]
            }
                \big \lfloor \sqrt[3] \iota \big \rfloor_{ \langle \lambda - 1 \rangle }
        \Bigg)
    \end{equation*}
    \bm{$\big \lfloor \sqrt[3] \iota \big \rfloor_\tau$}
    is the unique integer \bm{$\beta_\tau$},
    for \bm{$\tau = 0$} to 
    \bm{$\big \langle \lambda - 1 \big \rangle$}.
    Hence, by Lemma 13,
    by the partitioning from above, 
    and shifting the index of summation,
    \begin{equation*}
        \Theta
            =
        \Big \langle \beta_0 \mathrm{X} [\beta_0] \Big \rangle
            +
        \Big \langle \beta_1 \mathrm{X} [\beta_1] \Big \rangle
            + 
        \dots 
            +
        \Big \langle 
            \beta_{\langle \lambda - 1 \rangle}
            \mathrm{X} [\beta_{\langle \lambda - 1 \rangle}]
        \Big \rangle
    \end{equation*}
    By the distributive law for real numbers from the field axioms,
    and the commutative law for addition, that series is
    three times the finite summation of cubes,
    plus three times the finite summation of squares,
    plus the finite summation of integers from zero to lambda minus one.
    \begin{equation*}
        \Theta 
            = 
        \sum_{\tau=0}^{\lambda - 1} 3 \tau ^3
            +
        \sum_{\tau=0}^{\lambda - 1} 3 \tau ^2
            +
        \sum_{\tau=0}^{\lambda - 1} \tau
    \end{equation*}
    Lambda occurs exactly  
    \bm{$\big \langle \phi - \lambda^3 + 1 \big \rangle$}
    times in big phi. 
    Thus, by Lemma 14
    and the distributive law for real numbers from the field axioms,
    \begin{equation*}
        \bm{
            \Phi
                =
            \frac{
                \lambda ^3 - \lambda ^2
            }
            {4}
            \Bigg[ 3 \lambda + 1 \Bigg]
                + 
            \lambda
            \Bigg[
                \phi - \lambda ^3 + 1
            \Bigg]
        }
    \end{equation*}
\end{proof}


\end{document}
\sep
\pagebreak


% ============================== 0106 Theorem 2436 ================================
\subsection[Subsets of countable sets are countable.]{
    \color{section} Theorem 106. \color{black} Subsets of countable sets are countable.
}
\documentclass[preview]{standalone}
\usepackage{amssymb, amsthm}
\usepackage{mathtools}
\usepackage{bm}


\newtheorem{theorem}{Theorem}
\renewcommand\qedsymbol{$\blacksquare$}


\begin{document}


\begin{theorem}[\textbf{2436}]
    A subset of a countable set is countable.
\end{theorem}

\begin{proof}
    Let \bm{$\mathrm{A}$} and \bm{$\Lambda$} be sets such that 
    \bm{$\mathrm{A}$} is a subset of the countable set \bm{$\Lambda$}. 
    By the definition for countability,
    the cardinality of \bm{$\Lambda$} is less than or equal to \bm{$\aleph_0$}. 
    By the definition for subsets, 
    the cardinality of \bm{$\mathrm{A}$}
    is less than or equal to \bm{$\Lambda$}.
    Hence,
    the cardinality of \bm{$\mathrm{A}$} is less than or equal to \bm{$\aleph_0$}
    \bm{$\therefore$} the subset of a countable set is countable.
\end{proof}


\end{document}
\sep


% ============================== 0107 Theorem 2437 ================================
\subsection[Uncountable subsets imply an uncountable superset.]{
    \color{section} Theorem 107. \color{black} Uncountable subsets.
}
\documentclass[preview]{standalone}
\usepackage{amssymb, amsthm}
\usepackage{mathtools}
\usepackage{bm}


\newtheorem{theorem}{Theorem}
\renewcommand\qedsymbol{$\blacksquare$}


\begin{document}


\begin{theorem}[\textbf{2437}]
    Let \bm{$\mathrm{A}$}, and \bm{$\Lambda$} be sets such that 
    \bm{$\mathrm{A}$} is a subset of \bm{$\Lambda$}. 
    If \bm{$\mathrm{A}$} is uncountable, 
    then \bm{$\Lambda$} is uncountable.
\end{theorem}

\begin{proof}
    Direct proof.
    The cardinality for \bm{$\mathrm{A}$} is greater than \bm{$\aleph_0$},
    by the definition for countability. 
    By the definition of subsets, 
    the cardinality of \bm{$\Lambda$} is at least the cardinality 
    of \bm{$\mathrm{A}$}. 
    Hence, \bm{$\Lambda$} is uncountable, by the definition for countability.
\end{proof}


\end{document}
\sep


% ============================== 0108 Theorem 2438 ================================
\subsection[Power set cardinality.]{
    \color{section} Theorem 108. \color{black} Power set cardinality.
}
\documentclass[preview]{standalone}
\usepackage{amssymb, amsthm}
\usepackage{mathtools}
\usepackage{bm}


\newtheorem{theorem}{Theorem}
\renewcommand\qedsymbol{$\blacksquare$}


\begin{document}


\begin{theorem}[\textbf{2438}]
    Let \bm{$\mathrm{A}$}, and \bm{$\Lambda$} be sets with equal cardinality. 
    \begin{equation*}
        \bm{
            \Big| 
                \mathcal{P}\big \langle \mathrm{A} \big \rangle 
            \Big| 
                = 
            \Big|
                \mathcal{P} \big \langle \Lambda \big \rangle 
            \Big|
        }
    \end{equation*}
\end{theorem}

\begin{proof}
    By the hypothesis, and by the defintion for set cardinality,
    there exists an integer \bm{$\iota$} such that 
    \bm{$
        \big| \mathrm{A} \big| 
            = 
        \big| \Lambda \big| 
            = \iota
    $}. 
    The cardinality of a power set is $2$ to the power of the set cardinality. 
    Thus,
    \begin{equation*}
        \bm{
            \big | \mathcal{P} \big \langle \mathrm{A} \big \rangle \big | 
                = 
            \big | \mathcal{P} \big \langle \Lambda \big \rangle \big | 
                = 
            2^{\iota}
        }        
    \end{equation*} 
\end{proof}


\end{document}
\sep
\pagebreak


% ============================== 0109 Theorem 2440 ================================
\subsection[Union of countable sets is countable.]{
    \color{section} Theorem 109. \color{black} Union of countable sets is countable.
}
\input{../resources/LaTeX/2.4.sequences.and.summations/0109_2440/0109_2440.tex}
\sep
\pagebreak


% ============================== 0110 Theorem 2441 ================================
\subsection[A countable union of countable sets is countable.]{
    \color{section} Theorem 110. \color{black} A countable union.
}
\input{../resources/LaTeX/2.4.sequences.and.summations/0110_2441/0110_2441.tex}
\sep
\pagebreak


% ============================== 0111 Theorem 2442 ================================
\subsection[The Cartesian product of positive integers.]{
    \color{section} Theorem 111. \color{black} The cross product of positive integers.
}
\documentclass[preview]{standalone}
\usepackage{amssymb, amsthm}
\usepackage{mathtools}
\usepackage{bm}


\newtheorem{theorem}{Theorem}
\renewcommand\qedsymbol{$\blacksquare$}


\begin{document}


\begin{theorem}[\textbf{2442}]
    The cardinality of $\mathbb{Z^{+}} \times \mathbb{Z^{+}}$ is aleph null.
\end{theorem}

\begin{proof}
    $\mathbb{Z^{+}} \times \mathbb{Z^{+}}$ is defined as 
    $\{ \langle x, y \rangle | (x \in \mathbb{Z^{+}}) \land (y \in \mathbb{Z^{+}})\}$. 
    Since $x$ and $y$ are positive integers, for every ordered pair $\langle x, y \rangle$ in 
    $\mathbb{Z^{+}} \times \mathbb{Z^{+}}$, $\langle x, y \rangle$ exists if and only if the rational number 
    $\frac{x}{y}$ exists. Thus, $\frac{x}{y}$ exists, and all elements in 
    $\mathbb{Z^{+}} \times \mathbb{Z^{+}}$ can be represented by the two dimensional list
    $$\langle 1,1 \rangle \iff \frac{1}{1}, \langle 1,2 \rangle \iff \frac{1}{2}, \langle 1,3 \rangle \iff \frac{1}{3}, \dots$$
    $$\langle 2,1 \rangle \iff \frac{2}{1}, \langle 2,2 \rangle \iff \frac{2}{2}, \langle 2,3 \rangle \iff \frac{2}{3}, \dots$$
    $$\langle 3,1 \rangle \iff \frac{3}{1}, \langle 3,2 \rangle \iff \frac{3}{2}, \langle 3,3 \rangle \iff \frac{3}{3}, \dots$$
    $$\vdots$$
    The hypotheses in the biconditional converse statements for each list entry are the list 
    elements in the proof for the countability of rational numbers. That means 
    $\mathbb{Z^{+}} \times \mathbb{Z^{+}}$ is countable if and only if the rational numbers are 
    countable. We know the rational numbers are countable. Therefore the cardinality of 
    $\mathbb{Z^{+}} \times \mathbb{Z^{+}}$ is $\aleph_0$.
\end{proof}


\end{document}
\sep


% ============================== 0112 Theorem 2443 ================================
\subsection[The set of all finite bit strings is countable.]{
    \color{section} Theorem 110. \color{black} The set of all finite bit strings.
}
\input{../resources/LaTeX/2.4.sequences.and.summations/0112_2443/0112_2443.tex}


\section{Lemmas}
% ================================ 0010L Lemma 2401 =================================
\subsection[Lemma 10]{\color{section}Lemma 10}
\documentclass[preview]{standalone}
\usepackage{amssymb, amsthm}
\usepackage{mathtools}
\usepackage{bm}


\newtheorem{lemma}{Lemma}
\renewcommand\qedsymbol{$\blacksquare$}


\begin{document}


\begin{lemma}[\textbf{2401}]
    Let \bm{$\lambda$} be a positive integer.
    \begin{equation*}
        \bm{
            \frac{1}{3} 
            \bigg[ 
                \lambda^3 
                    + 
                3 \bigg \langle 
                    \frac{ \lambda [ \lambda + 1 ] }{2} 
                \bigg \rangle
                    - 
                    \lambda 
            \bigg] 
                =
            \bigg[
                \frac{
                    \lambda
                    \langle \lambda + 1 \rangle 
                    \langle 2 \lambda + 1 \rangle}
                    {6}
            \bigg]
        }
    \end{equation*}
\end{lemma}

\begin{proof}
    By the distributive laws for real numbers, 
    and by the associative law for multiplication,
    \begin{equation*}
        3 \bigg \langle \frac{ \lambda [ \lambda + 1 ] }{2} \bigg \rangle 
            = 
        \bigg \langle \frac{3 \lambda ^2 + 3 \lambda }{2} \bigg \rangle
    \end{equation*}
    Since the rational number \bm{$\frac{2}{2} = 1$} 
    by the inverse law for multiplication, 
    by the multiplicative identity law from the field axioms,
    (and by the identity established above,)
    \begin{equation*} 
            \bigg[ 
                \lambda ^3 
                    + 
                3 \bigg \langle \frac{ \lambda [ \lambda + 1 ] }{2} \bigg \rangle
                    - 
                \lambda 
            \bigg]
            =
            \bigg[
                \frac{ 2 \lambda ^3 }{2} \
                    + 
                \frac{ 3 \lambda ^2 + 3 \lambda }{2} 
                    - 
                \frac{ 2 \lambda }{2}
            \bigg]
    \end{equation*}
    By the distributive law for real numbers
    \begin{equation*}
            \frac{1}{3}
            \bigg[
                \frac{ 2 \lambda ^3 }{2} 
                    + 
                \frac{ 3 \lambda ^2 + 3 \lambda }{2} 
                    - 
                \frac{ 2 \lambda }{2}
            \bigg]
                =
            \bigg[
                \frac{ 2 \lambda ^3 + 3 \lambda ^2 + 3 \lambda - 2 \lambda }
                {6}
            \bigg]
    \end{equation*}
    Repeated factoring, 
    by the distributive laws for real numbers,
    completes the proof.
    \begin{equation*}
        \bigg[
            \frac{ 2 \lambda ^3 + 3 \lambda ^2 + 3 \lambda - 2 \lambda }{6}
        \bigg]
            =
        \bigg[
            \frac{ \lambda \langle 2 \lambda ^2 + 2 \lambda + \lambda + 1 \rangle }{6}
        \bigg]
            =
        \bigg[
            \frac{ \lambda \langle 2 \lambda [\lambda + 1] + [\lambda + 1] \rangle }{6}
        \bigg]
    \end{equation*}
    \begin{equation*}
            =
        \bm{
            \bigg[
                \frac{ \lambda \langle \lambda + 1 \rangle \langle 2 \lambda + 1 \rangle }{6}
            \bigg]
        }
    \end{equation*}
\end{proof}


\end{document}
\sep
\pagebreak


% ================================ 0011L Lemma 2402 =================================
\subsection[Lemma 11]{\color{section}Lemma 11}
\documentclass[preview]{standalone}
\usepackage{amssymb, amsthm}
\usepackage{mathtools}
\usepackage{bm}


\newtheorem{lemma}{Lemma}
\renewcommand\qedsymbol{$\blacksquare$}


\begin{document}


\begin{lemma}[\textbf{2402}]
    Let \bm{$\iota$} be a positive integer such that
    \bm{$\big \lfloor \sqrt{\iota} \big \rfloor = \lambda$}.
    \begin{equation*}
        \bm{
            \Bigg( 2\sum_{\phi=0}^{\lambda - 1} \phi ^2 \Bigg)
                + 
            \Bigg( \sum_{\phi=0}^{\lambda - 1} \phi \Bigg)
                \equiv
            \frac{
                \lambda
                \big \langle \lambda - 1 \big \rangle
            }
            {6}
            \Bigg[
                4 \lambda + 1
            \Bigg]
        }
    \end{equation*}
\end{lemma}

\begin{proof}
    Let \bm{$\Phi$} be two times the sum of squares from zero to \bm{$\lambda$} minus one.
    Let \bm{$\mathrm{X}$} be the sum of integers from zero to \bm{$\lambda$} minus one.
    By theorems 103, and 102, and by shifting the index of summation,
    \begin{equation*}
        \Bigg\{
            \Phi + \mathrm{X}
        \Bigg\}
            \equiv
        2
        \Bigg[
            \frac{ 
                \lambda [ \lambda - 1 ]
                [ 2 \langle \lambda - 1 \rangle + 1 ] 
            }
            {6}
        \Bigg]
            +
        \Bigg[
            \frac{\lambda [ \lambda - 1 ] }
            {2}
        \Bigg]
    \end{equation*}
    Since the rational number \bm{$\frac{3}{3} = 1$}
    by the inverse law for multiplication,
    by the multiplicative identity law from the field axioms
    \begin{equation*}
        \bigg \{ \mathrm{X} \bigg\}   
            \equiv
        \frac{
            3
            \lambda [ \lambda - 1 ]
        }
        {6}
    \end{equation*}
    Thus, 
    by the identity \bm{$\mathrm{X}$}, 
    factoring
    \bm{$\frac{1}{6} \lambda \big \langle \lambda - 1 \big \rangle$}
    out from the sum of \bm{$\Phi$} and \bm{$\mathrm{X}$}, 
    by the distributive laws for real numbers,
    \begin{equation*}
        \Bigg\{
            \Phi
                +
            \mathrm{X}
        \Bigg\}
            \equiv
        \frac{
            \lambda
            \big \langle \lambda - 1 \big \rangle
        }
        {6}
        \Bigg[
            2 \bigg \langle 2 [ \lambda - 1 ] + 1 \bigg \rangle
                +
            3
        \Bigg]
    \end{equation*}
    By the distributive laws for real numbers, 
    and by the associative law for addition from the field axioms,
    \begin{equation*}
        \bigg \{
            2 \Big \langle 2 [ \lambda - 1 ] + 1 \Big \rangle + 3
        \bigg \}
            =
        \bigg \{
            4 [ \lambda - 1 \big ] + 5
        \bigg \}
            =
        \bigg \{
            4 \lambda + 1
        \bigg \}
    \end{equation*}
    \bm{$\therefore$}
    \begin{equation*}
        \bm{
            \Bigg \{
                \Phi
                    +
                \mathrm{X}
            \Bigg \}
                \equiv
            \frac{
                \lambda
                \big \langle \lambda - 1 \big \rangle
            }
            {6}
            \Bigg[
                4 \lambda + 1
            \Bigg]
        }
    \end{equation*}
\end{proof}


\end{document}
\sep
\pagebreak


% ================================ 0012L Lemma 2403 =================================
\subsection[Lemma 12]{\color{section}Lemma 12}
\documentclass[preview]{standalone}
\usepackage{amssymb, amsthm}
\usepackage{mathtools}
\usepackage{bm}


\newtheorem{lemma}{Lemma}
\renewcommand\qedsymbol{$\blacksquare$}


\begin{document}


\begin{lemma}[\textbf{2403}]
    Let \bm{$\iota$} be a natural number such that
    \bm{$\big \lfloor \sqrt \iota \big \rfloor = \lambda$}.
    \begin{equation*}
        \bm{
            \sum_{\iota=1}^{2 \lambda + 1} 
                    \big \lfloor \sqrt \iota \big \rfloor
                =
            \lambda \big[ 2 \lambda + 1 \big]
        }
    \end{equation*}
\end{lemma}

\begin{proof}
    It is trivial that
    \begin{equation*}
        \sum_{\iota=1}^{2 \lambda + 1} 
                1 = 2 \lambda + 1
    \end{equation*}
    By the inverse law for multiplication from the field axioms,
    by the distributive law for real numbers, and by the identity \bm{$\lambda$},
    that is
    \begin{equation*}
        \Bigg\{
            \lambda \sum_{\iota=1}^{2 \lambda + 1} 1 = \lambda \big[ 2 \lambda + 1 \big]
        \Bigg\}
                \equiv
        \Bigg\{
            \sum_{\iota=1}^{2 \lambda + 1} \lambda = \lambda \big[ 2 \lambda + 1 \big]
        \Bigg\}
            \equiv
        \Bigg\{
            \sum_{\iota=1}^{2 \lambda + 1} \big \lfloor \sqrt \iota \big \rfloor
                = 
            \lambda \big[ 2 \lambda + 1 \big]
        \Bigg\}
    \end{equation*}
\end{proof}


\end{document}
\sep


% ================================ 0013L Lemma 2404 =================================
\subsection[Lemma 13]{\color{section}Lemma 13}
\documentclass[preview]{standalone}
\usepackage{amssymb, amsthm}
\usepackage{mathtools}
\usepackage{bm}


\newtheorem{lemma}{Lemma}
\renewcommand\qedsymbol{$\blacksquare$}


\begin{document}


\begin{lemma}[\textbf{2404}]
    Let \bm{$\iota$} be a natural number such that
    \bm{$\big \lfloor \sqrt[3] \iota \big \rfloor = \lambda$}.
    \begin{equation*}
        \bm{
            \sum_{\iota=1}^{ 3 \lambda ^2 + 3 \lambda + 1 } 
                    \big \lfloor \sqrt[3] \iota \big \rfloor
                =
            \lambda \big [ 3 \lambda ^2 + 3 \lambda + 1 \big ]
        }
    \end{equation*}
\end{lemma}

\begin{proof}
    Similiar to Lemma 2403, it is trivial that
    \begin{equation*}
        \sum_{\iota=1}^{ 3 \lambda ^2 + 3 \lambda + 1 } 
                    1
                =
            3 \lambda ^2 + 3 \lambda + 1
    \end{equation*}
    By the inverse law for multiplication from the field axioms,
    by the distributive law for real numbers, 
    and by the identity lambda, that is
    \begin{equation*}
        \Bigg \{
            \lambda \sum_{\iota=1}^{ 3 \lambda ^2 + 3 \lambda + 1 } 
                        1
                    =
            \lambda \big [ 3 \lambda ^2 + 3 \lambda + 1 \big ]
        \Bigg \}
            \equiv
        \Bigg \{
            \sum_{\iota=1}^{ 3 \lambda ^2 + 3 \lambda + 1 } 
                        \lambda
                    =
            \lambda \big [ 3 \lambda ^2 + 3 \lambda + 1 \big ]
        \Bigg \}
            \equiv
    \end{equation*}
    \begin{equation*}
        \Bigg \{
            \sum_{\iota=1}^{ 3 \lambda ^2 + 3 \lambda + 1 } 
                        \big \lfloor \sqrt[3] \iota \big \rfloor
                    =
            \lambda \big [ 3 \lambda ^2 + 3 \lambda + 1 \big ]
        \Bigg \}
    \end{equation*}
\end{proof}


\end{document}
\sep
\pagebreak


% ================================ 0014L Lemma 2405 =================================
\subsection[Lemma 14]{\color{section}Lemma 14}
\documentclass[preview]{standalone}
\usepackage{amssymb, amsthm}
\usepackage{mathtools}
\usepackage{bm}


\newtheorem{lemma}{Lemma}
\renewcommand\qedsymbol{$\blacksquare$}


\begin{document}


\begin{lemma}[\textbf{2405}]
    Let \bm{$\iota$} be a positive integers such that
    \bm{$\big \lfloor \sqrt[3] \iota \big \rfloor = \lambda$}.
    \begin{equation*}
        \bm{
            \Bigg(
                3 \sum_{\phi=0}^{\lambda - 1} \phi ^3
            \Bigg)
                +
            \Bigg(
                3 \sum_{\phi=0}^{\lambda - 1} \phi ^2
            \Bigg)
                +
            \Bigg(
                \sum_{\phi=0}^{\lambda - 1} \phi
            \Bigg)
                \equiv
            \frac{
                \lambda ^3 - \lambda ^2
            }
            {4}
            \Bigg[ 3 \lambda + 1 \Bigg]
        }
    \end{equation*}
\end{lemma}

\begin{proof}
    Let \bm{$\Phi$} be three times the summation of cubes from zero to lambda minus one.
    Let \bm{$\mathrm{X}$} be three times the summation of squares from zero to lambda minus one.
    Let \bm{$\Omega$} be the summation of integers from zero to lambda minus one.
    By theorems 2422 and 2421b, 
    by the closed formula for the summation of cubes,
    and by shifting the index of summation,
    \begin{equation*}
        \Bigg \{
            \Phi + \mathrm{X} + \Omega
        \Bigg \}
            \equiv
        3
        \Bigg[
            \frac{
                \lambda ^2
                \big[ \lambda - 1 \big] ^2
            }
            {4}
        \Bigg]
            +
        3
        \Bigg[
            \frac{
                \lambda 
                \big[ \lambda - 1 \big]
                \big[ 2 \langle \lambda - 1 \rangle + 1 \big]
            }
            {6}
        \Bigg]
            +
        \Bigg[
            \frac{\lambda \big [ \lambda - 1 ]}
            {2}
        \Bigg]
    \end{equation*} 
    By the multiplicative identity law from the field axioms,
    \begin{equation*}
        \bigg \langle 
            \frac{3}{6}
        \bigg \rangle
             = 
        \bigg \langle
            \frac{2 \cdot 3}{2 \cdot 6} 
        \bigg \rangle 
            = 
        \bigg \langle
            \frac{6}{12} 
        \bigg \rangle 
            = 
        \bigg \langle
            \frac{2}{4}
        \bigg \rangle
    \end{equation*}
    \begin{equation*}
        \bigg \langle 
            \frac{1}{2}
        \bigg \rangle
            =
        \bigg \langle
            \frac{2 \cdot 1}{2 \cdot 2}
        \bigg \rangle
            =
        \bigg \langle 
            \frac{2}{4}
        \bigg \rangle
    \end{equation*}
    Thus, by the identities for \bm{$\mathrm{X}$} and \bm{$\Omega$},
    factoring 
    \bm{$\frac{1}{4} \lambda \big \langle \lambda - 1 \big \rangle$}
    out from the sum of \bm{$\Phi$}, \bm{$\mathrm{X}$}, and \bm{$\Omega$},
    by the distributive laws from real numbers,
    \begin{equation*}
        \Bigg \{
            \Phi + \mathrm{X} + \Omega
        \Bigg \}
                \equiv
            \frac{
                \lambda \big \langle \lambda - 1 \big \rangle
            }
            {4}
            \Bigg[
                3 \lambda \big[ \lambda - 1 \big]
                    +
                2 \bigg \langle 
                    2 \big[ \lambda - 1 \big] + 1
                \bigg \rangle
                    +
                2
            \Bigg]    
    \end{equation*}
    By the distributive law for real numbers, 
    and by the inverse law for addition from the field axioms,
    \begin{equation*}
        \bigg \{
            2 \Big \langle 2 \big[ \lambda - 1 \big] + 1 \Big \rangle + 2
        \bigg \}
            =
        \bigg \{
            4 \big[ \lambda - 1 \big] + 2 + 2
        \bigg \}
            =
        \bigg \{
            4 \lambda - 4 + 4
        \bigg \}
            = 
        \bigg \{
            4 \lambda
        \bigg \}
    \end{equation*}
    Hence, by the identity four lambda,
    \begin{equation*}
        \Bigg \{
            \Phi + \mathrm{X} + \Omega
        \Bigg \}
            \equiv
        \frac{
            \lambda \big \langle \lambda - 1 \big \rangle
        }
        {4}
        \Bigg[
            3 \lambda \big[ \lambda - 1 \big]
                +
            4 \lambda
        \Bigg]
    \end{equation*}
    Factoring out lambda and distributing three,
    by the distributive laws for real numbers,
    \begin{equation*}
        \Bigg\{
            \Phi + \mathrm{X} + \Omega
        \Bigg\}
                \equiv
            \frac{
                \lambda ^2 \big \langle \lambda - 1 \big \rangle
            }
            {4}
            \Bigg[
                3 \lambda - 3 + 4
            \Bigg]
    \end{equation*}
    The proof is complete, 
    by the distributive law for real numbers distributing
    the second power of lambda,
    and the inverse law for addition from the field axioms.
\end{proof}


\end{document}
\pagebreak


\end{document}