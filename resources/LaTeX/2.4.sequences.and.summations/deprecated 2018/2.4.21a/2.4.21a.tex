\documentclass[a4paper, 12pt]{article}
\usepackage[utf8]{inputenc}
\usepackage[english]{babel}
\usepackage{amssymb, amsmath, amsthm}
\theoremstyle{plain}
\newtheorem*{theorem*}{Theorem}
\newtheorem{theorem}{Theorem}

\usepackage{mathtools}
\renewcommand\qedsymbol{$\blacksquare$}
\DeclarePairedDelimiter{\floor}{\lfloor}{\rfloor}
\DeclarePairedDelimiter{\ceil}{\lceil}{\rceil}

\begin{document}
	
\begin{theorem*}[\textbf{2.4.21a}]
    The summation of odd numbers from 1 to n is $n^{2}$.
\end{theorem*}

\begin{proof}
    The summation of odd numbers from $1$ to $n$ is given by, $$\sum_{k=1}^{n} 2k - 1$$
    by the
    definition for odd numbers. The identity of $2k - 1$ is the difference of squares 
    $k^2 - (k-1)^2$. This identity can be demonstrated by the statement 
    $$k^2 - (k-1)^2 = [k + (k-1)][k - (k-1)] = (2k - 1)[k + (-k + 1)] = (2k-1)1$$ 
    So the summation of odd numbers from $1$ to $n$ is the telescoping summation
    $$\sum_{k=1}^{n} k^2 - (k-1)^2$$
    By Theorem 2.4.19, that is $n^2 - 0^2 = n^2$. Thus, 
    $$\sum_{k=1}^{n} 2k - 1 = n^2$$ 
    and indeed the summation of odd numbers from $1$ to $n$ is $n^2$.
\end{proof}

\end{document}
