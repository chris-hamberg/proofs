\documentclass[a4paper, 12pt]{article}
\usepackage[utf8]{inputenc}
\usepackage[english]{babel}
\usepackage{amssymb, amsmath, amsthm}
\theoremstyle{plain}
\newtheorem*{theorem*}{Theorem}
\newtheorem{theorem}{Theorem}

\usepackage{mathtools}
\renewcommand\qedsymbol{$\blacksquare$}
\DeclarePairedDelimiter{\floor}{\lfloor}{\rfloor}
\DeclarePairedDelimiter{\ceil}{\lceil}{\rceil}

\begin{document}
	
\begin{theorem*}[\textbf{2.4.25}]
    Let m be a positive integer. The closed form formula for $\sum_{k=0}^{m} \floor{\sqrt{k}}$ 
    is $\floor{\sqrt{m}}[\frac{1}{6}(\floor{\sqrt{m}}-1)(4\floor{\sqrt{m}}+1) + 
    (m - \floor{\sqrt{m}}^{2} + 1)]$.
\end{theorem*}

\begin{proof}
    By the properties for floor functions, there exists an integer $n = \floor{\sqrt{k}}$ 
    if and only if $n^{2} \le k < n^{2} + 2n + 1$. Thus, each integer value 
    $n < \floor{\sqrt{m}}$ occurs exactly $2n + 1$ times, in the terms of summation. The value
    $n = \floor{\sqrt{m}}$ occurs exactly $(m - \floor{\sqrt{m}}^{2} + 1)$ times. Subtracting 
    those terms $n = \floor{\sqrt{m}}$ from $\sum_{k=0}^{m} \floor{\sqrt{k}}$ produces the 
    sequence $$\floor{\sqrt{0}}(2\floor{\sqrt{0}} + 1) + \dots + 
    (\floor{\sqrt{m}}-1)[2(\floor{\sqrt{m}}-1) + 1]$$
    Summarily expressed as $$\sum_{n=0}^{\floor{\sqrt{m}}-1} n(2n + 1)$$ Thus,
    $$\sum_{k=0}^{m} \floor{\sqrt{k}} = 
    \left(\sum_{n=0}^{\floor{\sqrt{m}}-1} n(2n + 1)\right) + 
    \floor{\sqrt{m}}(m - \floor{\sqrt{m}}^{2} + 1)$$
    That is, the summation of squares, and integers
    $$\sum_{k=0}^{m} \floor{\sqrt{k}} = 
    \left(2\sum_{n=0}^{\floor{\sqrt{m}}-1} n^{2}\right) + 
    \left(\sum_{n=0}^{\floor{\sqrt{m}}-1} n\right) + 
    [\floor{\sqrt{m}}(m - \floor{\sqrt{m}}^{2} + 1)]$$
    By Theorem 2.4.22, and by Theorem 2.4.21b
    $$\sum_{k=0}^{m} \floor{\sqrt{k}} = 
    \left\{\frac{2}{6}\floor{\sqrt{m}}(\floor{\sqrt{m}}-1)[2(\floor{\sqrt{m}}-1)+1]\right\}+$$
    $$\left\{\frac{3}{6}\floor{\sqrt{m}}(\floor{\sqrt{m}}-1)\right\} + \floor{\sqrt{m}}(m - 
    \floor{\sqrt{m}}^{2} + 1)$$
    Factoring $\frac{1}{6}\floor{\sqrt{m}}(\floor{\sqrt{m}}-1)$ out of the first two terms 
    yields 
    $$\frac{1}{6}\floor{\sqrt{m}}(\floor{\sqrt{m}}-1)\{2[2(\floor{\sqrt{m}}-1)+1] +3\}+
    \floor{\sqrt{m}}(m - \floor{\sqrt{m}}^{2} + 1)$$
    And by arithmetic simplification that is
    $$\frac{1}{6}\floor{\sqrt{m}}(\floor{\sqrt{m}}-1)(4\floor{\sqrt{m}} + 1) + 
    \floor{\sqrt{m}}(m - \floor{\sqrt{m}}^{2} + 1)$$
    Factoring $\floor{\sqrt{m}}$ from the outer sum completes the derivation
    $$\sum_{k=0}^{m} \floor{\sqrt{k}} = 
    \floor{\sqrt{m}}\left[\frac{1}{6}(\floor{\sqrt{m}}-1)(4\floor{\sqrt{m}}+1) + 
    (m - \floor{\sqrt{m}}^{2} + 1)\right]$$.
\end{proof}


\end{document}
