\documentclass[a4paper, 12pt]{article}
\usepackage[utf8]{inputenc}
\usepackage[english]{babel}
\usepackage{amssymb, amsmath, amsthm}
\theoremstyle{plain}
\newtheorem*{theorem*}{Theorem}
\newtheorem{theorem}{Theorem}

\usepackage{mathtools}
\renewcommand\qedsymbol{$\blacksquare$}
\DeclarePairedDelimiter{\floor}{\lfloor}{\rfloor}
\DeclarePairedDelimiter{\ceil}{\lceil}{\rceil}

\begin{document}
	
\begin{theorem*}[\textbf{2.4.41}]
    The union of a countable number of countable sets is countable.
\end{theorem*}

\begin{proof}
    Let $A_i$ be a countable set, for integers $i=0$ to $n \leq \infty$ 
    such that
    $$S = \bigcup_{i=0}^{n} A_i$$ 
    The function $f: \mathbb{N} \rightarrow A_i$ is the sequence 
    $\{a_{ij}\} = a_{i0}, a_{i1}, a_{i2}, \dots$. 
    Thus, by $f$, all elements $a_{ij}$ in $S$ can be listed in the second dimension
    $$a_{00}, a_{01}, a_{02}, \dots$$
    $$a_{10}, a_{11}, a_{12}, \dots$$
    $$a_{20}, a_{21}, a_{22}, \dots$$
    $$\vdots$$
    By tracing the diagonal path along the two 
    dimensional listing for $S$ we get the countable order 
    $$a_{00}, a_{01}, a_{10}, a_{20}, a_{11}, a_{02}, \dots$$
    $\therefore |S| \leq \aleph_0$, and indeed the union of a countable number of 
    countable sets is countable.
\end{proof}

\end{document}
