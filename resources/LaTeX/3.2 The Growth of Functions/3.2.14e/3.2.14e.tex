\documentclass[a4paper, 12pt]{article}
\usepackage[utf8]{inputenc}
\usepackage[english]{babel}
\usepackage{amssymb, amsmath, amsthm}
\theoremstyle{plain}
\newtheorem*{theorem*}{Theorem}
\newtheorem{theorem}{Theorem}

\usepackage{mathtools}
\renewcommand\qedsymbol{$\blacksquare$}
\DeclarePairedDelimiter{\floor}{\lfloor}{\rfloor}
\DeclarePairedDelimiter{\ceil}{\lceil}{\rceil}

\begin{document}
	
	\begin{theorem*}[3.2.14e]
		Let f be the function defined by f(x) = $x^{3}$, and let g be the function defined by g(x) = $3^{x}$. f(x) is $\mathcal{O}(g(x))$.
	\end{theorem*}
	
	\begin{proof}
		If $f(x)$ is $\mathcal{O}(g(x))$, then there exists constant witnesses $C$ and $k$ such that $x^{3} \le C \cdot 3^{x}$, for all $x > k$. If $C = 1$ and $k = 1$, $x^{3}$ is a decreasing function with respect to $3^{x}$. Thus, $f(x)$ is $\mathcal{O}(g(x))$.
	\end{proof}

\end{document}
