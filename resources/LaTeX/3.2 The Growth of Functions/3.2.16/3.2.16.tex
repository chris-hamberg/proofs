\documentclass[a4paper, 12pt]{article}
\usepackage[utf8]{inputenc}
\usepackage[english]{babel}
\usepackage{amssymb, amsmath, amsthm}
\theoremstyle{plain}
\newtheorem*{theorem*}{Theorem}
\newtheorem{theorem}{Theorem}

\usepackage{mathtools}
\renewcommand\qedsymbol{$\blacksquare$}
\DeclarePairedDelimiter{\floor}{\lfloor}{\rfloor}
\DeclarePairedDelimiter{\ceil}{\lceil}{\rceil}

\begin{document}
	
	\begin{theorem*}[3.2.16]
		If f is a function such that f(x) is $\mathcal{O}(x)$, then f(x) is $\mathcal{O}(x^{2})$.
	\end{theorem*}
	
	\begin{proof}
		If $f(x)$ is $\mathcal{O}(x)$, then there exists constant witnesses $C$ and $k$ such that $|f(x)| \le C|x|$, for all $x > k$. Clearly, $C|x| \le C|x^{2}|$. Thus, $|f(x)| \le C|x^{2}|$, for all $x > k$. It follows that $f(x)$ is $\mathcal{O}(x^{2})$.
	\end{proof}

\end{document}
