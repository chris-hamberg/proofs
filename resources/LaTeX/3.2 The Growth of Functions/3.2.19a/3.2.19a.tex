\documentclass[a4paper, 12pt]{article}
\usepackage[utf8]{inputenc}
\usepackage[english]{babel}
\usepackage{amssymb, amsmath, amsthm}
\theoremstyle{plain}
\newtheorem*{theorem*}{Theorem}
\newtheorem{theorem}{Theorem}

\usepackage{mathtools}
\renewcommand\qedsymbol{$\blacksquare$}
\DeclarePairedDelimiter{\floor}{\lfloor}{\rfloor}
\DeclarePairedDelimiter{\ceil}{\lceil}{\rceil}

\begin{document}
	
	\begin{theorem*}[3.2.19a]
		Let f be the function defined by f(n) = $(n^{2} + 8)(n + 1)$. \newline f(n) is $\mathcal{O}(n^{3})$.
	\end{theorem*}
	
	\begin{proof}
		$f(n)$ is the product of functions $(f^\prime f^{\prime\prime})(n)$ where $f^\prime(n) = (n^{2} + 8)$, and $f^{\prime\prime} = (n + 1)$. Since a $k \textsuperscript{th}$ degree polynomial is $\mathcal{O}(x^{k})$, it follows that $f^\prime(n)$ is $\mathcal{O}(n^{2})$, and $f^{\prime\prime}(n)$ is $\mathcal{O}(n)$. The upper bound for a product of functions is the product of the bounding functions for each function occurring in the product of functions. Hence, the upper bound for $f(n)$ is $\mathcal{O}(n(n^{2}))$. This means that $f(n)$ is $\mathcal{O}(n^{3})$.
	\end{proof}

\end{document}
