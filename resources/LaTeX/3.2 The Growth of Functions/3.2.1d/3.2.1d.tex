\documentclass[a4paper, 12pt]{article}
\usepackage[utf8]{inputenc}
\usepackage[english]{babel}
\usepackage{amssymb, amsmath, amsthm}
\theoremstyle{plain}
\newtheorem*{theorem*}{Theorem}
\newtheorem{theorem}{Theorem}

\usepackage{mathtools}
\renewcommand\qedsymbol{$\blacksquare$}
\DeclarePairedDelimiter{\floor}{\lfloor}{\rfloor}
\DeclarePairedDelimiter{\ceil}{\lceil}{\rceil}

\begin{document}
	
	\begin{theorem*}[3.2.1d]
		Let f be the function defined by f(x) = 5 $\log x$. \newline f(x) is $\mathcal{O} (x)$.
	\end{theorem*}
	
	\begin{proof}
		Let $g$ be the function defined by $g(x) = x$. It is clear that the inequality $|5 \log x| \le 5|x|$ is true for all $x > 1$. Therefore, $|f(x)| \le 5|g(x)|$, for all $x > 1$. It follows from the definition of big-O notation that $f(x)$ is $\mathcal{O} (x)$ with constant witnesses $C = 5$, and $k = 1$.
	\end{proof}

\end{document}
