\documentclass[a4paper, 12pt]{article}
\usepackage[utf8]{inputenc}
\usepackage[english]{babel}
\usepackage{amssymb, amsmath, amsthm}
\theoremstyle{plain}
\newtheorem*{theorem*}{Theorem}
\newtheorem{theorem}{Theorem}

\usepackage{mathtools}
\renewcommand\qedsymbol{$\blacksquare$}
\DeclarePairedDelimiter{\floor}{\lfloor}{\rfloor}
\DeclarePairedDelimiter{\ceil}{\lceil}{\rceil}

\begin{document}
	
	\begin{theorem*}[3.2.20b]
		Let f be the function defined by $f(n) = (2^n + n^2)(n^3 + 3^n)$. $f(x)$ is $\mathcal{O}(6^n)$.
	\end{theorem*}
	
	\begin{proof}
		Let $g$ be the function defined by $g(n) = 6^n$. If $n \ge 4$, then \newline \newline \indent $f(n) = \left( 2^n n^3 + 2^n 3^n + n^2 n^3 + 3^n n^2 \right) \le \left( 2^n 3^n + 2^n 3^n + 2^n 3^n + 2^n 3^n \right).$ \newline \newline $4(2^n 3^n) = 4(6^n)$, so $f(n)$ is $\mathcal{O}(6^n)$ with constant witnesses $C = 4$ and $k = 4$.
	\end{proof}

\end{document}
