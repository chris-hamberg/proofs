\documentclass[a4paper, 12pt]{article}
\usepackage[utf8]{inputenc}
\usepackage[english]{babel}
\usepackage{amssymb, amsmath, amsthm}
\theoremstyle{plain}
\newtheorem*{theorem*}{Theorem}
\newtheorem{theorem}{Theorem}

\usepackage{mathtools}
\renewcommand\qedsymbol{$\blacksquare$}
\DeclarePairedDelimiter{\floor}{\lfloor}{\rfloor}
\DeclarePairedDelimiter{\ceil}{\lceil}{\rceil}

\begin{document}
	
	\begin{theorem*}[3.2.21a]
		Let f be the function f(n) = $n \log(n^2 + 1) + n^2 \log n$. \newline f(n) is $\mathcal{O}(n^2 \log n)$.
	\end{theorem*}
	
	\begin{proof}
		$f(n)$ is $\mathcal{O}(n^2 \log n)$ follows directly from the fact that a $k^\textsuperscript{th}$ degree polynomial is $\mathcal{O}(x^k)$. Since $n \log(n^2 + 1) + n^2 \log n$ is a $2^\textsuperscript{nd}$ degree polynomial in $n^2 \log n$ with a constant coefficient $1$, $f(n)$ is $\mathcal{O}(n^2 \log n)$.
	\end{proof}

\end{document}
