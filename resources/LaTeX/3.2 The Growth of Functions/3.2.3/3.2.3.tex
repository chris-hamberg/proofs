\documentclass[a4paper, 12pt]{article}
\usepackage[utf8]{inputenc}
\usepackage[english]{babel}
\usepackage{amssymb, amsmath, amsthm}
\theoremstyle{plain}
\newtheorem*{theorem*}{Theorem}
\newtheorem{theorem}{Theorem}

\usepackage{mathtools}
\renewcommand\qedsymbol{$\blacksquare$}
\DeclarePairedDelimiter{\floor}{\lfloor}{\rfloor}
\DeclarePairedDelimiter{\ceil}{\lceil}{\rceil}

\begin{document}
	
	\begin{theorem*}[3.2.3]
		Let f be the function defined by f(x) = $x^{4} + 9x^{3} + 4x + 7$. \newline f(x) is $\mathcal{O}(x^{4})$.
	\end{theorem*}
	
	\begin{proof}
		Let $g$ be the function defined by $g(x) = x^{4}$. If $x \ge 2$, then \newline $x^{4} + 9x^{3} + 4x + 7 \le x^{4} + 9x^{4} + 4x^{4} + x^{4} = 15x^{4}$. Thus, $|f(x)| \le 15|g(x)|$, for all $x > 2$. It immediately follows from the definition that $f(x)$ is $\mathcal{O}(x^{4})$ with constant witnesses $C = 15$, and $k = 2$.
	\end{proof}

\end{document}
